\chapter{Analytical mechanics of particle systems}
\paragraph{Goal.}
Established approaches of analytical mechanics commonly rely on the parameterization of the system in terms of \textit{minimal} generalized coordinates $\genCoord$ and their derivatives $\genCoordd$.
In this section we like to generalize this to handle redundant configuration coordinates $\sysCoord(t) \in \configSpace$ and nonholonomic velocity coordinates $\sysVel(t)\in\RealNum^\dimConfigSpace$ as introduced in the previous section.
%In contrast to this we will allow \textit{redundant} generalized coordinates $\sysCoord$, i.e.\ coordinates that may not be completely free, but may themselves be constrained by equations of the form $\geoConstraint(\sysCoord) = \tuple{0}$.
%Furthermore we will consider \textit{velocity coordinates} $\sysVel$ that parameterize the change $\sysCoordd$ of the configuration coordinates.
%In this chapter we like to review some basic concepts of analytical mechanics for systems of constrained particles with this more general parameterization.
The resulting formulations might become more cumbersome, but with some examples we like to show that it is worth it.
%\cite[§1]{Landau:Mechanics}: One of the fundamental concepts of mechanics is that of a \textit{particle}.

\section{System under consideration}\label{sec:DefParticleSys}
\paragraph{Particle positions.} For this chapter we consider a system of $\numParticles$ particles under geometric constraints:
The \textit{position} of a particle in a given reference frame at a given time $t$ is $\particlePos{\PidxI}(t) \in \RealNum^3, \PidxI = 1,\ldots,\numParticles$ and the collection of all particle positions is $\particleCoord = [\particlePos{1}^\top,\ldots,\particlePos{\numParticles}^\top]^\top \in \RealNum^{3\numParticles}$.
\textit{Geometric constraints} on the particles are captured in $\numParticleConstraints \geq 0$ smooth functions of the form $\particleGeoConstraint(\particleCoord) = [\particleGeoConstraintCoeff{1}(\particleCoord), \ldots, \particleGeoConstraintCoeff{\numParticleConstraints}(\particleCoord)]^\top = \tuple{0}$.
The set of all mutually admissible particle positions 
\begin{align}
\particleConfigSpace = \{ \particleCoord \in \RealNum^{3\numParticles} \, | \, \particleGeoConstraint(\particleCoord) = \tuple{0} \} 
\end{align}
is called the \textit{configuration space}.
We require $\pdiff[\particleGeoConstraint]{\particleCoord}(\particleCoord)$ to have a constant, though not necessarily full rank.
The dimension of the configuration space is $\dim\particleConfigSpace = 3\numParticles - \rank\pdiff[\particleGeoConstraint]{\particleCoord} = \dimConfigSpace$.

\paragraph{Coordinates.}
Now let the admissible particle positions $\particleCoord \in \particleConfigSpace$ be parameterized $\particlePos{\PidxI} = \particlePos{\PidxI}(\sysCoord)$ by possibly redundant coordinates $\sysCoord \in \configSpace$.
This means $\geoConstraint(\sysCoord) = \tuple{0}\,\Rightarrow\,\particleGeoConstraint(\particleCoord(\sysCoord)) = \tuple{0}$ and consequently $\sysCoord \in \configSpace \, \Rightarrow \, \particleCoord(\sysCoord) \in \particleConfigSpace$.

With a suitable kinematics matrix $\kinMat(\sysCoord)$, as discussed in the previous section, we can parameterize the particle velocity as 
\begin{align}
\particlePosd{\PidxI} = \pdiff[\particlePos{\PidxI}]{\sysCoord} \kinMat \sysVel = \dirDiff{\LidxI} \particlePos{\PidxI} \sysVelCoeff{\LidxI} 
\end{align}
with minimal velocity coordinates $\sysVel \in \RealNum^{\dimConfigSpace}$.
Finally we may express the particle accelerations as
\begin{align}
 \particlePosdd{\PidxI} &= \dirDiff{\LidxI} \particlePos{\PidxI}(\sysCoord) \sysVelCoeffd{\LidxI} + \dirDiff{\LidxII} \dirDiff{\LidxI} \particlePos{\PidxI}(\sysCoord) \sysVelCoeff{\LidxI} \sysVelCoeff{\LidxII},
\end{align}
From the parameterization it is evident that
\begin{align}\label{eq:IdentityDifferentials}
 \dirDiff{\LidxI} \particlePos{\PidxI} = \pdiff[\particlePosd{\PidxI}]{\sysVelCoeff{\LidxI}} = \pdiff[\particlePosdd{\PidxI}]{\sysVelCoeffd{\LidxI}}.
\end{align}


% \begin{Example}\label{Example:ThreeParticles}
% Consider a system of $\numParticles = 3$ particles as illustrated in \autoref{fig:OrientationExample}.
% 
% \begin{minipage}{\textwidth}
%  \centering
%  \input{graphics/OrientationExample.pdf_tex}
%  \captionof{figure}{Three particles}
%  \label{fig:OrientationExample}
% \end{minipage}
% 
% Their distance to the coordinate origin and to each other are constrained by the geometric constraints
% \begin{subequations}\label{eq:ThreeParticleConstraint}
% \begin{align}
%  \label{eq:ThreeParticleConstraint1}
%  \norm{\particlePos{1}} &= l_1,&
%  \norm{\particlePos{2}} &= l_2,&
%  \norm{\particlePos{3}} &= l_3,
% \\
%  \label{eq:ThreeParticleConstraint2}
%  \norm{\particlePos{2} - \particlePos{3}}^2 &= l_2^2 + l_3^2,&
%  \norm{\particlePos{1} - \particlePos{3}}^2 &= l_1^2 + l_3^2,&
%  \norm{\particlePos{1} - \particlePos{2}}^2 &= l_1^2 + l_2^2.
% %  \norm{\particlePos{2} - \particlePos{3}} &= \sqrt{l_2^2 + l_3^2},&
% %  \norm{\particlePos{1} - \particlePos{3}} &= \sqrt{l_1^2 + l_3^2},&
% %  \norm{\particlePos{1} - \particlePos{2}} &= \sqrt{l_1^2 + l_2^2}.
% \end{align}
% % \begin{align}
% %  \sProd{\particlePos{2}}{\particlePos{3}} &= 0,&
% %  \sProd{\particlePos{1}}{\particlePos{3}} &= 0,&
% %  \sProd{\particlePos{1}}{\particlePos{2}} &= 0.
% % \end{align}
% Note that these constraints also imply $\sProd{\particlePos{\PidxI}}{\particlePos{\PidxII}} = 0, \PidxI \neq \PidxII$.
% Unfortunately \eqref{eq:ThreeParticleConstraint1} and \eqref{eq:ThreeParticleConstraint2} only fixes the shape of the system up to a mirroring.
% To overcome this we impose the constraint
% \begin{align}
%  \label{eq:ThreeParticleConstraint3}
%  \sProd{\particlePos{1} \times \particlePos{2}}{\particlePos{3}} = l_1 l_2 l_3.
% \end{align}
% \end{subequations}
% These $\numParticleConstraints = 7$ geometric can be rewritten to match the form $\particleGeoConstraint(\particleCoord) = \tuple{0}$.
% Computing the Jacobian one can check that $\rank \pdiff[\particleGeoConstraint]{\particleCoord} = 6$ and consequently $\dim\particleConfigSpace = 3$.
% \end{Example}




