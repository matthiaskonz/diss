\clearpage
\section{Differential geometry}

\subsection{Embedded manifold}
Consider the set $\configSpace$ defined by
\begin{align}
 \configSpace = \{ \sysCoord \in \RealNum^{\numCoord} \, | \, \geoConstraint(\sysCoord) = \tuple{0} \}
\end{align}
with smooth functions $\geoConstraintCoeff{\CidxI}(\sysCoord) = 0, \CidxI=1,\ldots\numGeoConst$.
Let their Jacobian have constant rank
\begin{align}
 \geoConstraintMatCoeff{\CidxI}{\GidxI}(\sysCoord) = \pdiff[\geoConstraintCoeff{\CidxI}]{\sysCoordCoeff{\GidxI}}(\sysCoord),
\qquad
 \rank\geoConstraintMat(\sysCoord) = \numCoord-\dimConfigSpace = \const \ \forall \ \sysCoord\in\configSpace.
\end{align}
Due to the rank condition, the implicit function theorem guarantees the existence of a local chart around each point $\sysCoord\in\configSpace$.
Consequently $\configSpace$ is a $\dimConfigSpace$ dimensional \textit{embedded submanifold} of $\RealNum^{\numCoord}$.
If there exits a global chart, then the manifold is \textit{linear}, i.e.\ homeomorphic to $\RealNum^\dimConfigSpace$.
If this is not the case, the manifold is called \textit{nonlinear}.

\paragraph{Tangent space.}
Consider a parametrized curve $\sysCoord : \RealNum \rightarrow \configSpace : t \mapsto \sysCoord(t)$.
Since we have $\geoConstraint(\sysCoord) = \tuple{0}$, we also have
\begin{align}
 \tdiff{t} \geoConstraint(\sysCoord(t)) = \geoConstraintMat(\sysCoord(t)) \tdiff[\sysCoord]{t}(t) = \tuple{0},
\end{align}
which has to hold for any curve through $\sysCoord$.
The set of tangent vectors $\tuple{v} = \sysCoordd$ of arbitrary curves through a particular point $\sysCoord\in\configSpace$ is the \textit{tangent space}
\begin{align}
 \Tangent[\sysCoord]\configSpace = \{ \tuple{v} \in \RealNum^{\numCoord} \, | \, \geoConstraintMat(\sysCoord) \tuple{v} = \tuple{0} \}.
\end{align}
Since $\rank\geoConstraintMat = \numCoord-\dimConfigSpace$ there is a matrix $\kinMat(\sysCoord) \in \RealNum^{\numCoord\times\dimConfigSpace}$ with $\geoConstraintMat\kinMat \equiv 0$, $\rank\kinMat=\dimConfigSpace$ with which we can give an explicit statement of the tangent space
\begin{align}
 \Tangent[\sysCoord]\configSpace = \{ \tuple{v} = \kinMat(\sysCoord)\sysVel \, | \, \sysVel \in \RealNum^{\dimConfigSpace} \}
\end{align}
i.e.\ the column space of the matrix $\kinMat$.

Linear algebra guarantees the existence of a matrix $\kinMat(\sysCoord)$ at each point $\sysCoord\in\configSpace$ and the implicit function theorem guarantees that there is a neighborhood in which $\kinMat$ is smooth. 
However, there is no guarantee for $\kinMat$ to be smooth over all $\configSpace$, i.e.\ to be global.
In the most important case for this work $\configSpace=\SpecialOrthogonalGroup(3)$, thought the manifold is nonlinear, there is a global matrix $\kinMat$, see Example \autoref{Example:KinMatSO3}.
The conjecture is that this is true for any \textit{Lie group}.
In other cases, e.g.\ the 2-sphere $\configSpace = \Sphere^2$, the \textit{hairy-ball theorem}, e.g.\ \cite{Poincare:HairyBall}, states that no global matrix $\kinMat$ can exist.
In these cases one may resort to having overlapping, local patches for $\kinMat$.

\paragraph{Some identities.}
Consider the matrices $\kinMat(\sysCoord) \in \RealNum^{\numCoord\times\dimConfigSpace}, \numCoord \geq \dimConfigSpace$ and $\geoConstraintMat(\sysCoord)\in\RealNum^{\numGeoConst\times\numCoord}, \numGeoConst \geq \numCoord-\dimConfigSpace$ with
\begin{align}
 \rank\kinMat = \dimConfigSpace,
\qquad
 \rank\geoConstraintMat = \numCoord - \dimConfigSpace,
\qquad
 \geoConstraintMat\kinMat = \mat{0}.
\end{align}
Let $\kinBasisMat(\sysCoord)\in\RealNum^{\dimConfigSpace\times\numCoord}$ and $\InvGeoConstraintMat(\sysCoord)\in\RealNum^{\numCoord\times\numGeoConst}$ be matrices that fulfill the following identities
\begin{align}
%  \begin{bmatrix} \kinBasisMat \\ \geoConstraintMat \end{bmatrix} \left[ \kinMat \ \InvGeoConstraintMat \right]
%  &= \begin{bmatrix} \kinBasisMat\kinMat & \kinBasisMat \InvGeoConstraintMat \\ \geoConstraintMat\kinMat & \geoConstraintMat \InvGeoConstraintMat \end{bmatrix}
%  = \begin{bmatrix} \idMat[\dimConfigSpace] & \mat{0} \\ \mat{0} & \idMat[\numGeoConst] \end{bmatrix}
% \\
% \left[ \kinMat \ \InvGeoConstraintMat \right] \begin{bmatrix} \kinBasisMat \\ \geoConstraintMat \end{bmatrix}
 \kinBasisMat \kinMat = \idMat[\dimConfigSpace],
\qquad
 \kinMat \kinBasisMat + \InvGeoConstraintMat \geoConstraintMat = \idMat[\numCoord]
\end{align}
Recalling the properties of the Moore-Penrose-pseudoinverse \eqref{eq:AppendixPenroseConditions} and the identity \eqref{eq:OrthogonalProjectorsFromBasis}, one possible solution to this is
\begin{align}
 \kinBasisMat = \kinMat^+, \qquad \InvGeoConstraintMat = \geoConstraintMat^+
\end{align}

\subsection{Euclidean geometry}
\begin{figure}[h]
 \input{graphics/EuclideanGeometryBasisVectors.pdf_tex}
 \caption{Curve and basis vectors on a torus.}
 \label{fig:EuclideanGeometryBasisVectors}
\end{figure}

\paragraph{Basis vectors.}
Let $(\basisVect[1],\ldots,\basisVect[\numCoord])$ be the standard basis for $\RealNum^{\numCoord}$.
Using the relations from the previous paragraph, we define tangent $\tangentBasisVect[1],\ldots,\tangentBasisVect[\dimConfigSpace]$ and normal vectors $\normalBasisVect[1],\ldots,\normalBasisVect[\numCoord]$ to any point $\sysCoord \in \configSpace$ as
\begin{subequations}
\begin{align}
 \tangentBasisVect[\LidxI](\sysCoord) &= \basisVect[\GidxI] \kinMatCoeff{\GidxI}{\LidxI}(\sysCoord),&
 \LidxI &= 1,\ldots,\dimConfigSpace,
\\
 \normalBasisVect[\CidxI](\sysCoord) &= \basisVect[\GidxI] \InvGeoConstraintMatCoeff{\GidxI}{\CidxI}(\sysCoord),&
 \CidxI &= 1,\ldots,\numGeoConst,
\\
 \basisVect[\GidxI] &= \tangentBasisVect[\LidxI](\sysCoord) \kinBasisMatCoeff{\LidxI}{\GidxI}(\sysCoord) + \normalBasisVect[\CidxI](\sysCoord) \geoConstraintMatCoeff{\CidxI}{\GidxI}(\sysCoord),&
 \GidxI &= 1,\ldots,\numCoord.
\end{align}
\end{subequations}

\paragraph{Euclidean metric}
For the Euclidean space $\RealNum^\numCoord$ we have the standard inner product $\sProd{\cdot}{\cdot}$ whose coefficients \wrt the standard basis $\basisVect[\GidxI]$ are
\begin{align}
 \sProd{\basisVect[\GidxI]}{\basisVect[\GidxII]} = \delta_{\GidxI\GidxII} = \begin{cases} 1 \ : & \GidxI = \GidxII \\ 0 \ : & \GidxI \neq \GidxII \end{cases}.
\end{align}
This induces a metric on the tangent space
\begin{align}
 \sProd{\tangentBasisVect[\LidxI]}{\tangentBasisVect[\LidxII]} = \sProd{\basisVect[\GidxI]\kinMatCoeff{\GidxI}{\LidxI}}{\basisVect[\GidxII]\kinMatCoeff{\GidxII}{\LidxII}} = \kinMatCoeff{\GidxI}{\LidxI} \kinMatCoeff{\GidxII}{\LidxII} \delta_{\GidxI\GidxII} =: \sysInertiaMatCoeff{\LidxI\LidxII}
\end{align}
Note that the metric coefficients $\sysInertiaMatCoeff{\LidxI\LidxII}(\sysCoord)\in\RealNum$ may vary over the configuration space $\configSpace$.

\paragraph{Velocity vector.}
\fixme{
Define the configuration vector $\sysCoordVect = \basisVect[\GidxI] \sysCoordCoeff{\GidxI}$ and the velocity vector $\sysVelVect = \tangentBasisVect[\LidxI] \sysVelCoeff{\LidxI}$.
They are related by
}
\begin{align}
 \sysCoordVectd &= \basisVect[\GidxI] \sysCoordCoeffd{\GidxI} = \basisVect[\GidxI] \kinMatCoeff{\GidxI}{\LidxI} \sysVelCoeff{\LidxI} = \tangentBasisVect[\LidxI] \sysVelCoeff{\LidxI} = \sysVelVect
\\
 \sysCoordVectdd &= \basisVect[\GidxI] \sysCoordCoeffdd{\GidxI} = \basisVect[\GidxI] \big( \kinMatCoeff{\GidxI}{\LidxI} \sysVelCoeffd{\LidxI} + \tpdiff[\kinMatCoeff{\GidxI}{\LidxI}]{\sysCoordCoeff{\GidxII}}\kinMatCoeff{\GidxII}{\LidxII} \sysVelCoeff{\LidxII} \sysVelCoeff{\LidxI} \big)
 = \underbrace{\tangentBasisVect[\LidxI] \big( \sysVelCoeffd{\LidxI} + \kinBasisMatCoeff{\LidxI}{\GidxI} \tpdiff[\kinMatCoeff{\GidxI}{\LidxIII}]{\sysCoordCoeff{\GidxII}} \kinMatCoeff{\GidxII}{\LidxII} \sysVelCoeff{\LidxII} \sysVelCoeff{\LidxIII}\big)}_{\frac{\nabla \sysVelVect}{\d t}}
 + \, \normalBasisVect[\CidxI] \geoConstraintMatCoeff{\CidxI}{\GidxI} \tpdiff[\kinMatCoeff{\GidxI}{\LidxIII}]{\sysCoordCoeff{\GidxII}} \kinMatCoeff{\GidxII}{\LidxII} \sysVelCoeff{\LidxII} \sysVelCoeff{\LidxIII}
\end{align}
The crucial observation here is that, even though $\sysVelVect \in \Tangent[\sysCoord]\configSpace$, its time derivative is not.
The tangential part of the derivative $\dot{\sysVelVect}$, here denoted as $\frac{\nabla \sysVelVect}{\d t}$, is called the \textit{intrinsic derivative} in \cite[sec.\,8.6b]{Frankel:GeometryOfPhysics} or the \textit{covariant derivative} in \cite[p.\,305]{Boothby:DiffGeo}).
%One may check by direct calculation that its definition remains unchanged under a change of coordinates, thus 

\subsection{Euclidean geometry and mechanics}
Consider a mechanical system of particles with positions $\particlePos{\PidxI}(t) \in \RealNum^3, \PidxI,\ldots,\numParticles$.

\paragraph{Special coordinates.}
To establish a connection to Euclidean geometry, we take the special choice of configuration coordinates (see \eg \cite[sec.\,I.5]{Lanczos:Variational} or \cite[sec.\,7.5]{Lurie:AnalyticalMechanics})
\begin{align}\label{eq:EuclideanGeometryCoord}
 \sysCoord = [\sqrt{\particleMass{1}}\, \particlePos{1}^\top, \sqrt{\particleMass{2}}\, \particlePos{2}^\top, \ldots, \sqrt{\particleMass{\numParticles}}\, \particlePos{\numParticles}^\top ]^\top \in \RealNum^{3\numParticles}
 .
\end{align}

\paragraph{Kinetic energy.}
This choice of coordinates is because now the kinetic energy $\kineticEnergy$ corresponds to the Euclidean metric:
\begin{align}\label{eq:EuclideanGeometryMetric}
 \kineticEnergy = \sum_{\PidxI=1}^{\numParticles} \tfrac{1}{2} \particleMass{\PidxI} \norm{\particlePosd{\PidxI}}^2
 = \tfrac{1}{2} \norm{\sysCoordd}^2
 = \tfrac{1}{2} \sysCoordCoeffd{\GidxI} \delta_{\GidxI\GidxII} \sysCoordCoeffd{\GidxII}
 = \tfrac{1}{2} \sysVelCoeff{\LidxI} \kinMatCoeff{\GidxI}{\LidxI} \delta_{\GidxI\GidxII} \kinMatCoeff{\GidxII}{\LidxII} \sysVelCoeff{\LidxII}
 .
\end{align}
The inertia matrix $\sysInertiaMat$ corresponds to the coefficients of the \textit{induced metric} on the tangent space, \ie $\sysInertiaMatCoeff{\LidxI\LidxII} = \sProd{\tangentBasisVect[\LidxI]}{\tangentBasisVect[\LidxII]}$.
% From \eqref{eq:blubb123432321} we find
% \begin{align}
%  \sysInertiaMatCoeff{\LidxI\LidxII} &= \sum_{\PidxI=1}^{\numParticles} \particleMass{\PidxI} \sProd{\partial_\LidxI \particlePos{\PidxI}}{\partial_\LidxII \particlePos{\PidxI}}
%  = \kinMatCoeff{\GidxI}{\LidxI} \delta_{\GidxI\GidxII} \kinMatCoeff{\GidxII}{\LidxII}
% \\
%  \ConnCoeffL{\LidxI}{\LidxII}{\LidxIII} &= \sum_{\PidxI=1}^{\numParticles} \particleMass{\PidxI} \sProd{\partial_\LidxI \particlePos{\PidxI}}{\partial_\LidxIII \partial_\LidxII \particlePos{\PidxI}}
%  = \kinMatCoeff{\GidxI}{\LidxI} \delta_{\GidxI\GidxII} \partial_\LidxIII \kinMatCoeff{\GidxII}{\LidxII}
%  .
% \end{align}

\paragraph{Inertia force.}
Plugging this formulation of the kinetic energy into the Lagrangian form of the inertia force yields
\begin{align}
  \genForceInertiaCoeff{\LidxI}
 &= \diff{t} \pdiff[\kineticEnergy]{\sysVelCoeff{\LidxI}} + \BoltzSym{\LidxIII}{\LidxI}{\LidxII} \sysVelCoeff{\LidxII} \pdiff[\kineticEnergy]{\sysVelCoeff{\LidxIII}} - \kinMatCoeff{\GidxI}{\LidxI} \pdiff[\kineticEnergy]{\sysCoordCoeff{\GidxI}}
\nonumber\\
 &= \diff{t} \big( \kinMatCoeff{\GidxI}{\LidxI} \delta_{\GidxI\GidxII} \kinMatCoeff{\GidxII}{\LidxII} \sysVelCoeff{\LidxII} \big)
 + \BoltzSym{\LidxIII}{\LidxI}{\LidxII} \sysVelCoeff{\LidxII} \big( \kinMatCoeff{\GidxI}{\LidxIII} \delta_{\GidxI\GidxII} \kinMatCoeff{\GidxII}{\LidxIV} \sysVelCoeff{\LidxIV} \big)
 - \kinMatCoeff{\GidxIII}{\LidxI} \big( \sysVelCoeff{\LidxIII} \tpdiff[\kinMatCoeff{\GidxI}{\LidxIII}]{\sysCoordCoeff{\GidxIII}} \delta_{\GidxI\GidxII} \kinMatCoeff{\GidxII}{\LidxII} \sysVelCoeff{\LidxII} \big)
\end{align}

\begin{multline}\label{eq:InertiaForceeuclideanGeo}
 \genForceInertiaCoeff{\LidxI}
 = \diff{t} \pdiff[\kineticEnergy]{\sysVelCoeff{\LidxI}} + \BoltzSym{\LidxIII}{\LidxI}{\LidxII} \sysVelCoeff{\LidxII} \pdiff[\kineticEnergy]{\sysVelCoeff{\LidxIII}} - \kinMatCoeff{\GidxI}{\LidxI} \pdiff[\kineticEnergy]{\sysCoordCoeff{\GidxI}}
 = \underbrace{\kinMatCoeff{\GidxI}{\LidxI} \delta_{\GidxI\GidxII} \kinMatCoeff{\GidxII}{\LidxII}}_{\sysInertiaMatCoeff{\LidxI\LidxII}} \sysVelCoeffd{\LidxII} + \underbrace{\kinMatCoeff{\GidxI}{\LidxI} \delta_{\GidxI\GidxII} \partial_\LidxIII \kinMatCoeff{\GidxII}{\LidxII} \, \sysVelCoeff{\LidxII} \sysVelCoeff{\LidxIII}}_{\gyroForceCoeff{\LidxI}}
\\
 = \kinMatCoeff{\GidxI}{\LidxI} \delta_{\GidxI\GidxII} \underbrace{\big( \kinMatCoeff{\GidxII}{\LidxII} \sysVelCoeffd{\LidxII} + \partial_\LidxIII \kinMatCoeff{\GidxII}{\LidxII} \, \sysVelCoeff{\LidxII} \sysVelCoeff{\LidxIII} \big)}_{\sysCoorddd[\GidxII]}
 = \sProd{\basisVect[\GidxI] \kinMatCoeff{\GidxI}{\LidxI}}{ \basisVect[\GidxII] \sysCoorddd[\GidxII]}
 = \sProd{\tangentBasisVect[\LidxI]}{\sysCoordVectdd}
 .
\end{multline}
The computation of the gyroscopic forces $\gyroForce$ would actually take some extra lines, but it is essentially just the derivation (??) of Lagrange's equations backwards.
With this said, it is not a surprise that the result $\genForceInertiaCoeff{\LidxI} = \sProd{\tangentBasisVect[\LidxI]}{\sysCoordVectdd}$ could immediately be concluded form the original definition (??) of the generalized inertia force $\genForceInertiaCoeff{\LidxI}$.
Nevertheless we can use this result for a beautiful interpretation:


\clearpage

\paragraph{Euclidean metric}
For the Euclidean space $\RealNum^\numCoord$ we have the standard inner product $\sProd{\cdot}{\cdot}$ whose coefficients \wrt the standard basis $\basisVect[\GidxI]$ are
\begin{align}
 \sProd{\basisVect[\GidxI]}{\basisVect[\GidxII]} = \delta_{\GidxI\GidxII} = \begin{cases} 1 \ : & \GidxI = \GidxII \\ 0 \ : & \GidxI \neq \GidxII \end{cases}
 .
\end{align}
% From this we can induce a metric for the tangent space
% \begin{align}
%  \sProd{\tangentBasisVect[\LidxI]}{\tangentBasisVect[\LidxII]} &= \sProd{\basisVect[\GidxI] \kinMatCoeff{\GidxI}{\LidxI}}{\basisVect[\GidxII] \kinMatCoeff{\GidxII}{\LidxII}} = \kinMatCoeff{\GidxI}{\LidxI} \delta_{\GidxI\GidxII} \kinMatCoeff{\GidxII}{\LidxII}
%  .
% \end{align}
% With the metric we have a notion of \textit{orthogonality} and notice that by the relation \eqref{eq:DefBasisOfVelocity} the vectors $\normalBasisVect[\CidxI]$ span the space normal to the configuration space, but are not necessarily orthogonal to the tangent vectors $\tangentBasisVect[\LidxI]$, \ie normal to the surface:
% \begin{align}
%  \sProd{\tangentBasisVect[\LidxI]}{\normalBasisVect[\CidxI]} = \kinMatCoeff{\GidxI}{\LidxI} \delta_{\GidxI\GidxII} \InvGeoConstraintMatCoeff{\GidxII}{\LidxII}
% \end{align}
% The orthogonality $\kinMat^\top \InvGeoConstraintMatCoeff{}{} = 0$ is not necessary but it simplifies several equations and leads to clearer interpretations.
% Since interpretation is what this section is all about, we will require the orthogonality (\textit{only!}) for this section.
% This has the consequence that $\kinMat$ and $\kinBasisMat$ are related by the \textit{(Moore–Penrose) pseudoinverse}:
% \begin{align}
%  {\underbrace{[\kinMat \ \InvGeoConstraintMatCoeff{}{}]}_{\kinMatSquare}}^{-1}
%  =
%  \underbrace{\begin{bmatrix} \kinMat^\top \kinMat & \kinMat^\top \InvGeoConstraintMatCoeff{}{}  \\ \InvGeoConstraintMatCoeff{}{}^\top \kinMat & \InvGeoConstraintMatCoeff{}{}^\top \InvGeoConstraintMatCoeff{}{} \end{bmatrix}^{-1}}_{(\kinMatSquare^\top \kinMatSquare)^{-1}}
%  \underbrace{\begin{bmatrix} \kinMat^\top \\ \InvGeoConstraintMatCoeff{}{}^\top \end{bmatrix}}_{\kinMatSquare^\top}
%  \overset{\kinMat^\top \InvGeoConstraintMatCoeff{}{} = 0}{=}
%  \underbrace{\begin{bmatrix} (\kinMat^\top \kinMat)^{-1} \kinMat^\top \\ (\InvGeoConstraintMatCoeff{}{}^\top \InvGeoConstraintMatCoeff{}{})^{-1} \InvGeoConstraintMatCoeff{}{}^\top \end{bmatrix}}_{\begin{bmatrix} \kinMat^+ \\ \InvGeoConstraintMatCoeff{}{}^+ \end{bmatrix}}
%  =
%  \underbrace{\begin{bmatrix} \kinBasisMat \\ \geoConstraintMat \end{bmatrix}}_{\kinBasisMatSquare}
%  .
% \end{align}
% We can also reverse the statements as
% \begin{align}
%  \geoConstraintMat \kinBasisMat^\top = 0
% \qquad \Rightarrow \qquad
%  \kinMat = \kinBasisMat^+ = (\kinBasisMat \kinBasisMat^\top)^{-1} \kinBasisMat^\top,
% \qquad
%  \InvGeoConstraintMatCoeff{}{} = \geoConstraintMat^+ = (\geoConstraintMat \geoConstraintMat^\top)^{-1} \geoConstraintMat^\top.
% \end{align}
% The statement might make more sense this way round since we argued in \eqref{eq:DefBasisOfVelocity} that we construct $\kinBasisMat$ as a complement for $\geoConstraintMat$ and now we require in addition that the complement is an \textit{orthogonal complement}.
% Note that we do not require that the rows of $\kinBasisMat$ are orthogonal to each other.
% 
% This requirement is not really a restriction since one can start with an arbitrary $\kinBasisMat$, obtain $\kinMat$ from $\kinBasisMatSquare^{-1}$ and then replace $\kinBasisMat$ by $\kinMat^+$ and $\InvGeoConstraintMatCoeff{}{}$ by $\geoConstraintMat^+$.

\paragraph{Configuration and velocity vector}
Define the configuration vector as $\sysCoordVect = \basisVect[\GidxI] \sysCoordCoeff{\GidxI}$ and the velocity vector as $\sysVelVect = \tangentBasisVect[\LidxI] \sysVelCoeff{\LidxI}$.
They are obviously related by
\begin{align}
 \sysCoordVectd &= \basisVect[\GidxI] \sysCoordCoeffd{\GidxI}
 = \underbrace{\basisVect[\GidxI] \kinMatCoeff{\GidxI}{\LidxI}}_{\tangentBasisVect[\LidxI]} \sysVelCoeff{\LidxI} = \sysVelVect
\end{align}
For the following it is crucial to notice that even though the velocity vector $\sysVelVect$ is tangent to the configuration space, its derivative is in general not:
\begin{align}
 \label{eq:IntrinsicDerivativeEmbedding}
 \sysVelVectd
 &= \tangentBasisVect[\LidxI] \sysVelCoeffd{\LidxI} + \tangentBasisVectd[\LidxI] \sysVelCoeff{\LidxI}
 = \tangentBasisVect[\LidxI] \sysVelCoeffd{\LidxI} + \basisVect[\GidxI] \partial_\LidxII \kinMatCoeff{\GidxI}{\LidxI} \sysVelCoeff{\LidxII} \sysVelCoeff{\LidxI}
 = \underbrace{\tangentBasisVect[\LidxI] \big( \sysVelCoeffd{\LidxI} + \kinBasisMatCoeff{\LidxI}{\GidxI} \partial_\LidxII \kinMatCoeff{\GidxI}{\LidxIII} \sysVelCoeff{\LidxIII} \sysVelCoeff{\LidxII}\big)}_{\frac{\nabla \sysVelVect}{\d t}}
 + \, \normalBasisVect[\CidxI] \geoConstraintMatCoeff{\CidxI}{\GidxI} \partial_\LidxII \kinMatCoeff{\GidxI}{\LidxIII} \, \sysVelCoeff{\LidxII} \sysVelCoeff{\LidxIII} .
\end{align}
This is a consequence of the tangent vectors $\tangentBasisVect[\LidxI]$ varying over the configuration space.

The tangential part of the derivative $\tdiff{t}\sysVelVect$, denoted as $\frac{\nabla \sysVelVect}{\d t}$, which also contains the change of the tangent vectors in the tangent direction, is \fixme{(for this context of an isometrically embedded manifold)} referred as the \textit{intrinsic} or \textit{covariant derivative} (\eg \cite[sec.\,8.6b]{Frankel:GeometryOfPhysics} or \cite[p.\,305]{Boothby:DiffGeo}).

\paragraph{A special choice of coordinates.}
To establish a connection to mechanics we will restrict to the discrete case of a finite number $\numParticles$ of particles and by taking a special choice of configuration coordinates (\eg \cite[sec.\,I.5]{Lanczos:Variational} or \cite[sec.\,7.5]{Lurie:AnalyticalMechanics})
\begin{align}\label{eq:EuclideanGeometryCoord}
 \sysCoord = [\sqrt{\particleMass{1}}\, \particlePos{1}^\top, \sqrt{\particleMass{2}}\, \particlePos{2}^\top, \ldots, \sqrt{\particleMass{\numParticles}}\, \particlePos{\numParticles}^\top ]^\top \in \RealNum^{3\numParticles}
 .
\end{align}
This (quite ugly) choice of coordinates is because now the kinetic energy $\kineticEnergy$ corresponds to the Euclidean metric of the embedding space $\RealNum^\numCoord$:
\begin{align}\label{eq:EuclideanGeometryMetric}
 \kineticEnergy = \sum_{\PidxI=1}^{\numParticles} \tfrac{1}{2} \particleMass{\PidxI} \norm{\particlePosd{\PidxI}}^2
 = \tfrac{1}{2} \norm{\sysCoordd}^2
 = \tfrac{1}{2} \sysCoordCoeffd{\GidxI} \delta_{\GidxI\GidxII} \sysCoordCoeffd{\GidxII}
 = \tfrac{1}{2} \sysVelCoeff{\LidxI} \underbrace{\kinMatCoeff{\GidxI}{\LidxI} \delta_{\GidxI\GidxII} \kinMatCoeff{\GidxII}{\LidxII}}_{\sysInertiaMatCoeff{\LidxI\LidxII}} \sysVelCoeff{\LidxII}
 .
\end{align}
The inertia matrix $\sysInertiaMat$ corresponds to the coefficients of the \textit{induced metric} on the tangent space, \ie $\sysInertiaMatCoeff{\LidxI\LidxII} = \sProd{\tangentBasisVect[\LidxI]}{\tangentBasisVect[\LidxII]}$.
% From \eqref{eq:blubb123432321} we find
% \begin{align}
%  \sysInertiaMatCoeff{\LidxI\LidxII} &= \sum_{\PidxI=1}^{\numParticles} \particleMass{\PidxI} \sProd{\partial_\LidxI \particlePos{\PidxI}}{\partial_\LidxII \particlePos{\PidxI}}
%  = \kinMatCoeff{\GidxI}{\LidxI} \delta_{\GidxI\GidxII} \kinMatCoeff{\GidxII}{\LidxII}
% \\
%  \ConnCoeffL{\LidxI}{\LidxII}{\LidxIII} &= \sum_{\PidxI=1}^{\numParticles} \particleMass{\PidxI} \sProd{\partial_\LidxI \particlePos{\PidxI}}{\partial_\LidxIII \partial_\LidxII \particlePos{\PidxI}}
%  = \kinMatCoeff{\GidxI}{\LidxI} \delta_{\GidxI\GidxII} \partial_\LidxIII \kinMatCoeff{\GidxII}{\LidxII}
%  .
% \end{align}
With this special form of the inertia matrix the generalized force from inertia (??) takes the form
\begin{multline}\label{eq:InertiaForceeuclideanGeo}
 \genForceInertiaCoeff{\LidxI}
 = \diff{t} \pdiff[\kineticEnergy]{\sysVelCoeff{\LidxI}} + \BoltzSym{\LidxIII}{\LidxI}{\LidxII} \sysVelCoeff{\LidxII} \pdiff[\kineticEnergy]{\sysVelCoeff{\LidxIII}} - \kinMatCoeff{\GidxI}{\LidxI} \pdiff[\kineticEnergy]{\sysCoordCoeff{\GidxI}}
 = \underbrace{\kinMatCoeff{\GidxI}{\LidxI} \delta_{\GidxI\GidxII} \kinMatCoeff{\GidxII}{\LidxII}}_{\sysInertiaMatCoeff{\LidxI\LidxII}} \sysVelCoeffd{\LidxII} + \underbrace{\kinMatCoeff{\GidxI}{\LidxI} \delta_{\GidxI\GidxII} \partial_\LidxIII \kinMatCoeff{\GidxII}{\LidxII} \, \sysVelCoeff{\LidxII} \sysVelCoeff{\LidxIII}}_{\gyroForceCoeff{\LidxI}}
\\
 = \kinMatCoeff{\GidxI}{\LidxI} \delta_{\GidxI\GidxII} \underbrace{\big( \kinMatCoeff{\GidxII}{\LidxII} \sysVelCoeffd{\LidxII} + \partial_\LidxIII \kinMatCoeff{\GidxII}{\LidxII} \, \sysVelCoeff{\LidxII} \sysVelCoeff{\LidxIII} \big)}_{\sysCoorddd[\GidxII]}
 = \sProd{\basisVect[\GidxI] \kinMatCoeff{\GidxI}{\LidxI}}{ \basisVect[\GidxII] \sysCoorddd[\GidxII]}
 = \sProd{\tangentBasisVect[\LidxI]}{\sysCoordVectdd}
 .
\end{multline}
The computation of the gyroscopic forces $\gyroForce$ would actually take some extra lines, but it is essentially just the derivation (??) of Lagrange's equations backwards.
With this said, it is not a surprise that the result $\genForceInertiaCoeff{\LidxI} = \sProd{\tangentBasisVect[\LidxI]}{\sysCoordVectdd}$ could immediately be concluded form the original definition (??) of the generalized inertia force $\genForceInertiaCoeff{\LidxI}$.
Nevertheless we can use this result for a beautiful interpretation:

\paragraph{Interpretation.}
Suppose the system moves only under its own inertia, \ie there are no applied forces.
For a single particle constraint to a surface in $\RealNum^3$ d'Alembert's principle states that the acceleration of the particle in directions tangent to the surface has to vanish.
This is visualized in \autoref{fig:ParticleOnTorus} for the example of a particle moving on the surface of a torus.
We can imagine the system consisting of $\numParticles$ particles as a \textit{single} particle moving on the configuration space $\configSpace$, a $\dimConfigSpace$-dimensional surface embedded in the $3\numParticles$ dimensional Euclidean space.
Now \eqref{eq:InertiaForceeuclideanGeo} states just the same, that the acceleration has to vanish in the $\dimConfigSpace$ independent tangent directions.

The fact that the derivative of the velocity \textit{vector} $\sysVelVect$ in the tangent directions vanish does not imply that the derivatives of the corresponding \textit{coefficients} $\sysVel$ vanish, as the basis vectors do vary as well.
This should be evident form \eqref{eq:IntrinsicDerivativeEmbedding} and from the example in \autoref{fig:ParticleOnTorus} and should give some understanding for the origin of, what we called, the generalized gyroscopic force $\gyroForce$.

\begin{figure}[t]
 \centering
 \input{graphics/ParticleOnTorus.pdf_tex}
 \caption{A particle moving freely on the surface of a torus [background image: \texttt{commons.wikimedia.org}]}
 \label{fig:ParticleOnTorus}
\end{figure}

\paragraph{\fixme{Conclusions.}}
The key aspect for the relation between Euclidean geometry and a mechanical system was the relation between the Euclidean metric and the kinetic energy in \eqref{eq:EuclideanGeometryMetric}.
To achieve this we took a special choice of coordinates \eqref{eq:EuclideanGeometryCoord}.
This is exactly what we do \textit{not} want to do in Lagrangian mechanics!
Also this is \textit{not} the motivation behind the use of redundant coordinates.

However, the results did lead to a nice mechanical interpretation and will be valuable for interpretation of the much more abstract notion of Riemannian geometry in the next section.


\subsection{Riemannian geometry}
The relation of analytical mechanics and Riemannian geometry is well known and can be found in standard textbooks on the subject \eg \cite{Frankel:GeometryOfPhysics} or \cite{Abraham:FoundationsOfMechanics}.
For deeper mathematical background the standard textbooks \cite{Boothby:DiffGeo} and \cite{Spivak:DiffGeo} can be consulted.

Unfortunately all textbooks on the subject known to the author use local charts (minimal configuration coordinates: $\genCoord$) and restrict to coordinate basis vectors (velocity coordinates: $\sysVel = \genCoordd$) for the content relevant to this work.
Consequently these coordinate expressions are not applicable for redundant configuration coordinates $\sysCoord$ and arbitrary velocity coordinates $\sysVel$ considered in this work. 
The only textbook known to the author that states some formulas for non-coordinate basis vectors is \cite{Misner:Gravitation} but for the rather different application to general relativity.

However, as the main theorems are usually stated in a \textit{coordinate free} fashion, we can derive the coordinate expressions for our case rather easily.
For doing so we first need to relate the concepts of the previous sections to the language of differential geometry.
%\cite{Synge:GeometryOfDynamics}

\paragraph{Configuration manifold.}
We previously introduced the configuration space as
\begin{align}
 \configSpace = \{ \sysCoord \in \RealNum^\numCoord \,|\, \geoConstraint(\sysCoord) = 0 \},
\qquad
 \dim \configSpace = \numCoord - \rank \tpdiff[\geoConstraint]{\sysCoord} = \dimConfigSpace
\end{align}
but this is also an (embedded) \textit{smooth manifold} and we call $\configSpace$ for this section the \textit{configuration manifold}.

% Formally we could solve the constraint equations for $\numCoord-\dimConfigSpace$ of the $\numCoord$ variables $\sysCoord$ and the remaining $\dimConfigSpace$ can serve as (minimal) coordinates for a coordinate patch.
% We can do this for different combinations of coordinates generating different coordinate patches and ensure that the complete manifold is covered, thus generating an atlas for the manifold.
% 
% As we motivated before this is not the path we like to go here, we will stick to the embedding.
% However, the following of course includes $\configSpace = \RealNum^\dimConfigSpace$ as a special case.

\paragraph{Tangent vectors.}
In differential geometry it is common to identify tangent vectors with differential operators $\basisVect[\GidxI] = \pdiff{\sysCoordCoeff{\GidxI}}, \GidxI = 1,\ldots,\numCoord$.
For the sake of readability we adopt the notation of \cite{Frankel:GeometryOfPhysics} and write them as $\pdiffVect{\sysCoordCoeff{\GidxI}}$ when they should be regarded as basis vectors.
The expression of a vector $\vect{a} \in \Tangent \RealNum^\numCoord$ and its application $\vect{a}(f)$ as a directional derivative to a function $f:\RealNum^\numCoord \rightarrow \RealNum$ reads
\begin{align}
 \vect{a} = a^\GidxI \pdiffVect{\sysCoordCoeff{\GidxI}},
\qquad
 \vect{a} (f) = a^\GidxI \pdiff[f]{\sysCoordCoeff{\GidxI}}.
\end{align}
As in the previous subsection we define
the tangent $\tangentBasisVect[\LidxI] = \partialVect_{\LidxI}, \LidxI = 1,\ldots,\dimConfigSpace$ and normal vectors $\normalBasisVect[\CidxI] = \partialVect_{\CidxI}, \CidxI = 1,\ldots,\numCoord-\dimConfigSpace$ to the configuration manifold $\configSpace$ as%
\footnote{
Note that the normal vectors are distinguished only by the index: $\CidxI$, \ie $\partialVect_{\CidxI}$, indicates the normal vector, whereas Latin letters, commonly $\partialVect_{\LidxI}, \partialVect_{\LidxII}, \partialVect_{\LidxIII}, \partialVect_{\LidxIV}, \partialVect_{\LidxV}$ indicate tangent vectors.
We also drop the explicit statement of the domain of the indices: They are always $\GidxI, \GidxII = 1,\ldots,\numCoord$ and $\LidxI, \LidxII, \LidxIII, \LidxIV, \LidxV = 1,\ldots,\dimConfigSpace$ and $\CidxI = 1,\ldots,\numCoord - \dimConfigSpace$. 
}
\begin{align}
 \partialVect_{\LidxI} &= \kinMatCoeff{\GidxI}{\LidxI}(\sysCoord) \pdiffVect{\sysCoordCoeff{\GidxI}},
\qquad
 \partialVect_{\CidxI} = \InvGeoConstraintMatCoeff{\GidxI}{\CidxI}(\sysCoord) \pdiffVect{\sysCoordCoeff{\GidxI}},&
&\Leftrightarrow&
 \pdiffVect{\sysCoordCoeff{\GidxI}} &= \kinBasisMatCoeff{\LidxI}{\GidxI}(\sysCoord) \partialVect_{\LidxI} + \geoConstraintMatCoeff{\CidxI}{\GidxI}(\sysCoord) \partialVect_{\CidxI}
\end{align}
With this the derivative of a function $f:\configSpace \rightarrow \RealNum$ along a tangent vector $\vect{v} \in \Tangent[\sysCoord] \configSpace$ reads
\begin{align}
 \vect{v} = v^\LidxI \partialVect_{\LidxI},
\qquad
 \vect{v} (f) = v^\LidxI \partial_{\LidxI} f = v^\LidxI \kinMatCoeff{\GidxI}{\LidxI} \pdiff[f]{\sysCoordCoeff{\GidxI}}
\end{align}

\paragraph{Lie bracket.}
Whereas the basis vectors $\pdiffVect{\sysCoordCoeff{\GidxI}}$ are considered constant, the tangent vectors $\partialVect_{\LidxI}$ form a vector \textit{field} that is usually not constant.
One consequence of this is that directional derivatives along the tangent vectors $\partialVect_{\LidxI}$ do usually not commute.
This is one way to define the \textit{Lie bracket} $\LieBracket{\cdot}{\cdot} : \Tangent[\sysCoord]\configSpace \times \Tangent[\sysCoord]\configSpace \rightarrow \Tangent[\sysCoord]\configSpace$ of vector fields:
\begin{align}
 \LieBracket{\vect{u}}{\vect{v}}(f) = \vect{u}(\vect{v}(f)) - \vect{v}(\vect{u}(f))
\end{align}
For the Lie bracket of the basis vectors we can derive in total analogy to (??): 
\begin{align}
 \LieBracket{\partialVect_\LidxI}{\partialVect_\LidxII}(f) &= \partialVect_\LidxI (\partialVect_\LidxII (f)) - \partialVect_\LidxII (\partialVect_\LidxI (f))
\nonumber\\
 &= \kinMatCoeff{\GidxI}{\LidxI} \pdiff{\sysCoordCoeff{\GidxI}} \bigg(\kinMatCoeff{\GidxII}{\LidxII} \pdiff[f]{\sysCoordCoeff{\GidxII}} \bigg) - \kinMatCoeff{\GidxII}{\LidxII} \pdiff{\sysCoordCoeff{\GidxII}} \bigg( \kinMatCoeff{\GidxI}{\LidxI} \pdiff[f]{\sysCoordCoeff{\GidxI}} \bigg)
\nonumber\\
 &= \kinMatCoeff{\GidxI}{\LidxI} \pdiff[\kinMatCoeff{\GidxII}{\LidxII}]{\sysCoordCoeff{\GidxI}} \pdiff[f]{\sysCoordCoeff{\GidxII}} - \kinMatCoeff{\GidxII}{\LidxII} \pdiff[\kinMatCoeff{\GidxI}{\LidxI}]{\sysCoordCoeff{\GidxII}} \pdiff[f]{\sysCoordCoeff{\GidxI}}
 + \kinMatCoeff{\GidxI}{\LidxI} \kinMatCoeff{\GidxII}{\LidxII} \underbrace{\bigg(\frac{\partial^2 f}{\partial \sysCoordCoeff{\GidxI} \partial \sysCoordCoeff{\GidxII}} - \frac{\partial^2 f}{\partial \sysCoordCoeff{\GidxII} \partial \sysCoordCoeff{\GidxI}} \bigg)}_{0}
\nonumber\\[-2ex]
 &= \bigg( \kinMatCoeff{\GidxI}{\LidxI} \pdiff[\kinMatCoeff{\GidxII}{\LidxII}]{\sysCoordCoeff{\GidxI}} - \kinMatCoeff{\GidxI}{\LidxII} \pdiff[\kinMatCoeff{\GidxII}{\LidxI}]{\sysCoordCoeff{\GidxI}} \bigg) 
 \underbrace{\big( \kinBasisMatCoeff{\LidxIII}{\GidxII} \partialVect_{\LidxIII} + \geoConstraintMatCoeff{\CidxI}{\GidxII} \partialVect_{\CidxI} \big) (f)}_{\pdiff[f]{\sysCoordCoeff{\GidxII}}}
\nonumber\\
 &= \underbrace{\bigg(\pdiff[\kinBasisMatCoeff{\LidxIII}{\GidxI}]{\sysCoordCoeff{\GidxII}} - \pdiff[\kinBasisMatCoeff{\LidxIII}{\GidxII}]{\sysCoordCoeff{\GidxI}}\bigg) \kinMatCoeff{\GidxI}{\LidxI} \kinMatCoeff{\GidxII}{\LidxII}}_{\BoltzSym{\LidxIII}{\LidxI}{\LidxII}} \partialVect_{\LidxIII}(f)
%  + \underbrace{\bigg(\frac{\partial^2 \geoConstraintCoeff{\CidxI}}{\partial \sysCoordCoeff{\GidxII} \partial \sysCoordCoeff{\GidxI}} - \frac{\partial^2 \geoConstraintCoeff{\CidxI}}{\partial \sysCoordCoeff{\GidxI} \partial \sysCoordCoeff{\GidxII}} \bigg)}_{0} \kinMatCoeff{\GidxI}{\LidxI} \kinMatCoeff{\GidxII}{\LidxII} \partialVect_{\CidxI}(f)
 + \underbrace{\bigg(\pdiff[\geoConstraintMatCoeff{\CidxI}{\GidxI}]{\sysCoordCoeff{\GidxII}} - \pdiff[\geoConstraintMatCoeff{\CidxI}{\GidxII}]{\sysCoordCoeff{\GidxI}} \bigg)}_{0} \kinMatCoeff{\GidxI}{\LidxI} \kinMatCoeff{\GidxII}{\LidxII} \partialVect_{\CidxI}(f)
\label{eq:LieBracketCoeff}
\end{align}
and find again the \textit{commutation coefficients} $\BoltzSym{\LidxIII}{\LidxI}{\LidxII}$.
%The last step used the differentiation of the identities $\kinMatCoeff{\GidxI}{\LidxI} \kinBasisMatCoeff{\LidxII}{\GidxI} = \delta^\LidxI_\LidxII$, $\kinMatCoeff{\GidxI}{\LidxI} \geoConstraintMatCoeff{\CidxI}{\GidxI} = 0$ and $\geoConstraintMatCoeff{\CidxI}{\GidxI} = \pdiff[{\geoConstraintCoeff{\CidxI}}]{\sysCoordCoeff{\GidxI}}$.
Note that the fact that the Lie bracket of the tangent vectors remains in the tangent space is a consequence of $\geoConstraintMat$ being a differential.
This would not be the case if $\geoConstraintMat$ would represent a nonholonomic constraint.
For the special case of coordinate basis vectors $\partial_\LidxI = \tpdiff{\genCoordCoeff{\LidxI}}$ the connection coefficients vanish $\BoltzSym{\LidxIII}{\LidxI}{\LidxII} = 0$.

Since \eqref{eq:LieBracketCoeff} holds for any function we can write
\begin{align}
 \LieBracket{\partialVect_\LidxI}{\partialVect_\LidxII} = \BoltzSym{\LidxIII}{\LidxI}{\LidxII} \partialVect_\LidxIII
 .
\end{align}
With this it is straight forward to derive a coordinate expression for the Lie bracket:
\begin{align}
 \LieBracket{u^\LidxI \partialVect_\LidxI}{v^\LidxII \partialVect_\LidxII}
% &= u^\LidxI \partialVect_\LidxI (v^\LidxII \partialVect_\LidxII) - v^\LidxII \partialVect_\LidxII(u^\LidxI \partialVect_\LidxI)
% &= u^\LidxI (\partial_\LidxI v^\LidxII \partialVect_\LidxII + v^\LidxII \partialVect_\LidxI \partialVect_\LidxII) - v^\LidxII (\partial_\LidxII u^\LidxI \partialVect_\LidxI +  u^\LidxI \partialVect_\LidxII \partialVect_\LidxI)
%  &= u^\LidxI \partial_\LidxI v^\LidxII \partialVect_\LidxII + u^\LidxI v^\LidxII \partialVect_\LidxI \partialVect_\LidxII - v^\LidxII \partial_\LidxII u^\LidxI \partialVect_\LidxI + v^\LidxII u^\LidxI \partialVect_\LidxII \partialVect_\LidxI
 &= \big( u^\LidxI \partial_\LidxI v^\LidxIII - v^\LidxI \partial_\LidxI u^\LidxIII + \BoltzSym{\LidxIII}{\LidxI}{\LidxII} u^\LidxI v^\LidxII \big) \partialVect_\LidxIII
\end{align}


\paragraph{Connection and covariant derivative.}
In the previous section, in \eqref{eq:IntrinsicDerivativeEmbedding}, we called the orthogonal projection of the derivative of a tangent vector back into the tangent space, the \textit{covariant derivative}.
For this to be related with mechanics we had to make a special choice of coordinates, such that the induced metric corresponds to the kinetic energy.
Since this is exactly what we do not want in Lagrangian mechanics, we need a more abstract notion for the covariant derivative.
%For the context of this work the terms \textit{covariant derivative} and \textit{connection} are synonym, the term connection is mainly used to be consistent with the literature.

\cite[Def.\,VII.3.1]{Boothby:DiffGeo}: A \textit{connection} $\covDiffVect{}{}$ on a manifold $\configSpace$ is a mapping $\covDiffVect{}{} : \Tangent[\sysCoord]\configSpace \times \Tangent[\sysCoord]\configSpace \rightarrow \Tangent[\sysCoord]\configSpace$
denoted by $\covDiffVect{}{} : (\vect{u}, \vect{v}) \mapsto \covDiffVect{\vect{v}}{\vect{u}}$ which has the properties:
\begin{subequations}
\begin{align}
 \label{eq:LeviCivitaLinear}
% &\text{linear in first argument}&
 \covDiffVect{f\vect{v}+g\vect{w}}{\vect{u}} &= f \covDiffVect{\vect{v}}{\vect{u}} + g \covDiffVect{\vect{w}}{\vect{u}}
\\
 \label{eq:LeviCivitaProduct}
% &\text{product rule}&
 \covDiffVect{\vect{v}}{(f\,\vect{u})} &= f \covDiffVect{\vect{v}}{\vect{u}} + \vect{v}(f) \vect{u}
\\
 \covDiffVect{\vect{v}}{(\vect{u} + \vect{w})} &= \covDiffVect{\vect{v}}{\vect{u}} + \covDiffVect{\vect{v}}{\vect{w}}
\end{align}
\end{subequations}
For all $f,g : \configSpace \rightarrow \RealNum$ and $\vect{u}, \vect{v}, \vect{w} \in \Tangent[\sysCoord]\configSpace$.

Introduce the \textit{connection coefficients} $\ConnCoeff{\LidxIII}{\LidxI}{\LidxII}(\sysCoord) \in \RealNum, \LidxI,\LidxII,\LidxIII = 1\ldots\dimConfigSpace$, \ie the coefficients of the connection \wrt to the basis vectors $\partialVect_\LidxI$, defined as\footnote{The order of the indices corresponds to the choice in \cite[eq.\,8.19a]{Misner:Gravitation} as this is the only textbook known to the author that states \eqref{eq:DefConnCoeffLeviCivita}. For the prevalent description of the Christoffel symbols the indices $\LidxI$ and $\LidxII$ are commonly swapped which does not matter since they are symmetric in these indices anyway.}
\begin{align}\label{eq:DefConnCoeff}
 \covDiffVect{\partialVect_{\LidxII}}{\partialVect_{\LidxI}} &= \ConnCoeff{\LidxIII}{\LidxI}{\LidxII} \partialVect_{\LidxIII}
 .
\end{align}
With this we can give a coordinate expression for the covariant derivative
\begin{align}\label{eq:CovariantDerivaitveCoordinate}
 \covDiffVect{\vect{v}}{\vect{u}} = \covDiffVect{v^\LidxII \partialVect_\LidxII}{(u^\LidxI \partialVect_\LidxI)}
 = \big( (\partial_\LidxII u^\LidxI) \partialVect_\LidxI + u^\LidxI \underbrace{\covDiffVect{\partialVect_\LidxII}{\partialVect_\LidxI}}_{\ConnCoeff{\LidxIII}{\LidxI}{\LidxII} \partialVect_\LidxIII} \big) v^\LidxII
% = \big( (\partial_\LidxII u^\LidxI) \partialVect_\LidxI + u^\LidxI \covDiffVect{\partialVect_\LidxII}{\partialVect_\LidxI} \partialVect_\LidxIII} \big) v^\LidxII
 = \big(\underbrace{ \partial_\LidxII u^\LidxI + \ConnCoeff{\LidxI}{\LidxIII}{\LidxII} u^\LidxIII}_{\covDiff{\LidxII}{u^\LidxI}} \big) v^\LidxII \partialVect_\LidxI
 .
\end{align}
The notation $\covDiff{\LidxII}{u^\LidxI}$ or ($u^\LidxI_{;\LidxII}$) is common in the literature, but we wont pick it up here.
For the coordinate expression \eqref{eq:CovariantDerivaitveCoordinate} to make sense, it has to be the same in any basis:
Consider another set of basis vectors $\accW{\partialVect}_\LidxWI, \LidxWI = 1,\ldots,\dimConfigSpace$ and the relations
\begin{align}\label{eq:BasisTransformation}
 \vect{u} &= u^\LidxI \partialVect_\LidxI = \accW{u}^\LidxWI \accW{\partialVect}_\LidxWI,&
 u^\LidxI &= \BasisChangeCoeff{\LidxI}{\LidxWI} \accW{u}^\LidxWI, \quad \partialVect_\LidxI = \iBasisChangeCoeff{\LidxWI}{\LidxI} \accW{\partialVect}_\LidxWI, \quad \iBasisChange = \BasisChange^{-1}
 .
\end{align}
Then the following has to hold
\begin{multline}
 \covDiffVect{v^\LidxII \partialVect_\LidxII}{(u^\LidxI \partialVect_\LidxI)} 
 = \big( \partial_\LidxII u^\LidxI + \ConnCoeff{\LidxI}{\LidxIII}{\LidxII} u^\LidxIII \big) v^\LidxII \partialVect_\LidxI
 = \big( \iBasisChangeCoeff{\LidxWIV}{\LidxII} \accW{\partial}_\LidxWIV (\BasisChangeCoeff{\LidxI}{\LidxWIII} \accW{u}^\LidxWIII) + \ConnCoeff{\LidxI}{\LidxIII}{\LidxII} \BasisChangeCoeff{\LidxIII}{\LidxWIII} \accW{u}^\LidxWIII \big) \BasisChangeCoeff{\LidxII}{\LidxWII} \accW{v}^\LidxWII \iBasisChangeCoeff{\LidxWI}{\LidxI} \accW{\partialVect}_\LidxWI
% = \big( \iBasisChangeCoeff{\LidxWI}{\LidxI} \accW{\partial}_\LidxWII (\BasisChangeCoeff{\LidxI}{\LidxWIII} \accW{u}^\LidxWIII) + \ConnCoeff{\LidxI}{\LidxIII}{\LidxII} \BasisChangeCoeff{\LidxIII}{\LidxWIII} \BasisChangeCoeff{\LidxII}{\LidxWII} \iBasisChangeCoeff{\LidxWI}{\LidxI} \accW{u}^\LidxWIII \big) \accW{v}^\LidxWII \accW{\partialVect}_\LidxWI
\\
 = \big( \accW{\partial}_\LidxWII \accW{u}^\LidxWI + \underbrace{\iBasisChangeCoeff{\LidxWI}{\LidxI} \big( \accW{\partial}_\LidxWII \BasisChangeCoeff{\LidxI}{\LidxWIII} + \ConnCoeff{\LidxI}{\LidxIII}{\LidxII} \BasisChangeCoeff{\LidxIII}{\LidxWIII} \BasisChangeCoeff{\LidxII}{\LidxWII}\big)}_{\ConnCoeffW{\LidxWI}{\LidxWIII}{\LidxWII}} \accW{u}^\LidxWIII \big) \accW{v}^\LidxWII \accW{\partialVect}_\LidxWI
 = \covDiffVect{\accW{v}^\LidxWII \accW{\partialVect}_\LidxWII}{(\accW{u}^\LidxWI \accW{\partialVect}_\LidxWI)} 
 .
\label{eq:TrafoRuleConnCoeff}
\end{multline}
This implies the transformation law for the connection coefficients.

From this law it should be clear that the connection coefficients $\ConnCoeff{\LidxIII}{\LidxI}{\LidxII}$ do \textit{not} form a tensor despite being indexed in the same way.
The covariant derivative does form a tensor, thus giving the justification for the name.

\begin{Remark}
In the special case of coordinate basis vectors the transformation law \eqref{eq:TrafoRuleConnCoeff} simplifies to
\begin{align}
 \partial_\LidxI = \tpdiff{\genCoordCoeff{\LidxI}}, \ \accW{\partial}_\LidxWI = \tpdiff{\genCoordCoeffW{\LidxWI}}
\quad \Rightarrow \quad
 \BasisChangeCoeff{\LidxI}{\LidxWI} = \tpdiff[{\genCoordCoeff{\LidxI}}]{\genCoordCoeffW{\LidxWI}}, \ \iBasisChangeCoeff{\LidxWI}{\LidxI} = \tpdiff[{\genCoordCoeffW{\LidxWI}}]{\genCoordCoeff{\LidxI}}, \
 \ConnCoeffW{\LidxWI}{\LidxWIII}{\LidxWII} = \tpdiff[{\genCoordCoeffW{\LidxWI}}]{\genCoordCoeff{\LidxI}} \Big( \tfrac{\partial^2 \genCoordCoeff{\LidxI}}{\partial\genCoordCoeffW{\LidxWII} \partial\genCoordCoeffW{\LidxWIII}} + \ConnCoeff{\LidxI}{\LidxIII}{\LidxII} \tpdiff[{\genCoordCoeff{\LidxIII}}]{\genCoordCoeffW{\LidxWIII}} \tpdiff[{\genCoordCoeff{\LidxII}}]{\genCoordCoeffW{\LidxWII}} \Big)
\end{align}
that can be found in \eg \cite[Vol.\,2, p.\,221]{Spivak:DiffGeo} or \cite[p.\,145]{Abraham:FoundationsOfMechanics}.
\end{Remark}


\paragraph{Riemannian metric and the Levi-Civita connection.}
So far we have only reviewed some basic concepts of differential geometry.
For the relation to mechanics we need one more crucial concept: a \textit{Riemannian metric}.
That is a (bilinear, symmetric and positive definite) inner product $\metricProd{\cdot}{\cdot}: \Tangent[\sysCoord]\configSpace \times \Tangent[\sysCoord]\configSpace \rightarrow \RealNum$ that may vary smoothly over the manifold.
The relation to mechanics is done by defining the metric $\metricProd{\cdot}{\cdot}$ such that it corresponds to the \textit{kinetic energy} $\kineticEnergy$:
\begin{align}\label{eq:RiemannianMetricKineticEnergy}
 \sysVelVect = \sysVelCoeff{\LidxI}\partialVect_\LidxI, \ \metricProd{\partialVect_\LidxI}{\partialVect_\LidxII} &= \sysInertiaMatCoeff{\LidxI\LidxII}&
&\Rightarrow&
 \tfrac{1}{2} \metricProd{\sysVelVect}{\sysVelVect} &= \tfrac{1}{2} \sysInertiaMatCoeff{\LidxI\LidxII} \sysVelCoeff{\LidxI} \sysVelCoeff{\LidxII} = \kineticEnergy&
\end{align}
A smooth manifold equipped with a metric is called \textit{Riemannian manifold}.

\cite[Theo.\,VII.3.3 (\textit{Fundamental Theorem of Riemannian geometry})]{Boothby:DiffGeo}: For a Riemannian manifold there exists a unique connection $\covDiffVect{}{}$, commonly called the \textit{Levi-Civita connection}, sometimes the Riemannian connection, determined by the additional properties
\begin{subequations}\label{eq:LeviCivitaProp}
\begin{align}
 \label{eq:LeviCivitaMetric}
 &\text{compatible with metric}&
 \vect{w} \metricProd{\vect{u}}{\vect{v}} &= \metricProd{\covDiffVect{\vect{w}}{\vect{u}}}{\vect{v}} + \metricProd{\vect{u}}{\covDiffVect{\vect{w}}{\vect{v}}}
\\
 \label{eq:LeviCivitaSymmetry}
 &\text{no torsion}&
 \LieBracket{\vect{u}}{\vect{v}} &= \covDiffVect{\vect{u}}{\vect{v}} - \covDiffVect{\vect{v}}{\vect{u}}.
\end{align}
\end{subequations}
%This unique connection is commonly called the \textit{Levi-Civita connection}, sometimes also the Riemannian connection.

To get a more explicit statement for the Levi-Civita connection we combine permutations of \eqref{eq:LeviCivitaMetric} and \eqref{eq:LeviCivitaSymmetry}:
\begin{align}
 \metricProd{\vect{u}}{\covDiffVect{\vect{w}}{\vect{v}}} &= \vect{w} \metricProd{\vect{u}}{\vect{v}} - \metricProd{\covDiffVect{\vect{w}}{\vect{u}}}{\vect{v}}
\nonumber\\
 &= \vect{w} \metricProd{\vect{u}}{\vect{v}} - \metricProd{\LieBracket{\vect{w}}{\vect{u}}}{\vect{v}} - \metricProd{\covDiffVect{\vect{u}}{\vect{w}}}{\vect{v}}
\nonumber\\
 &= \vect{w} \metricProd{\vect{u}}{\vect{v}} - \metricProd{\LieBracket{\vect{w}}{\vect{u}}}{\vect{v}} - \vect{u} \metricProd{\vect{w}}{\vect{v}} - \metricProd{\vect{w}}{\covDiffVect{\vect{u}}{\vect{v}}}
\nonumber\\
 &= \vect{w} \metricProd{\vect{u}}{\vect{v}} - \metricProd{\LieBracket{\vect{w}}{\vect{u}}}{\vect{v}} - \vect{u} \metricProd{\vect{w}}{\vect{v}} + \metricProd{\vect{w}}{\LieBracket{\vect{u}}{\vect{v}}} - \metricProd{\vect{w}}{\covDiffVect{\vect{v}}{\vect{u}}}
\nonumber\\
 &= \vect{w} \metricProd{\vect{u}}{\vect{v}} - \metricProd{\LieBracket{\vect{w}}{\vect{u}}}{\vect{v}} - \vect{u} \metricProd{\vect{w}}{\vect{v}} + \metricProd{\vect{w}}{\LieBracket{\vect{u}}{\vect{v}}} + \vect{v} \metricProd{\vect{w}}{\vect{u}} - \metricProd{\covDiffVect{\vect{v}}{\vect{w}}}{\vect{u}}
\nonumber\\
 &= \vect{w} \metricProd{\vect{u}}{\vect{v}} - \metricProd{\LieBracket{\vect{w}}{\vect{u}}}{\vect{v}} - \vect{u} \metricProd{\vect{w}}{\vect{v}} + \metricProd{\vect{w}}{\LieBracket{\vect{u}}{\vect{v}}} + \vect{v} \metricProd{\vect{w}}{\vect{u}} + \metricProd{\LieBracket{\vect{v}}{\vect{w}}}{\vect{u}}
\nonumber\\
 &\qquad - \metricProd{\covDiffVect{\vect{w}}{\vect{v}}}{\vect{u}}
 .
\end{align}
With some reordering this is
\begin{align}
 \metricProd{\vect{u}}{\covDiffVect{\vect{w}}{\vect{v}}} = \tfrac{1}{2} \big(
   &\vect{w} \metricProd{\vect{u}}{\vect{v}}
 + \vect{v} \metricProd{\vect{u}}{\vect{w}}
 - \vect{u} \metricProd{\vect{v}}{\vect{w}}
\nonumber\\
 + &\metricProd{\vect{w}}{\LieBracket{\vect{u}}{\vect{v}}}
 + \metricProd{\vect{v}}{\LieBracket{\vect{u}}{\vect{w}}}
 - \metricProd{\vect{u}}{\LieBracket{\vect{v}}{\vect{w}}}
 \big).
\label{eq:LeviCivitaVect}
\end{align}
This result \eqref{eq:LeviCivitaVect} can also be found in \cite[proof of Theorem 2.7.6]{Abraham:FoundationsOfMechanics}, but is evaluated in terms of coordinate basis vectors so that the Lie brackets cancel.

Recall the definition \eqref{eq:DefConnCoeff} of the connection coefficients $\ConnCoeff{\LidxI}{\LidxII}{\LidxIII}$ and introduce their completely covariant\footnote{Covariant refers to the placement of the indices, $\ConnCoeffL{\LidxI}{\LidxII}{\LidxIII}$ is not a tensor just as $\ConnCoeff{\LidxI}{\LidxII}{\LidxIII}$ is not.} version $\ConnCoeffL{\LidxI}{\LidxII}{\LidxIII}$ as
\begin{align}
 \covDiffVect{\partialVect_{\LidxII}}{\partialVect_{\LidxI}} &= \ConnCoeff{\LidxIII}{\LidxI}{\LidxII} \partialVect_{\LidxIII},&
 \metricProd{\partialVect_\LidxI}{\covDiffVect{\partialVect_{\LidxIII}}{\partialVect_{\LidxII}}} &= \sysInertiaMatCoeff{\LidxI\LidxV} \ConnCoeff{\LidxV}{\LidxII}{\LidxIII} = \ConnCoeffL{\LidxI}{\LidxII}{\LidxIII}
\end{align}
Evaluate \eqref{eq:LeviCivitaVect} for the basis vectors, \ie $\vect{u}=\partialVect_\LidxI$, $\vect{v}=\partialVect_\LidxII$ and $\vect{w}=\partialVect_\LidxIII$, and recall that $\metricProd{\partialVect_\LidxI}{\partialVect_\LidxII} = \sysInertiaMatCoeff{\LidxI\LidxII}$ and $\LieBracket{\partialVect_{\LidxI}}{\partialVect_{\LidxII}} = \BoltzSym{\LidxIII}{\LidxI}{\LidxII} \partialVect_{\LidxIII}$ we get the explicit formula for the \textit{Levi-Civita connection coefficients}:
\begin{align}\label{eq:DefConnCoeffLeviCivita}
% \sysInertiaMatCoeff{\LidxI\LidxV} \ConnCoeff{\LidxV}{\LidxII}{\LidxIII}
 \ConnCoeffL{\LidxI}{\LidxII}{\LidxIII} &= \tfrac{1}{2}\big( \partial_\LidxIII \sysInertiaMatCoeff{\LidxI\LidxII} + \partial_\LidxII \sysInertiaMatCoeff{\LidxI\LidxIII} - \partial_\LidxI \sysInertiaMatCoeff{\LidxII\LidxIII} + \BoltzSym{\LidxV}{\LidxI}{\LidxII} \sysInertiaMatCoeff{\LidxV\LidxIII} + \BoltzSym{\LidxV}{\LidxI}{\LidxIII} \sysInertiaMatCoeff{\LidxV\LidxII} - \BoltzSym{\LidxV}{\LidxII}{\LidxIII} \sysInertiaMatCoeff{\LidxV\LidxI} \big)
\end{align}
Note that the coordinate expressions for \eqref{eq:LeviCivitaMetric} and \eqref{eq:LeviCivitaSymmetry} are
\begin{subequations}
\begin{align}
 \partial_\LidxIII \sysInertiaMatCoeff{\LidxI\LidxII} &= \ConnCoeffL{\LidxI}{\LidxII}{\LidxIII} + \ConnCoeffL{\LidxII}{\LidxI}{\LidxIII}
\\
 \label{eq:blah21341313}
 \BoltzSym{\LidxIII}{\LidxI}{\LidxII} &= \ConnCoeff{\LidxIII}{\LidxII}{\LidxI} - \ConnCoeff{\LidxIII}{\LidxI}{\LidxII}
\end{align}
\end{subequations}

\begin{Remark}
In the special case of coordinate basis vectors the Levi-Civita connection coefficients are called the \textit{Christoffel symbols} (\eg \cite[sec.\,9.2]{Frankel:GeometryOfPhysics}):
\begin{align}
 \partial_\LidxI = \tpdiff{\genCoordCoeff{\LidxI}}
\quad \Rightarrow \quad
 \BoltzSym{\LidxIII}{\LidxI}{\LidxII} = 0, \quad
 \ConnCoeffL{\LidxI}{\LidxII}{\LidxIII} &= \tfrac{1}{2} \Big( \tpdiff[{\sysInertiaMatCoeff{\LidxI\LidxII}}]{\genCoordCoeff{\LidxIII}} + \tpdiff[{\sysInertiaMatCoeff{\LidxI\LidxIII}}]{\genCoordCoeff{\LidxII}} - \tpdiff[{\sysInertiaMatCoeff{\LidxII\LidxIII}}]{\genCoordCoeff{\LidxI}} \Big) = \ConnCoeffL{\LidxI}{\LidxIII}{\LidxII}
\end{align}
In \cite[eq.\,4.10.9]{Lurie:AnalyticalMechanics} only the first three terms of \eqref{eq:DefConnCoeffLeviCivita} are introduced as the ``generalized Christoffel symbols'' in the context of minimal coordinates and non-coordinate basis vectors.
This might be misleading since these quantities do not obey the transformation rule \eqref{eq:TrafoRuleConnCoeff}, so do not define a connection.
\end{Remark}

\paragraph{Relation to mechanics.}
Note that we have found the expression \eqref{eq:DefConnCoeffLeviCivita} for the connection coefficients $\ConnCoeffL{\LidxI}{\LidxII}{\LidxIII}$ already in the context of the explicit equations of motion in (??).
Furthermore consider the covariant derivative of the velocity vector $\sysVelVect$ along itself
\begin{align}
 \covDiffVect{\sysVelVect}{\sysVelVect}
 = \big( \partial_\LidxII \sysVelCoeff{\LidxI} + \ConnCoeff{\LidxI}{\LidxIII}{\LidxII} \sysVelCoeff{\LidxIII} \big) \sysVelCoeff{\LidxII} \partialVect_\LidxI
 = \big( \tpdiff[{\sysVelCoeff{\LidxI}}]{\sysCoordCoeff{\GidxI}} \underbrace{\kinMatCoeff{\GidxI}{\LidxII} \sysVelCoeff{\LidxII}}_{\sysCoordCoeffd{\GidxI}} + \ConnCoeff{\LidxI}{\LidxIII}{\LidxII} \sysVelCoeff{\LidxIII} \sysVelCoeff{\LidxII} \big) \partialVect_\LidxI
 = \big( \sysVelCoeffd{\LidxI} + \ConnCoeff{\LidxI}{\LidxII}{\LidxIII} \sysVelCoeff{\LidxII} \sysVelCoeff{\LidxIII} \big) \partialVect_\LidxI
 .
\end{align}
Consequently we have
\begin{align}\label{eq:InertiaForceRiemannianGeo}
 \metricProd{\partialVect_\LidxI}{\covDiffVect{\sysVelVect}{\sysVelVect}}
 = \sysInertiaMatCoeff{\LidxI\LidxII} \sysVelCoeffd{\LidxII} + \ConnCoeffL{\LidxI}{\LidxII}{\LidxIII} \sysVelCoeff{\LidxII} \sysVelCoeff{\LidxIII}
 = \genForceInertiaCoeff{\LidxI},
\qquad \LidxI = 1\ldots\dimConfigSpace
\end{align}
\ie the coefficients $\genForceInertiaCoeff{\LidxI}$ of the generalized inertia force are the projections of the covariant derivative $\covDiffVect{\sysVelVect}{\sysVelVect}$ of the velocity vector along itself to the tangent vectors $\partialVect_\LidxI$.
Equivalently we could say $\genForceInertiaCoeff{\LidxI}$ are the coefficients of the co-vector corresponding to $\covDiffVect{\sysVelVect}{\sysVelVect}$.

\paragraph{Relation to Euclidean geometry.}
In the previous subsection on Euclidean geometry we considered a special choice of coordinates, such that the kinetic energy $\kineticEnergy$ resembles the Euclidean metric and the inertia matrix $\sysInertiaMatCoeff{\LidxI\LidxII} = \kinMatCoeff{\GidxI}{\LidxI} \delta_{\GidxI\GidxII} \kinMatCoeff{\GidxII}{\LidxII}$ is the \textit{induced metric} for the tangent space.
In total we have
\begin{subequations}
\begin{align}
 \sysInertiaMatCoeff{\LidxI\LidxII} &= \kinMatCoeff{\GidxI}{\LidxI} \delta_{\GidxI\GidxII} \kinMatCoeff{\GidxII}{\LidxII},& 
 \ConnCoeffL{\LidxI}{\LidxII}{\LidxIII} &= \kinMatCoeff{\GidxI}{\LidxI} \delta_{\GidxI\GidxII} \partial_\LidxIII \kinMatCoeff{\GidxII}{\LidxII}
\\
 \isysInertiaMatCoeff{\LidxI\LidxII} &= (\kinMat^+)^{\LidxI}_{\GidxI} \delta^{\GidxI\GidxII} (\kinMat^+)^{\LidxII}_{\GidxII},&
 \ConnCoeff{\LidxI}{\LidxII}{\LidxIII} = \isysInertiaMatCoeff{\LidxI\LidxV} \ConnCoeffL{\LidxV}{\LidxII}{\LidxIII} &= (\kinMat^+)^{\LidxI}_{\GidxI} \partial_\LidxIII \kinMatCoeff{\GidxI}{\LidxII}
\end{align}
\end{subequations}
The computations here might take some extra lines but it is all pretty straight forward when expressing the commutation symbols (??) with the pseudo inverse and using the defining criteria of the pseudo inverse.

\fixme{Probably remove section on Euclidean geometry and put the conclusions here.}

\fixme{
The Levi-Civita connection resembles the result we would get from Euclidean geometry but \textit{without requiring an isometric embedding}
Due to the arbitrary choice of configuration coordinates the standard metric for the (embedded) configuration space is meaningless for the mechanics.
}

% \fixme{
% \paragraph{Geodesics}
% A geodesic is a curve $t \mapsto \sysCoordVect(t)$ for which $\covDiffVect{\sysCoordVectd}{\sysCoordVectd} = 0$.
% In terms of redundant coordinates and minimal velocity coordinates the geodesic equations is
% \begin{align}
%  \sysCoordCoeffd{\GidxI} = \kinMatCoeff{\GidxI}{\LidxI} \sysVelCoeff{\LidxI},
% \qquad
%  \sysVelCoeffd{\LidxI} + \ConnCoeff{\LidxI}{\LidxII}{\LidxIII} \sysVelCoeff{\LidxII} \sysVelCoeff{\LidxIII} = 0,
% \end{align}
% \cite[ch.\,10]{Frankel:GeometryOfPhysics}
% }

\section{Gradient and Hessian}
\paragraph*{Gradient.}
The gradient vector is commonly defined as
\begin{align}\label{eq:DefGradient}
 \vect{v}(f) = \metricProd{\vect{v}}{\gradVect f}
\end{align}
which yields the coordinate expression
\begin{align}
 (\gradVect f)^\LidxI = \isysInertiaMatCoeff{\LidxI\LidxII} \, \partial_\LidxII f
 .
\end{align}

\paragraph{Hessian.}
The Hessian tensor can be defined as (\eg \cite[sec.\,6.1.4]{Bullo:GeometricControl})
\begin{align}
 (\Hess f)(\vect{u},\vect{v}) = \metricProd{\vect{u}}{\covDiffVect{\vect{v}}{\gradVect f}}
 .
\end{align}
The symmetry of the Hessian might be not that obvious, but can be shown by using the definition of the gradient \eqref{eq:DefGradient} and the properties \eqref{eq:LeviCivitaProp} of the Levi-Civita connection:
\begin{align}
 &\metricProd{\vect{u}}{\covDiffVect{\vect{v}}{\gradVect f}} - \metricProd{\vect{v}}{\covDiffVect{\vect{u}}{\gradVect f}}
\nonumber\\
 &\qquad\overset{\eqref{eq:LeviCivitaMetric}}{=} \vect{v} \metricProd{\vect{u}}{\gradVect f} - \metricProd{\covDiffVect{\vect{v}}{\vect{u}}}{\gradVect f} 
  - \vect{u} \metricProd{\vect{v}}{\gradVect f} + \metricProd{\covDiffVect{\vect{u}}{\vect{v}}}{\gradVect f}
\nonumber\\
 &\qquad\overset{\eqref{eq:DefGradient}}{=} \vect{v} (\vect{u}(f)) - \vect{u} (\vect{v}(f)) - \metricProd{\covDiffVect{\vect{v}}{\vect{u}} - \covDiffVect{\vect{u}}{\vect{v}}}{\gradVect f}
\nonumber\\
 &\qquad\overset{\eqref{eq:LeviCivitaSymmetry}}{=} \LieBracket{\vect{v}}{\vect{u}}(f) - \metricProd{\LieBracket{\vect{v}}{\vect{u}}}{\gradVect f}
 \overset{\eqref{eq:DefGradient}}{=} 0
\end{align}

We get the coefficients by evaluating for the basis vectors
\begin{multline}
 (\Hess f)_{\LidxI\LidxII} = (\Hess f) (\partialVect_\LidxI, \partialVect_\LidxII)
 = \metricProd{\partialVect_\LidxI}{\covDiffVect{\partialVect_\LidxII}{\gradVect f}}
\\
 \overset{\eqref{eq:LeviCivitaMetric}}{=} \partial_\LidxII \metricProd{\partialVect_\LidxI}{\gradVect f} - \metricProd{\covDiffVect{\partialVect_\LidxII}{\partialVect_\LidxI}}{\gradVect f}
 \overset{\eqref{eq:DefGradient}, \eqref{eq:DefConnCoeff}}{=} \partial_\LidxII (\partial_\LidxI f) - \ConnCoeff{\LidxIII}{\LidxI}{\LidxII} \partial_\LidxIII f
\end{multline}
Note that the symmetry condition for the coefficients follows directly from the condition \eqref{eq:blah21341313}.
