\chapter{On nonholonomic systems}
Nonholonomic systems are characterized by having constraints expressed in terms of the \textit{velocity} $\sysCoordd$.
We restrict to linear constraints, \ie
\begin{align}\label{eq:nonholonomicConstraint}
 \kinConstraintMatCoeff{\NidxI}{\GidxI}(\sysCoord) \, \sysCoordCoeffd{\GidxI} = 0, \qquad \NidxI = 1\ldots\numKinConst
 .
\end{align}
since practical examples of nonlinear nonholonomic constraints seem to not exist, see \cite[p.\,499]{Hamel:TheoretischeMechanik} or \cite[ch.\,IV]{Neimark:NonholonomicSystems}.

\paragraph{Nonholonomy.}
So far \eqref{eq:nonholonomicConstraint} should rather be called a kinematic constraint.
It is nonholonomic if it is \textit{nonintegrable}, \ie there exists no function $\psi(\sysCoord)$ such that $\ddt \psi = \pdiff[\psi]{\sysCoord} \sysCoordd = \kinConstraintMat \sysCoordd$.
The reverse would imply
\begin{align}
 \kinConstraintMatCoeff{\NidxI}{\GidxI} = \pdiff[\psi^\NidxI]{\sysCoordCoeff{\GidxI}}
\qquad \Rightarrow \qquad
 \frac{\partial \kinConstraintMatCoeff{\NidxI}{\GidxI}}{\partial \sysCoordCoeff{\GidxII}} = \frac{\partial^2 \psi^\NidxI}{\partial\sysCoordCoeff{\GidxII} \partial \sysCoordCoeff{\GidxI}} = \frac{\partial \kinConstraintMatCoeff{\NidxI}{\GidxII}}{\partial \sysCoordCoeff{\GidxI}}
\end{align}
If this condition is fulfilled, we call \eqref{eq:nonholonomicConstraint} a holonomic kinematic constraint and the only difference to a geometric constraint (as treated in \eqref{eq:diffGeoConstraint}) is a missing initial condition.


\paragraph{Some trouble.}
It is well documented in the dedicated literature (\eg \cite{Hamel:HamiltonsNonhol}, \cite[p.\,208]{Bloch:NonholonomicMechanics}, probably first mentioned in \cite{Korteweg:UnrichtigeBehandlungsweise}) that in the case of nonholonomic constraints, the Lagrange-d'Alembert principle \eqref{eq:dAlembert} and the calculus of variations \eqref{eq:StatAction} lead to \textit{different} results.
In the context of the Lagrange-d'Alembert principle, the \textit{virtual displacements} $\delta \sysCoord$ have to be compatible with system constraints which implies
\begin{align}\label{eq:NonholConstVariation}
 \kinConstraintMatCoeff{\NidxI}{\GidxI} \,\delta \sysCoordCoeff{\GidxI} = 0
 .
\end{align}
This results in the same dynamical equations as derived with a Newton-Eulerian approach.
%The virtual displacements in the case of nonholonomic constraints are extensively discussed in \cite{Hamel:virtuelleVerschiebungen}.

On the other hand to get the solution to an associated variational problem the \textit{varied path} $t \mapsto \sysCoordVar(t) = \sysCoord(t) + \varParam \varFkt(t)$ has to obey the constraint, \ie
\begin{multline}\label{eq:NonholConstVakonomic}
%  0 = \kinConstraintMatCoeff{\NidxI}{\GidxI}(\underbrace{\sysCoord + \varParam\varFkt}_{\sysCoordVar}) \underbrace{(\sysCoordCoeffd{\GidxI} + \varParam \varFktCoeffd[\GidxI])}_{\sysCoordVarCoeffd[\GidxI]}
%  = \underbrace{\kinConstraintMatCoeff{\NidxI}{\GidxI} \sysCoordCoeffd{\GidxI}}_{0} + \pdiff[\kinConstraintMatCoeff{\NidxI}{\GidxI}]{\sysCoordCoeff{\GidxII}} \sysCoordCoeffd{\GidxI} \underbrace{\varParam\varFktCoeff{\GidxII}}_{\delta \sysCoordCoeff{\GidxII}} + \kinConstraintMatCoeff{\NidxI}{\GidxI} \underbrace{\varParam \varFktCoeffd[\GidxI]}_{\delta \sysCoordCoeffd{\GidxI}} + \mathcal{O}(\varParam^2),
 0 = \kinConstraintMatCoeff{\NidxI}{\GidxI}(\sysCoordVar) \sysCoordVarCoeffd[\GidxI]
 = \underbrace{\kinConstraintMatCoeff{\NidxI}{\GidxI} \sysCoordCoeffd{\GidxI}}_{0} + \kinConstraintMatCoeff{\NidxI}{\GidxI} \underbrace{\varParam \varFktCoeffd[\GidxI]}_{\delta \sysCoordCoeffd{\GidxI}} + \pdiff[\kinConstraintMatCoeff{\NidxI}{\GidxI}]{\sysCoordCoeff{\GidxII}} \underbrace{\varParam \varFktCoeff{\GidxII}}_{\delta \sysCoordCoeff{\GidxII}} \sysCoordCoeffd{\GidxI} + \mathcal{O}(\varParam^2),
\\
 = \diff{t} \left( \kinConstraintMatCoeff{\NidxI}{\GidxI} \delta \sysCoordCoeff{\GidxI} \right) + \left( \pdiff[\kinConstraintMatCoeff{\NidxI}{\GidxI}]{\sysCoordCoeff{\GidxII}} - \pdiff[\kinConstraintMatCoeff{\NidxI}{\GidxII}]{\sysCoordCoeff{\GidxI}} \right) \sysCoordCoeffd{\GidxI} \delta \sysCoordCoeff{\GidxII} + \mathcal{O}(\varParam^2)
\end{multline}
which is obviously a different constraint on the \textit{variation} $\delta \sysCoord$.
This branch is called \textit{vakonomic mechanics}, see \eg \cite[sec.\,1.4]{ArnoldKozlov:Mechanics}, and leads to different equations of motion unless the constraint is holonomic, \ie the term in the brackets in \eqref{eq:NonholConstVakonomic} vanishes.
See also \cite{Hamel:HamiltonsNonhol} for a deeper analysis.

There is consensus that the ``correct'' equations for physics, \ie the ones that are backed by experiments (see \cite{Lewis:Variational}), are the ones derived with Lagrange-d'Alembert principle with the constraint \eqref{eq:NonholConstVariation}.
Hamilton's principle can still be used when the constraints \eqref{eq:NonholConstVakonomic} that arise form the calculus of variations are replaced by the ``correct'' constraints $\kinConstraintMatCoeff{\NidxI}{\GidxI} \,\delta \sysCoordCoeff{\GidxI} = 0$ for the variation $\delta \sysCoord$, see \cite[sec.\,5.2]{Bloch:NonholonomicMechanics}
The vakonomic approach however, is interesting in its own right and has application for optimal control, \eg \cite{Bloch:NonholonomicMechanics}.

\paragraph{Equations of motion with Lagrange multipliers}
Assume we have a given holonomic mechanical system we have chosen possibly redundant coordinates $\sysCoord$ and a kinematic relation $\sysCoordd = \kinMat \sysVel$.
Now we like to add the nonholonomic constraint \eqref{eq:nonholonomicConstraint} that can be expressed in terms of the velocity coordinates $\sysVel$ as
\begin{align}
 \kinConstraintMatCoeff{\NidxI}{\GidxI} \, \sysCoordCoeffd{\GidxI} = 0
\qquad\Leftrightarrow\qquad
 \underbrace{\kinConstraintMatCoeff{\NidxI}{\GidxI} \kinMatCoeff{\GidxI}{\LidxI}}_{\accConstraintMatCoeff{\NidxI}{\LidxI}} \, \sysVelCoeffd{\LidxI} = 0, \qquad \NidxI = 1\ldots\numKinConst
\end{align}
Using the concept of the Lagrange multipliers from \eqref{eq:GaussPrincipleOfLeastConstraintP} we can write the kinetic equation as
\begin{align}
 \genForceInertiaCoeff{\LidxI} + \accConstraintMatCoeff{\NidxI}{\LidxI} \LagrangeMult[\NidxI] = \genForceExCoeff{\LidxI} - \partial_\LidxI \potentialEnergy, \qquad \LidxI = 1\ldots\dimConfigSpace
 .
\end{align}
Expressing the inertia forces $\genForceInertia$ by the kinetic energy \eqref{eq:GenForceInertiaKineticEnergy} and introducing the Lagrangian $\Lagrangian = \kineticEnergy-\potentialEnergy$ we have
\begin{align}\label{eq:blub12313}
 \diff{t} \pdiff[\Lagrangian]{\sysVelCoeff{\LidxI}} + \BoltzSym{\LidxIII}{\LidxI}{\LidxII} \sysVelCoeff{\LidxII} \pdiff[\Lagrangian]{\sysVelCoeff{\LidxIII}} - \kinMatCoeff{\GidxI}{\LidxI} \pdiff[\Lagrangian]{\sysCoordCoeff{\GidxI}} + \accConstraintMatCoeff{\NidxI}{\LidxI} \LagrangeMult[\NidxI] = \genForceExCoeff{\LidxI}, \qquad \LidxI = 1\ldots\dimConfigSpace
 .
\end{align}
This form for the special case that $\sysCoord = \genCoord$ and $\sysVel = \genCoordd$, so $\BoltzSym{}{}{}$, is widely used for nonholonomic systems and can be found in \eg \cite[eq.\,7.1.6]{Lurie:AnalyticalMechanics} or \cite[eq.\,2.29]{Goldstein:ClassicalMechanics}.
However, we could have expressed the inertia forces $\genForceInertiaCoeff{\LidxI} = \spdiff[\accEnergy]{\sysVelCoeffd{\LidxI}}$ by the acceleration energy (??) or directly $\genForceInertiaCoeff{\LidxI} = \sysInertiaMatCoeff{\LidxI\LidxII}\sysVelCoeffd{\LidxI} + \ConnCoeffL{\LidxI}{\LidxII}{\LidxIII} \sysVelCoeff{\LidxII} \sysVelCoeff{\LidxIII}$ by the inertia matrix just as well.

\paragraph{Equations of motion without Lagrange multipliers}
Usually we do not want to calculate the Lagrange multipliers explicitly.
Instead we like to use the kinematic constraint to reduce the order of the resulting equations of motion.
The crucial idea\footnote{Actually Hamel and Appell derived their formulations directly from Lagrange-d'Alembert's principle without the use of Lagrange multipliers, but the main idea remains the same.} of Appell \cite{Appell:formeGenerale} and Hamel \cite{Hamel:LagrangeEuler} is to chose the velocity coordinates $\sysVel$ such that the last $\numKinConst$ coincide with the kinematic constraint\footnote{This seemed to be their motivation for the use of velocity coordinates in the first place.}, \ie $\sysVel[\dimConfigSpace - \numKinConst + \NidxI] = \kinConstraintMatCoeff{\NidxI}{\GidxI} \sysCoordCoeffd{\GidxI}, \NidxI = 1\ldots\numKinConst$.
Using the structure from \eqref{eq:DefBasisOfVelocity} this is
\begin{align}\label{eq:kinNonholonimc}
 \underbrace{\begin{bmatrix} \ \kinBasisMatF{}{} \ \ \\ \kinConstraintMat \\ \geoConstraintMat \end{bmatrix}}_{\kinBasisMatSquare} \sysCoordd
 = \begin{bmatrix} \, \sysVelF \,\,  \\ \sysVelKC \\ 0 \end{bmatrix},
\quad \rank \kinBasisMatSquare = \numCoord.
\qquad \Leftrightarrow \qquad
 \sysCoordd = [ \, \underbrace{\kinMatF{}{} \ \kinMatKC{}{}}_{\kinMatA{}{}} \ \kinMatGC{}{} \,] \begin{bmatrix} \, \sysVelF \,\, \\ \sysVelKC \\ 0 \end{bmatrix}
\end{align}
With this choice we can express the nonholonomic constraint simply by $\sysVelKC = 0$ and we have $\accConstraintMat = [\,0 \ \idMat[\numKinConst]\,]$.
Consequently the last $\numKinConst$ components of the force balance \eqref{eq:blub12313} define the reaction forces $\LagrangeMult[1]\ldots\LagrangeMult[\numKinConst]$ while the first $\dimConfigSpace - \numKinConst$ are independent of them.

Combining these considerations and expressing the generalized inertia force $\genForceInertia$ in terms of the \textit{acceleration energy} $\accEnergy$ from (??) we find \textit{Appell's equation}
\begin{align}\label{eq:blub12aasdsasdd313}
 \pdiff[\accEnergy]{\sysVelCoeffd{\LidxI}} \Big|_{\sysVel[\dimConfigSpace-\numKinConst+1]\ldots\sysVel[\dimConfigSpace] = 0} + \partial_\LidxI \potentialEnergy
 = \genForceExCoeff{\LidxI},
\qquad \LidxI = 1\ldots\dimConfigSpace-\numKinConst.
\end{align}
Expressing the generalized inertia force $\genForceInertia$ in terms of the \textit{kinetic energy} $\kineticEnergy$ from (??) and introducing the Lagrangian $\Lagrangian = \kineticEnergy-\potentialEnergy$ we find \textit{Hamel's equation}
\begin{align}\label{eq:HamelNonhol}
 \bigg( \diff{t} \pdiff[\Lagrangian]{\sysVelCoeff{\LidxI}} + \BoltzSym{\LidxIII}{\LidxI}{\LidxII} \sysVelCoeff{\LidxII} \pdiff[\Lagrangian]{\sysVelCoeff{\LidxIII}} - \kinMatCoeff{\GidxI}{\LidxI} \pdiff[\Lagrangian]{\sysCoordCoeff{\GidxI}} \bigg) \Big|_{\sysVel[\dimConfigSpace-\numKinConst+1]\ldots\sysVel[\dimConfigSpace]=0}
 = \genForceExCoeff{\LidxI}, \qquad \LidxI = 1\ldots\dimConfigSpace-\numKinConst.
\end{align}
Note that the dummy index $\LidxIII$ still runs over $1\ldots\dimConfigSpace$ so we still need to differentiate $\Lagrangian$ \wrt the velocity coordinates $\sysVel[\dimConfigSpace-\numKinConst+1]\ldots\sysVel[\dimConfigSpace]$ that will be set to zero \textit{afterwards}.

We like to investigate this a bit further:
First introduce the \textit{constraint Lagrangian} $\LagrangianC$ :
\begin{align}
 \LagrangianC = \Lagrangian \big|_{\sysVel[\dimConfigSpace-\numKinConst+1]\ldots\sysVel[\dimConfigSpace]=0}.
\end{align}
Then recall the splitting of the velocity coordinates $\sysVelA = [\sysVelF^\top \, \sysVelKC^\top]^\top$ from \eqref{eq:kinNonholonimc} and split the commutation coefficients in the same way:
\begin{subequations}
\begin{align}
 \BoltzSymA{\LidxIII}{\LidxI}{\LidxII}
 = \bigg(\pdiff[\kinBasisMatF{\LidxIII}{\GidxI}]{\sysCoordCoeff{\GidxII}} - \pdiff[\kinBasisMatF{\LidxIII}{\GidxII}]{\sysCoordCoeff{\GidxI}} \bigg) \kinMatF{\GidxI}{\LidxI} \kinMatF{\GidxII}{\LidxII}
 &= \BoltzSymF{\LidxIII}{\LidxI}{\LidxII},&
 \LidxI,\LidxII,\LidxIII &=1\ldots\dimConfigSpace-\numKinConst
\\
 \label{eq:BoltzSymNonholonomic}
 \BoltzSymA{\dimConfigSpace-\numKinConst+\NidxI}{\LidxI}{\LidxII}
 = \bigg(\pdiff[\kinConstraintMatCoeff{\NidxI}{\GidxI}]{\sysCoordCoeff{\GidxII}} - \pdiff[\kinConstraintMatCoeff{\NidxI}{\GidxII}]{\sysCoordCoeff{\GidxI}} \bigg) \kinMatF{\GidxI}{\LidxI} \kinMatF{\GidxII}{\LidxII}
 &= \BoltzSymKC{\NidxI}{\LidxI}{\LidxII},&
 \LidxI,\LidxII &= 1\ldots\dimConfigSpace-\numKinConst, \ \ \NidxI = 1,\ldots,\numKinConst
\end{align}
\end{subequations}
Then we can write \eqref{eq:HamelNonhol} as
\begin{align}\label{eq:HamelNonholSplit}
 \diff{t} \pdiff[\LagrangianC]{\sysVelF[\LidxI]} + \BoltzSymF{\LidxIII}{\LidxI}{\LidxII} \sysVelF[\LidxII] \pdiff[\LagrangianC]{\sysVelCoeff{\LidxIII}}
 + \underbrace{\BoltzSymKC{\NidxI}{\LidxI}{\LidxII} \sysVelF[\LidxII] \pdiff[\Lagrangian]{\sysVelKC[\NidxI]}\Big|_{\sysVelKC=0}}_{\gyroForceNonholonomicCoeff{\LidxI}}
 - \kinMatF{\GidxI}{\LidxI} \pdiff[\LagrangianC]{\sysCoordCoeff{\GidxI}}
 = \genForceExCoeff{\LidxI}, \qquad \LidxI = 1\ldots\dimConfigSpace-\numKinConst.
\end{align}
Note that if the kinematic constraint \eqref{eq:nonholonomicConstraint} is holonomic then $\BoltzSymKC{}{}{} = 0$ and consequently $\gyroForceNonholonomic = 0$.

\fixme{Discussion}

\paragraph{EoM with minimal coordinates.}
Assume that for a special choice of minimal coordinates $\genCoord = [\genCoordFTranspose, \genCoordCTranspose]^\top$ the kinematic constraint \eqref{eq:nonholonomicConstraint} can be transformed to
\begin{align}\label{eq:LagrangeDAlembertKinematic}
 \genCoordCd[\NidxI] = -W^{\NidxI}_{\LidxI} \genCoordFd[\LidxI], \qquad \NidxI = 1\ldots\numKinConst .
\end{align}
Choose the velocity coordinates $\sysVelF = \genCoordFd$ and $\sysVelKC = \genCoordCd + W \genCoordFd$ then we find that $\BoltzSymF{}{}{} = 0$ and
\begin{align}\label{eq:LagrangeDAlembertBoltz}
 \BoltzSymKC{\NidxI}{\LidxI}{\LidxII} &= \pdiff[W^\NidxI_\LidxI]{\genCoordF[\LidxII]} - \pdiff[W^\NidxI_\LidxII]{\genCoordF[\LidxI]} + \pdiff[W^\NidxI_\LidxII]{\genCoordC[\NidxII]} W^\NidxII_\LidxI - \pdiff[W^\NidxI_\LidxI]{\genCoordC[\NidxII]} W^\NidxII_\LidxII,
\qquad
 \NidxI = 1,\ldots,\numKinConst \quad \LidxI,\LidxII = 1,\ldots,\dimConfigSpace-\numKinConst
 .
\end{align}
The kinetic equation \eqref{eq:HamelNonholSplit} takes the form
\begin{align}\label{eq:LagrangeDAlembertKinetic}
 \diff{t} \pdiff[\LagrangianC]{\genCoordFd[i]} - \pdiff[\LagrangianC]{\genCoordF[i]}
 + \underbrace{\BoltzSymKC{\NidxI}{\LidxI}{\LidxII} \genCoordFd[j] \pdiff[\Lagrangian]{\genCoordCd[a]}|_{\genCoordCd=-S\genCoordFd}}_{\gyroForceNonholonomicCoeff{\LidxI}}
 + W^\NidxI_\LidxI \pdiff[\LagrangianC]{\genCoordC[\NidxI]} &= \genForceExCoeff{\LidxI},
\quad i=1,\ldots,\dimConfigSpace-\numKinConst .
\end{align}
The combination of \eqref{eq:LagrangeDAlembertKinematic}, \eqref{eq:LagrangeDAlembertBoltz} and \eqref{eq:LagrangeDAlembertKinetic} are called \textit{the Lagrange-d'Alembert equations of motion} in \cite[Def.\,2.1]{bloch:NonholonomicMechanicalSystemsWithSymmetry}.

% \begin{figure}[ht]
%  \centering
%  \input{graphics/RollingEllipsoid.pdf_tex}
%  \caption{Ellipsoid rolling on a plane}
%  \label{fig:RollingEllipsoid}
% \end{figure}

\begin{Example}{A rolling rigid body}
Consider a rigid body, \eg the ellipsoid from (??), rolling on a plane.
We use the position $\r$ of a body fixed point and the orientation $\R$ as redundant configuration coordinates $\sysCoord \cong (\r, \R)$.
The condition of rolling without slipping can be formulated as
\begin{align}
 \rd + \Rd p = 0
\qquad \cong \qquad
 \sysVelKC = \kinConstraintMat(\sysCoord) \sysCoordd = 0
\end{align}
where $p$ are the coordinates of the contact point \wrt the body fixed frame which depends on the current configuration $p = p(\sysCoord)$.

As velocity coordinates we choose $\sysVel = \w$ and obtain the kinematic relations
\begin{align}
 \w = (\R^\top \Rd)^\vee
\qquad \Rightarrow \quad
 \rd = \R \widehat{p} \w, \ \Rd = \R \widehat{\w}
\qquad \cong \qquad \sysCoordd = \kinMatF{}{}(\sysCoord) \sysVel 
\end{align}


The kinetic energy of the rigid body in terms of the chosen coordinates is
\begin{align}
 \kineticEnergy &= \tfrac{1}{2} m \norm{\rd}^2 + \tfrac{1}{2} \w^\top \Theta \w
 = \underbrace{\tfrac{1}{2} \sysVelT \overbrace{(\Theta + m \widehat{p}^\top \widehat{p})}^{\tilde{\sysInertiaMat}} \sysVel}_{\tilde{\kineticEnergy}}
 - m (R \widehat{\sysVel} p)^\top \sysVelKC
 + \tfrac{1}{2} m \norm{\sysVelKC}^2
\end{align}
potential energy from gravity
\begin{align}
 \potentialEnergy = -m\gravityAcc^\top \r 
\end{align}

With these ingredients we can evaluate \eqref{eq:HamelNonhol} to obtain the kinetic equation.
Using $\partial_\LidxI p \sysVelCoeff{\LidxI} = \dot{p}$ we can write them in a matrix vector form
\begin{align}
 \left[ \diff{t} \pdiff[\tilde{\kineticEnergy}]{\sysVelF[\LidxI]} + \BoltzSymF{\LidxIII}{\LidxI}{\LidxII} \sysVelF[\LidxII] \pdiff[\tilde{\kineticEnergy}]{\sysVelCoeff{\LidxIII}} - \kinMatF{\GidxI}{\LidxI} \pdiff[\tilde{\kineticEnergy}]{\sysCoordCoeff{\GidxI}} \right]_{\LidxI = 1\ldots3} 
 &= \tilde{\sysInertiaMat} \sysVeld + \big(\underbrace{m (\dot{\widehat{p}}^\top \widehat{p} + \widehat{p}^\top \dot{\widehat{p}})}_{\dot{\tilde{\sysInertiaMat}}} +  \widehat{\sysVel} \tilde{\sysInertiaMat} \big) \sysVel
\\
 \gyroForceNonholonomic = \left[ \BoltzSymKC{\NidxI}{\LidxI}{\LidxII} \sysVelF[\LidxII] \pdiff[\kineticEnergy]{\sysVelKC[\NidxI]}\Big|_{\sysVelKC=0} \right]_{\LidxI = 1\ldots3}
 &= m \, \dot{\widehat{p}} \, \widehat{p} \, \sysVel
\\
 \genForcePotential = \differential \potentialEnergy
 &= m \widehat{p}^\top \R^\top \gravityAcc
\end{align}


\textbf{Rolling ellipsoid:}
Constraint $\psi(p)$ for the ellipsoid surface (with $D = \diag (d_1^2, d_2^2, d_3^2)$) and the tangent condition yield
\begin{align}
 \psi(p) = p^\top D^{-1} p - 1 = 0,
\quad
 \nabla \psi(p) \times R^\top e_3 = 0
\quad \Rightarrow \quad
 p = \frac{D R^\top e_3}{\sqrt{e_3^\top R D R^\top e_3}}.
\end{align}
\end{Example}

Newton-Euler
\begin{align}\label{RollingRBNewton}
 m \ddot{r} &= F^R + F^A,&
 \Theta \dot{\omega} + \widehat{\omega}\Theta \omega &= \widehat{p} R^\top F^R + \tau^A
\end{align}
The condition of rolling without slipping implies
\begin{align}
 \dot{r} + R \widehat{\omega} p = 0
\quad \Rightarrow \quad
 \ddot{r} = R (\widehat{p} \dot{\omega} - \widehat{\omega}^2 p - \widehat{\omega} \dot{p})
\end{align}
where $p$ are the coordinates of the contact point \wrt the body fixed frame.
We can eliminate $\ddot{r}$ and the reaction force $F^R$ from \eqref{RollingRBNewton} to obtain (cf. \cite{Borisov:RollingRB})
\begin{align}
 \tilde{\Theta} \dot{\omega} + \widehat{\omega} \tilde{\Theta} \omega + m \widehat{p}^\top \dot{\widehat{p}} \, \omega &= \tau^A - \widehat{p} R^\top F^A
\end{align}
where $\tilde{\Theta} = \Theta + m \widehat{p}^\top \widehat{p}$.
