\subsection{Two masses connected by a spring}
In order to illustrate the control approach for underactuated systems we consider the minimal example:
Two bodies in prismatic joints connected by a linear spring but where only one is directly actuated by the force $F$ as illustrated in \autoref{fig:TwoMassSpring}.
\begin{figure}[ht]
 \centering
 \input{graphics/TwoMassSpring.pdf_tex}
 \caption{Model of two bodies connected by a spring}
 \label{fig:TwoMassSpring}
\end{figure}

\paragraph{Model.}
We choose the absolute positions of the bodies as configuration coordinates $\sysCoord = [z_1, z_2]^\top$ and their derivative as velocity coordinates $\sysVel = \sysCoordd$.
With this the body configurations are
\begin{align}
 \bodyHomoCoord{0}{1} = \begin{bmatrix} 1 & 0 & 0 & z_1 \\ 0 & 1 & 0 & 0 \\ 0 & 0 & 1 & 0 \\ 0 & 0 & 0 & 1\end{bmatrix},
\qquad
 \bodyHomoCoord{0}{2} = \begin{bmatrix} 1 & 0 & 0 & z_2 \\ 0 & 1 & 0 & 0 \\ 0 & 0 & 1 & 0 \\ 0 & 0 & 0 & 1\end{bmatrix}.
\end{align}
With the total mass $\bodyMass{0}{1}$, $\bodyMass{0}{2}$ of the individual bodies and the spring stiffness $\bodyStiffness{1}{2}$ the resulting equations of motion may be written as
\begin{align}\label{eq:TwoMassSpringModel}
 \underbrace{\begin{bmatrix} \bodyMass{0}{1} & 0 \\ 0 & \bodyMass{0}{2} \end{bmatrix}}_{\sysInertiaMat}
 \begin{bmatrix} \ddot{z}_1 \\ \ddot{z}_2 \end{bmatrix}
 +
 \underbrace{\begin{bmatrix} \bodyStiffness{1}{2} & -\bodyStiffness{1}{2} \\ -\bodyStiffness{1}{2} & \bodyStiffness{1}{2} \end{bmatrix}}_{\sysStiffMat}
 \begin{bmatrix} z_1 \\ z_2 \end{bmatrix} 
 =
 \underbrace{\begin{bmatrix} 1 \\ 0 \end{bmatrix}}_{\sysInputMat} F
 .
\end{align}
% Obviously the second equation is independent of the control force $\sysInput = F$:
% \begin{align}\label{eq:accConstTwoMassSpring}
%  \sysInputMatLComp = [0 \ 1], \qquad \sysInputMatLComp (\sysInertiaMat \sysVeld + \d_{\sysVel} \potentialEnergy) &= 0&
% &\Rightarrow&
%  \ddot{z}_2 + \underbrace{\tfrac{\bodyStiffness{1}{2}}{\bodyMass{0}{2}}}_{\varpi} (z_2 - z_1) &= 0
%  .
% \end{align}
% From this its obvious that $z_2$ is a \textit{flat output} of the system and we can utilize this to compute suitable reference trajectories $t \mapsto (\sysCoord[1](t), \sysCoord[2](t))$ which obey the EOM.

\paragraph{Desired closed loop.}
We assume general body inertia $\bodyInertiaMat{0}{1},\ldots$, damping and stiffness.
The three proposed control approaches lead to identical desired closed loop dynamics:
\begin{align}\label{eq:TwoMassSpringClosedLoop}
 \sysInertiaMatC \sysCoordEdd + \sysDissMatC \sysCoordEd + \sysStiffMatC \sysCoordE = \tuple{0},
\qquad
 \sysCoordE = \sysCoord - \sysCoordR
\end{align}
where
\begin{align}
 \sysInertiaMatC &= \begin{bmatrix} \bodyMassC{0}{1} + \bodyMassC{1}{2} & -\bodyMassC{1}{2} \\ -\bodyMassC{1}{2} & \bodyMassC{0}{2}+\bodyMassC{1}{2} \end{bmatrix},&
 \sysDissMatC &= \begin{bmatrix} \bodyDampingC{0}{1} + \bodyDampingC{1}{2} & -\bodyDampingC{1}{2} \\ -\bodyDampingC{1}{2} & \bodyDampingC{0}{2}+\bodyDampingC{1}{2} \end{bmatrix},&
 \sysStiffMatC &= \begin{bmatrix} \bodyStiffnessC{0}{1} + \bodyStiffnessC{1}{2} & -\bodyStiffnessC{1}{2} \\ -\bodyStiffnessC{1}{2} & \bodyStiffnessC{0}{2}+\bodyStiffnessC{1}{2} \end{bmatrix}.
\end{align}
The corresponding potential $\potentialEnergyC = \tfrac{1}{2} \sysCoordE^\top \sysStiffMatC \sysCoordE$ has the obvious transport map $\sysTransportMap = \idMat[2]$.


\paragraph{Matching.}
The matching condition \eqref{eq:MatchingCondition} for this example may be written as
\begin{align}
 \tuple{\lambda} = (\sysInputMatLComp)^\top (\sysInertiaMat \sysInertiaMatC^{-1} (-\sysInertiaMatC \sysCoordRdd + \sysDissMatC (\sysCoordd-\sysCoordRd) + \sysStiffMatC (\sysCoord-\sysCoordR)) - \sysStiffMat \sysCoord) &= \tuple{0}.
\end{align}
Since this equation is linear in the system coordinates we can separate the into
\begin{subequations}
\begin{align}
 (\sysInputMatLComp)^\top (\sysInertiaMat \sysCoordRdd + \sysStiffMat \sysCoordR) &= \tuple{0}.
\\
 (\sysInputMatLComp)^\top \sysInertiaMat \sysInertiaMatC^{-1} \sysDissMatC &= \mat{0},
\\
 (\sysInputMatLComp)^\top \sysInertiaMat \sysInertiaMatC^{-1} (\sysStiffMatC - \sysStiffMat) &= \mat{0},
\end{align}
\end{subequations}
Choosing $\sysInputMatLComp = [0, 1]^\top$ this is explicitly
\begin{subequations}
\begin{align}
 &\ddot{z}_{2\idxRef} + \varpi (z_{2\idxRef} - z_{1\idxRef}) = 0
 \label{eq:TwoMassSpringFlatnessCondition}
\\
 &\left\{ \begin{matrix}
  \bodyMassC{0}{1} \bodyDampingC{1}{2} - \bodyMassC{1}{2} \bodyDampingC{0}{1} = 0 \\
  \bodyMassC{0}{1} \bodyDampingC{1}{2} + (\bodyMassC{0}{1} + \bodyMassC{1}{2}) \bodyDampingC{0}{2} = 0
 \end{matrix} \right.
\\
 &\left\{ \begin{matrix}
  \bodyMassC{0}{1} \bodyStiffnessC{1}{2} - \bodyMassC{1}{2} \bodyStiffnessC{0}{1} = (\bodyMassC{0}{1} \bodyMassC{0}{2}+\bodyMassC{0}{1} \bodyMassC{1}{2}+\bodyMassC{0}{2} \bodyMassC{1}{2}) \varpi \\
 \bodyMassC{0}{1} \bodyStiffnessC{1}{2} + (\bodyMassC{0}{1} + \bodyMassC{1}{2}) \bodyStiffnessC{0}{2} = (\bodyMassC{0}{1} \bodyMassC{0}{2}+\bodyMassC{0}{1} \bodyMassC{1}{2}+\bodyMassC{0}{2} \bodyMassC{1}{2}) \varpi
 \end{matrix} \right.
\end{align}
\end{subequations}
where $\varpi = \sfrac{\bodyStiffness{1}{2}}{\bodyMass{0}{2}}$ is the sole model parameter relevant for the matching condition.
The first part \eqref{eq:TwoMassSpringFlatnessCondition} is a constraint on the reference trajectory as it is independent of tunable parameters.
It can be resolved by acknowledging that $z_2$ is a \textit{flat output} of the system and planing the reference trajectory accordingly, i.e.
\begin{align}
 z_{1\idxRef} = z_{2\idxRef} - \sfrac{\ddot{z}_{2\idxRef}}{\varpi}. 
\end{align}
The other conditions can be resolved by setting
\begin{align}\label{eq:TwoMassSpringMatchingConditionSolution}
 \bodyStiffnessC{1}{2} &=  \tfrac{\bodyMassC{1}{2}}{\bodyMassC{0}{1}} \bodyStiffnessC{0}{1} + \tfrac{\bodyMassC{0}{1} \bodyMassC{0}{2} + \bodyMassC{0}{1} \bodyMassC{1}{2} + \bodyMassC{0}{2} \bodyMassC{1}{2}}{\bodyMassC{0}{1}} \varpi,&
 \bodyStiffnessC{0}{2} &= -\tfrac{\bodyMassC{1}{2}}{\bodyMassC{0}{1}+\bodyMassC{1}{2}} \bodyStiffnessC{0}{1},&
 \bodyDampingC{1}{2} &= \tfrac{\bodyMassC{1}{2}}{\bodyMassC{0}{1}} \bodyDampingC{0}{1},&
 \bodyDampingC{0}{2} &= -\tfrac{\bodyMassC{1}{2}}{\bodyMassC{0}{1}+\bodyMassC{1}{2}} \bodyDampingC{0}{1},
\end{align}
which leaves the 5 tuning parameters $\bodyStiffnessC{0}{1}$, $\bodyDampingC{0}{1}$, $\bodyMassC{0}{1}$, $\bodyMassC{0}{2}$ and $\bodyMassC{1}{2}$.

The resulting control law is
\begin{multline}
 F = \bodyMass{0}{1} \ddot{z}_{1\idxRef}  +  \bodyStiffness{1}{2} (z_{1\idxRef} - z_{2\idxRef})
 + \big( \bodyStiffness{1}{2} + \tfrac{\bodyMass{0}{1} (\bodyMassC{1}{2} \bodyStiffness{1}{2} + \bodyMass{0}{2} (\bodyStiffnessC{0}{1} - \bodyStiffnessC{1}{2}))}{\bodyMass{0}{2} (\bodyMassC{0}{1} + \bodyMassC{1}{2})} \big) e_1
\\
 - \big( \bodyStiffness{1}{2} - \tfrac{\bodyMass{0}{1} (\bodyMassC{1}{2} \bodyStiffness{1}{2} - \bodyMass{0}{2} \bodyStiffnessC{1}{2})}{\bodyMass{0}{2} (\bodyMassC{0}{1} + \bodyMassC{1}{2})} \big) e_2
 - \tfrac{\bodyMass{0}{1} (\bodyDampingC{0}{1} + \bodyDampingC{1}{2})}{\bodyMassC{0}{1} + \bodyMassC{1}{2}} \dot{e}_1
 + \tfrac{\bodyMass{0}{1} \bodyDampingC{1}{2}}{\bodyMassC{0}{1} + \bodyMassC{1}{2}} \dot{e}_2
 .
\end{multline}

\paragraph{Pole placement.}
Tuning the design parameters under the given matching conditions might not be intuitive for this example.
To resolve this we can fall back to the classical approach of placing the eigenvalues of the closed loop system \eqref{eq:TwoMassSpringClosedLoop}.
Taking into account the matching condition \eqref{eq:TwoMassSpringMatchingConditionSolution}, the characteristic polynomial of \eqref{eq:TwoMassSpringClosedLoop} is
\begin{align}
 \tfrac{\det (\sysInertiaMatC \EigenVal^2 + \sysDissMatC \EigenVal + \sysStiffMatC)}{\det \sysInertiaMatC}
 =
 \EigenVal^4 
 + \underbrace{\tfrac{\bodyDampingC{0}{1}}{\bodyMassC{0}{1}}}_{p_3} \EigenVal^3
 + \underbrace{\tfrac{\bodyStiffnessC{0}{1} + (\bodyMassC{0}{1}+\bodyMassC{0}{2})\varpi}{\bodyMassC{0}{1}}}_{p_2} \EigenVal^2
 + \underbrace{\tfrac{\varpi \bodyDampingC{0}{1}}{\bodyMassC{0}{1}+\bodyMassC{1}{2}}}_{p_1} \EigenVal
 + \underbrace{\tfrac{\varpi \bodyStiffnessC{0}{1}}{\bodyMassC{0}{1}+\bodyMassC{1}{2}}}_{p_0}
 .
\end{align}
This can be solved for
\begin{subequations}
\begin{align}
 \bodyStiffnessC{0}{1}  &= \tfrac{\bodyMassC{1}{2} p_0 p_3}{\varpi p_3 - p_1},&
 \bodyDampingC{0}{1}    &= \tfrac{\bodyMassC{1}{2} p_1 p_3}{\varpi p_3 - p_1},&
 \bodyMassC{0}{1}       &= \tfrac{\bodyMassC{1}{2} p_1}{\varpi p_3 - p_1},&
 \bodyMassC{0}{2}       &= \tfrac{\bodyMassC{1}{2} (p_1 p_2 - p_0 p_3 - \varpi p_1)}{\varpi(\varpi p_3 - p_1)}
 .
\end{align}
\end{subequations}
and $\bodyMassC{1}{2} \in \RealNum \neq 0$.
Choosing any Hurwitz polynomial for the coefficients $p_i$ guarantees the asymptotic stability of the closed loop.
In order to conclude $\sysInertiaMatC > 0$, $\sysDissMatC \geq 0$ and $\sysStiffMatC > 0$ from the Hurwitz criterion ($p_0, p_1, p_2, p_3, p_1 p_2 - p_0 p_3, p_1 p_2 p_3 - p_1^2 - p_0 p_3^2 > 0$) we need $\sign \bodyMassC{1}{2} = \sign (\varpi p_3 - p_1)$.

\paragraph{Conclusions.}
The resulting controller is equivalent to one that could be designed by standard linear state-feedback methods.
However, this approach here might give some \textit{physical} insight to the resulting closed loop system.
For example that the closed loop system must have an inertial coupling ($\bodyMassC{1}{2} \neq 0$) of the two bodies, if one wants to tune all 4 poles.
