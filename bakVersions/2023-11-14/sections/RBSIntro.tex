\chapter{Rigid body systems}
A \textit{rigid body} can be regarded as a special case of a particle system whose particles have constant distance to each other.
Several rigid bodies that may be constrained to each other and/or to the surrounding space constitute a \textit{rigid body system}.
This chapter applies the results from the previous section to them.
As the configuration space of a free rigid body is nonlinear, the use of redundant coordinates is very appropriate.

\paragraph{Goal.}
The goal for this section is to motivate a recipe or formalism for the derivation of the equations of motion for rigid body systems.
In contrast to established formalisms, the approach here should be most flexible w.r.t.\ the parameterization of the configuration and velocity.
Furthermore, it should not only apply to the inertia of a system but also to damping and stiffness, as motivated in the previous chapter.
Overall, this should give deeper insight into the structure of rigid body systems which will be exploited in the next chapter on control of rigid body systems.


\fixme{
\cite[chap. 4]{Goldstein:ClassicalMechanics}: A rigid body is defined as a system of mass points subject top the holonomic constraints thet the distances between all pairs of points remain constant throughout the motion.
\\
\cite[§31]{Landau:Mechanics}: A rigid body may be defined in mechanics as a system of particles such that the distances between the particles do not vary.
\\
\cite[§44]{Boltzmann:PrincipeDerMechanik}: \ldots starrer Körper, d.h.\ für ein System materieller Punkte, welche so verbunden sind, dass sich während ihrer ganzen Bewegung ihre relative Lage nicht ändern kann.
\\
\cite[§28]{Arnold:MathematicalMethodsOfClassicalMechanics}: A rigid body is a system of point masses, constrained by holonomic relations expressed by the fact that the distance between points is constant.
\\
\cite[sec.\ 4.1]{Bremer:ElasticMultibodyDynamics}: A body can be defined as a number N of particles or mass points with N going to infinity.
\\
\cite[sec.\ 23]{Hamel:TheoretischeMechanik}: nicht wirklich hilfreich
\\
\cite[sec.\ 3.1.1]{Schwertassek:MultibodySystems}: definition by frames
\\
\cite{Kane:Dynamics} no direct definition
\\
\cite{Shabana:MultibodySystems} no direct definition
\\
\cite[p.\ 229]{Abraham:FoundationsOfMechanics} no direct definition, but ``The constrained system par excellence is the rigid body''
}
