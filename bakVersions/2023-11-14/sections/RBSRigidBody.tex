\section{A single free rigid body}\label{sec:RB}
Established textbooks on physics (e.g.\ \cite[§31]{Landau:Mechanics} or \cite[§44]{Boltzmann:PrincipeDerMechanik}) define a \textit{rigid body} as a a system of $\numParticles$ particles such that the distances $d_{\PidxI\PidxII} = \norm{\particlePos{\PidxI} - \particlePos{\PidxII}}$ between their positions $\particlePos{\PidxI}$ are constant.
For a system with $\numParticles > 3$ particles that do not lie in a plane, this would still allow a mirroring of the body about an arbitrary plane.
Since this does change the shape of the body, additional constraints are necessary:
For all particles we require $\sProd{(\particlePos{\PidxII}-\particlePos{\PidxI}) \times (\particlePos{\PidxIII}-\particlePos{\PidxI})}{\particlePos{\PidxIV}-\particlePos{\PidxI}} = c_{\PidxI\PidxII\PidxIII\PidxIV} = \const$.
The magnitude $|c_{\PidxI\PidxII\PidxIII\PidxIV}|$ is already fixed by the corresponding particle distances $d_{\PidxI\PidxII}$, but the sign of $c_{\PidxI\PidxII\PidxIII\PidxIV}$ is positive (negative) if the vectors $\particlePos{\PidxII}-\particlePos{\PidxI}$, $\particlePos{\PidxIII}-\particlePos{\PidxI}$ and $\particlePos{\PidxIV}-\particlePos{\PidxI}$ form a right (left) handed basis of $\RealNum^3$ and $c_{\PidxI\PidxII\PidxIII\PidxIV} = 0$ if the four particles lie on a plane.
For the following we require that for at least one quadruple of particles $c_{\PidxI\PidxII\PidxIII\PidxIV} \neq 0$, i.e.\ not all particles lie in a common plane. 

Based on the discussion above the configuration space for a rigid body as a system of constrained particles can be written as
\begin{align}
% \particleConfigSpace = \{ [\particlePos{1}^\top, \ldots \particlePos{\numParticles}^\top]^\top \in \RealNum^{3\numParticles} \, | \, \norm{\particlePos{\PidxI} - \particlePos{\PidxII}} = d_{\PidxI\PidxII}, \sProd{(\particlePos{\PidxII}-\particlePos{\PidxI}) \times (\particlePos{\PidxIII}-\particlePos{\PidxI})}{\particlePos{\PidxIV}-\particlePos{\PidxI}} = c_{\PidxI\PidxII\PidxIII\PidxIV} \}
 \particleConfigSpace = \left\{ \begin{bmatrix}\particlePos{1} \\ \vdots \\ \particlePos{\numParticles}\end{bmatrix} \in \RealNum^{3\numParticles} \ \Bigg| \ \begin{matrix} \norm{\particlePos{\PidxI} - \particlePos{\PidxII}} = d_{\PidxI\PidxII}, \\ \sProd{(\particlePos{\PidxII}-\particlePos{\PidxI}) \times (\particlePos{\PidxIII}-\particlePos{\PidxI})}{\particlePos{\PidxIV}-\particlePos{\PidxI}} = c_{\PidxI\PidxII\PidxIII\PidxIV}, \\ \PidxI,\PidxII,\PidxIII,\PidxIV = 1,\ldots,\numParticles \end{matrix} \right\}
\end{align}
Evidently, for large $\numParticles$ the number of constraints surpasses the count $3\numParticles$ of particle coordinates, so the geometric constraints cannot be independent.

\subsection{Blah}
\begin{figure}[ht]
 \centering
 \input{graphics/RBCoordinates.pdf_tex}
 \caption{RigidBodyIllustration}
 \label{fig:RigidBodyIllustration}
\end{figure}
For a given set of particle positions $(\particlePos{1},\ldots,\particlePos{\numParticles}) \in \particleConfigSpace$ and the body configuration $(\r,\R)\in\RealNum^3 \times \SpecialOrthogonalGroup(3)$ we can compute the quantities
\begin{align}
 \particleBodyPos{\PidxI} = \R^\top (\particlePos{\PidxI}-\r), \qquad \PidxI=1,\ldots,\numParticles.
\end{align}



There is a mapping $h\,:\,\configSpace = \RealNum^3\times \SpecialOrthogonalGroup(3) \rightarrow \particleConfigSpace$ defined as
\begin{align}
 \begin{bmatrix}\particlePos{1} \\ \vdots \\ \particlePos{\numParticles}\end{bmatrix}
 &=
 \underbrace{\begin{bmatrix} \r + \R \particleBodyPos{1} \\ \vdots \\ \r + \R \particleBodyPos{\numParticles} \end{bmatrix}}_{h}
\end{align}
For the inverse function we use the result from \autoref{sec:RBStiffness}:
\begin{align}
 \mathcal{H} = \sumParticles \norm{\r + \R \particleBodyPos{\PidxI} - \particlePos{\PidxI}}^2 \geq 0
\end{align}
has a minimum at
\begin{align}
 (\r, \R)  = (\tuple{p} - \bar{\mat{Y}} \bar{\mat{X}}^\top \bodyCOS{}{}, \bar{\mat{Y}} \bar{\mat{X}}^\top)
\end{align}
where
\begin{align}
 \bodyStiffness{}{} &= \sumParticles \particleStiffness{\PidxI},&
 \bodyCOS{}{} &= \bodyStiffness{}{}^{-1} \sumParticles \particleStiffness{\PidxI} \particleBodyPos{\PidxI},&
 \tuple{p} &= \bodyStiffness{}{}^{-1} \sumParticles \particleStiffness{\PidxI} \springHubPos{\PidxI},&
 \mat{P} &= \sumParticles \particleStiffness{\PidxI} \particleBodyPos{\PidxI} \springHubPos{\PidxI}^\top.
\end{align}



\subsection{Coordinates}
\paragraph{Configuration coordinates.}
Position $\r(t) \in \RealNum^3$ and orientation $\R \in \SpecialOrthogonalGroup(3)$.

The key argument here is
\begin{align}
 \particlePos{\PidxI} = \r + \R \particleBodyPos{\PidxI}, \quad \PidxI=1,\ldots,\numParticles
\end{align}
where $\r(t) \in \RealNum^3$, $\R(t) \in\SpecialOrthogonalGroup(3)$ and the \textit{constant} particle position $\particleBodyPos{\PidxI} \in \RealNum^3$ w.r.t.\ the body fixed frame.

\begin{align}
 \sysCoord = [\r^\top, \Rx^\top, \Ry^\top, \Rz^\top]^\top \ \in \
 \configSpace \cong \RealNum^3 \times \SpecialOrthogonalGroup(3) \cong \particleConfigSpace.
\end{align}

\begin{align}
 d_{\PidxI\PidxII} = \norm{\particlePos{\PidxI} - \particlePos{\PidxII}} = \norm{\R(\particleBodyPos{\PidxI} - \particleBodyPos{\PidxII})}
\\
 c_{\PidxI\PidxII\PidxIII\PidxIV} = \sProd{(\R(\particleBodyPos{\PidxII} - \particleBodyPos{\PidxI})) \times (\R(\particleBodyPos{\PidxIII} - \particleBodyPos{\PidxI}))}{\R(\particleBodyPos{\PidxIV} - \particleBodyPos{\PidxI})},
\end{align}




\paragraph{Velocity coordinates.}
\begin{align}\label{eq:KinematicsRigidBody}
 \underbrace{\begin{bmatrix} \rd \\ \Rxd \\ \Ryd \\ \Rzd  \end{bmatrix}}_{\sysCoordd}
 = 
 \underbrace{\begin{bmatrix}
  \Rx & \Ry & \Rz & 0 & 0 & 0 \\
  0 & 0 & 0 & 0 & -\Rz & \Ry \\
  0 & 0 & 0 & \Rz & 0 & -\Rx \\
  0 & 0 & 0 & -\Ry & \Rx & 0
 \end{bmatrix}}_{\kinMat}
 \underbrace{\begin{bmatrix} \vx \\ \vy \\ \vz \\ \wx \\ \wy \\ \wz \end{bmatrix}}_{\sysVel}
 .
\end{align}
This can be written in the probably more familiar form
\begin{align}
 \rd = \R \v, \qquad \Rd = \R \wedOp(\w).
\end{align}
With this we can express the particle velocities $\particlePosd{\PidxI}$ and accelerations $\particlePosdd{\PidxI}$ in terms of the configuration $(\r, \R)$ and velocity coordinates $(\v, \w)$ as
\begin{subequations}
\begin{align}
 \label{eq:RBParticlePos}
 \particlePos{\PidxI} &= \r + \R \particleBodyPos{\PidxI},
\\
 \label{eq:RBParticleVel}
 \particlePosd{\PidxI} &= \R \big( \v - \wedOp(\particleBodyPos{\PidxI}) \w \big),
\\
 \label{eq:RBParticleAcc}
 \particlePosdd{\PidxI} &= \R \big( \vd - \wedOp(\particleBodyPos{\PidxI})\wdot + \wedOp(\w)(\v - \wedOp(\particleBodyPos{\PidxI})\w) \big), \quad \PidxI=1,\ldots,\numParticles
\end{align}
\end{subequations}
and
\begin{align}
 \differential \particlePos{\PidxI} = \pdiff[\particlePosd{\PidxI}]{\sysVel} = \pdiff[\particlePosdd{\PidxI}]{\sysVeld} = \big[ \R, \, -\R\wedOp(\particleBodyPos{\PidxI})\big] 
\end{align}


\paragraph{Commutation coefficients.}
For the next steps we will also need the commutation coefficients $\BoltzSym{}{}{}$ for the chosen coordinates.
To indicate the special case for the rigid body we use the sans font $\bodyBoltzSym{}{}{}$ here.
Plugging the kinematic matrix $\kinMat$ from \eqref{eq:KinematicsRigidBody} into the definition \eqref{eq:DefCommutationCoeff} yields
\begin{align}\label{eq:RBCommutationCoeff}
 \bodyBoltzSym{1}{2}{6} = \bodyBoltzSym{1}{5}{3} = 
 \bodyBoltzSym{2}{3}{4} = \bodyBoltzSym{2}{6}{1} =
 \bodyBoltzSym{3}{1}{5} = \bodyBoltzSym{3}{4}{2} =
 \bodyBoltzSym{4}{5}{6} =
 \bodyBoltzSym{5}{6}{4} = 
 \bodyBoltzSym{6}{4}{5} &= 1 \ ,
% \nonumber\\
%  \bodyBoltzSym{1}{6}{2} = \bodyBoltzSym{1}{3}{5} = 
%  \bodyBoltzSym{2}{4}{3} = \bodyBoltzSym{2}{1}{6} =
%  \bodyBoltzSym{3}{5}{1} = \bodyBoltzSym{3}{2}{4} =
%  \bodyBoltzSym{4}{6}{5} =
%  \bodyBoltzSym{5}{4}{6} = 
%  \bodyBoltzSym{6}{5}{4} &= -1 \ 
\end{align}
and the remaining coefficients vanish.
It will be useful to define the matrix
\begin{align}
 \big[ \BoltzSym{\LidxIII}{\LidxI}{\LidxII} \sysVelCoeff{\LidxII} \big]^{\LidxIII=1\ldots6}_{\LidxI=1\ldots6} =
 \begin{bmatrix} \wedOp(\bodyAngVel{}{}) & 0 \\ \wedOp(\bodyLinVel{}{}) & \wedOp(\bodyAngVel{}{}) \end{bmatrix}
 = -\ad{\sysVel}^\top
\end{align}
whose naming will be discussed later.

\subsection{Inertia}
\paragraph{Kinetic energy.}
Expressing the particle velocities $\particlePosd{\PidxI}$ in terms of the chosen coordinates \eqref{eq:RBParticleVel} the kinetic energy $\kineticEnergy$ of a free rigid body is 
\begin{align}\label{eq:RigidBodyKineticEnergy}
 \kineticEnergy &= \tfrac{1}{2} \sumParticles \particleMass{\PidxI} \norm{\overbrace{\R \big( \v - \wedOp(\particleBodyPos{\PidxI}) \w \big)}^{\particlePosd{\PidxI}}}^2
\nonumber\\[1ex]
 &= \tfrac{1}{2} \underbrace{\sumParticles \particleMass{\PidxI}}_{\bodyMass{}{}} \norm{\v}^2
  - \v^\top \underbrace{\sumParticles \particleMass{\PidxI} \wedOp(\particleBodyPos{\PidxI})}_{\bodyMass{}{} \wedOp(\bodyCOM{}{})} \w
  + \tfrac{1}{2} \w^\top \underbrace{\sumParticles \particleMass{\PidxI} \wedOp(\particleBodyPos{\PidxI})^\top \wedOp(\particleBodyPos{\PidxI})}_{\bodyMOI{}{}} \w
\nonumber\\
 &= \tfrac{1}{2} \underbrace{\big[ \v^\top \ \w^\top \big]}_{\bodyVel{}{}^\top}
 \underbrace{\begin{bmatrix} \bodyMass{}{}\idMat[3] & \bodyMass{}{} \wedOp(\bodyCOM{}{})^\top \\ \bodyMass{}{} \wedOp(\bodyCOM{}{}) & \bodyMOI{}{} \end{bmatrix}}_{\bodyInertiaMat{}{}}
 \underbrace{\begin{bmatrix} \v \\ \w \end{bmatrix}}_{\bodyVel{}{}}.
\end{align}
Here we have substituted some well established inertia parameters: the total mass $\bodyMass{}{}$, the center of mass $\bodyCOM{}{} = \bodyMass{}{}^{-1} \sumParticles \particleMass{\PidxI} \particleBodyPos{\PidxI}$ and the moment of inertia $\bodyMOI{}{} = \bodyMOI{}{}^\top$.
Assuming that the particle masses are positive $\particleMass{\PidxI} > 0, \PidxI=1,\ldots,\numParticles$ implies that the total mass is positive $\bodyMass{}{} > 0$.
Furthermore, if the rigid body has at least three particles that do not lie on a line, the inertia matrix is positive definite $\bodyMOI{}{} > 0$.
It is important to notice that the inertia matrix $\bodyInertiaMat{}{}$ for the chosen coordinates is \textit{constant}\footnote{
One reason for the choice of $\v$ as velocity coordinates it the fact that the inertia matrix $\bodyInertiaMat{}{}$ is \textit{constant}.
If we choose instead $\rd$ as velocity coordinates we have
\begin{align*}
 \kineticEnergy = \tfrac{1}{2} [\rd^\top, \w^\top] \begin{bmatrix} \bodyMass{}{} \idMat[3] & \bodyMass{}{} \R \wedOp(\bodyCOM{}{})^\top \\ \bodyMass{}{} \wedOp(\bodyCOM{}{}) \R^\top & \bodyMOI{}{} \end{bmatrix} \begin{bmatrix} \rd \\ \w \end{bmatrix}.
\end{align*}
Obviously the body inertia matrix depends on the orientation $\R$ of the body and is not constant unless the reference position $\r$ coincides with the \textit{center of mass}, i.e.\ $\bodyCOM{}{}{} = 0$.
Actually many textbooks, \eg \cite[p.\,167]{Murray:Robotic} or \cite[p.\,153]{Shabana:MultibodySystems}, restrict to this case for their expressions of the kinetic energy.
In the next section on rigid body systems we will see that it can be quite useful to use \textit{geometrically} meaningful body fixed frames rather than restricting to the center of mass.
}.

\paragraph{Connection coefficients.}
As the inertia matrix $\bodyInertiaMat{}{}$ in \eqref{eq:RigidBodyKineticEnergy} is constant, the corresponding connection coefficients $\bodyConnCoeffL{}{}{\LidxI}{\LidxII}{\LidxIII}$ only consist of the terms with the commutation coefficients $\bodyBoltzSym{}{}{}$:
\begin{align}\label{eq:RBConnCoeff}
 \bodyConnCoeffL{}{}{\LidxI}{\LidxII}{\LidxIII} &= \tfrac{1}{2}\big(\bodyBoltzSym{\LidxV}{\LidxI}{\LidxII} \bodyInertiaMatCoeff{}{}{\LidxV\LidxIII} + \bodyBoltzSym{\LidxV}{\LidxI}{\LidxIII} \bodyInertiaMatCoeff{}{}{\LidxV\LidxII} - \bodyBoltzSym{\LidxV}{\LidxII}{\LidxIII} \bodyInertiaMatCoeff{}{}{\LidxV\LidxI} \big)
 = -\bodyConnCoeffL{}{}{\LidxII}{\LidxI}{\LidxIII}.
\end{align}
Taking into account this skew symmetry and the commutation coefficients $\bodyBoltzSym{}{}{}$ given in \eqref{eq:RBCommutationCoeff}, the non-zero connection coefficients are
\begin{subequations}\label{eq:RBConnCoeffRes}
\begin{align}
 \bodyConnCoeffL{}{}{3}{2}{4} = \bodyConnCoeffL{}{}{1}{3}{5} = \bodyConnCoeffL{}{}{2}{1}{6} \ &= \ \bodyMass{}{},
\\
 \bodyConnCoeffL{}{}{2}{5}{4} = \bodyConnCoeffL{}{}{3}{6}{4} = \bodyConnCoeffL{}{}{5}{1}{5} = \bodyConnCoeffL{}{}{6}{1}{6} \ &= \ \bodyMass{}{} \bodyCOMCoeff{}{}{\idxX},
\\
 \bodyConnCoeffL{}{}{4}{2}{4} = \bodyConnCoeffL{}{}{1}{4}{5} = \bodyConnCoeffL{}{}{3}{6}{5} = \bodyConnCoeffL{}{}{6}{2}{6} \ &= \ \bodyMass{}{} \bodyCOMCoeff{}{}{\idxY},
\\
 \bodyConnCoeffL{}{}{4}{3}{4} = \bodyConnCoeffL{}{}{5}{3}{5} = \bodyConnCoeffL{}{}{1}{4}{6} = \bodyConnCoeffL{}{}{2}{5}{6} \ &= \ \bodyMass{}{} \bodyCOMCoeff{}{}{\idxZ},
\\
 \bodyConnCoeffL{}{}{6}{5}{4} \ &= \ \bodyMOICoeffp{}{}{\idxX\idxX} = \tfrac{1}{2}\big( \bodyMOICoeff{}{}{\idxY\idxY} + \bodyMOICoeff{}{}{\idxZ\idxZ} - \bodyMOICoeff{}{}{\idxX\idxX} \big),
\\
 \bodyConnCoeffL{}{}{4}{6}{5} \ &= \ \bodyMOICoeffp{}{}{\idxY\idxY} = \tfrac{1}{2}\big( \bodyMOICoeff{}{}{\idxX\idxX} + \bodyMOICoeff{}{}{\idxZ\idxZ} - \bodyMOICoeff{}{}{\idxY\idxY} \big),
\\
 \bodyConnCoeffL{}{}{5}{4}{6} \ &= \ \bodyMOICoeffp{}{}{\idxZ\idxZ} = \tfrac{1}{2}\big( \bodyMOICoeff{}{}{\idxX\idxX} + \bodyMOICoeff{}{}{\idxY\idxY} - \bodyMOICoeff{}{}{\idxZ\idxZ} \big),
\\
 \bodyConnCoeffL{}{}{4}{6}{4} = \bodyConnCoeffL{}{}{6}{5}{5} \ &= \ \bodyMOICoeffp{}{}{\idxX\idxY} = -\bodyMOICoeff{}{}{\idxX\idxY},
\\
 \bodyConnCoeffL{}{}{5}{4}{4} = \bodyConnCoeffL{}{}{6}{5}{6} \ &= \ \bodyMOICoeffp{}{}{\idxX\idxZ} = -\bodyMOICoeff{}{}{\idxX\idxZ},
\\
 \bodyConnCoeffL{}{}{5}{4}{5} = \bodyConnCoeffL{}{}{4}{6}{6} \ &= \ \bodyMOICoeffp{}{}{\idxY\idxZ} = -\bodyMOICoeff{}{}{\idxY\idxZ}.
\end{align}
\end{subequations}
The parameters $\bodyMOIp{}{} = \tfrac{1}{2} \tr(\bodyMOI{}{})\idMat[3] - \bodyMOI{}{}$ will appear several times in the following.

% In addition to the symmetry of the connection coefficients for the rigid body we have
% \begin{align}
%  \gyroForceCoeff{\LidxI}
%  = \bodyConnCoeffL{}{}{\LidxI}{\LidxII}{\LidxIII} \sysVelCoeff{\LidxII} \sysVelCoeff{\LidxIII}
%  = \tfrac{1}{2}\big(\bodyBoltzSym{\LidxV}{\LidxI}{\LidxII} \bodyInertiaMatCoeff{}{}{\LidxV\LidxIII} + \bodyBoltzSym{\LidxV}{\LidxI}{\LidxIII} \bodyInertiaMatCoeff{}{}{\LidxV\LidxII} - \bodyBoltzSym{\LidxV}{\LidxII}{\LidxIII} \bodyInertiaMatCoeff{}{}{\LidxV\LidxI} \big) \sysVelCoeff{\LidxII} \sysVelCoeff{\LidxIII}
%  = \bodyBoltzSym{\LidxV}{\LidxI}{\LidxII} \bodyInertiaMatCoeff{}{}{\LidxV\LidxIII} \sysVelCoeff{\LidxII} \sysVelCoeff{\LidxIII}
% \end{align}

\paragraph{Acceleration energy.}
Expressing the particle accelerations $\particlePosdd{\PidxI}$ in terms of the coordinates \eqref{eq:RBParticleAcc} we find the acceleration energy $\accEnergy$ for the free rigid body as\footnote{using the Jacobi identity $\tuple{a},\tuple{b},\tuple{c} \in \RealNum^3$: $\wedOp(\tuple{a}) \wedOp(\tuple{b}) \tuple{c} + \wedOp(\tuple{b}) \wedOp(\tuple{c}) \tuple{a} + \wedOp(\tuple{c}) \wedOp(\tuple{a}) \tuple{b} = \tuple{0}$}
\begin{align}\label{eq:RigidBodyAccEnergy}
 \accEnergy &= \tfrac{1}{2} \sumParticles \particleMass{\PidxI} \norm{\overbrace{\R \big( \vd - \wedOp(\particleBodyPos{\PidxI})\wdot + \wedOp(\w)(\v - \wedOp(\particleBodyPos{\PidxI})\w) \big)}^{\particlePosdd{\PidxI}}}^2
\nonumber\\[1ex]
 &= \tfrac{1}{2} \underbrace{\sumParticles \particleMass{\PidxI}}_{\bodyMass{}{}} \norm{\vd}^2
  - \vd^\top \underbrace{\sumParticles \particleMass{\PidxI} \wedOp(\particleBodyPos{\PidxI})}_{\bodyMass{}{} \wedOp(\bodyCOM{}{})} \wdot
  + \tfrac{1}{2} \wdot^\top \underbrace{\sumParticles \particleMass{\PidxI} \wedOp(\particleBodyPos{\PidxI})^\top \wedOp(\particleBodyPos{\PidxI})}_{\bodyMOI{}{}} \wdot
\nonumber\\
 &\qquad+ \vd^\top \wedOp(\w) \Big( \underbrace{\sumParticles \particleMass{\PidxI}}_{\bodyMass{}{}}\v - \underbrace{\sumParticles \particleMass{\PidxI} \wedOp(\particleBodyPos{\PidxI})}_{\bodyMass{}{} \wedOp(\bodyCOM{}{})} \w \Big)
\nonumber\\
 &\qquad+ \wdot^\top \Big( \underbrace{\sumParticles \particleMass{\PidxI} \wedOp(\particleBodyPos{\PidxI})}_{\bodyMass{}{} \wedOp(\bodyCOM{}{})} \wedOp(\w) \v +  \wedOp(\w) \underbrace{\sumParticles \particleMass{\PidxI} \wedOp(\particleBodyPos{\PidxI})^\top \wedOp(\particleBodyPos{\PidxI})}_{\bodyMOI{}{}} \w \Big)
\nonumber\\
 &\qquad+ \tfrac{1}{2} \underbrace{\sumParticles \particleMass{\PidxI}}_{\bodyMass{}{}} \norm{\wedOp(\w)\v}^2
 + \v^\top \wedOp(\w)^2 \underbrace{\sumParticles \particleMass{\PidxI} \wedOp(\particleBodyPos{\PidxI})}_{\bodyMass{}{} \wedOp(\bodyCOM{}{})} \w
\nonumber\\
 &\qquad+ \tfrac{1}{2} \tr \Big( \underbrace{\sumParticles \particleMass{\PidxI} \particleBodyPos{\PidxI} \particleBodyPos{\PidxI}^\top}_{\bodyMOIp{}{}} \wedOp(\w)^4 \Big)
\end{align}
As for the kinetic energy $\kineticEnergy$, the acceleration energy $\accEnergy$ for a free rigid body can be formulated using the inertia parameters $(\bodyMass{}{}, \bodyCOM{}{}, \bodyMOI{}{})$ as
\begin{multline}
 \accEnergy = \tfrac{1}{2} \underbrace{\big[ \vd^\top \ \wdot^\top \big]}_{\bodyVeld{}{}^\top}
 \underbrace{\begin{bmatrix} \bodyMass{}{}\idMat[3] & \bodyMass{}{} \wedOp(\bodyCOM{}{})^\top \\ \bodyMass{}{} \wedOp(\bodyCOM{}{}) & \bodyMOI{}{} \end{bmatrix}}_{\bodyInertiaMat{}{}}
 \underbrace{\begin{bmatrix} \vd \\ \wdot \end{bmatrix}}_{\bodyVeld{}{}}
\\
 + \underbrace{\big[ \vd^\top \ \wdot^\top \big]}_{\bodyVeld{}{}^\top} 
% \underbrace{\begin{bmatrix} \wedOp(\w) \bodyMass{}{}(\v - \wedOp(\bodyCOM{}{})\w) \\ \bodyMass{}{} \wedOp(\bodyCOM{}{}) \wedOp(\w) \v + \wedOp(\w)\bodyMOI{}{} \w \end{bmatrix}}_{\gyroForce}
 \underbrace{\begin{bmatrix} \wedOp(\w) & \mat{0} \\ \wedOp(\v) & \wedOp(\w) \end{bmatrix}}_{-\ad{\bodyVel{}{}}^\top} 
 \underbrace{\begin{bmatrix} \bodyMass{}{}\idMat[3] & \bodyMass{}{} \wedOp(\bodyCOM{}{})^\top \\ \bodyMass{}{} \wedOp(\bodyCOM{}{}) & \bodyMOI{}{} \end{bmatrix}}_{\bodyInertiaMat{}{}} 
 \underbrace{\begin{bmatrix} \v \\ \w \end{bmatrix}}_{\bodyVel{}{}}
\\
 + \underbrace{\tfrac{1}{2} \bodyMass{}{} \norm{\wedOp(\w)\v}^2 + \v^\top \wedOp(\w)^2 \bodyMass{}{} \wedOp(\bodyCOM{}{}) \w + \tfrac{1}{2} \tr \big( \bodyMOIp{}{} \wedOp(\w)^4 \big)}_{\accEnergy_0}
 .
\end{multline}
Note that $\accEnergy_0$ is independent of the generalized acceleration $\bodyVeld{}{}$ and consequently does not contribute to the generalized inertia force $\bodyGenForceInertia{}{} = \spdiff[\accEnergy]{\bodyVeld{}{}}$.

\paragraph{Inertia force.}
Finally, the inertia force of the free rigid body $\bodyGenForceInertia{}{}$, derived from any of the of the previous equivalent formulations, reads
\begin{align}
 \bodyGenForceInertia{}{} = \bodyInertiaMat{}{} \bodyVeld{}{} - \ad{\bodyVel{}{}}^\top \bodyInertiaMat{}{} \bodyVel{}{}.
\end{align}

% \begin{align}
%  \bodyGenForceInertia{}{}
%  &= \underbrace{\begin{bmatrix} \bodyMass{}{}\idMat[3] & \bodyMass{}{} \wedOp(\bodyCOM{}{})^\top \\ \bodyMass{}{} \wedOp(\bodyCOM{}{}) & \bodyMOI{}{} \end{bmatrix}}_{\bodyInertiaMat{}{}} \underbrace{\begin{bmatrix} \vd \\ \wdot \end{bmatrix}}_{\bodyVeld{}{}}
%  + \underbrace{\begin{bmatrix} \wedOp(\w) \bodyMass{}{}(\v - \wedOp(\bodyCOM{}{})\w) \\ \bodyMass{}{} \wedOp(\bodyCOM{}{}) \wedOp(\w) \v + \wedOp(\w)\bodyMOI{}{} \w \end{bmatrix}}_{\gyroForce}
% %  + \underbrace{\begin{bmatrix} \wedOp(\w) & \mat{0} \\ \wedOp(\v) & \wedOp(\w) \end{bmatrix}}_{-\ad{\bodyVel{}{}}^\top} 
% %  \underbrace{\begin{bmatrix} \bodyMass{}{}\idMat[3] & \bodyMass{}{} \wedOp(\bodyCOM{}{})^\top \\ \bodyMass{}{} \wedOp(\bodyCOM{}{}) & \bodyMOI{}{} \end{bmatrix}}_{\bodyInertiaMat{}{}} 
% %  \underbrace{\begin{bmatrix} \v \\ \w \end{bmatrix}}_{\bodyVel{}{}}
% \end{align}

\subsection{Gravitation}
The potential energy $\potentialGravity$ of a rigid body due to a gravitational acceleration $\gravityAcc$ according to \eqref{eq:ParticlePotentialGravity} is
\begin{align}\label{eq:RBPotentialEnergyGravity}
 \potentialGravity 
% &= \sumParticles \sProd{\particlePos{\PidxI}}{\particleMass{\PidxI} \gravityAcc} 
 = \sumParticles \sProd{\r + \R \particleBodyPos{\PidxI}}{\particleMass{\PidxI} \gravityAcc}
 = \sProd{\underbrace{\sumParticles \particleMass{\PidxI}}_{\bodyMass{}{}} \r + \R \underbrace{\sumParticles \particleMass{\PidxI} \particleBodyPos{\PidxI}}_{\bodyMass{}{}\bodyCOM{}{}}}{\gravityAcc},
\end{align}
and the resulting generalized force is
\begin{align}
 \bodyGenForceGravity{}{} = \differential \potentialGravity = \begin{bmatrix} \R^\top \bodyMass{}{} \gravityAcc \\ \wedOp(\bodyCOM{}{}) \R^\top \bodyMass{}{} \gravityAcc \end{bmatrix}.
\end{align}


\subsection{Stiffness}\label{sec:RBStiffness}
\begin{figure}[ht]
 \centering
 \input{graphics/RBStiffness.pdf_tex}
 \caption{Rigid body stiffness}
 \label{fig:RBStiffness}
\end{figure}
Assume that every particle of the rigid body is connected to a position $\springHubPos{\PidxI} \in \RealNum^3$ by a linear spring with stiffness $\particleStiffness{\PidxI} \in\RealNum^{(+)}$, see \autoref{fig:RBStiffness}.
The resulting potential energy in terms of the rigid body coordinates $\sysCoord \cong (\r, \R)$ is
\begin{align}\label{eq:RBStiffnessPotential}
 \potentialStiff(\sysCoord) &= \tfrac{1}{2} \sumParticles \particleStiffness{\PidxI} \norm{\r + \R \particleBodyPos{\PidxI} - \springHubPos{\PidxI}}^2.
\end{align}

\paragraph{Some identities.}
In the following we use some basic identities
\begin{subequations}
\begin{align}
 \tuple{a}, \tuple{b} \in \RealNum^{n}  \ : \ &\tuple{a}^\top \tuple{b} = \tr(\tuple{a}\tuple{b}^\top)
\\
 \tuple{a}, \tuple{b} \in \RealNum^{3}  \ : \ &\wedOp(\tuple{a}) \tuple{b} = 2\veeOp(\tuple{b} \tuple{a}^\top)
\\
 &\wedOp(\tuple{a}) \wedOp(\tuple{b}) = \tuple{b} \tuple{a}^\top - (\tuple{b}^\top \tuple{a}) \idMat[3]
\\
 \mat{A}\in\RealNum^{3\times3}, \tuple{b}\in\RealNum^3 \ : \ &\tr(\mat{A} \wedOp(\tuple{b})) = -2 \tuple{b}^\top \veeOp(\mat{A})
\\
 &2\veeOp(\wedOp(\tuple{b}) \mat{A}) = (\tr(\mat{A})\idMat[3] - \mat{A}) \tuple{b}
\end{align} 
\end{subequations}
Each may be checked by direct computation.

\paragraph{Stiffness parameters.}
Rearrange \eqref{eq:RBStiffnessPotential} to
\begin{align}\label{eq:RBStiffnessPotential2}
 \potentialStiff(\sysCoord) &= \tfrac{1}{2} \sumParticles \particleStiffness{\PidxI} \norm{\r + \R \particleBodyPos{\PidxI} - \springHubPos{\PidxI}}^2
\nonumber\\
 &= \tfrac{1}{2} \sumParticles \particleStiffness{\PidxI} \big( \norm{\r}^2 + \underbrace{\norm{\R \particleBodyPos{\PidxI}}^2}_{\const} + \underbrace{\norm{\springHubPos{\PidxI}}^2}_{\const} + 2\sProd{\r}{\R \particleBodyPos{\PidxI}} - 2\sProd{\r}{\springHubPos{\PidxI}} - 2\sProd{\R \particleBodyPos{\PidxI}}{\springHubPos{\PidxI}}\big)
\nonumber\\
 &= \sumParticles \particleStiffness{\PidxI} \big( \tfrac{1}{2} \norm{\r}^2 + \sProd{\r}{\R \particleBodyPos{\PidxI}} - \sProd{\r}{\springHubPos{\PidxI}} - \tr(\R \particleBodyPos{\PidxI} \springHubPos{\PidxI}^\top )\big)
  + \underbrace{\tfrac{1}{2} \sumParticles \particleStiffness{\PidxI} \big( \norm{\particleBodyPos{\PidxI}}^2 + \norm{\springHubPos{\PidxI}}^2)}_{\potentialStiff_c\,=\,\const}
\nonumber\\
 &= \tfrac{1}{2} \bodyStiffness{}{}\norm{\r}^2 + \bodyStiffness{}{} \sProd{\r}{\R \bodyCOS{}{}} - \bodyStiffness{}{} \sProd{\r}{\tuple{p}} - \tr(\mat{P} \R) + \potentialStiff_c
\end{align}
with substitution of the constant parameters
\begin{align}
 \bodyStiffness{}{} &= \sumParticles \particleStiffness{\PidxI},&
 \bodyCOS{}{} &= \bodyStiffness{}{}^{-1} \sumParticles \particleStiffness{\PidxI} \particleBodyPos{\PidxI},&
 \tuple{p} &= \bodyStiffness{}{}^{-1} \sumParticles \particleStiffness{\PidxI} \springHubPos{\PidxI},&
 \mat{P} &= \sumParticles \particleStiffness{\PidxI} \particleBodyPos{\PidxI} \springHubPos{\PidxI}^\top.
\end{align}
Since there is no assumption on the particle and spring constellation, any constellation may be captured by the $1+3+3+9+1=17$ parameters within $(\bodyStiffness{}{}, \bodyCOS{}{}, \tuple{p}, \mat{P}, \potentialStiff_c)$.

\paragraph{Critical points.}
The time derivatives of the potential may be written as
\begin{align}\label{eq:RBStiffnessPotentialDifferential}
 \tdiff{t} \potentialStiff &= \bodyStiffness{}{} \sProd{\r}{\R\v} + \bodyStiffness{}{} \sProd{\R\v}{\R \bodyCOS{}{}} + \bodyStiffness{}{} \sProd{\r}{\R\wedOp(\w) \bodyCOS{}{}} - \bodyStiffness{}{} \sProd{\R\v}{\tuple{p}} - \tr(\mat{P} \R \wedOp(\w))
\nonumber\\
 &= \sysVel^\top \underbrace{\begin{bmatrix} \bodyStiffness{}{} (\R^\top (\r - \tuple{p}) + \bodyCOS{}{}) \\ \bodyStiffness{}{} \wedOp(\bodyCOS{}{}) \R^\top \r + 2\veeOp(\mat{P} \R) \end{bmatrix}}_{\differential \potentialStiff}
\\
 \tfrac{\d^2}{\d t^2} \potentialStiff 
 &= \sysVeld^\top \differential \potentialStiff + \sysVel^\top \underbrace{\begin{bmatrix} \bodyStiffness{}{} \idMat[3] & \bodyStiffness{}{}\wedOp\big(\R^\top (\r - \tuple{p})\big)^\top \\ \bodyStiffness{}{}\wedOp(\bodyCOS{}{}) & \bodyStiffness{}{} \wedOp(\bodyCOS{}{}) \wedOp(\R^\top \r) + \tr(\mat{P} \R)\idMat[3] - (\mat{P} \R)^\top \end{bmatrix}}_{\differential^2 \potentialStiff} \sysVel
\end{align}
We are interested in configurations $\sysCoordR\cong (\rR, \RR)$ at which the potential is stationary $\differential \potentialStiff(\sysCoordR) = \tuple{0}$.
From the upper part of \eqref{eq:RBStiffnessPotentialDifferential} we get the condition
\begin{align}
 \rR = \tuple{p} - \RR \bodyCOS{}{}.
\end{align}
Plugging this into the lower part of \eqref{eq:RBStiffnessPotentialDifferential} we obtain
\begin{align}
 \bodyStiffness{}{} \wedOp(\bodyCOS{}{}) \RR^\top (\tuple{p} - \RR \bodyCOS{}{}) + 2\veeOp(\mat{P} \RR)
% = \bodyStiffness{}{} \wedOp(\bodyCOS{}{}) \RR^\top \tuple{p} + 2\veeOp(\mat{P} \RR)
 = 2\veeOp(\underbrace{(\mat{P} - \bodyStiffness{}{}\bodyCOS{}{}\tuple{p}^\top)}_{\mat{P}_{\!s}} \RR ) = \tuple{0}.
\end{align}
So we need $\RR \in \SpecialOrthogonalGroup(3)$ such that $\mat{P}_{\!s} \RR$ is symmetric.
Let $\mat{P}_s = \mat{X} \mat{\varSigma} \mat{Y}^\top$ with $\mat{X},\mat{Y} \in \OrthogonalGroup(3)$ and $\mat{\varSigma} = \diag(\sigma_1, \sigma_2, \sigma_3)$, $\sigma_1 \geq \sigma_2 \geq \sigma_3 \geq 0$ be a singular value decomposition and define
\begin{subequations}
\begin{align}
 \bar{\mat{X}} &= \mat{X} \diag(1, 1, \det\mat{X}) \in \SpecialOrthogonalGroup(3),
\\
 \bar{\mat{Y}} &= \mat{Y} \diag(1, 1, \det\mat{Y}) \in \SpecialOrthogonalGroup(3),
\\
 \bar{\mat{\varSigma}} &= \diag(\sigma_1, \sigma_2, \det\mat{X}\det\mat{Y} \sigma_3).
\end{align} 
\end{subequations}
%where $\sigma_3$ is the smallest singular value.
With this we may write $\mat{P}_{\!s} \RR = \bar{\mat{X}} \bar{\mat{\varSigma}} \bar{\mat{X}}^\top \bar{\mat{X}} \bar{\mat{Y}}^\top \RR$ which is clearly symmetric for $\RR = \bar{\mat{Y}} \bar{\mat{X}}^\top$.
The Hessian at this critical point is
\begin{align}
 \differential^2 \potentialStiff(\sysCoordR) 
 &= \begin{bmatrix} \bodyStiffness{}{} \idMat[3] & \bodyStiffness{}{}\wedOp(\bodyCOS{}{})^\top \\ \bodyStiffness{}{}\wedOp(\bodyCOS{}{}) & \tr(\mat{P}_{\!s} \RR)\idMat[3] - \mat{P}_{\!s} \RR + \bodyStiffness{}{} \wedOp(\bodyCOS{}{}) \wedOp(\bodyCOS{}{})^\top \end{bmatrix}
\nonumber\\
 &= \begin{bmatrix} \idMat[3] & \mat{0} \\ \wedOp(\bodyCOS{}{}) & \idMat[3] \end{bmatrix}
 \begin{bmatrix} 
  \bodyStiffness{}{} \idMat[3] & \mat{0} \\
  \mat{0} & \bodyMOSs{}{}
 \end{bmatrix}
 \begin{bmatrix} \idMat[3] & \wedOp(\bodyCOS{}{})^\top \\ \mat{0} & \idMat[3] \end{bmatrix}
\end{align}
with $\bodyMOSs{}{} = \tr(\mat{P}_{\!s} \RR)\idMat[3] - \mat{P}_{\!s} \RR = \bar{\mat{X}} ( \tr(\bar{\mat{\varSigma}})\idMat[3] - \bar{\mat{\varSigma}} ) \bar{\mat{X}}^\top \in \SymMatSP(3)$.
The positive semi-definiteness of $\tr(\bar{\mat{\varSigma}})\idMat[3] - \bar{\mat{\varSigma}}$ is a consequence of $\sigma_3$ being the smallest singular value.

The rotational part of $\potentialStiff$ is analyzed thoroughly in \autoref{sec:AppendixAttitudePotential} which also discusses saddle points and maxima.
From the conclusions there we may deduct: $\sysCoordR \cong (\rR, \RR) = (\tuple{p} - \bar{\mat{Y}} \bar{\mat{X}}^\top \bodyCOS{}{}, \bar{\mat{Y}} \bar{\mat{X}}^\top)$ is a minimum of $\potentialStiff$.
It is the strict, global minimum, if and only if, $\bodyMOSs{}{}$ is positive definite.


\paragraph{Stiffness parameters cont'd.}
We may express the parameters $\tuple{p}$ and $\mat{P}$ in terms of the minimum configuration $(\rR, \RR)$ and the moment of stiffness matrix $\bodyMOSs{}{}$ as
\begin{align}
 \tuple{p} &= \rR + \RR \bodyCOS{}{}
\\
 \mat{P} &= \mat{P}_{\!s} + \bodyStiffness{}{}\bodyCOS{}{}\tuple{p}^\top
 = (\tfrac{1}{2}\tr(\bodyMOSs{}{})\idMat[3]-\bodyMOSs{}{})\RR^\top + \bodyStiffness{}{}\bodyCOS{}{}(\rR + \RR \bodyCOS{}{})^\top
\end{align}
Plugging this into \eqref{eq:RBStiffnessPotential2} yields
\begin{align}\label{eq:RBStiffnessPotential3}
 \potentialStiff(\sysCoord) &= \tfrac{1}{2} \bodyStiffness{}{}\norm{\r}^2 + \bodyStiffness{}{} \sProd{\r}{\R \bodyCOS{}{}} - \bodyStiffness{}{} \sProd{\r}{\rR + \RR \bodyCOS{}{}}
\nonumber\\
 &\qquad -\tr\big( ((\tfrac{1}{2}\tr(\bodyMOSs{}{})\idMat[3]-\bodyMOSs{}{})\RR^\top + \bodyStiffness{}{}\bodyCOS{}{}(\rR + \RR \bodyCOS{}{})^\top) \R \big) + \potentialStiff_c
\nonumber\\
 &= \tfrac{1}{2} \bodyStiffness{}{}\norm{\r}^2
 - \bodyStiffness{}{} \sProd{\r}{\rR}
 + \bodyStiffness{}{} \sProd{\r}{\R \bodyCOS{}{}}
 - \bodyStiffness{}{} \sProd{\r}{\RR \bodyCOS{}{}}
 -\bodyStiffness{}{} \sProd{\rR}{\R\bodyCOS{}{}} 
\nonumber\\
 &\qquad -\tr\big( \underbrace{(\tfrac{1}{2}\tr(\bodyMOSs{}{})\idMat[3]-\bodyMOSs{}{} + \bodyStiffness{}{}\bodyCOS{}{}\bodyCOS{}{}^\top)}_{\bodyMOSp{}{}} \RR^\top \R \big) + \potentialStiff_c
\nonumber\\
 &= \tfrac{1}{2} \bodyStiffness{}{}\norm{\r-\rR}^2
 + \bodyStiffness{}{} \sProd{\r-\rR}{(\R-\RR) \bodyCOS{}{}}
 +\tr\big(\bodyMOSp{}{} (\idMat[3]-\RR^\top \R) \big) 
\nonumber\\
 &\qquad
 + \underbrace{\potentialStiff_c
 - \tfrac{1}{2} \bodyStiffness{}{}\norm{\rR}^2
 - \bodyStiffness{}{} \sProd{\rR}{\RR\bodyCOS{}{}} 
 - \tr(\bodyMOSp{}{}) 
 }_{\potentialStiff_0\,=\,\const}
\end{align}
The potential $\potentialStiff_0$ at the minimum is
\begin{align}
 \potentialStiff_0 = \potentialStiff(\sysCoordR) &= \tfrac{1}{2} \sumParticles \particleStiffness{\PidxI} \norm{\rR + \RR \particleBodyPos{\PidxI} - \springHubPos{\PidxI}}^2 \geq 0.
\end{align}

The differential may be written as
\begin{align}\label{eq:RBStiffnessForce}
 \differential\potentialStiff(\sysCoord) = \begin{bmatrix} \bodyStiffness{}{} \R^\top (\r\!-\!\rR) + (\idMat[3]\!-\!\R^\top\RR) \bodyStiffness{}{}\bodyCOS{}{} \\ \bodyStiffness{}{}\wedOp(\bodyCOS{}{}) \R^\top (\r\!-\!\rR) + \veeTwoOp\big(\bodyMOSp{}{} \RR^\top\R\big) \end{bmatrix}
\end{align}
and the Hessian at the minimum is
\begin{align} 
 \differential^2\potentialStiff(\sysCoordR) &= \begin{bmatrix} \bodyStiffness{}{}\idMat[3] & \bodyStiffness{}{} \wedOp(\bodyCOS{}{})^\top \\ \bodyStiffness{}{}\wedOp(\bodyCOS{}{}) & \bodyMOS{}{} \end{bmatrix} \geq 0
\end{align}
where $\bodyMOS{}{} = \tr(\bodyMOSp{}{})\idMat[3]-\bodyMOSp{}{} = \bodyMOSs{}{} + \bodyStiffness{}{} \wedOp(\bodyCOS{}{}) \wedOp(\bodyCOS{}{})^\top \in \SymMatSP(3)$.

\paragraph{Conclusion.}
The conclusion of this subsection is that any constellation of linear springs attached to a rigid body may be captured by the potential $\potentialStiff$ from \eqref{eq:RBStiffnessPotential3} and the resulting force $\genForceStiff = \differential\potentialStiff$ from \eqref{eq:RBStiffnessForce}.
It is parameterized by 6 parameters within $(\rR,\RR) \in \RealNum^3 \times \SpecialOrthogonalGroup(3)$ which describe the minimum, the $1+3+6=10$ parameters within the total stiffness $\bodyStiffness{}{}\in\RealNum^+_0$, the center of stiffness $\bodyCOS{}{} \in \RealNum^3$ and the moment of stiffness $\bodyMOS{}{} \in \SymMatSP(3)$ and the offset $\potentialStiff_0$: $\potentialStiff(\sysCoord) \geq \potentialStiff_0 \geq 0$.
The naming is due to the obvious analogy to the inertia parameters from the previous section.
The minimum is strict and global if, and only if, $\bodyStiffness{}{} > 0$ and $\bodyMOSs{}{} > 0$.



\subsection{Dissipation}\label{sec:DampingSE3}
Since the dissipation function $\dissFkt$ has the structure we already saw for the kinetic energy \eqref{eq:RigidBodyKineticEnergy} we can immediately conclude
\begin{align}
 \dissFkt &= \tfrac{1}{2} \sumParticles \particleDamping{\PidxI} \norm{\particlePosd{\PidxI}}^2
\nonumber\\
%  &= \tfrac{1}{2} \sumParticles \particleDamping{\PidxI} \norm{\v - \wedOp(\particleBodyPos{\PidxI})\w}^2
% \nonumber\\
 &= \tfrac{1}{2} \underbrace{\sumParticles \particleDamping{\PidxI}}_{\bodyDamping{}{}} \norm{\v}^2
  - \v^\top \underbrace{\sumParticles \particleDamping{\PidxI} \wedOp(\particleBodyPos{\PidxI})}_{\bodyDamping{}{} \wedOp(\bodyCOD{}{})} \w
  + \tfrac{1}{2} \w^\top \underbrace{\sumParticles \particleDamping{\PidxI} \wedOp(\particleBodyPos{\PidxI})^\top \wedOp(\particleBodyPos{\PidxI})}_{\bodyMOD{}{}} \w
\nonumber\\
 &= \tfrac{1}{2} \underbrace{\big[ \v^\top \ \w^\top \big]}_{\bodyVel{}{}^\top}
 \underbrace{\begin{bmatrix} \bodyDamping{}{}\idMat[3] & \bodyDamping{}{} \wedOp(\bodyCOD{}{})^\top \\ \bodyDamping{}{} \wedOp(\bodyCOD{}{}) & \bodyMOD{}{} \end{bmatrix}}_{\bodyDissMat{}{}}
 \underbrace{\begin{bmatrix} \v \\ \w \end{bmatrix}}_{\bodyVel{}{}}
\label{eq:RBDissFkt}
\end{align}
The resulting generalized force is 
\begin{align}
 \bodyGenForceDiss{}{} = \pdiff[\dissFkt]{\bodyVel{}{}} = \bodyDissMat{}{} \bodyVel{}{}
\end{align}




\subsection{External forces.}
\fixme{
Let external forces $\particleForceEx{\PidxI} = \particleForceEx{\PidxI}(\sysCoord, \sysVel, \sysInput)$
\begin{align}
 \genForceEx(\sysCoord, \sysVel, \sysInput) = \Big[\sumParticles \sProd{\dirDiff{\LidxI} \particlePos{\PidxI}(\sysCoord)}{\particleForceEx{\PidxI}(\sysCoord, \sysVel, \sysInput)} \Big]_{\LidxI=1,\ldots,6}
 = \sumParticles \begin{bmatrix} \R \\ \wedOp(\particleBodyPos{\PidxI}) \R \end{bmatrix} \particleForceEx{\PidxI}(\sysCoord, \sysVel, \sysInput)
\end{align}
}

