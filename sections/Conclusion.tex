\chapter{Conclusion}

This work discussed the modeling and tracking control of rigid body systems.
Both subjects are well covered in the dedicated literature.
In the modeling aspect, this work extended some well established formalisms by allowing for redundnant configuration coordinates and velocity coordinates. 

rigid body attiutde is the most common case. rotation matrix for theory, quaternion for efficient implementation, euler angles for user interface.

\fixme{
Many machines, vehicles and robots may be modeled as rigid body systems, i.e.\ a number of interconnected, undeformable bodies subject to inertia, gravity, and other forces.
Energy-based methods for derivation of their equations of motion, like the Lagrange formalism, are standard in engineering education and well established in the dedicated literature.
These algorithms commonly rely on the use of a minimal set of generalized coordinates.
This is appropriate for many applications, e.g. machines containing only one-dimensional joints.
For systems whose configuration space is nonlinear, e.g. mobile robots whose configiguration space contains the rigid body attitude, the use of minimal coordinates necessarily leads to singularities.
From the point of view of differential geometry, this is a well known fact.
}

\cite{Konz:Mathmod}
\cite{Konz:Mathmod2018}
\cite{Irscheid:HeavyRopesTricopter}
\cite{Konz:GaussTrackingControl}
\cite{Kastelan:Tricopter}
\cite{Konz:Mechatronics}
\cite{Konz:MechatronicsBook}
