\subsection{Bicopter}

\begin{figure}[ht]
 \centering
% \def\svgwidth{.5\textwidth}
 \input{graphics/MechModelBicopter.pdf_tex}
 \caption{Model of a bicopter (background image: \url{www.poppaper.net/80155164809})}
 \label{fig:BicopterModel}
\end{figure}

\paragraph{Equations of motion.}
The bicopter considered here is a single rigid body with two tiltable propellers as illustrated in \autoref{fig:QuadModel}.
With the same coordinates as for the previous examples, the equations of motion are identical as well up to the generalized force from the propellers 
\begin{align}
 \genForceInput
 =
 \underbrace{\begin{bmatrix}
  1 & 0 & 1 & 0 \\
  0 & \sin\PropInc & 0 & -\sin\PropInc \\
  0 & \cos\PropInc & 0 & \cos\PropInc \\
  0 & \PropPosY' & 0 & -\PropPosY' \\
  \PropPosZ & 0 & \PropPosZ & 0 \\
  -\PropPosY & 0 & \PropPosY & 0 \\
 \end{bmatrix}}_{\sysInputMat}
 \underbrace{\begin{bmatrix} \PropForce[1] \sin\PropTilt[1] \\ \PropForce[1] \cos\PropTilt[1] \\ \PropForce[2] \sin\PropTilt[2] \\ \PropForce[2] \cos\PropTilt[2] \end{bmatrix}}_{\sysInput}
 .
\end{align}
with $\PropPosY' = \PropPosY \cos\PropInc - \PropPosZ \sin\PropInc$.
The transformation of the actual control inputs $\PropForce[1]$, $\PropForce[2]$ and $\PropTilt[1]$, $\PropTilt[2]$ to a auxiliary input $\sysInput$ is used to achieve the linear form $\genForceInput = \sysInputMat\sysInput$.
Within the input constraints $2\,\unit{N} \leq \PropForce[i] \leq 14\,\unit{N}$, $-30^\circ \leq \PropTilt[i] \leq 30^\circ$, $i=1,2$ this transformation is bijective.
To account for the original constraints in the transformed input $\sysInput\in \RealNum^4$, a convex approximation illustrated in \autoref{fig:BicopterInputConstraintApprox} is used.
These constraints can be written in the required form $\sysInputConstMat \sysInput \leq \sysInputConstVec$.

\begin{figure}[ht]
 \centering
 \includegraphics[]{Bicopter/BicopterInputConstraintApprox.pdf}
 \caption{Approximation of the Bicopter input constraints}
 \label{fig:BicopterInputConstraintApprox}
\end{figure}

\paragraph{Reference trajectory.}
A possible left complement for $\sysInputMat$ is
\begin{align}
 (\sysInputMatLComp)^\top = \begin{bmatrix} \PropPosZ & 0 & 0 & 0 & -1 & 0 \\ 0 & \PropPosY' & 0 & -\sin\PropInc & 0 & 0 \end{bmatrix}
\end{align}
With this, the matching condition for the reference from \eqref{eq:MatchingForceZeroError} is
\begin{align}
 \tuple{\lambda}^{\text{ZeroError}} = 
% \begin{bmatrix} 
%  m \PropPosZ (\vxd + \vy\wz \!-\! \vz\wy + \Rzx\gravityAccConst ) - (\Jy\wyd + (\Jx-\Jz)\wx\wz)  \\
%  \PropPosY' m (\vyd + \vx\wz \!-\! \vz\wx + \Rzy\gravityAccConst) - \sin\PropInc (\Jx \wxd + (\Jz-\Jy)\wy\wz)
% \end{bmatrix}
 \begin{bmatrix} 
  m \PropPosZ  \big( \vxd + \vy\wz \!-\! \vz\wy - \epsx (\wyd + \tfrac{\Jx-\Jz}{\Jy} \wx\wz) + \Rzx\gravityAccConst \big)  \\
  m \PropPosY' \big( \vyd + \vx\wz \!-\! \vz\wx + \epsy (\wxd + \tfrac{\Jz-\Jy}{\Jx} \wy\wz) + \Rzy\gravityAccConst \big)
 \end{bmatrix}
 = \tuple{0}
 .
\end{align}
where
\begin{align}
 \epsx = -\tfrac{\Jy}{\m \PropPosZ},
\qquad
 \epsy = \tfrac{\Jx \sin\PropInc}{\m (\PropPosY \cos\PropInc - \PropPosZ \sin\PropInc)}.
\end{align}
For the general case\footnote{Even in the case $\epsx=\epsy\neq0$ we would need $\Jx=\Jy=\Jz$ for $\r+\R \tuple{\eps}$ to be a flat output. The case $\epsx=0$ implies $\Jy=0$, which does not make physical sense.} 
$\epsx\neq\epsy$ this system is probably not flat, so parameterization of a feasible reference trajectory is not trivial.

\paragraph{Closed loop template.}
As for the quadcopter, the closed loop templates are that of a free rigid body i.e.\ \autoref{sec:CtrlApproachParticlesSingleBody}, \autoref{sec:CtrlApproachBodySingleBody} and \autoref{sec:CtrlApproachEnergySingleBody}.
However, in contrast to the quadcopter there are no assumtions on symmetries in the constitutive parameters.

\paragraph{Matching.}
As before we first consider the linearized matching condtions:
It turns out that asymmetries in the constitutive parameters are not useful for resolving matching constraints.
So we set $\Jcxy=\Jcxz=\Jcyz=0$, $\scx=\scy=0$ and the same for damping and stiffness.
The remaining linearized matching conditions may be fulfilled by constraining the parameters as
\begin{subequations}
\begin{align} 
 \lcz = \hcz,
\qquad
 \sigcx = \sigcy = 0,
\qquad
 \kapcx = \kapcy = \mc \gravityAccConst (\hcz-\scz),
\\
 \Jcx = \mc (\hcz-\scz)(\scz-\epsy),
\qquad
 \Jcy = \mc(\hcz-\scz) (\scz-\epsx),
\end{align}
\end{subequations}
which leaves the tuning parameters $\mc, \dc, \kc, \scz, \hcz$ and $\Jcz, \sigcz, \kapcz$.
Note, in contrast to the quadcopter, that $\epsx \neq \epsy$ implies $\Jcx \neq \Jcy$ and since the remaining relevant parameters are identical, the closed loop dynamics for x and y are different from another.

The remaining matching force in the stabilization case $\sysVelR = \tuple{0}$ is
\begin{align}
 \tilde{\sysForce} = \frac{\mc}{m}
 \begin{bmatrix} \frac{1}{\hcz-\epsx} & 0 \\ 0 & \frac{1}{\hcz-\epsy} \\ 0 & 0 \\ 0 & -\frac{\epsy}{\hcz-\epsy} \\ \frac{\epsx}{\hcz-\epsx} & 0 \\ 0 & 0 \end{bmatrix}
 \begin{bmatrix}
  \big( \Jcz - \mc(\hcz-\scz) (\tfrac{\Jx-\Jz}{\Jy}\epsx - \epsy) \big) \wy\wz - \tfrac{1}{2} \kapcz (\RExz+\REzx) \\
  \big( \Jcz - \mc(\hcz-\scz) (\tfrac{\Jy-\Jz}{\Jx}\epsy - \epsx) \big) \wx\wz - \tfrac{1}{2} \kapcz (\REyz+\REzy)
 \end{bmatrix}
\end{align}

\begin{figure}
 \centering
 \includegraphics[]{Bicopter/BicopterCircleSimSnapshots.pdf}
 \caption{Snapshots for the simulation of the bicopter}
 \label{fig:BicopterCircleSimSnapshots}
\end{figure}

\begin{figure}[p]
 \centering
 \includegraphics[]{Bicopter/BicopterCircleSimRes.pdf}
 \caption{Simulation result for the bicopter}
 \label{fig:BicopterCircleSimRes}
\end{figure}


\paragraph{Simulation result.}
Since the bicopter model is probably not flat, generation of a feasible reference trajectory is not trivial. 
For the simulation example here we exploited that, if we set $\rRx = 0$ and $\Rxx=1$, the motion is constrained to the yz plane and the remaining model is essentially a PVTOL.

As a challanging reference trajectory a vertical circle was chosen.


