\subsection{PVTOL}\label{sec:CtrlExamplePVTOL}
The planar vertical take off landing aircraft (PVTOL) as depicted in \autoref{fig:PVTOLModel}, is a common benchmark problem discussed in e.g.\ \cite{Hauser:PVTOL} or \cite{Fliess:LieBacklund}.
\begin{figure}
 \centering
% \def\svgwidth{.5\textwidth}
 \input{graphics/PVTOLModel.pdf_tex}
 \caption{Model of the PVTOL}
 \label{fig:PVTOLModel}
\end{figure}

\paragraph{Model.}
The model of the PVTOL is a planar rigid body already discussed in \autoref{sec:CtrlPlanarRigidBody} supplemented with gravity and the specific actuation $F_\text{r}$, $F_\text{l}$ depicted in \autoref{fig:PVTOLModel}.
Here, we chose the body fixed frame to be positioned at the center of mass, i.e.\ $\sx = \sy = 0$.
The kinetic equation takes the form
\begin{align}
 \underbrace{\begin{bmatrix} \m & 0 & 0 \\ 0 & \m & 0 \\ 0 & 0 & \Jz \end{bmatrix}}_{\sysInertiaMat}
 \underbrace{\begin{bmatrix} \vxd \\ \vyd \\ \wzd \end{bmatrix}}_{\sysVeld}
 +
%  \underbrace{\begin{bmatrix} -m \vy \wz \\ m \vx \wz \\ 0 \end{bmatrix}}_{\gyroForce}
%  +
%  \underbrace{\begin{bmatrix} m \gravityAccConst \sin\tiltAngle \\ m \gravityAccConst \cos\tiltAngle \\ 0 \end{bmatrix}}_{\differential \potentialEnergy}
 \underbrace{\begin{bmatrix} m(\gravityAccConst \sin\tiltAngle - \vy \wz) \\ m (\gravityAccConst \cos\tiltAngle + \vx \wz) \\ 0 \end{bmatrix}}_{\sysForce}
 =
 \underbrace{\begin{bmatrix} \sin\beta & -\sin\beta \\ \cos\beta & \cos\beta \\ \ArmRadius \cos\beta & -\ArmRadius\cos\beta \end{bmatrix}}_{\sysInputMat}
 \underbrace{\begin{bmatrix} F_\text{r} \\ F_\text{l} \end{bmatrix}}_{\sysInput}
 .
\end{align}
% \fixme{Input trafo}
% \begin{align}
% %  F_m &= \cos\beta (F_\text{r} + F_\text{l}),
% %  \tau = a \cos\beta (F_\text{r} - F_\text{l})
% % \qquad \Leftrightarrow \qquad
% %  F_\text{r} = \tfrac{1}{2 \cos\beta} (F_m + \tfrac{\tau}{a}), \
% %  F_\text{l} = \tfrac{1}{2 \cos\beta} (F_m - \tfrac{\tau}{a})
%  \begin{bmatrix} F_m \\ \tau \end{bmatrix}
%  = \begin{bmatrix} \cos\beta & \cos\beta \\ a \cos\beta & -a \cos\beta \end{bmatrix}
%  \begin{bmatrix} F_\text{r} \\ F_\text{l} \end{bmatrix}
% \end{align}

\paragraph{Reference trajectory.}
For the following we chose $\sysInputMatLComp = [1, 0, -\tfrac{\sin\beta}{l \cos\beta}]$ as a left complement to $\sysInputMat$.
With this, the matching condition for the reference from \eqref{eq:MatchingForceZeroError} reads
\begin{align}
 \lambda^{\text{ZeroError}} = \m \big(\vRxd - \underbrace{\tfrac{\Jz \sin\beta}{\m \ArmRadius \cos\beta}}_{\eps} \wRzd - \wRz \vRy + \gravityAccConst \sin\tiltAngleR \big) = 0
\end{align}
This can be fulfilled by parameterizing the configuration through the flat output $y_{1\idxRef} = \rRx - \eps \sin\tiltAngleR$, $y_{2\idxRef} = \rRy + \eps \cos\tiltAngleR$ (see e.g.\ \cite{Fliess:LieBacklund}), i.e.
\begin{align}
 \rxR &= y_{1\idxRef} - \eps \frac{\ddot{y}_{1\idxRef}}{\sqrt{\ddot{y}_{1\idxRef}^2 + (\ddot{y}_{2\idxRef}+\gravityAccConst)^2}},
\\
 \ryR &= y_{2\idxRef} - \eps \frac{\ddot{y}_{2\idxRef}-\gravityAccConst}{\sqrt{\ddot{y}_{1\idxRef}^2 + (\ddot{y}_{2\idxRef}+\gravityAccConst)^2}},
\\
 \tiltAngle_{\idxRef} &= \atanTwo(\ddot{y}_{1\idxRef}, \ddot{y}_{2\idxRef}+\gravityAccConst).
\end{align}
Note that this parameterization fails if $\ddot{y}_{1\idxRef} = \ddot{y}_{2\idxRef}+\gravityAccConst = 0$, i.e. the body is in free fall.

% \begin{align}
%  \matchingForceR = \m (\ddot{y}_{1\idxRef} \cos\tiltAngleR + (\ddot{y}_{2\idxRef} + \gravityAccConst) \sin\tiltAngleR) = 0
% \end{align}

\paragraph{Closed loop.}
As for the model, the closed loop templates are the ones the planar rigid body from \autoref{sec:CtrlPlanarRigidBody}. 
Due to symmetry reasons we set the parameters $\hcx=\lcx=\scx=0$.
% \begin{align}
%  \potentialEnergyC &= \tfrac{1}{2} \kc \big( (\rEx - \hcy \sin\tiltAngleE)^2 + (\rEy + \hcy(\cos\tiltAngleE-1))^2 \big) + \kapcz (1-\cos\tiltAngleE),
% \\
%  \sysDissMatC &= \begin{bmatrix} \dc & 0 & -\dc \lcy \\ 0 & \dc & 0 \\ -\dc \lcy & 0 & \sigcz + \dc \lcy^2 \end{bmatrix},
%  \qquad
%  \sysInertiaMatC = \begin{bmatrix} \mc & 0 & -\mc \scy \\ 0 & \mc & 0 \\ -\mc \scy & 0 & \Jcz + \mc \scy^2 \end{bmatrix}
% \end{align}
% This leads to the desired closed loop kinetics
% \begin{multline}
%  \underbrace{\begin{bmatrix} \mc & 0 & -\mc \scy \\ 0 & \mc & 0 \\ -\mc \scy & 0 & \Jcz + \mc \scy^2 \end{bmatrix}}_{\sysInertiaMatC}
%  \underbrace{\begin{bmatrix} \vExd \\ \vEyd \\ \wEzd \end{bmatrix}}_{\sysVelEd}
%  +
%  \underbrace{\begin{bmatrix} -\mc \wz \vEy \\ \mc \wz (\vEx - \scy \wEz) \\ \mc \scy \wz \vEy \end{bmatrix}}_{\gyroForceC}
% \\
%  +
%  \underbrace{\begin{bmatrix} \dc & 0 & -\dc \lcy \\ 0 & \dc & 0 \\ -\dc \lcy & 0 & \sigcz + \dc \lcy^2 \end{bmatrix}}_{\sysDissMatC}
%  \underbrace{\begin{bmatrix} \vEx \\ \vEy \\ \wEz \end{bmatrix}}_{\sysVelE}
%  +
%  \underbrace{\begin{bmatrix} \kc (\hcy \sin\tiltAngleE - \rEx) \\ \kc (\hcy (1-\cos\tiltAngleE) - \rEy) \\ \kc \hcy (\rEx - \hcy \sin\tiltAngleE) - \kapcz \sin\tiltAngleE \end{bmatrix}}_{\differential \potentialEnergyC}
%  =
%  \tuple{0}.
% \end{multline}

\paragraph{Matching.}
The matching force $\tuple{\lambda}$ from \eqref{eq:DefMatchingForce} with the orthogonal complement from above takes a rather cumbersome form and is not given explicitly here.
Instead we will investigate its linear approximation about any reference trajectory with $\tiltAngle_\idxRef = 0$: The matrices of the linearized model and desired closed loop are
\begin{subequations}
\begin{align}
 \sysInertiaMatLin &= \begin{bmatrix} m & 0 & 0 \\ 0 & m & 0 \\ 0 & 0 & J \end{bmatrix},&
 \sysInertiaMatCLin &= \begin{bmatrix} \mc & 0 & -\mc\scy \\ 0 & \mc & 0 \\ -\mc\scy & 0 & \Jcz + \mc\scy^2 \end{bmatrix},
\\
 \sysDissMatLin &= \begin{bmatrix} 0 & 0 & 0 \\ 0 & 0 & 0 \\ 0 & 0 & 0 \end{bmatrix},&
 \sysDissMatCLin &= \begin{bmatrix} \dc & 0 & -\dc\lcy \\ 0 & \dc & 0 \\ -\dc\lcy & 0 & \sigcz + \dc\lcy^2 \end{bmatrix},
\\
 \sysStiffMatLin &= \begin{bmatrix} 0 & 0 & m \gravityAccConst \\ 0 & 0 & 0 \\ 0 & 0 & 0 \end{bmatrix},&
 \sysStiffMatCLin &= \begin{bmatrix} \kc & 0 & -\kc\hcy \\ 0 & \kc & 0 \\ -\kc\hcy & 0 & \kapcz + \kc\hcy^2 \end{bmatrix}.
\end{align}
\end{subequations}
The conditions for $(\sysInputMatLComp)^\top \big( \sysInertiaMatLin \sysInertiaMatCLin^{-1} \sysDissMatCLin - \sysDissMatLin\big) = \mat{0}$ and $(\sysInputMatLComp)^\top \big( \sysInertiaMatLin \sysInertiaMatCLin^{-1} \sysStiffMatCLin - \sysStiffMatLin\big) = \mat{0}$ for the linearized matching force from \eqref{eq:MatchingForceLin} to vanish are equivalent to
\begin{subequations}\label{eq:PVTOLMatchingCondLin}
\begin{align}
 \kc \big( \Jcz - \mc(\hcy - \scy)(\scy - \eps) \big) &= 0,
\\
 \mc \kapcz (\scy-\eps) - \kc\hcy \big( \Jcz - \mc (\hcy - \scy)(\scy - \eps) \big) &= \Jcz \mc \gravityAccConst, 
\\
 \dc \big( \Jcz - \mc(\lcy - \scy)(\scy - \eps) \big) &= 0,
\\
 \mc \sigcz (\scy-\eps) - \dc\,\lcy \big( \Jcz - \mc (\lcy - \scy)(\scy - \eps) \big) &= 0.
\end{align}
\end{subequations}
One solution for this is
\begin{align}\label{eq:PVTOLMatchingSolution}
 \Jcz &= \mc(\hcy - \scy)(\scy - \eps),&
 \lcy &= \hcy,&
 \sigcz &= 0,&
 \kapcz &= \mc \gravityAccConst (\hcy-\scy)
\end{align}
which leaves the parameters $\kc, \dc, \mc, \hcy, \scy$ for tuning.
For the energies to be positive (semi) definite, i.e.\ $\sysInertiaMatC > 0$, $\sysDissMatC \geq 0$ and $\potentialEnergyC > 0$ we need
\begin{align}
 \kc, \dc, \mc > 0, \qquad \hcy > \scy > \eps.
\end{align}

The remaining matching force is
\begin{align}
 \tilde{\sysForce} = \frac{\mc(\hcy-\scy)}{m(\hcy-\eps)} \begin{bmatrix} 1 \\ 0 \\ -\eps \end{bmatrix} \lambda
\end{align}
where $\lambda$ for the corresponding approach is
\begin{subequations}
\begin{align}
 \lambda^{\text{ParticleBased}} &= m(a_{\idxY\idxRef} - \eps\wzR^2)\sin\tiltAngleE
\\
 \lambda^{\text{BodyBased}} &= m(a_{\idxY\idxRef}\sin\tiltAngleE - \rxE\wzR^2 + 2\vyE\wzR + (\eps(1-\cos\tiltAngleE)-\ryE)\wzRd)
\\
 \lambda^{\text{EnergyBased}} &= m(a_{\idxY\idxRef}\sin\tiltAngleE - \rxE\wzR^2 +\phantom{2}\vyE\wzR + (\eps(1-\cos\tiltAngleE)-\ryE)\wzRd)
\end{align} 
\end{subequations}
where $a_{\idxY\idxRef} = \vyRd + \vxR\wzR + \gravityAccConst(\cos\tiltAngleR - 1)$. 
Note that for a constant reference motion $\vRd = 0$ (which implies $\tiltAngleR = 0$) the matching force vanishes $\lambda = 0$ for all approaches.


\begin{figure}
 \centering
 \input{graphics/PVTOLPendulum2.pdf_tex}
 \caption{Interpretation of the controlled PVTOL as a mechanical system} 
 \label{fig:PVTOLPendulum}
\end{figure}

\paragraph{Mechanical interpretation.}
The closed loop template is a planar rigid body with attached springs and dampers as motivated in \autoref{sec:RB} but without gravity.
Taking into account the parameter constraints \eqref{eq:PVTOLMatchingSolution}, the controlled PVTOL corresponds to the mechanical system illustrated in \autoref{fig:PVTOLPendulum}.
The condition $\hcy > \scy$ implies that the center of mass must be below the center of stiffness.
As discussed in \autoref{sec:DecouplingRigidBody}, $\hcy \neq \scy$ results in a coupling between translational and rotational motion.
From a stability/attractiveness point of view this is necessary: As there is no inherent rotational dissipation $\sigcz = 0$, the corresponding energy is dissipated through its coupling to the translational motion.
The coupling also reflects the characteristic behavior of the PVTOL that tilting is required for sideways motion.



\paragraph{Pole placement.}
Even if the tuning parameters might be mechanically intuitive, a classic pole placement approach can be instructive:
The characteristic polynomial of the linearized system is
% \begin{align}
%  \det (\sysInertiaMatC \EigenVal^2 + \sysDissMatC \EigenVal + \sysStiffMatC) = 0
% \nonumber\\
% \Leftrightarrow \quad
%  \det \begin{bmatrix}
%   \bodyMassC{}{} \EigenVal^2 + \bodyDampingC{}{} \EigenVal + \bodyStiffnessC{}{} & 0 & \bodyMassC{}{}(\bodyCOSC{}{}{2}-\bodyCOMC{}{}{2}) \EigenVal^2 \\
%   0 & \bodyMassC{}{} \EigenVal^2 + \bodyDampingC{}{} \EigenVal + \bodyStiffnessC{}{} & 0 \\
%   \bodyMassC{}{}(\bodyCOSC{}{}{2}-\bodyCOMC{}{}{2}) \EigenVal^2 & 0 & \bodyMassC{}{}(\bodyCOSC{}{}{2} - \varepsilon)(\bodyCOSC{}{}{2}-\bodyCOM{}{}{2}) \EigenVal^2+ \bodyMassC{}{} g (\bodyCOSC{}{}{2}-\bodyCOMC{}{}{2}) \\
%  \end{bmatrix}
%  = 0
% \end{align}
\begin{multline}
 \sfrac{\det(\sysInertiaMatC \EigenVal^2 + \sysDissMatC \EigenVal + \sysStiffMatC)}{\det\sysInertiaMatC}
\\
 = (\EigenVal^2 + \tfrac{\dc}{\mc} \EigenVal + \tfrac{\kc}{\mc})
 \big(\EigenVal^4
 + \tfrac{\dc(\hcy-\eps)}{\mc(\scy-\eps)} \EigenVal^3
 + \tfrac{\kc(\hcy-\eps) + \mc \gravityAccConst}{\mc(\scy-\eps)} \EigenVal^2
 + \tfrac{\dc \gravityAccConst}{\mc (\scy-\eps)} \EigenVal
 + \tfrac{\kc \gravityAccConst}{\mc (\scy-\eps)} \big)
\end{multline}
From a desired polynomial of forth degree $\EigenVal^4 + p_3 \EigenVal^3 + p_2 \EigenVal^2 + p_1 \EigenVal + p_0$ we get the parameters
\begin{align}
 \kc &= \frac{\mc \, p_0 p_1}{p_1 p_2 - p_0 p_3},&
 \hcy &= \eps + \frac{\gravityAccConst\, p_3}{p_1},&
 \Jcz &= \frac{\mc \gravityAccConst^2 (p_1 p_2 p_3 - p_1^2 - p_0 p_3^2)}{(p_1 p_2 - p_0 p_3)^2},
\\
 \dc &= \frac{\mc\, p_1^2}{p_1 p_2 - p_0 p_3},&
 \scy &= \varepsilon + \frac{\gravityAccConst\, p_1}{p_1 p_2 - p_0 p_3},&
 \kapcz &= \frac{\mc \gravityAccConst^2 (p_1 p_2 p_3 - p_1^2 - p_0 p_3)}{p_1 (p_1 p_2 - p_0 p_3)}, 
\end{align}
and leaves an arbitrary choice for $\mc > 0$.
Note that the Hurwitz criterion ($p_0, p_1, p_2, p_3, p_1 p_2 - p_0 p_3, p_1 p_2 p_3 - p_1^2 - p_0 p_3^2 > 0$) implies that $\kc, \dc, \Jcz, \kapcz > 0$ which ultimately implies positive definiteness of the total energy and non-negativity of its derivative.

\paragraph{Simulation.}
A very similar control design approach for the PVTOL was presented in \cite{Konz:AT}.
Therein on finds two exemplary simulation results for stabilization and tracking or a looping trajectory.

The PVTOL is a simpler version of the 3d bicopter which will be discussed in the following and the simulation results there also apply to the PVTOL.


% \paragraph{Flatness based control}
% It is well known that the PVTOL is a flat system, see \eg \cite{Fliess:LieBacklund}.
% A flat output is
% \begin{align}
%  \yx = \rx - \eps \sa, 
% \qquad
%  \yy = \ry + \eps \ca, 
% \end{align}
% Rewriting the the acceleration constraint \eqref{eq:PVTOLAccConstraint} in terms of the flat output yields the equation for the tilt angle $\tiltAngle$:
% \begin{align}
%  \ca \yxdd + \sa (\yydd + g) &= 0&
%  &\Rightarrow&
%  \tiltAngle &= -\arctan\frac{\yxdd}{\yydd + g}.
% \end{align}
% However, naive feedback linearization might lead to a singularity when $\tiltAngle = \pm \tfrac{\pi}{2}$ as pointed out in \cite{Rudolph:InvariantTrackingPAMM}.
% This can be resolved by the position error $(\yxE, \yyE)$ \wrt to the (moving) reference frame:
% \begin{align}
%  \yxE &= \caR (\yx - \yxD) + \saR(\yy - \yyD),&
%  \yyE &= -\saR (\yx - \yxD) + \caR(\yy - \yyD).
% \end{align}
% Since this is just a time variant transformation, the position error $(\yxE, \yyE)$ is as well a flat output of the system.
% We can define the error dynamics
% \begin{align}
%  (\yxE)^{(4)} + p_3 (\yxE)^{(3)} + p_2 \yxEdd + p_1 \yxEd + p_0 \yxE &= 0
% \\
%  \yyEdd + \tfrac{\bodyDampingC{}{}}{\bodyMassC{}{}} \yyEd + \tfrac{\bodyStiffnessC{}{}}{\bodyMassC{}{}} \yyE &= 0
% \end{align}
% which essentially defines a quasi-static state feedback that stabilizes the motion of the PVTOL to the reference trajectory.
% See \cite{Rudolph:InvariantTrackingPAMM} for further details.
