\subsection{Planar rigid body}\label{sec:CtrlPlanarRigidBody}
A planar rigid body is a free rigid body in two dimensional space, i.e.\ it can translate in two dimensions and rotate about an perpendicular axis as illustrated in \autoref{fig:PlanarRigidBody}.
The model equations as well as the closed loop equations could be directly derived from the three dimensional rigid body by setting e.g.\ $\vz=0$, $\wx=\wy=0$ and removing the trivial equations.
However it might be still instructive to display the resulting equations.

\begin{figure}[ht]
 \centering
 \input{graphics/PlanarRigidBody.pdf_tex}
 \caption{model of the planar rigid body}
 \label{fig:PlanarRigidBody}
\end{figure}

\paragraph{Coordinates and kinematics.}
As configuration coordinates $\sysCoord$ we use the position $\rx,\ry$ and the sine $\sa$ and cosine $\ca$ of the angle $\alpha$.
Consequently we have to impose the constraint $\ca^2+\sa^2-1 = 0$ on the configuration coordinates.
As velocity coordinates $\sysVel$ we use the components $\vx, \vy$ of the translational velocity w.r.t.\ the body fixed frame as illustrated in \autoref{fig:PlanarRigidBody} and the angular velocity $\wz=\dot{\alpha}$.
This kinematic relation is
\begin{align}
 \ddt
 \underbrace{\begin{bmatrix} \rx \\ \ry \\ \sa \\ \ca \end{bmatrix}}_{\sysCoord}
 = 
 \underbrace{\begin{bmatrix} \ca & -\sa & 0 \\ \sa & \ca & 0 \\ 0 & 0 & \ca \\ 0 & 0 & -\sa \end{bmatrix}}_{\kinMat}
 \underbrace{\begin{bmatrix} \vx \\ \vy \\ \wz \end{bmatrix}}_{\sysVel}
\end{align}
The rigid body configuration $\bodyHomoCoord{0}{1}$ and the resulting body Jacobian $\bodyJac{0}{1}$ w.r.t.\ the chosen velocity coordinates are
\begin{align}
 \bodyHomoCoord{0}{1} = \begin{bmatrix} \ca & -\sa & 0 & \rx \\ \sa & \ca & 0 & \ry \\ 0 & 0 & 1 & 0 \\ 0 & 0 & 0 & 1 \end{bmatrix},
\qquad
 \bodyJac{0}{1} = \begin{bmatrix} 1 & 0 & 0 \\ 0 & 1 & 0 \\ 0 & 0 & 0 \\ 0 & 0 & 0 \\ 0 & 0 & 0 \\ 0 & 0 & 1 \end{bmatrix}
\end{align}

\paragraph{Kinetic equation.}
Let the rigid body have the total mass $\m$, the moment of inertia $\Jz$ and the coordinates $\sx,\sy$ of the center of mass w.r.t.\ the body fixed frame.
As control input consider the forces $\Fx, \Fy$ and the torque $\tauz$ as displayed in \autoref{fig:PlanarRigidBody}.
The resulting kinetic equation is
\begin{align}
 \underbrace{\begin{bmatrix} \m & 0 & -\m\sy \\ 0 & \m & \m\sx \\ -\m\sy & \m\sx & \Jz \end{bmatrix}}_{\sysInertiaMat}
 \underbrace{\begin{bmatrix} \vxd \\ \vyd \\ \wzd \end{bmatrix}}_{\sysVeld}
 +
% \underbrace{
 \begin{bmatrix} -m(\vy + \sx\wz) \wz \\ m(\vx-\sy\wz)\wz \\ \m(\sx\vx+\sy\vy)\wz \end{bmatrix}
% }_{\gyroForce}
%  +
%  \underbrace{\begin{bmatrix} m g \sin\tiltAngle \\ m g \cos\tiltAngle \\ 0 \end{bmatrix}}_{\differential \potentialEnergy}
 =
 \underbrace{\begin{bmatrix} \Fx \\ \Fy \\ \tauz \end{bmatrix}}_{\sysInput}.
\end{align}

\paragraph{Control parameters.}
For the controlled kinetics we chose the following non-zero parameters
\begin{align}\label{eq:CtrlParamPlanarRigidBody}
 \bodyMassC{0}{1}, \bodyDamping{0}{1}, \bodyStiffness{0}{1} \in \RealNum^+,
\quad
 \bodyCOMCoeffC{0}{1}{\idxX}, \bodyCOMCoeffC{0}{1}{\idxY}, \bodyCODCoeffC{0}{1}{\idxX}, \bodyCODCoeffC{0}{1}{\idxY}, \bodyCOSCoeffC{0}{1}{\idxX}, \bodyCOSCoeffC{0}{1}{\idxY} \in \RealNum,
\quad
 \bodyMOICoeffC{0}{1}{\idxZ}, \bodyMODCoeffC{0}{1}{\idxZ}, \bodyMOSCoeffC{0}{1}{\idxZ} \in \RealNum^+ .
\end{align}
Since all parameters are associated with the configuration $\bodyHomoCoord{0}{1}$, we drop the indices in the following, i.e.\ $\mc = \bodyMassC{0}{1}$.

\paragraph{Potential.}
The potential, resulting from the chosen parameters \eqref{eq:CtrlParamPlanarRigidBody}, and its derivatives are
\begin{subequations}
\begin{align}
 \potentialEnergyC &= \tfrac{1}{2} \kc (\rx\!-\!\rxR)^2 + \tfrac{1}{2} \kc (\ry\!-\!\ryR)^2 + \kapcz (1-\caE)
\nonumber\\
 &\qquad+ \kc\hcx(\ca\!-\!\caR)(\rx\!-\!\rxR) - \kc\hcy(\sa\!-\!\saR)(\rx\!-\!\rxR)
\nonumber\\
 &\qquad+ \kc\hcy(\ca\!-\!\caR)(\ry\!-\!\ryR) - \kc\hcx(\sa\!-\!\saR)(\ry\!-\!\ryR)
\label{eq:CtrlPotentialPlanarRigidBody}\\
 \differential \potentialEnergyC &= 
 \begin{bmatrix}
  \kc( \ca(\rx\!-\!\rxR) + \sa(\ry\!-\!\ryR) + \hcx(1-\caE) - \hcy\saE) \\
  \kc(-\sa(\rx\!-\!\rxR) + \ca(\ry\!-\!\ryR) + \hcx\saE + \hcy(1-\caE)) \\
  \kc( (\hcx\ca+\hcy\sa) (\ry\!-\!\ryR) - (\hcy\ca+\hcx\sa) (\rx\!-\!\rxR) ) + \kapcz\saE
 \end{bmatrix}
\\
 \differential^2 \potentialEnergyC |_{\idxRef} &=
 \begin{bmatrix}
  \kc & 0 & -\kc\hcy \\
  0 & \kc & \kc\hcx \\
  -\kc\hcy & \kc\hcx & \kapcz
 \end{bmatrix}
\end{align}
\end{subequations}
The sine and cosine of the angle error $\alpha-\alpha_{\idxRef}$ are introduced just for readability
\begin{align}\label{eq:ErrorCoordPlanarRigidBody}
 \caE = \ca \caR + \sa \saR = \cos(\alpha-\alpha_{\idxRef}),
\qquad
 \saE = \sa \caR - \ca \saR = \sin(\alpha-\alpha_{\idxRef}).
\end{align}
From the Hessian $\differential^2 \potentialEnergyC |_{\idxRef}$ at the critical point $\sysCoord=\sysCoordR$ one can see that (local) positive definiteness requires $\kapcz > \kc (\hcx^2+\hcy^2)$.
We will encounter the analog requirement for the controlled moment of inertia $\Jcz$ and damping $\sigcz$.

A transport map for \eqref{eq:CtrlPotentialPlanarRigidBody} is given by\footnotemark
\begin{align}\label{eq:sysVelEPlanarRigidBody}
 \underbrace{\begin{bmatrix} \vxE \\ \vyE \\ \wzE \end{bmatrix}}_{\sysVelE}
 =
 \underbrace{\begin{bmatrix} \vx \\ \vy \\ \wz \end{bmatrix}}_{\sysVel}
 - 
 \underbrace{\begin{bmatrix}
  \caE & \saE & \sa(\rx \!-\! \rxR) - \ca(\ry \!-\! \ryR) \\
  -\saE & \caE & \ca(\rx \!-\! \rxR) + \sa(\ry \!-\! \ryR) \\
  0 & 0 & 1 
 \end{bmatrix}}_{\sysTransportMap}
 \underbrace{\begin{bmatrix} \vxR \\ \vyR \\ \wzR \end{bmatrix}}_{\sysVelR}.
\end{align}
\footnotetext{
An alternative transport map corresponding to \eqref{eq:RigidBodyAlternativeTransportMap} is
\begin{align}
 \sysTransportMap = \begin{bmatrix}
   \caE & \saE & \hcx\saE - \hcy(\caE - 1) \\
  -\saE & \caE & \hcx(\caE - 1) + \hcy\saE \\
  0 & 0 & 1 
 \end{bmatrix}.
\end{align}
}

\paragraph{Particle-based approach.}
The damping and inertia force using the particle based approach are:
\begin{subequations}
\begin{align}
 \genForceDissC &=
 \underbrace{\begin{bmatrix} \dc & 0 &-\dc\,\lcy \\ 0 & \dc & \dc\,\lcx \\ -\dc\,\lcy & \dc\,\lcx & \sigcz \end{bmatrix}}_{\sysDissMat}
 \underbrace{\begin{bmatrix} \vx \\ \vy \\ \wz \end{bmatrix}}_{\sysVel}
\nonumber\\
 &\qquad-
 \begin{bmatrix} \dc\caE & \dc\saE & \dc(\lcx\saE-\lcy\caE) \\ -\dc\saE & \dc\caE & \dc(\lcx\caE+\lcy\saE) \\ -\dc(\lcx\saE+\lcy\caE) & \dc(\lcx\caE-\lcy\saE) & \sigcz\caE \end{bmatrix}
 \underbrace{\begin{bmatrix} \vxR \\ \vyR \\ \wzR \end{bmatrix}}_{\sysVelR},
\\
 \genForceInertiaC &=
 \underbrace{\begin{bmatrix} \mc & 0 &-\mc\,\scy \\ 0 & \mc & \mc\,\scx \\ -\mc\,\scy & \mc\,\scx & \sigcz \end{bmatrix}}_{\sysDissMat}
 \underbrace{\begin{bmatrix} \vxd \\ \vyd \\ \wzd \end{bmatrix}}_{\sysVeld}
 +
 \begin{bmatrix} -\mc(\vy + \scx\wz) \wz \\ \mc(\vx-\scy\wz)\wz \\ \mc(\scx\vx+\scy\vy)\wz \end{bmatrix}
\nonumber\\
 &\qquad-
 \begin{bmatrix} \mc\caE & \mc\saE & \mc(\scx\saE-\scy\caE) \\ -\mc\saE & \mc\caE & \mc(\scx\caE+\scy\saE) \\ -\mc(\scx\saE+\scy\caE) & \mc(\scx\caE-\scy\saE) & \sigcz\caE \end{bmatrix}
 \underbrace{\begin{bmatrix} \vxRd \\ \vyRd \\ \wzRd \end{bmatrix}}_{\sysVelRd}
\nonumber\\
 &\qquad- 
 \begin{bmatrix} -\mc((\vyR + \scx\wzR)\caE - (\vxR-\scy\wzR)\saE) \wzR \\ \mc((\vyR + \scx\wzR)\saE + (\vxR-\scy\wzR)\caE) \wzR \\ \mc((\scx\vxR+\scy\vyR)\caE - (\scy\vxR-\scx\vyR)\saE)\wzR + \Jcz\saE\wzR^2 \end{bmatrix}.
\end{align}
\end{subequations}
The corresponding total energy as defined in \eqref{eq:SolCtrlGaussPrinciple}, is not a Lyapunov function for the closed loop. 

\paragraph{Body-based approach.}
The damping and inertia force using the body-based approach are:
\begin{subequations}
\begin{align}
 \genForceDissC &= \underbrace{\begin{bmatrix} \dc & 0 &-\dc\,\lcy \\ 0 & \dc & \dc\,\lcx \\ -\dc\,\lcy & \dc\,\lcx & \sigcz \end{bmatrix}}_{\sysDissMat}
 \underbrace{\begin{bmatrix} \vxE \\ \vyE \\ \wzE \end{bmatrix}}_{\sysVelE}
\\
 \genForceInertiaC &= \underbrace{\begin{bmatrix} \mc & 0 & -\mc\scy \\ 0 & \mc & \mc\scx \\ -\mc\scy & \mc\scx & \Jcz \end{bmatrix}}_{\sysInertiaMat}
 \underbrace{\begin{bmatrix} \vxEd \\ \vyEd \\ \wzEd \end{bmatrix}}_{\sysVelEd}
 + \begin{bmatrix} 0 & -\mc\wzE & -\mc\scx\wzE \\ \mc\wzE & 0 & -\mc\scy\wzE \\ \mc\scx\wzE & \mc\scy\wzE & 0 \end{bmatrix}
 \begin{bmatrix} \vxE \\ \vyE \\ \wzE \end{bmatrix}
\end{align} 
\end{subequations}
where the velocity error $\sysVelE$ was defined in \eqref{eq:sysVelEPlanarRigidBody}.
The total energy $\totalEnergyC = \tfrac{1}{2} \sysVelE^\top \sysInertiaMat \sysVelE + \potentialEnergyC$ is a Lyapunov function for the closed loop.

\paragraph{Energy-based approach.}
The damping and inertia forces using the energy-based approach are:
\begin{subequations}
\begin{align}
 \genForceDissC &= \underbrace{\begin{bmatrix} \dc & 0 &-\dc\,\lcy \\ 0 & \dc & \dc\,\lcx \\ -\dc\,\lcy & \dc\,\lcx & \sigcz \end{bmatrix}}_{\sysDissMat}
 \underbrace{\begin{bmatrix} \vxE \\ \vyE \\ \wzE \end{bmatrix}}_{\sysVelE},
\\
 \genForceInertiaC &= \underbrace{\begin{bmatrix} \mc & 0 & -\mc\scy \\ 0 & \mc & \mc\scx \\ -\mc\scy & \mc\scx & \Jcz \end{bmatrix}}_{\sysInertiaMat}
 \underbrace{\begin{bmatrix} \vxEd \\ \vyEd \\ \wzEd \end{bmatrix}}_{\sysVelEd}
 + \begin{bmatrix} 0 & -\mc\wz & -\mc\scx\wz \\ \mc\wz & 0 & -\mc\scy\wz \\ \mc\scx\wz & \mc\scy\wz & 0 \end{bmatrix}
 \begin{bmatrix} \vxE \\ \vyE \\ \wzE \end{bmatrix}.
\end{align} 
\end{subequations}
The total energy $\totalEnergyC = \tfrac{1}{2} \sysVelE^\top \sysInertiaMat \sysVelE + \potentialEnergyC$ coincides with the total energy for the body-based approach and is a Lyapunov function for this closed loop as well.
Note that the two approaches do differ in the gyroscopic terms, so do lead to different solutions of the closed loop dynamics.

\paragraph{Simulation result.}
In \autoref{sec:DecouplingRigidBody} we will give and discuss a simulation result for the special parameter choice $\scx=\lcx=\hcx=0$ and $\scy=\lcy=\hcy=0$.

