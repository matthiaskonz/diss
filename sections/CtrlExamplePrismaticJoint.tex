\subsection{Prismatic joint}
\begin{figure}[h]
 \centering
 \input{graphics/PrismaticJoint.pdf_tex}
 \caption{Model of a prismatic joint (left) and the closed loop (right)}
 \label{fig:PrismaticJoint}
\end{figure}

\paragraph{Model.}
Probably the simplest example of a rigid body system is a single body moving in a prismatic joint, i.e. can only translate on one axis as illustrated on the left of \autoref{fig:PrismaticJoint}.
The corresponding rigid body transformation is simply
\begin{align}
 \bodyHomoCoord{0}{1} = \begin{bmatrix} 1 & 0 & 0 & x \\ 0 & 1 & 0 & 0 \\ 0 & 0 & 1 & 0 \\ 0 & 0 & 0 & 1\end{bmatrix}.
\end{align}
With the trivial choice of the velocity coordinate $\sysVel = \dot{x}$, i.e. $\kinMat = 1$, the equation of motion is
\begin{align}
 \m \ddot{x} = F.
\end{align}

\paragraph{Closed loop.}
Due to the geometry of the model, only the controlled total mass $\bodyMass{0}{1}$ within the controlled body inertia matrix $\bodyInertiaMat{0}{1}$ contributes to the controlled kinetics and analog for the dissipation and stiffness.
For the sake of readability we drop the body indices for the following examples of single bodies.
So the only parameters contributing to the controlled kinetics are $\mc, \dc, \kc \in \RealNum > 0$.

For this example all three proposed control approaches are identical.
With the displacement error $x_{\idxErr} = x - x_{\idxRef}$ the resulting energies are
\begin{align}
 \potentialEnergyC &= \tfrac{1}{2} \kc x_{\idxErr}^2,&
 \dissFktC &= \tfrac{1}{2} \dc \dot{x}_{\idxErr}^2,&
 \kineticEnergyC &= \tfrac{1}{2} \mc \dot{x}_{\idxErr}^2,&
 \accEnergyC &= \tfrac{1}{2} \mc \ddot{x}_{\idxErr}^2.
\end{align}
The potential has the obvious transport map $\sysTransportMap = 1$ and the resulting closed loop kinetics are
\begin{align}
 \mc \ddot{x}_{\idxErr} + \dc \dot{x}_{\idxErr} + \kc x_{\idxErr} = 0.
\end{align}
The corresponding explicit control law is
\begin{align}
 F = \m \ddot{x}_{\idxRef} - \tfrac{\m \dc}{\mc} \dot{x}_{\idxErr} - \tfrac{\m \kc}{\mc} x_{\idxErr}.
\end{align}
An interpretation of the closed loop is given on the right side of \autoref{fig:PrismaticJoint}:
The controlled body can be thought as being connected by a spring (stiffness $\kc$) and a damper (viscosity $\dc$) to its reference position $x_{\idxRef}$.
The inertial force $\mc \ddot{x}_{\idxErr}$ reacts to the error acceleration, i.e.\ to the acceleration of the body relative to its reference acceleration $\ddot{x}_{\idxRef}$.
One could say the body has an inertia w.r.t.\ its reference.
