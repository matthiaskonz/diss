\subsection{Rigid body orientation}\label{sec:CtrlExampleRigidBodyOrientation}
\begin{figure}[ht]
 \centering
 \input{graphics/RigidBodyOrientation.pdf_tex}
 \caption{rigid body fixed at one point}
 \label{fig:RigidBodyOrientation}
\end{figure}

\paragraph{Model.}
Consider a rigid body fixed at one point $\bodyPos{}{} = \const$ as illustrated in \autoref{fig:RigidBodyOrientation}.
Its orientation may be parameterized by the coefficients of the rotation matrix $\R = [\Rx,\Ry,\Rz] \in \SpecialOrthogonalGroup(3)$.
With the angular velocity $\w = \veeMatOp(\R^\top\Rd)$ as velocity coordinates, the inertia matrix $\J$ and the control torques $\tuple{\tau}$ about the body fixed axes, the equations of motion may be written as
\begin{align}
 \Rd = \R \wedOp(\w)
\qquad
 \J\wdot + \wedOp(\w) \J \w &= \tuple{\tau}.
\end{align}

\paragraph{Potential energy.}
Only the parameters $\Jc$, $\sigc$ and $\bodyMOSC{}{}$ contribute to the closed loop kinetics.
Using the attitude error $\RE = \RR^\top \R$ the error potential, its differential and Hessian are
\begin{subequations}
\begin{align}
 \potentialEnergyC &= \tr\big( \wedMatOp(\bodyMOSC{}{}) (\idMat[3] - \RE) \big),
\\
 \genForceStiffC = \differential \potentialEnergyC &= \veeTwoOp(\wedMatOp(\bodyMOSC{}{}) \RE),
\\
 (\differential^2 \potentialEnergyC)|_{\R=\RR} &= \bodyMOSC{}{}.
\end{align} 
\end{subequations}
The potential has the transport map $\sysTransportMap = \RE^\top$, so the velocity error for the energy based approach is $\wE = \w - \RE^\top \wR$ which coincides with the body velocity error.

\paragraph{Particle-based approach.}
The particle-based approach \eqref{eq:CtrlApproachParticles} leads to
\begin{subequations}
\begin{align}
 \genForceInertiaC &= \Jc \wdot - \veeMatOp(\RE^\top \wedMatOp(\Jc)) \RE^\top \wRd
\nonumber\\
 &\hspace{10em}
 + \wedOp(\w) \Jc \w + \veeTwoOp(\wedMatOp(\Jc)\wedOp(\wR)^2\RE)
\\
 \genForceDissC &= \sigc \w - \veeMatOp(\RE^\top \wedMatOp(\sigc)) \RE^\top \wR
\end{align} 
\end{subequations}

\paragraph{Body-based approach.}
The body-based approach \eqref{eq:CtrlApproachBody} leads to 
\begin{subequations}
\begin{align}
 \genForceInertiaC &= \Jc \wEd + \wedOp(\wE) \Jc \wE
\\
 \genForceDissC &= \sigc \wE.
\end{align} 
\end{subequations}
The corresponding control law coincides with the one proposed in \cite{Koditschek:TotalEnergy}.
The total energy $\totalEnergyC = \tfrac{1}{2} \wE^\top \Jc \wE + \potentialEnergyC$ with $\totalEnergyCd = -\wE^\top \sigc \wE$ serves as a Lyapunov function for the closed loop.

\paragraph{Energy-based approach.}
For the energy-based approach \eqref{eq:CtrlApprochEnergyKineticsFirst} leads to
\begin{subequations}
\begin{align}
 \genForceInertiaC &= \Jc \wEd + \wedOp(\wedMatOp(\Jc) \w) \wE
\\
 \genForceDissC &= \sigc \wE.
\end{align} 
\end{subequations}
The corresponding control law coincides with one proposed in \cite{Bullo:TrackingAutomatica}.
The total energy is the same as the one for the body based approach.
The two approaches only differ in the gyroscopic terms.
% \begin{align}
%  \gyroForceC 
% % &= \big[ \ConnCoeffLC{\LidxI}{\LidxII}{\LidxIII} \sysVelCoeff{\LidxIII} \sysVelCoeffE{\LidxII} \big]
% % = \big[ \tfrac{1}{2}\big( \BoltzSym{\LidxV}{\LidxI}{\LidxII} \sysInertiaMatCoeffC{\LidxV\LidxIII} + \BoltzSym{\LidxV}{\LidxI}{\LidxIII} \sysInertiaMatCoeffC{\LidxV\LidxII} - \BoltzSym{\LidxV}{\LidxII}{\LidxIII} \sysInertiaMatCoeffC{\LidxV\LidxI} \big) \sysVelCoeff{\LidxIII} \sysVelCoeffE{\LidxII} \big]
% %\nonumber\\
%  &= \tfrac{1}{2} \big( \wedOp(\wE)\Jc \w + \wedOp(\w) \Jc \wE - \Jc \wedOp(\w) \wE \big)
%  = \wedOp(\Jpc \w) \wE
% \end{align}
% Note that $\wedOp(\Jpc \w) \w = \wedOp(\w) \Jpc \w$.

% The explicit control law can be written as
% \begin{align}
%  \tuple{\tau} 
%  &= \J \wdot + \wedOp(\w) \J \w
% \nonumber\\
%  &= \J (\wEd + \RE^\top \wRd - \wedOp(\wE) \RE^\top \wR) + \wedOp(\w) \J \w
% \nonumber\\
%  &= \J (\RE^\top \wRd - \wedOp(\wE) \RE^\top \wR - \Jc^{-1}(\wedOp(\Jpc \w) \wE + \sigc \wE + 2 \veeOp(\kappc \RE))) + \wedOp(\w) \J \w
% \nonumber\\
%  &= \J \big( \RE^\top \wRd - \wedOp(\w) \RE^\top \wR - \Jc^{-1}\wedOp(\w)\Jc \w + \Jc^{-1}\wedOp(\Jpc \w) \RE^\top \wR \big) + \wedOp(\w) \J \w
% \nonumber\\ 
%  &\qquad - \J \Jc^{-1} \big( \sigc \wE + 2 \veeOp(\kappc \RE) \big)
% % \nonumber\\
% %  &= \Jc \R^\top \RR \wRd - \big(\Jc \wedOp(\w) + \wedOp(\Jpc \w)\big) \R^\top \RR \wR
% % \nonumber\\
% %  &\qquad- \sigc (\w - \R^\top \RR \wR) - 2 \veeOp(\kappc \RR^\top \R)
% \end{align}


\paragraph{Linearization.}
For a small error to the reference we have
\begin{align}
 \R = \RR + \RR \wedOp(\LinErrorCoord),
\qquad
 \w = \wR + \LinErrorCoordd,
\qquad
 \Jc \LinErrorCoorddd + \sigc \LinErrorCoordd + \kapc \LinErrorCoord = 0.
\end{align}
