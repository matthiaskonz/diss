\chapter{Tracking control of rigid body systems}\label{chap:Ctrl}
This chapter motivates and discusses several approaches for a model based design of a tracking controller for a rigid body system by static feedback.

\paragraph{System model.}
The previous chapter discussed the equations of motion of rigid body systems:
For chosen configuration coordinates $\sysCoord(t) \in \configSpace$, velocity coordinates $\sysVel(t) \in\RealNum^{\dimConfigSpace}$ and the control inputs $\sysInput(t)\in\RealNum^{\numInputs}$ these have the form
\begin{align}\label{eq:CtrlSysMdl}
 \sysCoordd = \kinMat(\sysCoord) \sysVel, 
\qquad
 \overbrace{\sysInertiaMat(\sysCoord) \sysVeld + \gyroForce(\sysCoord, \sysVel)}^{\genForceInertia(\sysCoord, \sysVel, \sysVeld)} + \overbrace{\sysDissMat(\sysCoord)\sysVel\vphantom{\sysVeld}}^{\genForceDiss(\sysCoord, \sysVel)} + \overbrace{\differential\potentialEnergy(\sysCoord)\vphantom{\sysVeld}}^{\genForceStiff(\sysCoord)} = \sysInputMat(\sysCoord) \sysInput.
% \\[-3.5ex]
%  \phantom{\sysCoordd = \kinMat(\sysCoord) \sysVel, \quad \sysInertiaMat(\sysCoord) \sysVeld +} \underbrace{\phantom{\gyroForce(\sysCoord, \sysVel) + \sysDissMat(\sysCoord)\sysVel + \differential\potentialEnergy(\sysCoord)}}_{\sysForce(\sysCoord, \sysVel)} \phantom{= \sysInputMat(\sysCoord) \sysInput}
% \nonumber
\end{align}
The forces $\genForceInertia, \genForceDiss, \genForceStiff$ may be computed from the rigid body configurations $\bodyHomoCoord{\BidxI}{\BidxII}(\sysCoord)$ and the constitutive parameters $\bodyInertiaMat{0}{\BidxII}, \bodyDissMat{\BidxI}{\BidxII}, \bodyStiffMat{\BidxI}{\BidxII}$.
This structure will be the main inspiration for the design of the controlled system.

However, mathematically, the control approach does not rely on the model having this structure.
We may assume any model of the form 
\begin{align}\label{eq:CtrlSysMdl2}
 \sysCoordd = \kinMat(\sysCoord) \sysVel, 
\qquad
 \sysInertiaMat(\sysCoord) \sysVeld + \sysForce(\sysCoord,\sysVel) = \sysInputMat(\sysCoord)\sysInput
\end{align}
where $\kinMat(\sysCoord) \in \RealNum^{\numCoord\times\dimConfigSpace}$ and $\sysInputMat(\sysCoord) \in \RealNum^{\dimConfigSpace\times\numInputs}$ are full rank, the system inertia matrix $\sysInertiaMat(\sysCoord) \in \SymMatP(\dimConfigSpace)$ is symmetric, positive definite, and $\sysForce(\sysCoord, \sysVel) \in \RealNum^{\dimConfigSpace}$ collects the remaining terms of the kinetic equation.
The system is called \textit{fully-actuated} for $\numInputs = \dimConfigSpace$ and \textit{underactuated} for $0 < \numInputs < \dimConfigSpace$.
Firstly we will restrict to fully-actuated systems and later try to expand the approach to underactuated systems.

% \paragraph{Reference trajectory.}
% The goal of a \textit{tracking controller} is to achieve controlled dynamics for which the actual configuration $\sysCoord(t)$ converges to a predefined \textit{reference trajectory} $t \mapsto \sysCoordR(t)$.
% It has to be feasible $\sysCoordR(t) \in \configSpace$ and sufficiently smooth, so we can define the reference velocity $\sysVelR = \kinMat^+(\sysCoordR) \sysCoordRd$ and acceleration $\sysVelRd$.

\paragraph{Reference trajectory and tracking controller.}
Let there be a \textit{reference trajectory} $t\mapsto\sysCoordR(t)$ which is compatible with the model \eqref{eq:CtrlSysMdl2}:
It must be feasible, i.e.\ $\sysCoordR(t)\in\configSpace$, and sufficiently smooth, so we can define the reference velocity $\sysVelR = \kinMat^+(\sysCoordR) \sysCoordRd$ and acceleration $\sysVelRd$.
For the underactuated case we also require the kinetic equation $\sysInertiaMat(\sysCoordR) \sysVelRd + \sysForce(\sysCoordR,\sysVelR) = \sysInputMat(\sysCoordR)\sysInputR$ to have a solution for $\sysInputR$.

The design task for a \textit{tracking controller} is: 
Find a function $\sysInput = \sysInput(\sysCoord, \sysVel, \sysCoordR, \sysVelR, \sysVelRd)$ (the controller) such that $t \mapsto \sysCoordR(t)$ is a stable and attractive trajectory of the closed loop which is the combination of model \eqref{eq:CtrlSysMdl2} and controller.

\paragraph{State of the art.}
This is a pretty general task and may be tackled by various standard approaches from control theory, see e.g.\ \cite[chap.\,7-10]{Spong:RobotModelingAndControl} for some overview.
For fully actuated systems a popular approach is \textit{computed torque}, see e.g.\ \cite[sec.\,4.5.2]{Murray:Robotic}, also called \textit{inverse dynamics} in \cite[sec.\,8.3]{Spong:RobotModelingAndControl}.
It can be regarded as a particularly simple case of feedback linearization utilizing the fact that any set of minimal generalized coordinates $\genCoord(t)\in\RealNum^{\dimConfigSpace}$ is a \textit{flat output} of a fully actuated mechanical system \cite[sec.\,7.1]{Martin:FlatSystems}.

For underactuated systems there is no standard textbook approach.
Examples for the flatness-based approach can be found in e.g.\ \cite{RathinamFlatness}, \cite{MurrayFlatCataloge} or \cite[sec.\,7.1]{Martin:FlatSystems}.
General Lyapunov designs can be found in \cite{OlfatiSaberDiss} and a approach called \textit{controlled Lagrangian} is proposed in \cite{bloch2000controlled}.

\paragraph{Outline for this chapter.}
With the computed torque method one might consider the topic to be solved for fully actuated systems.
However, for system whose configuration space is not isomorph to $\RealNum^{\dimConfigSpace}$ it is only local due to the requirement of minimal coordinates $\genCoord(t)\in\RealNum^{\dimConfigSpace}$.
Furthermore, applying linear dynamics in these coordinates may result in intrinsic singularities for the closed loop.
Recalling the satellite example from \autoref{sec:MotivationRigidBodyAttitude}, it should be clear that linear dynamics for the Euler angles would probably not be a good choice, also see \cite{Konz:AT} for further examples.

If linear dynamics are not a general choice, what \textit{is} a good choice for the closed loop dynamics?
This chapter motivates three approaches for designing a closed loop for fully actuated systems which rely on the underlying rigid body structure.
In addition, the result will be extended to underactuated systems.
Finally the approach will be applied to several example systems including the tricopter (fully-actuated), the quadcopter (underactuated but flat) and the bicopter (underactuated and probably not flat).


% In the next three sections we will motivate three possible answers which all share the structure
% \begin{align}\label{eq:CtrlClosedLoop}
%  \sysCoordd = \kinMat(\sysCoord) \sysVel, 
% \quad
%  \overbrace{\sysInertiaMatC(\sysCoord) \sysVeld + \gyroForceC(\sysCoord, \sysVel, \sysCoordR, \sysVelR, \sysVelRd)}^{\genForceInertiaC(\sysCoord, \sysVel, \sysVeld, \sysCoordR, \sysVelR, \sysVelRd)} + \genForceDissC(\sysCoord, \sysVel, \sysCoordR, \sysVelR) + \overbrace{\differential\potentialEnergyC(\sysCoord,\sysCoordR)}^{\genForceStiffC(\sysCoord,\sysCoordR)} = \tuple{0}.
% \\[-3.5ex]
%  \phantom{\sysCoordd = \kinMat(\sysCoord) \sysVel, \quad \sysInertiaMatC(\sysCoord) \sysVeld +} \underbrace{\phantom{\gyroForceC(\sysCoord, \sysVel, \sysCoordR, \sysVelR, \sysVelRd) + \genForceDissC(\sysCoord, \sysVel, \sysCoordR, \sysVelR) + \differential\potentialEnergyC(\sysCoord,\sysCoordR)}}_{\sysForceC(\sysCoord, \sysVel, \sysCoordR, \sysVelR, \sysVelRd)} \phantom{= \tuple{0}}
% \nonumber
% \end{align}
% The kinematic relation remains untouched and there is also a controlled system inertia $\sysInertiaMatC\in\SymMatP(\dimConfigSpace)$ and force $\genForceStiffC$ derived from a tracking potential $\potentialEnergyC$.
% However, the gyroscopic terms $\gyroForceC$ and the dissipative force $\genForceDissC$ take rather different forms in the different approaches.
% 
% Combining the desired closed loop \eqref{eq:CtrlClosedLoop} and the model \eqref{eq:CtrlSysMdl} and solving for the input, yields the actual control law
% \begin{align}
%  \sysInput = \sysInputMat^{-1} \big(\sysForce - \sysInertiaMat \sysInertiaMatC^{-1} \sysForceC \big).
% \end{align}
% This is only possible for fully-actuated systems $\numInputs = \dimConfigSpace$.
% In \autoref{sec:CtrlUnderactuated} we will discuss an extension to \textit{underactuated} systems $\numInputs < \dimConfigSpace$.
% Furthermore, in (??), we tackle input constraints. 



%A similar idea has been pursued in \cite{Koditschek:TotalEnergy} and \cite{Bullo:TrackingAutomatica}.

% \begin{figure}[ht]
%  \centering
%  \newcommand{\macCtrlBlockKinematics}{$\sysCoordd = \kinMat(\sysCoord)  \sysVel$}
%  \newcommand{\macCtrlBlockKinetics}{$\sysInertiaMat(\sysCoord) \sysVeld + \gyroForce(\sysCoord, \sysVel) = \genForceImpressed(\sysCoord, \sysVel, \sysInput)$}
%  \newcommand{\macCtrlBlockCtrl}{$\sysInput = \sysInput(\sysCoord, \sysVel, \sysCoordR, \sysVelR, \sysVelRd)$}
%  \newcommand{\macControlBlockClosedLoopKinetics}{$\sysInertiaMatC(\sysCoord) \sysVeld + \gyroForceC(\sysCoord, \sysVel) = \tuple{0}$}
%  \newcommand{\macCtrlBlockMeas}{$\sysCoord, \sysVel$}
%  \newcommand{\macCtrlBlockInput}{$\sysInput$}
%  \newcommand{\macCtrlBlockRef}{$\sysCoordR, \sysVelR, \sysVelRd$}
%  \newcommand{\macCtrlBlockInitials}{$\sysCoord_0, \geoConstraint(\sysCoord_0) = 0, \sysVel_0$}
%  \input{graphics/CtrlBlock.pdf_tex}
%  \caption{Model and tracking controller}
%  \label{fig:CtrlBlock}
% \end{figure}
