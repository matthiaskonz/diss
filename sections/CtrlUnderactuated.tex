\section{Underactuated systems}\label{sec:CtrlUnderactuated}
The first three sections of this chapter motivated different desired closed loop dynamics which share the structure
\begin{align}\label{eq:CtrlClosedLoopUnderactuated}
 \sysCoordd = \kinMat(\sysCoord) \sysVel, 
\quad
 \sysInertiaMatC(\sysCoord) \sysVeld + \sysForceC(\sysCoord, \sysVel, \sysCoordR, \sysVelR, \sysVelRd) = \tuple{0}.
\end{align}
The system model has still the form of \eqref{eq:CtrlSysMdl}:
\begin{align}\label{eq:CtrlSysMdlUnderactuated}
 \sysCoordd = \kinMat(\sysCoord) \sysVel, 
\quad
 \sysInertiaMat(\sysCoord) \sysVeld + \sysForce(\sysCoord, \sysVel) = \sysInputMat(\sysCoord) \sysInput.
\end{align}
For a fully actuated system, i.e.\ $\rank\sysInputMat = \dimConfigSpace$, the combination of \eqref{eq:CtrlClosedLoopUnderactuated} and \eqref{eq:CtrlSysMdlUnderactuated} can be solved for the system input $\sysInput$ yielding the actual control law.
For an underactuated system, i.e.\ $\rank\sysInputMat = \numInputs < \dimConfigSpace$, this is generally not possible.

%In this section we discuss an approach for the tracking control of underactuated mechanical systems.

%but with the input matrix $\sysInputMat(\sysCoord) \in \RealNum^{\dimConfigSpace\times\numInputs}$, $\rank \sysInputMat = \numInputs < \dimConfigSpace$, i.e.\ there are less control inputs $\sysInput(t)\in\RealNum^{\numInputs}$ than the degree of freedom $\dimConfigSpace$.

\subsection{Control law through static optimization}\label{sec:CtrlUnderactuatedOptim}
If the desired closed loop dynamics \eqref{eq:CtrlClosedLoopUnderactuated} cannot be achieved exactly, the next best thing is to get ``as close as possible'' while still obeying the model dynamics \eqref{eq:CtrlSysMdlUnderactuated}
This is done by computing the control input $\sysInput$ by means of the static optimization problem
\begin{align}\label{eq:MinProbUnderactuated}
 \begin{array}{rl}
  \text{minimize} & \GaussianConstraintC \, = \, \tfrac{1}{2} \norm[\sysInertiaMatC]{\sysVeld + \sysInertiaMatC^{-1} \sysForceC}^2
  \\[1ex]
  \text{subject to} & \sysInertiaMat \sysVeld + \sysForce = \sysInputMat \sysInput, \ \sysInput \in \RealNum^\numInputs 
 \end{array}
 .
\end{align}
Mathematically we do not have to use the metric coefficients $\sysInertiaMatC$ here for the optimization, any other symmetric positive definite matrix would serve the purpose as well.
However, from the physics point of view, the terms in the norm correspond to an acceleration, so the inertia of the desired closed loop is the reasonable choice.
Furthermore, from the control point of view, the additional parameters arising with a different matrix would not turn out to be really useful.

Elimination of the acceleration $\sysVeld = \sysInertiaMat^{-1}(\sysInputMat \sysInput - \sysForce)$ from \eqref{eq:MinProbUnderactuated} leads to
\begin{align}\label{eq:GaussianUnderactuated}
 \GaussianConstraintC &= \tfrac{1}{2} \norm[\sysInertiaMatC]{\sysInertiaMat^{-1}\sysInputMat \sysInput - \underbrace{(\sysInertiaMat^{-1}\sysForce - \sysInertiaMatC^{-1}\sysForceC)}_{\tilde{\tuple{a}}}}^2
\nonumber\\
 &= \tfrac{1}{2} \sysInput^\top \underbrace{\sysInputMat^\top \sysInertiaMat^{-1} \sysInertiaMatC \sysInertiaMat^{-1} \sysInputMat}_{\mat{H}} \sysInput - \sysInput^\top \sysInputMat^\top \sysInertiaMat^{-1} \sysInertiaMatC \tilde{\tuple{a}} + \tfrac{1}{2} \tilde{\tuple{a}}^\top \sysInertiaMatC \tilde{\tuple{a}}
\nonumber\\
 &= \tfrac{1}{2} \big( \sysInput - \underbrace{\mat{H}^{-1} \sysInputMat^\top \sysInertiaMat^{-1} \sysInertiaMatC \tilde{\tuple{a}}}_{\sysInput_0} \big)^\top \mat{H} \big(\sysInput - \underbrace{\mat{H}^{-1} \sysInputMat^\top \sysInertiaMat^{-1} \sysInertiaMatC \tilde{\tuple{a}}}_{\sysInput_0} \big)
\nonumber\\
 &\hspace{10em}+ \tfrac{1}{2} \tilde{\tuple{a}}^\top \sysInertiaMatC \underbrace{\big( \idMat[\dimConfigSpace] - \sysInertiaMat^{-1} \sysInputMat \mat{H}^{-1} \sysInputMat^\top \sysInertiaMat^{-1} \sysInertiaMatC \big)}_{\mat{P}^\bot} \tilde{\tuple{a}}
\nonumber\\
 &= \tfrac{1}{2} \norm[\mat{H}]{\sysInput - \sysInput_0}^2 + \underbrace{\tfrac{1}{2} \norm[\sysInertiaMatC]{\mat{P}^\bot \tilde{\tuple{a}}}^2}_{\GaussianConstraintC_0}.
\end{align}
For the formulation of $\GaussianConstraintC_0$ it is crucial to note that $\mat{P}^\bot$ is a projection matrix, which will be exploited in the next subsection.

The control law, i.e.\ the solution of the minimization problem, is obviously $\sysInput = \sysInput_0$.
The resulting closed loop kinetics are
\begin{align}\label{eq:UnderactuatedControlledKinetics}
 \sysInertiaMat \sysVeld + \sysForce &= \sysInputMat \overbrace{\mat{H}^{-1} \sysInputMat^\top \sysInertiaMat^{-1} \sysInertiaMatC (\sysInertiaMat^{-1}\sysForce - \sysInertiaMatC^{-1}\sysForceC)}^{\sysInput_0}
\nonumber\\
\Leftrightarrow \qquad \qquad 
 \sysVeld &= \sysInertiaMat^{-1} \sysInputMat \mat{H}^{-1} \sysInputMat^\top \sysInertiaMat^{-1} \sysInertiaMatC (\sysInertiaMat^{-1}\sysForce - \sysInertiaMatC^{-1}\sysForceC) - \sysInertiaMat^{-1} \sysForce
% \sysVeld &= \underbrace{\sysInputMatAst \mat{H}^{-1} \sysInputMatAst^\top \sysInertiaMatC}_{\idMat[\dimConfigSpace] - \mat{P}^\bot} (\sysInertiaMat^{-1}\sysForce - \sysInertiaMatC^{-1}\sysForceC) - \sysInertiaMat^{-1} \sysForce
\nonumber\\
\qquad \Leftrightarrow \qquad
 \sysInertiaMatC \sysVeld + \sysForceC &= \underbrace{\sysInertiaMatC \mat{P}^\bot \big( \sysInertiaMatC^{-1} \sysForceC - \sysInertiaMat^{-1} \sysForce \big)}_{\tilde{\sysForce}}.
\end{align}
This result means that there is an additional vector\footnote{The coefficients $\tilde{\tuple{b}}$ do indeed transform like a tensor, even though $\sysForce$ and $\sysForceC$ do not.} $\tilde{\tuple{b}}(\sysCoord, \sysVel, \sysCoordR, \sysVelR, \sysVelRd) \in \RealNum^{\dimConfigSpace}$ in the closed loop which allows for the closed loop \eqref{eq:UnderactuatedControlledKinetics} to be realizable with the available controls.

For the special case of a fully actuated system, i.e.\ $\sysInputMat$ is invertible, the control law simplifies to $\sysInput_0 = \sysInputMat^{-1} (\sysForce - \sysInertiaMat \sysInertiaMatC^{-1}\sysForceC)$. %which is the result we get from solving the combination of the desired closed loop \eqref{eq:CtrlClosedLoopUnderactuated} and the model \eqref{eq:CtrlSysMdlUnderactuated}.
Furthermore we have $\mat{P}^\bot = \mat{0}$ and consequently $\tilde{\tuple{b}} = \tuple{0}$ and $\GaussianConstraintC_0 = 0$.

In general, the value $\GaussianConstraintC_0$ is a measure of how much the resulting closed loop differs from the original desired system \eqref{eq:CtrlClosedLoopUnderactuated}.
One main goal when parameterizing the controller is to make $\GaussianConstraintC_0$ as small as possible.
Unfortunately the form of $\GaussianConstraintC_0$ in \eqref{eq:UnderactuatedControlledKinetics} is not handy, mainly due to $\rank \mat{P}^\bot = \dimConfigSpace - \numInputs$.
In the following we like to find a more handy formulation.

\subsection{Matching condition}\label{sec:MatchingCondition}
The matrices $\mat{P} = \sysInertiaMat^{-1} \sysInputMat \mat{H}^{-1} \sysInputMat^\top \sysInertiaMat^{-1} \sysInertiaMatC$ and its complementary $\mat{P} = \idMat[\dimConfigSpace] - \mat{P}^\bot$ from \eqref{eq:GaussianUnderactuated} are projection matrices, i.e. $\mat{P}^2 = \mat{P}$ and $(\mat{P}^\bot)^2 = \mat{P}^\bot$.
Furthermore they are orthogonal w.r.t.\ the inner product $\sProd[\sysInertiaMatC]{\cdot}{\cdot}$, i.e.\ $\sProd[\sysInertiaMatC]{\mat{P} \tuple{\xi}}{\mat{P}^\bot\tuple{\eta}} = 0 \, \forall\, \tuple{\xi},\tuple{\eta}\in \RealNum^{\dimConfigSpace}$.
The image of the projector $\mat{P}$ is the subspace of $\RealNum^{\dimConfigSpace}$ that is spanned by the columns of $\sysInertiaMat^{-1} \sysInputMat$.
Consequently we have $\rank\mat{P} = \numInputs$ and $\rank\mat{P}^\bot = \dimConfigSpace-\numInputs$.
So $\tilde{\tuple{b}}$ lies in a $(\dimConfigSpace-\numInputs)$-dimensional subspace.
The goal of this subsection is to construct a basis for this subspace.

Let $\sysInputMatLComp \in \RealNum^{\dimConfigSpace-\numInputs}$ be any orthogonal complement to $\sysInputMat$, i.e.\ $\rank \sysInputMatLComp = \dimConfigSpace-\numInputs$ and $\sysInputMat^\top \sysInputMatLComp = \mat{0}$.
The columns of $\sysInertiaMatC^{-1} \sysInertiaMat \sysInputMatLComp$ are orthogonal to the columns of $\sysInertiaMat^{-1} \sysInputMat$ under the metric $\sProd[\sysInertiaMatC]{\cdot}{\cdot}$ and span the image of $\mat{P}^\bot$.
Using some basic properties of orthogonal projectors (see \fixme{sec:OrthogonalProjectors}) we can formulate
\begin{align}
 \mat{P}^\bot = \sysInertiaMatC^{-1} \sysInertiaMat \sysInputMatLComp \underbrace{\big( (\sysInputMatLComp)^\top \sysInertiaMat \sysInertiaMatC^{-1} \sysInertiaMat \sysInputMatLComp \big)^{-1}}_{=\,\mat{S}\,\in\,\SymMatP(\dimConfigSpace-\numInputs)} (\sysInputMatLComp)^\top \sysInertiaMat,
% \quad
%  \mat{S} = (\sysInputMatLComp)^\top \sysInertiaMat \sysInertiaMatC^{-1} \sysInertiaMat \sysInputMatLComp \in \SymMatP(\dimConfigSpace-\numInputs).
\end{align}
Plugging this into the expressions from the previous subsection, we find 
\begin{align}\label{eq:DefMatchingForce}
 \tilde{\tuple{b}} = \sysInertiaMat \sysInputMatLComp \mat{S} \underbrace{(\sysInputMatLComp)^\top (\sysInertiaMat \sysInertiaMatC^{-1} \sysForceC - \sysForce)}_{=\,\tuple{\lambda}\,\in\,\RealNum^{\dimConfigSpace-\numInputs}}.
\qquad
 \GaussianConstraintC_0 = \tfrac{1}{2} \norm[\mat{S}]{\tuple{\lambda}}^2.
\end{align}
It is much simpler to analyse $\tuple{\lambda}$ which has only the dimension of the underactuation $\dimConfigSpace-\numInputs$ instead of $\tilde{\tuple{b}}$ which has the full dimension $\dimConfigSpace$ of the configuration space.
% The value of $\GaussianConstraintC_0$ is a measure of how much the actual closed loop \eqref{eq:UnderactuatedControlledKinetics} differs from the original desired closed loop \eqref{eq:CtrlClosedLoopUnderactuated}.
% With the use of a basis, the columns of $\sysInputMatLComp$, we are able to formulate this measure in terms of minimal coefficients $\tuple{\lambda}$.
% This will turn out very useful for the actual design.
Though it should be stressed that the values of $\GaussianConstraintC_0$ and $\tilde{\tuple{b}}$ are, as derived above, independent of the choice of $\sysInputMatLComp$.
The naming $\tuple{\lambda}$ is because we could have derived the same expressions by applying an acceleration constraint $\sysInputMatLComp(\sysInertiaMat \sysVeld + \sysForce)=\tuple{0}$ to the desired closed loop and using the method of \textit{Lagrangian multipliers}.

The best case is, of course, if we achieve 
\begin{align}\label{eq:MatchingCondition}
 \tuple{\lambda} = (\sysInputMatLComp)^\top (\sysInertiaMat \sysInertiaMatC^{-1} \sysForceC - \sysForce) = \tuple{0}
\qquad \Rightarrow \qquad
 \tilde{\tuple{b}} = \tuple{0}, \ \GaussianConstraintC_0 = 0
\end{align}
i.e.\ the desired closed loop is realized exactly.
An approach based on this is discussed in \cite{bloch2000controlled}.
A condition similar to \eqref{eq:MatchingCondition} is therein called the \textit{the matching condition} and is required to be fulfilled exactly.
However, the examples for which this approach is demonstrated restricts to stabilization tasks $\sysVelR = \tuple{0}$ for small academic systems.

An advantage of the presented approach is that the control law $\sysInput = \sysInput_0$ is defined independently of whether the matching condition is fulfilled or not.
Instead the quantity $\tuple{\lambda}$, which we will call the \textit{matching force} in the following, ensures that the control law is realizable.

\subsection{Approximations}
The matching force \eqref{eq:MatchingCondition} may become very cumbersome for complex systems and might even be impossible to vanish with the given parameters.
It might be instructive to analyse it for particular situations.

\paragraph{Zero error.}
Assume that the controller tracks the reference perfectly, i.e.\ $\sysCoord = \sysCoordR$ and $\sysVel=\sysVelR$.
One may check that for this case the three approaches all yield $\sysForceC = \sysInertiaMatC\sysVelRd$.
The resulting matching force $\tuple{\lambda}^{\text{ZeroError}}$ for this special case is
\begin{align}\label{eq:MatchingForceZeroError}
 \tuple{\lambda}^{\text{ZeroError}} = (\sysInputMatLComp(\sysCoordR))^\top (\sysInertiaMat(\sysCoordR) \sysVelRd - \sysForce(\sysCoordR, \sysVelR))
\end{align}
Evidently, this is independent of the closed loop parameters, and should rather be regarded as a constraint on the \textit{reference trajectory} $t\mapsto \sysCoordR(t)$.
The condition $\tuple{\lambda}^{\text{ZeroError}}=\tuple{0}$ is essentially the model equation after elimination of the control inputs.

A very useful approach here is to formulate the reference trajectory in terms of a \textit{flat output} \cite{Fliess:Flatness} of the model.
The first step for a systematic construction of a flat output is commonly the elimination of the control inputs (see e.g.\ \cite{Schlacher:ConstructionOfFlatOutputs}) i.e.\ $\tuple{\lambda}^{\text{ZeroError}}=\tuple{0}$.

\paragraph{Small error.}
Assume that we a small error $\LinErrorCoord = \kinMat^+(\sysCoordR)(\sysCoord-\sysCoordR)$ to a constant reference $\sysVelR=\tuple{0}$ as already considered in \autoref{sec:CtrlLinearization}.
Then the model and the closed loop template may be approximated by
\begin{subequations}
\begin{align}
 \sysInertiaMatLin \LinErrorCoorddd + \sysDissMatLin \LinErrorCoordd + \sysStiffMatLin \LinErrorCoord &= \sysInputMat(\sysCoordR) \Delta \tuple{u},
\\
 \sysInertiaMatCLin \LinErrorCoorddd + \sysDissMatCLin \LinErrorCoordd + \sysStiffMatCLin \LinErrorCoord &= \tuple{0}
\end{align}
\end{subequations}
and the matching force $\tuple{\lambda}^{\text{SmallError}}$ for this special case is
\begin{align}\label{eq:MatchingForceLin}
 \tuple{\lambda}^{\text{SmallError}} &= (\sysInputMatLComp(\sysCoordR))^\top \big( \sysInertiaMatLin \sysInertiaMatCLin^{-1} (\sysDissMatCLin \LinErrorCoordd + \sysStiffMatCLin \LinErrorCoord) - (\sysDissMatLin \LinErrorCoordd + \sysStiffMatLin \LinErrorCoord) \big)
\nonumber\\
 &= \underbrace{(\sysInputMatLComp(\sysCoordR))^\top \big( \sysInertiaMatLin \sysInertiaMatCLin^{-1} \sysDissMatCLin - \sysDissMatLin\big)}_{\mat{\Lambda}_{\sysDissMat}} \LinErrorCoordd
  + \underbrace{(\sysInputMatLComp(\sysCoordR))^\top \big( \sysInertiaMatLin \sysInertiaMatCLin^{-1} \sysStiffMatCLin - \sysStiffMatLin\big)}_{\mat{\Lambda}_{\sysStiffMat}} \LinErrorCoord
\end{align}
As $\LinErrorCoord$ and $\LinErrorCoordd$ can be arbitrary, the matrices $\mat{\Lambda}_{\sysStiffMat}$ and $\mat{\Lambda}_{\sysDissMat}$ have to vanish, for $\tuple{\lambda}^{\text{SmallError}}$ to vanish.
For the following examples it will turn out that we can always find suitable parameters within $\sysInertiaMatCLin$, $\sysDissMatCLin$ and $\sysStiffMatCLin$ such that $\mat{\Lambda}_{\sysStiffMat} = \mat{\Lambda}_{\sysDissMat} = \mat{0}$.
Thus ensuring that at least the first order approximation of the actual matching force $\tuple{\lambda}$ vanishes.

\subsection{Systems with input constraints}
In most control systems the control inputs $\sysInput$ can not take arbitrary values, but are constrained like e.g. $-\sysInputCoeff{a}^{\idxMax} \leq \sysInputCoeff{a} \leq \sysInputCoeff{a}^{\idxMax}, a=1,\ldots,\numInputs$ due to practical limitations.
In general we assume that the constraints can be written as $\sysInputConstMat \sysInput \leq \sysInputConstVec$ where the inequality is understood componentwise and the resulting set $\mathbb{U} = \{ \sysInput \in \RealNum^\numInputs \, | \, \sysInputConstMat \sysInput \leq \sysInputConstVec \}$ is assumed to be convex.

Here we can use just the same arguments as in \autoref{sec:CtrlUnderactuatedOptim} to motivate a control law defined by the solution of the optimization problem
\begin{align}\label{eq:MinProbInputConstraints}
 \begin{array}{rl}
  \text{minimize} & \GaussianConstraintC \, = \, \tfrac{1}{2} \norm[\mat{H}]{\sysInput - \sysInput_0}^2 + \GaussianConstraintC_0
  \\[1ex]
  \text{subject to} &\sysInputConstMat \sysInput \leq \sysInputConstVec, \ \sysInput \in \RealNum^\numInputs 
 \end{array}
\end{align}
with $\mat{H}$ and $\sysInput_0$ defined in \eqref{eq:GaussianUnderactuated}.
Given that $\mat{H} \in \SymMatP(\numInputs)$ is positive definite and $\mathbb{U}$ is convex, this problem has a unique solution, though it usually has to be computed numerically.
For the following simulation results the \textsc{Matlab} function \texttt{quadprog} was used and a C++ implementation of the Active-Set algorithm from \cite[Algorithm 16.3]{Nocedal:NumericalOptimization} was used for the real-time implementation on the Multicopters.


