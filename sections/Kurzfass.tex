\thispagestyle{plain}

% \section*{Kurzfassung}
% Der theoretische Teil dieser Arbeit befasst sich mit der Modellierung und Folgeregelung von Starrkörpersystemen.
% Ein Starrkörpersystem besteht aus miteinander verkoppelten starren Körpern, welche wiederum als Menge von speziell verkoppelten Massepunkten angesehen werden können.
% 
% Das erste Kapitel diskutiert allgemeine Systeme von verkoppelten Massepunkten.
% Ziel ist die Herleitung von Energien und Bewegungsgelichungen und deren Zusammenhang.
% Im Kontrast zum Großteil der Literator wird hier eine recht allgemeine Parametrierung in Positions- und Geschwindigkeitskoordinaten betrachtet.
% 
% Im zweiten Kapitel wird diese Theorie auf die spezielleren Fälle eines einzelnen freien Starrkörperes und allgemeiner Starrkörpersysteme angewendet.
% Als Resultat ergeben sich in kompakte Formulierungen für die Starrkörperenergien als auch für die Bewegungsgleichungen, welche in dieser Form wohl noch nicht in der Literatur etabliert sind.
% 
% Im dritten Kapitel wird ein Ansatz zur Folgeregelung von Starrköpersystemen durch statische Zustandsrückführung präsentiert.
% Zunächst wird ein, basierend auf den Erkentnissen der Modellierung, allgemeines, aber geometrisch passendes Wunschverhalten definiert.
% Das Regelgesetz ergibt sich aus der Minimierung der Differenz von realisierbarer Beschleunigung und Wunschbeschleunigung.
% Die Performance dieses Ansatzes wird an meheren Beispielen und Simulationsergebnissen diskutiert.
% 
% Zur praktischen Evaluation des Reglerkonzepts werden die am LSR entwickelten Tri- und Quadcopter verwendet.
% Die Performance des realisierten Regelkonzepts wird an meherern aerobatischen Beispielmaneuvern demonstriert.
% 
% \clearpage

\section*{Abstract}
Many machines, vehicles and robots may be modeled as rigid body systems, i.e.\ a number of interconnected, undeformable bodies subject to inertia, gravity, and other forces.
Energy-based methods for derivation of their equations of motion, like the Lagrange formalism, are standard in engineering education and well established in the dedicated literature.
These algorithms commonly rely on the use of a minimal set of generalized coordinates.
This is appropriate for many applications, e.g. machines containing only one-dimensional joints.
For systems whose configuration space is nonlinear, e.g. mobile robots whose configiguration space contains the rigid body attitude, the use of minimal coordinates necessarily leads to singularities.
From the point of view of differential geometry, this is a well known fact.

This work resolves this problem by the use of (possibly) redundant configuration coordinates and (possibly) nonholonomic velocity coordinates.
The second chapter reviews several established formalisms of analytical mechanics and states them in terms of these more general coordinates.
The third chapter applies these results to rigid body systems.
Though inertia is the crucial part of the dynamics, this work also investigates dissipation and stiffness.
Finally, it presents an algorithm for the derivation of global equations of motion of general rigid body systems.

The literature states computed-torque is a standard approach for tracking control of fully actuated mechanical systems.
However, this recipe relies on minimal coordinates and consequently suffers from the problems mentioned above.
There is no established ``standard'' approach for the control of underactuated systems.

This work presents three sightly different algorithms for tracking control of general rigid body systems by means of static state feedback.
These essentially minimize the distance between the actual realizable acceleration of the model and a desired acceleration computed from a stable prototype system.
The prototype system shares the geometry and kinematics of the actual model, but may have different constitutive properties (inertia, damping, stiffness).
The resulting control law can be computed globally and explicitly for any rigid body system.
The resulting closed loop system (which may differ from the prototype in the underactuated case), is invariant to the chosen coordinates, i.e.\ its formulation is covariant.
However, so far, there is no general proof of stability.
The performance of the proposed approaches are discussed on several examples and simulation results.

The last chapter of this work discusses the experimental realization of the control approach to two small UAVs.
The performance is demonstrated on tracking control for several aerobatic maneuvers.

