\section{An important function on the special orthogonal group}\label{sec:AppendixAttitudePotential}
\paragraph{Motivation.}
\begin{itemize}
\item In the context of satellite navigation, the following problem \cite{Wahba:WahbaProblem} arose, now commonly called \textit{Wahba's problem}: given $\tuple{u}_k, \tuple{v}_k \in \RealNum^3$, find $\R \in \SpecialOrthogonalGroup(3)$ that minimizes
\begin{multline}
 \potentialEnergy_1(\R) = \sum_k \norm{\tuple{u}_k - \R \tuple{v}_k}^2
 = \sum_k \big( \norm{\tuple{u}_k}^2 + \norm{\tuple{v}_k}^2 - \sProd{\tuple{u}_k}{\R \tuple{v}_k} \big)
\\
 = \underbrace{\sum_k \big( \norm{\tuple{u}_k}^2 + \norm{\tuple{v}_k}^2 \big)}_{\const} - \tr\big( \R \underbrace{\sum_k \tuple{v}_k \tuple{u}_k^\top}_{\mat{P}_1} \big).
\end{multline}

\item In \cite{Koditschek:TotalEnergy} the following function with parameters $\mat{K}\in\SymMatP(3)$ and $\RR\in\SpecialOrthogonalGroup(3)$ is called a \textit{navigation function on} $\SpecialOrthogonalGroup(3)$:
\begin{align}
 \potentialEnergy_2(\R) = \tr\big(\mat{K}(\idMat[3] - \RR^\top\R)\big) = \underbrace{\tr\mat{K}}_{\const} - \tr(\underbrace{\mat{K}\RR^\top}_{\mat{P}_2}\R).
\end{align}

\item Using the metric from \eqref{eq:DefMatrixMetric} with a weight $\mat{K}\in\SymMatP(3)$ we may ask for the rotation matrix $\R \in \SpecialOrthogonalGroup(3)$ which is closest to a given matrix $\mat{Q}\in\RealNum^{3\times3}$, i.e.\ which minimizes
\begin{multline}
 \potentialEnergy_3(\R) = \tfrac{1}{2} \metricSq[\mat{K}]{\mat{Q}}{\R} 
 = \tfrac{1}{2} \tr\big( (\mat{Q}-\R)^\top \mat{K} (\mat{Q}-\R) \big)
\\
 = \underbrace{\tfrac{1}{2} \tr\big( \mat{Q}^\top \mat{K} \mat{Q} + \mat{K} \big)}_{\const} - \tr\big( \underbrace{\mat{Q}^\top \mat{K}}_{\mat{P}_3} \R \big)
\end{multline}
\end{itemize}

Each of these problems essentially asks for an $\R\in\SpecialOrthogonalGroup(3)$ which minimizes $\potentialEnergy = -\tr(\mat{P} \R)$ for some given $\mat{P}\in\RealNum^{3\times3}$.

Solutions to Wahba's problem are given in \cite{Davenport:QMethod} using attitude quaternions and in \cite{Kabsch:SSVD} using the singular value decomposition.
A proof of a unique minimum in \cite{Bullo:TrackingAutomatica} relies on $\mat{P}$ having distinct singular values.
Within this work we will encounter similar problems.
In the following we extend theses results in that, we do not impose assumptions on $\mat{P}$ and are also interested other extrema of $\potentialEnergy$.

\paragraph{Problem definition.}
Given a matrix $\mat{P} \in \RealNum^{3\times3}$, what are the extrema and their nature of the function
\begin{align}
 \potentialEnergy: \SpecialOrthogonalGroup(3) \rightarrow \RealNum \, : \, \R \mapsto -\tr(\mat{P} \R).
\end{align}


\paragraph{Coordinate transformation.}
Let $\mat{P} = \mat{X} \mat{\varSigma} \mat{Y}^\top$ be a singular value decomposition with $\mat{X}, \mat{Y} \in \OrthogonalGroup(3)$ and $\mat{\varSigma} = \diag(\sigma_1, \sigma_2, \sigma_3)$, $\sigma_1 \geq \sigma_2 \geq \sigma_3 \geq 0$.
As proposed in \cite{Kabsch:SSVD}, define 
\begin{subequations}\label{eq:DefSSVD}
\begin{align}
 \bar{\mat{X}} &= \mat{X} \diag(1, 1, \det\mat{X}) \in \SpecialOrthogonalGroup(3),
\\
 \bar{\mat{Y}} &= \mat{Y} \diag(1, 1, \det\mat{Y}) \in \SpecialOrthogonalGroup(3),
\\
 \bar{\mat{\varSigma}} &= \diag(\sigma_1, \sigma_2, \bar{\sigma}_3), \ \bar{\sigma}_3 = \det\mat{X}\det\mat{Y} \sigma_3 .
\end{align} 
\end{subequations}
This means flipping the direction of the eigenvectors corresponding to the smallest singular value $\sigma_3$, if the corresponding matrix $\mat{X}, \mat{Y}$ contains a reflection.
This yields a decomposition $\mat{P} = \bar{\mat{X}} \bar{\mat{\varSigma}} \bar{\mat{Y}}^\top$ with proper rotations.
Using the cyclic permutation property of the trace we get
\begin{align}\label{eq:AttitudePotentialTrafo}
 \potentialEnergy(\R) = -\tr( \bar{\mat{X}} \bar{\mat{\varSigma}} \bar{\mat{Y}}^\top \R) = -\tr( \bar{\mat{\varSigma}} \underbrace{\bar{\mat{Y}}^\top \R \bar{\mat{X}}}_{\bar{\R}}) =: \bar{\potentialEnergy}(\bar{\R})
\end{align}
Since the SVD is not unique in general, the transformed function $\bar{\potentialEnergy}$ is neither.
However, since the coordinate transformation $\R = \bar{\mat{Y}} \bar{\R} \bar{\mat{X}}^\top$ is bijective, no information is lost here.
%The SVD is also the most robust approach for the numerical solution of the problem \cite{Markley:Wahba}.

\paragraph{Critical points.}
Using the operators defined above, we may formulate the differential of the transformed function as
\begin{align}
 \differential \bar{\potentialEnergy}(\bar{\R}) &= \veeTwoOp(\bar{\mat{\varSigma}} \bar{\R}).
\end{align}
For a critical point $\bar{\R}_0 : \differential \bar{\potentialEnergy}(\bar{\R}_0)=\tuple{0}$ we need the matrix $\bar{\mat{\varSigma}} \bar{\R}_0$ to be symmetric.
As $\bar{\mat{\varSigma}}$ is diagonal, this implies that $\bar{\R}_0$ must be symmetric.
The Hessian at a critical point is
\begin{align}
 \differential^2 \bar{\potentialEnergy}(\bar{\R}_0) &= \veeMatOp(\bar{\mat{\varSigma}} \bar{\R}_0).
\end{align}

For the following it will be useful to substitute the entries/eigenvalues of $\veeMatOp(\bar{\mat{\varSigma}}) = \diag(\lambda_1, \lambda_2, \lambda_3) = \mat{\Lambda}$ as
\begin{align}
 \left.\begin{matrix}[1.2]
 \lambda_1 = \sigma_2 + \bar{\sigma}_3, \\
 \lambda_2 = \bar{\sigma}_3 + \sigma_1, \\
 \lambda_3 = \sigma_1 + \sigma_2\hphantom{,}  
 \end{matrix}\right\}
\qquad\Leftrightarrow \qquad
 \left\{\begin{matrix}[1.2]
 \sigma_1 = \tfrac{1}{2} (\lambda_2 + \lambda_3 - \lambda_1), \\
 \sigma_2 = \tfrac{1}{2} (\lambda_3 + \lambda_1 - \lambda_2), \\
 \bar{\sigma}_3 = \tfrac{1}{2} (\lambda_1 + \lambda_2 - \lambda_3)\hphantom{,}  
 \end{matrix}\right.
\end{align} 
Its crucial to notice that $\sigma_1 \geq \sigma_2 \geq |\bar{\sigma}_3| \geq 0$ implies $\lambda_3 \geq \lambda_2 \geq \lambda_1 \geq 0$.
Depending on the constellation of the eigenvalues we have the following critical points:
\begin{subequations}
\begin{itemize}
\item Distinct eigenvalues: $\lambda_3 > \lambda_2 > \lambda_1 > 0$: We have the critical points
\begin{align}
 &\bar{\R}_{0} = \idMat[3] \ :
\nonumber\\
 &\quad
 \potentialEnergy(\bar{\R}_{0}) = -\tfrac{\lambda_1}{2} - \tfrac{\lambda_2}{2} - \tfrac{\lambda_3}{2}, 
 \quad
 \eig(\differential^2 \bar{\potentialEnergy}(\bar{\R}_{0})) = \{ \lambda_3, \lambda_2, \lambda_1 \}
\\[.5ex]
 &\bar{\R}_{1} = \diag(1,-1,-1) \ :
\nonumber\\
 &\quad
 \potentialEnergy(\bar{\R}_{1}) = \tfrac{3\lambda_1 - \lambda_2 - \lambda_3}{2}, 
 \quad
 \eig(\differential^2 \bar{\potentialEnergy}(\bar{\R}_{1})) = \{ \lambda_3-\lambda_1, \lambda_2-\lambda_1, -\lambda_1 \}
\\[.5ex]
 &\bar{\R}_{2} = \diag(-1,1,-1) \ :
\nonumber\\
 &\quad
 \potentialEnergy(\bar{\R}_{2}) = \tfrac{3\lambda_2 - \lambda_1 - \lambda_3}{2}, 
 \quad
 \eig(\differential^2 \bar{\potentialEnergy}(\bar{\R}_{2})) = \{ \lambda_3-\lambda_2, \lambda_1-\lambda_2, -\lambda_2 \}
\\[.5ex]
 &\bar{\R}_{3} = \diag(-1,-1,1) \ :
\nonumber\\
 &\quad
 \potentialEnergy(\bar{\R}_{3}) = \tfrac{3\lambda_3 - \lambda_1 - \lambda_2}{2}, 
 \quad
 \eig(\differential^2 \bar{\potentialEnergy}(\bar{\R}_{3})) = \{ \lambda_2-\lambda_3, \lambda_1-\lambda_3, -\lambda_3 \}
\end{align}
so $\bar{\R}_{0}$ is a minimum, $\bar{\R}_{1}$ and $\bar{\R}_{2}$ are saddle points, and $\bar{\R}_{3}$ is a maximum.

\item Double eigenvalue: $\lambda_3 > \lambda_2 = \lambda_1 > 0$: We have a minimum at $\bar{\R}_{0}$, a maximum at $\bar{\R}_{3}$ and a saddle on the circular manifold 
\begin{align}
 \bar{\R}_{4} &= \begin{bmatrix} -c & s & 0 \\ s & c & 0 \\ 0 & 0 & -1 \end{bmatrix}, \ c^2+s^2=1 \ : 
\nonumber\\
 &\quad
 \potentialEnergy(\bar{\R}_{4}) = \lambda_1 - \tfrac{\lambda_3}{2}, 
 \quad
 \eig(\differential^2 \bar{\potentialEnergy}(\bar{\R}_{4})) = \{ \lambda_3-\lambda_1, 0, -\lambda_1 \}
\end{align}
which includes the points $\bar{\R}_{1}$ and $\bar{\R}_{2}$.

\item Double eigenvalue: $\lambda_3 = \lambda_2 > \lambda_1 > 0$: Analog to above we have a minimum at $\bar{\R}_{0}$, a saddle at $\bar{\R}_{1}$ and a maximum on the circular manifold 
\begin{align}
 \bar{\R}_{5} &= \begin{bmatrix} -1 & 0 & 0 \\ 0 & c & s \\ 0 & s & -c \end{bmatrix}, \ c^2+s^2=1 \ : 
\nonumber\\
 &\quad
 \potentialEnergy(\bar{\R}_{5}) = \lambda_2 - \tfrac{\lambda_1}{2}, 
 \quad
 \eig(\differential^2 \bar{\potentialEnergy}(\bar{\R}_{5})) = \{ 0, \lambda_1-\lambda_2, -\lambda_2 \}
\end{align}
which includes the points $\bar{\R}_{2}$ and $\bar{\R}_{3}$.

\item Triple eigenvalue: $\lambda_3 = \lambda_2 = \lambda_1 > 0$: Minimum at $\bar{\R}_{0}$ and a maximum on the spherical manifold 
\begin{align}
 &\bar{\R}_{6} = \begin{bmatrix} \quatx^2 - \quaty^2 - \quatz^2 & 2\quatx\quaty & 2\quatx\quatz \\ 2\quatx\quaty & \quaty^2-\quatx^2+\quatz^2 & 2\quaty\quatz \\ 2\quatx\quatz & 2\quaty\quatz & \quatz^2-\quatx^2-\quaty^2 \end{bmatrix}, \ \quatx^2+\quaty^2+\quatz^2=1 :
\nonumber\\
 &\quad
 \potentialEnergy(\bar{\R}_{6}) = \tfrac{\lambda_1}{2},
 \quad
 \eig(\differential^2 \bar{\potentialEnergy}(\bar{\R}_{6})) = \{ 0, 0, -\lambda_1 \}
\end{align}
which includes the points $\bar{\R}_{1}$, $\bar{\R}_{2}$ and $\bar{\R}_{3}$ and the circles $\bar{\R}_4$ and $\bar{\R}_5$.
It corresponds to a $180^\circ$ rotation about an arbitrary axis $[\quatx,\quaty,\quatz]^\top \in \Sphere^2$.

\item One zero eigenvalue: $\lambda_3 > \lambda_2 > \lambda_1 = 0$: We have a minimum on the circular manifold
\begin{align}
 \bar{\R}_{7} &= \begin{bmatrix} 1 & 0 & 0 \\ 0 & c & -s \\ 0 & s & c \end{bmatrix}, \ c^2+s^2=1 \ : 
\nonumber\\
 \potentialEnergy(\bar{\R}_{7}) &= -\tfrac{\lambda_2+\lambda_3}{2}, \qquad
 \eig(\differential^2 \bar{\potentialEnergy}(\bar{\R}_{7})) = \{ \lambda_3, \lambda_2, 0\}
\end{align}
which includes $\bar{\R}_{0}$ and $\bar{\R}_{1}$.
Furthermore we have a saddle point at $\bar{\R}_{2}$ and a maximum at $\bar{\R}_{3}$.

\item Double eigenvalue and zero eigenvalue: $\lambda_3 = \lambda_2 > \lambda_1 = 0$: We have a minimum on $\bar{\R}_{7}$ and a maximum on $\bar{\R}_{5}$.

\item Two zero eigenvalues: $\lambda_3 > \lambda_2 = \lambda_1 = 0$: We have a minimum on the spherical manifold
\begin{align}
 \bar{\R}_{8} &= \begin{bmatrix} \quatw^2 + \quatx^2 - \quaty^2 & 2\quatx\quaty & 2\quatw\quaty \\ 2\quatx\quaty & \quatw^2-\quatx^2+\quaty^2 & -2\quatw\quatx \\ -2\quatw\quatx & 2\quatw\quatx & \quatw^2-\quatx^2-\quaty^2 \end{bmatrix}, \ \quatw^2+\quatx^2+\quaty^2=1 :
\nonumber\\
 \potentialEnergy(\bar{\R}_{8}) &= -\tfrac{\lambda_3}{2}, \qquad
 \eig(\differential^2 \bar{\potentialEnergy}(\bar{\R}_{8})) = \{ \lambda_3, 0, 0\}
\end{align}
which includes $\bar{\R}_{0}$, $\bar{\R}_{1}$ and $\bar{\R}_{2}$ and corresponds to an arbitrary rotation about an axis $[\quatx, \quaty, 0]^\top$.
Furthermore we have a maximum at $\bar{\R}_{3}$.

\item All zero eigenvalues: $\lambda_3 = \lambda_2 = \lambda_1 = 0$: for this we have $\bar{\mat{\varSigma}} = \mat{P} = \mat{0}$ and the function is $\potentialEnergy = 0$.

\end{itemize}
\end{subequations}
We may conclude that the function $\bar{\potentialEnergy}$ has a minimum at $\bar{\R}_0 = \idMat[3]$ and a maximum at $\bar{\R}_3 = \diag(-1,-1,1)$, though they may not be strict.
The minimum is strict if, and only if, $\lambda_1 > 0$. 
The maximum is strict if, and only if, $\lambda_3 > \lambda_2$.

It should also be noted that the results of this paragraph would be more ``symmetric'' if we would not have required the descending order of the singular values $\sigma_i$.
This did however reduce the number of cases to distinguish.

\paragraph{Original coordinates.}
Reversing the coordinate transformation from \eqref{eq:AttitudePotentialTrafo} we may conclude:
The original function $\potentialEnergy$ has a minimum at $\R_0 = \bar{\mat{Y}} \bar{\mat{X}}^\top$.
The Hessian at the minimum may be written as
\begin{align}
 \differential^2 \potentialEnergy (\R_0) 
 = \veeMatOp(\mat{P}\R_0) 
 = \veeMatOp(\bar{\mat{X}} \bar{\mat{\varSigma}} \bar{\mat{Y}}^\top \bar{\mat{Y}} \bar{\mat{X}}^\top)
 = \bar{\mat{X}} \veeMatOp(\bar{\mat{\varSigma}}) \bar{\mat{X}}^\top\!
 = \bar{\mat{X}} \mat{\Lambda} \bar{\mat{X}}^\top\!
 =: \mat{K}
\end{align}
with $\mat{K} \in \SymMatSP(3)$.
The minimum $\R_0$ is strict if, and only if, $\lambda_1 > 0$ which is equivalent to $\mat{K} > 0$.

\paragraph{Special polar decomposition.}
Using the \textit{special singular value decomposition} from \eqref{eq:DefSSVD} we may write
\begin{align}
 \mat{M} = \bar{\mat{X}} \bar{\mat{\varSigma}} \bar{\mat{Y}}^\top
 = \bar{\mat{X}} \bar{\mat{Y}}^\top \bar{\mat{Y}} \bar{\mat{\varSigma}} \bar{\mat{Y}}^\top
 = \underbrace{\bar{\mat{X}} \bar{\mat{Y}}^\top}_{\R} \wedMatOp\big(\underbrace{\bar{\mat{Y}} \veeMatOp(\bar{\mat{\varSigma}}) \bar{\mat{Y}}^\top}_{\mat{K}}\big).
\end{align}
From this we may conclude:
For any matrix $\mat{M} \in\RealNum^{3\times3}$ there is a matrix $\R \in \SpecialOrthogonalGroup(3)$ and a unique, positive semidefinite matrix $\mat{K} \in \SymMatSP(3)$, such that $\mat{M} = \R \wedMatOp(\mat{K})$.
The matrix $\R$ is unique if, and only if, the matrix $\mat{K}$ is positive definite.
Within this work, this will be called the \textit{special polar decomposition}.
Note that the matrix $\wedMatOp(\mat{K})$ is, in general, indefinite.


% \paragraph{Prototype for a positive definite function.}
% Subtracting the minimal value $\potentialEnergy(\R_0)$ from the function we obtain
% \begin{subequations}
% \begin{align}
%  \hat{\potentialEnergy}(\R) &= \potentialEnergy(\R) - \potentialEnergy(\R_0)
% \nonumber\\
%  &= -\tr(\bar{\mat{X}} \wedMatOp(\mat{\Lambda}) \bar{\mat{Y}}^\top \R) + \tfrac{1}{2}\tr(\mat{\Lambda})
% \nonumber\\
%  &= -\tr(\bar{\mat{X}} \wedMatOp(\mat{\Lambda}) \bar{\mat{X}}^\top \R_0^\top \R) + \tr(\wedMatOp(\mat{K}))
% \nonumber\\
% \label{eq:NavFunctionSO3}
%  &= \tr\big(\wedMatOp(\mat{K})(\idMat[3] - \R_0^\top \R)\big)
% \\
%  &= \tfrac{1}{2} \tr\big(\wedMatOp(\mat{K})(\R - \R_0)^\top (\R - \R_0)\big)
% \nonumber\\
%  &= \tfrac{1}{2}\norm[\wedMatOp(\mat{K})]{\R-\R_0}^2.
% \end{align}
% \end{subequations}
% We have the properties
% \begin{subequations}
% \begin{align}
%  \mat{K} &\geq 0& 
%  &\Leftrightarrow&
%  \hat{\potentialEnergy}(\R) &\geq 0
% \\
%  \mat{K} &> 0& 
%  &\Leftrightarrow&
%  \hat{\potentialEnergy}(\R) &\geq 0 \ \wedge \ \hat{\potentialEnergy}(\R) = 0 \, \Leftrightarrow \, \R = \R_0.
% \end{align}
% \end{subequations}
% The form \eqref{eq:NavFunctionSO3} is called the \textit{navigation function} for $\SpecialOrthogonalGroup(3)$ in \cite{Koditschek:TotalEnergy}.
% From its properties $\hat{\potentialEnergy}$ is an $\SpecialOrthogonalGroup(3)$ analog to $\tfrac{1}{2}(\tuple{x}-\tuple{x}_0)^\top\mat{K}(\tuple{x}-\tuple{x}_0), \tuple{x},\tuple{x}_0\in\RealNum^3$.
