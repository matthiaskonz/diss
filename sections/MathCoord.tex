\section{Coordinates}\label{sec:MathCoordinates}
The example of the rigid body orientation showed that, though its degree of freedom is $\dimConfigSpace=3$, it cannot be \textit{globally} parameterized by 3 coordinates without having singularities.
In other words, the configuration space of the rigid body orientation is not isomorphic to $\RealNum^3$ and is called a nonlinear manifold.

If interested in a global parameterization of a $\dimConfigSpace$ dimensional nonlinear manifold, there are two common approaches:
\begin{enumerate}
 \item Choose a finite number of overlapping local charts with with \textit{minimal} coordinates $\genCoord \in \RealNum^\dimConfigSpace$, e.g. four distinct sets of Euler angles for the rigid body attitude \cite{Grafarend:AtlasSO3}.
 As this is the common way of defining a smooth manifold, this is always possible.
 \item Choose one parameterization with \textit{redundant} coordinates $\sysCoord \in \RealNum^{\numCoord}$, i.e. coordinates that are constrained by smooth equations of the form $\geoConstraint(\sysCoord) = \tuple{0}$.
 E.g. the coefficients of the rotation matrix as done in \eqref{eq:ExampleEulerEq}.
 \textit{Whitney embedding theorem} states that this is always possible with at least $\numCoord = 2\dimConfigSpace$ coordinates.
\end{enumerate}
Both approaches have benefits and drawbacks depending on the application, but the first approach and the use of minimal coordinates is far more dominant in the literature.
This work utilizes the second approach.

%Another way for a global parameterization of nonlinear configuration manifolds is motivated from the \textit{Whitney embedding theorem} (see e.g.\ \cite[Theo.\,6.14]{Lee:SmoothManifolds}), that states: \textit{Every smooth manifold of dimension $\dimConfigSpace$ can be smoothly embedded in the Euclidean space $\RealNum^{2\dimConfigSpace}$.}
%Note that $2\dimConfigSpace$ is a worst case bound, i.e.\ for a particular example a lower dimension for the embedding space might work and a higher dimension is permitted anyway.

\subsection{Redundant configuration coordinates}\label{sec:MathConfigCoord}
In the notation of this work, we use $\numCoord > 0$ coordinates $\sysCoord(t) = [\sysCoordCoeff{1}(t),\ldots,\sysCoordCoeff{\numCoord}(t)]^\top \in \RealNum^\numCoord$ that might be constrained by $\numGeoConst \geq 0$ smooth functions of the form $\geoConstraint(\sysCoord) = [\geoConstraintCoeff{1}(\sysCoord), \ldots, \geoConstraintCoeff{\numGeoConst}(\sysCoord)]^\top = \tuple{0}$.
For $\numGeoConst > 0$ these coordinates are not independent and are commonly called \textit{redundant}.
The set of mutually admissible coordinates is called the configuration space $\configSpace$:
\begin{align}
 \configSpace = \{ \sysCoord \in \RealNum^{\numCoord} \, | \, \geoConstraint(\sysCoord) = \tuple{0} \}.
\end{align}
Assuming that the rank of $\pdiff[\geoConstraint]{\sysCoord}$ is constant, the dimension of the configuration space is
\begin{align}
 \dimConfigSpace = \dim \configSpace = \numCoord - \rank \pdiff[\geoConstraint]{\sysCoord}.
\end{align}For holonomic systems, $\dimConfigSpace$ is also called its degree of freedom.

Whitney embedding theorem (see e.g.\ \cite[Theo.\,6.14]{Lee:SmoothManifolds}) states that: \textit{Every smooth manifold of dimension $\dimConfigSpace$ can be smoothly embedded in the Euclidean space $\RealNum^{2\dimConfigSpace}$.}
The number $2\dimConfigSpace$ is a worst case bound, i.e.\ for a particular example a lower dimension for the embedding space $\RealNum^\numCoord$ might work and a higher dimension is permitted anyway.
For this work, it essentially guarantees the existence of a global parameterization by the set $\configSpace$ for any smooth manifold.


\subsection{Minimal velocity coordinates}\label{sec:secMathMinVelCoord}
For the following it is crucial to note that a geometric constraint is equivalent to its derivative supplemented with a suitable initial condition
\begin{subequations}\label{eq:DiffConstraint}
\begin{align}
 \label{eq:GeoConstraint}
 &&
 \geoConstraintCoeff{\CidxI}(\sysCoord) &= 0
\\
 \label{eq:DiffGeoConstraint}
 &\Leftrightarrow&
 \pdiff[\geoConstraintCoeff{\CidxI}]{\sysCoordCoeff{\GidxI}}(\sysCoord) \sysCoordCoeffd{\GidxI} &= 0,&
 \geoConstraintCoeff{\CidxI}(\sysCoord_0) &= 0
\\
 &\Leftrightarrow&
 \pdiff[\geoConstraintCoeff{\CidxI}]{\sysCoordCoeff{\GidxI}}(\sysCoord) \sysCoordCoeffdd{\GidxI} + \frac{\partial^2 \geoConstraintCoeff{\CidxI}}{\partial \sysCoordCoeff{\GidxI} \partial \sysCoordCoeff{\GidxII}}(\sysCoord) \sysCoordCoeffd{\GidxII} \sysCoordCoeffd{\GidxI} &= 0,&
 \geoConstraintCoeff{\CidxI}(\sysCoord_0) &= 0, \ \pdiff[\geoConstraintCoeff{\CidxI}]{\sysCoordCoeff{\GidxI}}(\sysCoord_0) \sysCoordCoeffd{\GidxI}_0 = 0
\\
 &&
 &\ldots \nonumber
\end{align}
\end{subequations}
where $\sysCoord_0 = \sysCoord(t_0)$.
Even though \eqref{eq:GeoConstraint} might be nonlinear, its derivative \eqref{eq:DiffGeoConstraint} is always \textit{linear} in the velocities $\sysCoordd$.
So here it is reasonable to choose \textit{minimal velocity coordinates}:
Let $\kinMat(\sysCoord) \in \RealNum^{\numCoord\times\dimConfigSpace}$ be a matrix with the properties $\pdiff[\geoConstraint]{\sysCoord} \kinMat = \mat{0}$ and $\rank \kinMat = \dimConfigSpace$.
The first property of $\kinMat(\sysCoord)$ is that these columns of $\kinMat(\sysCoord)$ are orthogonal to the rows of $\pdiff[\geoConstraint]{\sysCoord}(\sysCoord)$.
The second property implies that the columns of $\kinMat(\sysCoord)$ are linearly independent.
So the columns of $\kinMat(\sysCoord)$ can be interpreted as a \textit{basis vectors} for the tangent space $\Tangent[\sysCoord] \configSpace$.
We can capture all allowed velocities $\sysCoordd(t)$ by the minimal velocity coordinates $\sysVel(t) \in \RealNum^{\dimConfigSpace}$ through
\begin{align}\label{eq:MyKinematics}
 \sysCoordd = \kinMat(\sysCoord) \sysVel
\end{align}
This kinematic relation \eqref{eq:MyKinematics} ensures that the time derivative \eqref{eq:DiffGeoConstraint} of the geometric constraint is fulfilled, and consequently the geometric constraint only has to be imposed on the initial condition $\geoConstraint(\sysCoord(t_0)) = \tuple{0}$.

% \fixme{Existence for A? guaranteed for Lie groups: the Lie algebra at the identity can be translated around the manifold by the group operation}
% 
% \fixme{Construction of $\kinMat$ by matrix inversion}

\clearpage

\begin{Example}
Consider a single particle constrained to a circle of radius $\rho$ as illustrated in \autoref{fig:ParticleOnCircle}.

\begin{minipage}{\textwidth}
 \centering
 \input{graphics/ParticleOnCircle.pdf_tex}
 \captionof{figure}{Particle on a circle}
 \label{fig:ParticleOnCircle}
\end{minipage}

We use the its Cartesian position $[\sysCoordCoeff{1}, \sysCoordCoeff{2}]^\top \in \RealNum^2$ constrained by $\geoConstraintCoeff{} = (\sysCoordCoeff{1})^2 + (\sysCoordCoeff{2})^2 - \rho^2 = 0$ as configuration coordinates.
A reasonable choice for the kinematics matrix $\kinMat$ is
\begin{align}
 \underbrace{\big[ 2\sysCoordCoeff{1} \ 2\sysCoordCoeff{2} \big]}_{\pdiff[\geoConstraintCoeff{}]{\sysCoord}}
 \underbrace{\begin{bmatrix} -\sysCoordCoeff{2} \\ \sysCoordCoeff{1} \end{bmatrix}}_{\kinMat}
 = 0
\end{align}
\end{Example}
\vfill
\begin{Example}\label{Example:KinMatSO3}
Consider the motivation example of the rigid body orientation from \autoref{sec:MotivationRigidBodyAttitude}.
Instead of parameterizing the rotation matrix $\R$ by minimal coordinates, we take its 9 coefficients $\sysCoord = [\Rxx, \Ryx, \Rzx, \Rxy, \Ryy, \Rzy, \Rxz, \Ryz, \Rzz]^\top \in \RealNum^{9}$ as configuration coordinates.
The constraints $\R^\top \R = \idMat[3]$ and $\det\R = 1$ read
\begin{align}\label{eq:constraintSO3}
 \geoConstraint(\sysCoord) = 
  \begin{bmatrix}
  (\Rxx)^2 + (\Ryx)^2 + (\Rzx)^2 - 1 \\
  (\Rxy)^2 + (\Ryy)^2 + (\Rzy)^2 - 1 \\
  (\Rxz)^2 + (\Ryz)^2 + (\Rzz)^2 - 1 \\
  \Rxy \Rxz + \Ryy \Ryz + \Rzy \Rzz \\
  \Rxx \Rxz + \Ryx \Ryz + \Rzx \Rzz \\
  \Rxx \Rxy + \Ryx \Ryy + \Rzx \Rzy \\
  \Rxx \Ryy \Rzz + \Rxy \Ryz \Rzx + \Rxz \Ryx \Rzy - \Rxx \Ryz \Rzy - \Rxy \Ryx \Rzz - \Rxz \Ryy \Rzx - 1
 \end{bmatrix}
 = \tuple{0}.
\end{align}
The 9 conditions $\R^\top \R = \idMat[3]$ yields due to symmetry only 6 constraints and already imply $\det\R = \pm 1$.
Since the determinant is a smooth function, the corresponding manifold must consist of two disjoint components, one with $\det\R = +1$ (proper rotations) and one with $\det\R=-1$ (rotations with reflection).
So the additional constraint $\det\R = +1$ does not change the dimension of the configuration space.
Formally this means $\rank\pdiff[\geoConstraint]{\sysCoord} = 6$ and consequently $\dim\configSpace = 9-6 = 3$.
A kinematics matrix with $\pdiff[\geoConstraint]{\sysCoord} \kinMat = \mat{0}$ and $\rank\kinMat = 3$ is given by
\begin{align}\label{eq:KinMatSO3}
 \mat{A}(\sysCoord) =
 \begin{bmatrix}
  0 & -\Rxz & \Rxy \\
  0 & -\Ryz & \Ryy \\
  0 & -\Rzz & \Rzy \\
  \Rxz & 0 & -\Rxx \\
  \Ryz & 0 & -\Ryx \\
  \Rzz & 0 & -\Rzx \\
  -\Rxy & \Rxx & 0 \\
  -\Ryy & \Ryx & 0 \\
  -\Rzy & \Rzx & 0
 \end{bmatrix}.
\end{align}
The resulting kinematic equation $\sysCoordd = \kinMat(\sysCoord) \sysVel$ can be reordered to the matrix equation $\Rd = \R \wedOp(\sysVel)$ by introducing the \textit{wedge operator} defined as
\begin{align}
 \wedOp \begin{bmatrix} \sysVelCoeff{1} \\ \sysVelCoeff{2} \\ \sysVelCoeff{3} \end{bmatrix} = \begin{bmatrix} 0 & -\sysVelCoeff{3} & \sysVelCoeff{2} \\ \sysVelCoeff{3} & 0 & -\sysVelCoeff{1} \\ -\sysVelCoeff{2} & \sysVelCoeff{1} & 0 \end{bmatrix}.
\end{align}
\end{Example}

\paragraph{Pseudoinverse.}
\cite[Theo.\ 1]{Penrose:Pseudoinverse}:
For any matrix $\mat{S}\in\RealNum^{m\times n}$ there exists a unique \textit{(Moore-Penrose) pseudoinverse} $\mat{S}^+ \in \RealNum^{n\times m}$ determined by the following conditions:
\begin{subequations}\label{eq:PenroseConditions}
\begin{align}
 \mat{S} \mat{S}^+ \mat{S} &= \mat{S},
\\
 \mat{S}^+ \mat{S} \mat{S}^+ &= \mat{S}^+,
\\
 (\mat{S} \mat{S}^+)^\top &= \mat{S} \mat{S}^+,
\\
 (\mat{S}^+ \mat{S})^\top &= \mat{S}^+ \mat{S}.
\end{align}
\end{subequations}
If the matrix $\mat{S}$ has linearly independent columns, its pseudoinverse is $\mat{S}^+ = (\mat{S}^\top \mat{S})^{-1} \mat{S}^\top$.
Similarly, if $\mat{S}$ has linearly independent rows, its pseudoinverse is $\mat{S}^+ = \mat{S}^\top (\mat{S} \mat{S}^\top)^{-1}$.
Consequently, if $\mat{S}$ is invertible (independent rows and columns) the pseudoinverse coincides with the inverse $\mat{S}^+ = \mat{S}^{-1}$.

\paragraph{Some identities involving the pseudo-inverse.}
As $\kinMat(\sysCoord) \in \RealNum^{\numCoord\times\dimConfigSpace}$ is has independent columns, its pseudoinverse has the explicit formula $\kinMat^+ = (\kinMat^\top \kinMat)^{-1} \kinMat^\top =: \kinBasisMat$.
Evidently we have $\kinBasisMat \kinMat = \idMat[\dimConfigSpace]$, but, in general, $\kinMat \kinBasisMat \neq \idMat[\numCoord]$.
The matrix $\mat{P} = \kinMat \kinBasisMat$ is the unique orthogonal projector, (i.e.\ $\mat{P}^2 = \mat{P} = \mat{P}^\top$, see e.g.\ \cite[p.\ 38]{Horn:MatrixAnalysis}) from $\RealNum^\numCoord$ to the tangent space $\Tangent[\sysCoord]\configSpace$.

Introduce the shorthand notation $\geoConstraintMat = \pdiff[\geoConstraint]{\sysCoord}$.
As stated above, $\geoConstraintMat(\sysCoord) \in \RealNum^{\numGeoConst\times\numCoord}$ is not required to be full rank.
Consequently, there is no explicit expression for its pseudoinverse $\InvGeoConstraintMat(\sysCoord) = \geoConstraintMat^+(\sysCoord) \in \RealNum^{\numCoord\times\numGeoConst}$.
Nevertheless $\geoConstraintMat^+$ is unique and obeys the Penrose conditions \eqref{eq:PenroseConditions}.
With them its straight forward to show that $\mat{P}^\bot = \InvGeoConstraintMat \geoConstraintMat$ is also an orthogonal projection.

%The condition $\geoConstraintMat\kinMat = \mat{0}$ combined with the Penrose conditions \eqref{eq:PenroseConditions} implies\footnote{$\InvGeoConstraintMat^\top \kinMat = (\InvGeoConstraintMat \geoConstraintMat \InvGeoConstraintMat)^\top \kinMat = \InvGeoConstraintMat^\top (\InvGeoConstraintMat \geoConstraintMat)^\top \kinMat = \InvGeoConstraintMat^\top \InvGeoConstraintMat \geoConstraintMat \kinMat = \mat{0}$} that $\InvGeoConstraintMat^\top \kinMat = \mat{0}$ and $\kinBasisMat^\top \geoConstraintMat = \mat{0}$.
The condition $\geoConstraintMat\kinMat = \mat{0}$ combined with the Penrose conditions \eqref{eq:PenroseConditions} implies $\InvGeoConstraintMat^\top \kinMat = (\InvGeoConstraintMat \geoConstraintMat \InvGeoConstraintMat)^\top \kinMat = \InvGeoConstraintMat^\top (\InvGeoConstraintMat \geoConstraintMat)^\top \kinMat = \InvGeoConstraintMat^\top \InvGeoConstraintMat \geoConstraintMat \kinMat = \mat{0}$ and analog $\kinBasisMat^\top \geoConstraintMat = \mat{0}$.
Supplementing the fact $\rank\InvGeoConstraintMat = \rank\geoConstraintMat = \numCoord - \dimConfigSpace$ we may conclude that the columns of $\InvGeoConstraintMat$ span the complementary space $(\Tangent[\sysCoord]\configSpace)^\bot$ and $\mat{P}^\bot$ is the unique orthogonal projector from $\RealNum^\numCoord$ to $(\Tangent[\sysCoord]\configSpace)^\bot$.
As the projectors are related by $\mat{P}^\bot = \idMat - \mat{P}$ we have the identity
\begin{align}\label{eq:ProjectionIdentity}
 \kinMat \kinBasisMat +\, \InvGeoConstraintMat \geoConstraintMat = \idMat[\numCoord].
\end{align}

The deduction above simplifies significantly if $\geoConstraintMat$ is full rank: Then the columns of $\kinMat$ and $\InvGeoConstraintMat$ form a basis for $\RealNum^\numCoord$ and we can immediately conclude
\begin{subequations}
\begin{align}
 \underbrace{\begin{bmatrix} \kinBasisMat \\ \geoConstraintMat \end{bmatrix}}_{\kinMatSquare^{-1}} \underbrace{[\kinMat \ \InvGeoConstraintMat]}_{\kinMatSquare}
 &= \begin{bmatrix} \kinBasisMat \kinMat & \kinBasisMat\InvGeoConstraintMat\\ \geoConstraintMat\kinMat & \geoConstraintMat \InvGeoConstraintMat \end{bmatrix} 
 = \begin{bmatrix} \idMat[\dimConfigSpace] & \mat{0} \\ \mat{0} & \idMat[\numCoord-\dimConfigSpace] \end{bmatrix} 
\\
 \underbrace{[\kinMat \ \InvGeoConstraintMat]}_{\kinMatSquare}
 \underbrace{\begin{bmatrix} \kinBasisMat \\ \geoConstraintMat \end{bmatrix}}_{\kinMatSquare^{-1}}
 &= \kinMat \kinBasisMat +\, \InvGeoConstraintMat \geoConstraintMat
 = \idMat[\numCoord]
\end{align}
\end{subequations}
These identities will be important for the next sections. 

% In summary: Given a matrix $\geoConstraintMat \in \RealNum^{\numGeoConst\times\numCoord}$ and a matrix $\kinMat \in \RealNum^{\numCoord\times\dimConfigSpace}$ with full column rank and $\geoConstraintMat\kinMat = \mat{0}$ we have
% \begin{subequations}
% \begin{align}
%  \InvGeoConstraintMat &= \geoConstraintMat^+,
% \\
%  \kinBasisMat &= \kinMat^+ = (\kinMat^\top \kinMat)^{-1} \kinMat^\top
% \\
%  \kinBasisMat \kinMat &= \idMat[\dimConfigSpace]
% \\
%  \kinMat \kinBasisMat + \InvGeoConstraintMat \geoConstraintMat &= \idMat[\numCoord].
% \end{align}
% \end{subequations}
