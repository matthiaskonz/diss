\section{Systems of constrained particles}\label{sec:MechConstrainedParticles}

\paragraph{System under consideration.} For this section we consider a system of $\numParticles$ particles under geometric constraints:
The \textit{position} of a particle with respect to a given inertial frame at a given time $t$ is $\particlePos{\PidxI}(t) \in \RealNum^3, \PidxI = 1,\ldots,\numParticles$ and the collection of all particle positions is $\particleCoord = [\particlePos{1}^\top,\ldots,\particlePos{\numParticles}^\top]^\top \in \RealNum^{3\numParticles}$.
\textit{Geometric constraints} on the particles are captured in $\numParticleConstraints \geq 0$ smooth functions of the form $\particleGeoConstraint(\particleCoord) = [\particleGeoConstraintCoeff{1}(\particleCoord), \ldots, \particleGeoConstraintCoeff{\numParticleConstraints}(\particleCoord)]^\top = \tuple{0}$.
The set of all mutually admissible particle positions 
\begin{align}
\particleConfigSpace = \{ \particleCoord \in \RealNum^{3\numParticles} \, | \, \particleGeoConstraint(\particleCoord) = \tuple{0} \} 
\end{align}
is called the \textit{configuration space}.
We require $\pdiff[\particleGeoConstraint]{\particleCoord}(\particleCoord)$ to have a constant, though not necessarily full rank.
%The dimension of the configuration space is $\dim\particleConfigSpace = 3\numParticles - \rank\pdiff[\particleGeoConstraint]{\particleCoord} = \dimConfigSpace$.

\subsection{First principles}
\paragraph{Principle of constraint release.} 
The principle of constraint release (see e.g.\ \cite[sec.\ 32]{Hamel:TheoretischeMechanik} or \cite[sec.\ 6.1]{Lurie:AnalyticalMechanics}) states that the motion of system of geometrically constrained particles is governed by
\begin{align}\label{eq:ConstraintRelease}
 \particleGeoConstraint(\particleCoord) = \tuple{0}, 
\quad
 \particleMass{\PidxI} \particlePosdd{\PidxI} = \particleForceImpressed{\PidxI} + \LagrangeMultCoeff{\CidxI} \pdiff[\particleGeoConstraintCoeff{\CidxI}]{\particlePos{\PidxI}},
\quad \PidxI = 1,\ldots,\numParticles.
\end{align}

\paragraph{Lagrange-d'Alembert principle.} 
For a system of geometrically constrained particles states (e.g.\ \cite[sec.\,1.4]{Goldstein:ClassicalMechanics} or \cite[sec.\,6.3]{Lurie:AnalyticalMechanics}):
\begin{align}\label{eq:DAlembertPrinciple}
 \sumParticlesFull \sProd{\delta \particlePos{\PidxI}}{\particleForceImpressed{\PidxI} - \particleMass{\PidxI}\particlePosdd{\PidxI}} = 0 
\qquad \forall \quad \pdiff[\particleGeoConstraint]{\particleCoord} \delta\particleCoord = \tuple{0}.
\end{align}
The \textit{virtual displacements} $\delta \particlePos{\PidxI}$ are tangents to possible motions:
For particle positions constrained by $\particleGeoConstraint(\particleCoord) = \tuple{0}$ the displacements have to fulfill $\pdiff[\particleGeoConstraint]{\particleCoord} \delta\particleCoord = \tuple{0}$.

\paragraph{Gauß principle.}
Gauss's principle of least constraint was originally described in \cite{Gauss:Principle} in words rather than equations.
Maybe because of this, one finds somewhat different mathematical formulations in more contemporary sources, e.g.\ \cite[sec.\ 7]{Paesler:PrinzipeDerMechanik}, \cite[sec.\,IV.8]{Lanczos:Variational}, \cite[sec.\,2.2]{Bremer:ElasticMultibodyDynamics}. %\cite[sec.\,VII.8]{Hamel:TheoretischeMechanik} %\cite[p.\,82]{UdwadiaKalaba:AnalyticalDynamics}
%For some historical background see \cite[§6.6]{Papastavridis:AnalyticalMechanics}.

For the system given above, Gauss' principle states that the particle accelerations $\particlePosdd{\PidxI}, \PidxI=1,\ldots,\numParticles$ minimize the so-called Gaussian constrain $\GaussianConstraint$:
\begin{align}\label{eq:GaussPrinciple}
 \begin{array}{rl}
  \minOp[\particleCoorddd\in\RealNum^{3\numParticles}] & \GaussianConstraint = \tfrac{1}{2} \displaystyle\sumParticlesFull \particleMass{\PidxI} \norm{\particlePosdd{\PidxI} - \particlePosdd{\PidxI}^{\idxText{f}}}^2 \\
  \text{s.\ t.} & \ddot{\particleGeoConstraint}(\particleCoord, \particleCoordd, \particleCoorddd) = \tuple{0}
 \end{array}
\end{align}
where $\particlePosdd{\PidxI}^{\idxText{f}}$ are the \textit{unconstrained} particle accelerations, i.e.\ Newtons's second law $\particlePosdd{\PidxI}^{\idxText{f}}=\tfrac{\particleForceImpressed{\PidxI}}{\particleMass{\PidxI}}$.
Its crucial to note that the constraint equations $\ddot{\particleGeoConstraint}(\particleCoord, \particleCoordd, \particleCoorddd) = \tuple{0}$ are \textit{linear} in the accelerations $\particleCoorddd$.
Consequently, as stressed in \cite{Gauss:Principle}, the principle \eqref{eq:GaussPrinciple} is a (static) quadratic optimization problem with linear constraints.


% A system of particles with positions $\particlePos[\PidxI], \PidxI=1,\ldots,\numParticles$ and masses $\particleMass{\PidxI}$, constrainted by 
% \begin{align}\label{eq:GaussPrinciple}
%  \begin{array}{rl}
%   \minOp[\particleCoorddd\in\RealNum^{3\numParticles}] & \GaussianConstraint = \tfrac{1}{2} \displaystyle\sumParticlesFull \particleMass{\PidxI} \norm{\particlePosdd{\PidxI} - \particlePosdd{\PidxI}^{\idxText{f}}}^2 \\
%   \text{s.\ t.} & \ddot{\particleGeoConstraint}(\particleCoord, \particleCoordd, \particleCoorddd) = \tuple{0}
%  \end{array}
% \end{align}
% where $\particleCoord = [\particlePos{1}^\top, \ldots,\particlePos{\numParticles}^\top]^\top$ and $\particlePosdd{\PidxI}^{\idxText{f}}$ are the particle accelerations of the unconstrained system and we call $\GaussianConstraint$ the Gaussian constraint.
% Its crucial to note that the constraint equations $\ddot{\particleGeoConstraint}(\particleCoord, \particleCoordd, \particleCoorddd) = \tuple{0}$ are \textit{linear} in the accelerations $\particleCoorddd$.
% Consequently, as stressed in \cite{Gauss:Principle}, the principle \eqref{eq:GaussPrinciple} can be regarded as a (static) quadratic optimization problem with linear constraints. 


\paragraph{Hamilton's principle.}
%E.g. \cite[ch.\,V.1]{Lanczos:Variational}: \textit{Hamilton's principle} states that the trajectories of a mechanical system are such that the action functional $\mathcal{A} = \int_{t_0}^{t_1} \Lagrangian\, \d t$ is stationary in the sense of the calculus of variations \autoref{sec:CalculusOfVariations}.
\cite[p.\,113]{Lanczos:Variational}: \textit{Hamilton's principle states that the motion of a mechanical system occurs in such a way that the action} $\mathcal{A} = \int_{t_0}^{t_1} \Lagrangian\, \d t$ \textit{becomes stationary for arbitrary possible variations of the configuration of the system, provided the initial and final conditions are prescribed}.
\textit{The Lagrangian $\Lagrangian=\kineticEnergy-\potentialEnergy$ is the excess of kinetic energy $\kineticEnergy$ over potential energy $\potentialEnergy$.}
For the system considered here, the kinetic energy is $\kineticEnergy = \tfrac{1}{2} \sumParticlesFull \particleMass{\PidxI} \norm{\particlePosd{\PidxI}}^2$.
The potential energy may have various origins, some of them will be discussed later.

%For this context of mechanics it can be derived from Lagrange-d'Alembert's principle \eqref{eq:DAlembertPrinciple}, see \eg \cite[ch.\,V.1]{Lanczos:Variational}.
%However, the remarkable quality of the principle of stationary action is that it is used interdisciplinary:

The remarkable quality of the principle of stationary action is that it is used interdisciplinary:
With appropriate formulation of the Lagrangian $\Lagrangian$ it has applications in all branches of physics \eg general relativity \cite{Einstein:HamiltonsPrinciple}, electromagnetic field theory \cite[ch.\,4]{Landau:Fields} or optics, the original context of Hamilton \cite[p.\,192]{Klein:EntwicklungDerMathematik}.
Apart form physics, there is optimal control that builds up on essentially the same idea \eg \cite{Bryson:AppliedOptimalControl}.

% \cite[S.\ 233]{Hamel:TheoretischeMechanik} Prinzip der kleinsten Wirkung
% \cite[S.61]{Szabo:HM},
% \cite[p.\ 136]{Shabana:MultibodySystems},
% \cite[p.\,113]{Lanczos:Variational} Hamilton's principle, some history, excellent!
% \cite[p.\,92]{Arnold:MathematicalMethodsOfClassicalMechanics} In mechanics, tangent vectors to the configuration manifold are called virtual variations.

For the context of this work, the statement above is actually not that useful since it does not allow for generic impressed forces $\particleForceImpressed{\PidxI}$.
To mend this one finds a similar statement in e.g.\ \cite[eq.\,12.2.14]{Lurie:AnalyticalMechanics} or \cite[sec.\ I.3]{Szabo:HM}:
\begin{align}\label{eq:HamiltonPrinciple}
 \delta'\mathcal{A} = \int_{t_1}^{t_2} \big( \delta \kineticEnergy - \delta'\mathcal{W} \big) \d t = 0, 
\qquad
 \kineticEnergy = \tfrac{1}{2} \sumParticles \particleMass{\PidxI} \norm{\particlePosd{\PidxI}}^2, \ \delta'\mathcal{W} = \sProd{\delta \particlePos{\PidxI}}{\particleForceImpressed{\PidxI}}.
\end{align}
Here $\delta'\mathcal{W}$ is an variational quantity, but in general, there is no $\mathcal{W}$ that it is the variation of, unless the impressed forces $\particleForceImpressed{\PidxI}$ may be derived from a potential.
Consequently, there is generally no quantity $\mathcal{A}$ which becomes stationary, thus the naming is not suitable here.

\subsection{Coordinates}
\paragraph{Generalized coordinates.} 
In most cases we are not really interested in the motion of the individual particles but rather in the system as a whole.
Using the constraint equations it is possible to capture the configuration of the system by $\dim\particleConfigSpace = 3\numParticles - \rank\pdiff[\particleGeoConstraint]{\particleCoord} = \dimConfigSpace$ coordinates, commonly called \textit{generalized coordinates} and commonly denoted by $\genCoord$.
Its components, in contrast to the Cartesian particle coordinates, may be lengths, angles or some completely generic quantity stressed by the term ``generalized``.
What is more crucial but usually implicit, is that they have to be independent from another to parameterize the $\dimConfigSpace$ dimensional configuration space with its $\dimConfigSpace$ components.
Whenever used in this work, these coordinates are referred to as \textit{minimal} generalized coordinates $\genCoord\in\RealNum^\dimConfigSpace$.

\paragraph{Redundant configuration coordinates and velocity coordinates.} 
As motivated in \autoref{sec:MathCoordinates}, in some cases it can be beneficial to use use a slightly larger number of \textit{redundant} generalized coordinates $\sysCoord(t) \in \configSpace = \{ \sysCoord \in \RealNum^{\numCoord} \, | \, \geoConstraint(\sysCoord) = \tuple{0} \}$ and \textit{minimal} velocity coordinates $\sysVel(t) \in \RealNum^{\dimConfigSpace}$ related by $\sysCoordd = \kinMat \sysVel$.
Of course this includes common minimal parameterization $\sysCoord = \genCoord \in \RealNum^{\dimConfigSpace}$ and $\sysVel=\genCoordd \in \RealNum^{\dimConfigSpace}$ as the special case $\configSpace=\RealNum^{\dimConfigSpace}$ and $\kinMat=\idMat[\dimConfigSpace]$.

\paragraph{Particle parameterization.}
Let the admissible particle positions $\particleCoord \in \particleConfigSpace$ be parameterized $\particlePos{\PidxI} = \particlePos{\PidxI}(\sysCoord, t)$ by possibly redundant coordinates $\sysCoord \in \configSpace$.
This means $\geoConstraint(\sysCoord) = \tuple{0}\,\Rightarrow\,\particleGeoConstraint(\particleCoord(\sysCoord,t)) = \tuple{0}$ and consequently $\sysCoord \in \configSpace \, \Rightarrow \, \particleCoord(\sysCoord) \in \particleConfigSpace$.
The particle velocities and accelerations in terms of the coordinates we have 
\begin{subequations}\label{eq:particleVelAcc}
\begin{align}
 \particlePosd{\PidxI} &= \dirDiff{\LidxI}\particlePos{\PidxI} \sysVelCoeff{\LidxI} + \pdiff[\particlePos{\PidxI}]{t}
\\
 \particlePosdd{\PidxI} &= \dirDiff{\LidxI}\particlePos{\PidxI} \sysVelCoeffd{\LidxI} + \dirDiff{\LidxII}\dirDiff{\LidxI}\particlePos{\PidxI} \sysVelCoeff{\LidxI} \sysVelCoeff{\LidxII}
  + \underbrace{\Big( \dirDiff{\LidxI} \pdiff[\particlePos{\PidxI}]{t} + \pdiff{t}\dirDiff{i}\particlePos{\PidxI} \Big) \sysVelCoeff{\LidxI} + \frac{\partial^2 \particlePos{\PidxI}}{\partial t^2}}_{\particleAccTime{\PidxI}}.
\label{eq:particleAcc}
\end{align}
\end{subequations}

The following relations will be useful for the next steps of this section:
\begin{itemize}
\item From \eqref{eq:particleVelAcc} it is evident that
\begin{align}\label{eq:IdentityDifferentials}
 \dirDiff{\LidxI} \particlePos{\PidxI} = \pdiff[\particlePosd{\PidxI}]{\sysVelCoeff{\LidxI}} = \pdiff[\particlePosdd{\PidxI}]{\sysVelCoeffd{\LidxI}}.
 \qquad \LidxI=1,\ldots,\dimConfigSpace.
\end{align}
\item From
$
 \diff{t} \dirDiff{\LidxI} \particlePos{\PidxI}
 = \dirDiff{\LidxII} \dirDiff{\LidxI} \particlePos{\PidxI} \sysVelCoeff{\LidxII} + \pdiff{t} \dirDiff{\LidxI} \particlePos{\PidxI}
 = (\dirDiff{\LidxI} \dirDiff{\LidxII} - \BoltzSym{\LidxIII}{\LidxI}{\LidxII} \dirDiff{\LidxIII}) \particlePos{\PidxI} \sysVelCoeff{\LidxII} + \dirDiff{\LidxI} \pdiff[\particlePos{\PidxI}]{t}
 = \dirDiff{\LidxI} \particlePosd{\PidxI} - \BoltzSym{\LidxIII}{\LidxI}{\LidxII} \dirDiff{\LidxIII} \particlePos{\PidxI} \sysVelCoeff{\LidxII}
$
with the commutation coefficients $\BoltzSym{\LidxIII}{\LidxI}{\LidxII}$ established in \autoref{sec:CommutationCoeff}, we obtain the commutation relation
\begin{align}\label{eq:CommutationParticleVel}
 \dirDiff{\LidxI} \particlePosd{\PidxI} - \diff{t} \dirDiff{\LidxI} \particlePos{\PidxI} = \BoltzSym{\LidxIII}{\LidxI}{\LidxII} \dirDiff{\LidxIII} \particlePos{\PidxI} \sysVelCoeff{\LidxII},
 \qquad \LidxI=1,\ldots,\dimConfigSpace.
\end{align}
\item As
$
 \tdiff{t} \particleGeoConstraintCoeff{\CidxI}
 = \sumParticles \pdiff[\particleGeoConstraintCoeff{\CidxI}]{\particlePos{\PidxI}} \particlePosd{\PidxI}
 = \sumParticles \pdiff[\particleGeoConstraintCoeff{\CidxI}]{\particlePos{\PidxI}} \big( \dirDiff{\LidxI}\particlePos{\PidxI} \sysVelCoeff{\LidxI} + \pdiff[\particlePos{\PidxI}]{t} \big)
 = 0
$
has to hold for any $\sysVel$ we have
\begin{subequations}
\begin{align}
 \sumParticles \pdiff[\particleGeoConstraintCoeff{\CidxI}]{\particlePos{\PidxI}} \dirDiff{\LidxI}\particlePos{\PidxI} &= 0,
\qquad 
 \CidxI=1,\ldots,\numParticleConstraints, \ \LidxI=1,\ldots,\dimConfigSpace
 \label{eq:ParticleConstraintIdentity}
\\
 \sumParticles \pdiff[\particleGeoConstraintCoeff{\CidxI}]{\particlePos{\PidxI}} \pdiff[\particlePos{\PidxI}]{t} &= 0, 
\qquad 
 \CidxI=1,\ldots,\numParticleConstraints.
\end{align}
\end{subequations}
\end{itemize}

\paragraph{Application to the principle of constraint release.}
Summing up the projections of \eqref{eq:ConstraintRelease} on $\dirDiff{\LidxI}\particlePos{\PidxI}$ eliminates the constraint forces $\LagrangeMult$ due to \eqref{eq:ParticleConstraintIdentity}:
\begin{align}\label{eq:ConstraintReleaseEliminated}
 \underbrace{\sumParticles \sProd{\dirDiff{\LidxI} \particlePos{\PidxI}}{\particleMass{\PidxI} \particlePosdd{\PidxI}}}_{\genForceInertiaCoeff{\LidxI}}
 = \underbrace{\sumParticles \sProd{\dirDiff{\LidxI} \particlePos{\PidxI}}{\particleForceImpressed{\PidxI}}}_{\genForceImpressedCoeff{\LidxI}}
 + \LagrangeMultCoeff{\CidxI} \underbrace{\sumParticles \sProd{\dirDiff{\LidxI} \particlePos{\PidxI}}{\pdiff[\particleGeoConstraintCoeff{\CidxI}]{\particlePos{\PidxI}}}}_{0},
 \quad \LidxI = 1,\ldots,\dimConfigSpace.
\end{align}
We call $\genForceInertia$ the generalized inertia force and $\genForceImpressed$ the generalized applied force.

Parameterizing the particle accelerations $\particlePosdd{\PidxI}$ in terms of the chosen coordinates \eqref{eq:particleAcc} yields 
\begin{align}
 \sumParticles \particleMass{\PidxI} \sProd{\dirDiff{\LidxI} \particlePos{\PidxI}}{\overbrace{\dirDiff{\LidxII} \particlePos{\PidxI} \sysVelCoeffd{\LidxII} + \dirDiff{\LidxIII} \dirDiff{\LidxII} \particlePos{\PidxI} \sysVelCoeff{\LidxIII} \sysVelCoeff{\LidxII} + \particleAccTime{\PidxI}}^{\particlePosdd{\PidxI}}}
 = \sumParticles \sProd{\dirDiff{\LidxI} \particlePos{\PidxI}}{\particleForceImpressed{\PidxI}},
 \quad \LidxI = 1,\ldots,\dimConfigSpace
\nonumber\\
\Leftrightarrow \qquad
 \underbrace{\sumParticles \particleMass{\PidxI} \sProd{\dirDiff{\LidxI} \particlePos{\PidxI}}{\dirDiff{\LidxII} \particlePos{\PidxI}}}_{\sysInertiaMatCoeff{\LidxI\LidxII}} \sysVelCoeffd{\LidxII}
 = \underbrace{\sumParticles \sProd{\dirDiff{\LidxI} \particlePos{\PidxI}}{\particleForceImpressed{\PidxI} - \particleMass{\PidxI} \big(\dirDiff{\LidxIII} \dirDiff{\LidxII} \particlePos{\PidxI} \sysVelCoeff{\LidxIII} \sysVelCoeff{\LidxII} + \particleAccTime{\PidxI}}\big)}_{\sysForceCoeff{\LidxI}},
 \quad \LidxI = 1,\ldots,\dimConfigSpace.
\end{align}
Mathematically, the matrix $\sysInertiaMat$ is symmetric and (assuming $\particleMass{\PidxI} \geq 0$) is positive semidefinite.
Physically, we may assume that $\sysInertiaMat$ is actually positive definite, otherwise there is a degree of freedom with no inertia attached which would rather be an error in modeling.
For this work we will assume that $\sysInertiaMat$ is symmetric positive definite and consequently invertible.
Then the acceleration coordinates can be expressed as 
\begin{align}
 \sysVeld = \sysInertiaMat^{-1} \sysForce.
\end{align}

\paragraph{Application to the Lagrange-d'Alembert principle.}
Analog to the velocity, we may parameterize virtual displacements as $\delta \particlePos{\PidxI} = \dirDiff{\LidxI} \particlePos{\PidxI} \varCoordCoeff{\LidxI}, \PidxI = 1,\ldots,\numParticles$ in terms of \textit{minimal} displacements coordinates $\varCoord\in\RealNum^\dimConfigSpace$.
Plugging this into \eqref{eq:DAlembertPrinciple} we get
\begin{align}\label{eq:DAlembertPrincipleBasis}
 \varCoordCoeff{\LidxI} \sumParticles \sProd{\dirDiff{\LidxI} \particlePos{\PidxI} }{\particleForceImpressed{\PidxI} - \particleMass{\PidxI}\particlePosdd{\PidxI}} = 0 \qquad \forall \quad \varCoord\in\RealNum^\dimConfigSpace.
\end{align}
Since this has to hold for any $\varCoord$, we have the identical result as above \eqref{eq:ConstraintReleaseEliminated}.

\paragraph{Application to the Gauß principle.}
Parameterizing the particle accelerations $\particlePosdd{\PidxI} = \particlePosdd{\PidxI}(\sysCoord, \sysVel, \sysVeld)$ in terms of the chosen coordinates \eqref{eq:particleAcc} in Gauß' principle \eqref{eq:GaussPrinciple} eliminates the constraints.
So it essentially transforms it to an \textit{unconstrained} minimization problem.
The Gaussian constraint now reads
\begin{align}
 \GaussianConstraint &= \tfrac{1}{2} \sumParticles \particleMass{\PidxI} \norm{\overbrace{\dirDiff{\LidxII} \particlePos{\PidxI} \sysVelCoeffd{\LidxII} + \dirDiff{\LidxIII} \dirDiff{\LidxII} \particlePos{\PidxI} \sysVelCoeff{\LidxIII} \sysVelCoeff{\LidxII} + \particleAccTime{\PidxI}}^{\particlePosdd{\PidxI}} - \tfrac{\particleForceImpressed{\PidxI}}{\particleMass{\PidxI}}}^2
\nonumber\\
 &= \tfrac{1}{2} \underbrace{\sumParticles \particleMass{\PidxI} \sProd{\dirDiff{\LidxI} \particlePos{\PidxI}}{\dirDiff{\LidxII} \particlePos{\PidxI}}}_{\sysInertiaMatCoeff{\LidxI\LidxII}} \sysVelCoeffd{\LidxI}\sysVelCoeffd{\LidxII}
 - \underbrace{\sumParticles \sProd{\dirDiff{\LidxI} \particlePos{\PidxI}}{\particleForceImpressed{\PidxI} - \particleMass{\PidxI} (\dirDiff{\LidxIII} \dirDiff{\LidxII} \particlePos{\PidxI} \sysVelCoeff{\LidxIII} \sysVelCoeff{\LidxII} + \particleAccTime{\PidxI})}}_{\sysForceCoeff{\LidxI}} \sysVelCoeffd{\LidxI}
\nonumber\\
 &\qquad + \underbrace{\tfrac{1}{2} \sumParticles \tfrac{1}{\particleMass{\PidxI}}\norm{\particleForceImpressed{\PidxI} - \particleMass{\PidxI} (\dirDiff{\LidxIII} \dirDiff{\LidxII} \particlePos{\PidxI} \sysVelCoeff{\LidxIII} \sysVelCoeff{\LidxII} + \particleAccTime{\PidxI})}^2}_{\GaussianConstraint_0}
\nonumber\\
 &= \tfrac{1}{2} \sysVel^\top \sysInertiaMat \sysVel - \sysVel^\top \sysForce + \GaussianConstraint_0.
\end{align}
The necessary condition for a critical point is
\begin{align}
 \pdiff[\GaussianConstraint]{\sysVeld} = \tuple{0}
\qquad\Leftrightarrow\qquad
 \sysInertiaMat \sysVeld = \sysForce.
\end{align}
Which is, again, the same result as above.
Since $\sfrac{\partial^2 \GaussianConstraint}{\partial\sysVeld \partial \sysVeld} = \sysInertiaMat$ is positive definite, the solution $\sysVeld = \sysInertiaMat^{-1}\sysForce$ is a minimum of the Gaussian constraint $\GaussianConstraint$.

\paragraph{Application to Hamilton's principle.}
As discussed above, Hamilton's principle of stationary action is not applicable for generic forces as assumed here.
However, it might still be instructive to apply it to our system when setting $\genForceImpressed=\tuple{0}$ and $\Lagrangian = \kineticEnergy$.

The kinetic energy $\kineticEnergy$ in terms of the chosen coordinates is
\begin{align}
 \kineticEnergy(\sysCoord, \sysVel, t)
% = \tfrac{1}{2} \sumParticles \particleMass{\PidxI} \norm{\particlePosd{\PidxI}(\sysCoord, \sysVel, t)}^2
 = \tfrac{1}{2} \sumParticles \particleMass{\PidxI} \norm{\particlePosd{\PidxI}(\sysCoord, \sysVel, t)}^2,
\qquad
 \particlePosd{\PidxI}(\sysCoord, \sysVel, t) = \dirDiff{\LidxI} \particlePos{\PidxI}(\sysCoord) \sysVelCoeff{\LidxI} + \pdiff[\particlePos{\PidxI}]{t}(\sysCoord, t)
% \nonumber\\
%  &= \tfrac{1}{2} \underbrace{\sumParticles \particleMass{\PidxI} \sProd{\dirDiff{\LidxI} \particlePos{\PidxI}(\sysCoord)}{\dirDiff{\LidxII} \particlePos{\PidxI}(\sysCoord)}}_{\sysInertiaMatCoeff{\LidxI\LidxII}(\sysCoord)} \sysVelCoeff{\LidxI} \sysVelCoeff{\LidxII}
%  + \sumParticles \particleMass{\PidxI} \sProd{\dirDiff{\LidxI} \particlePos{\PidxI}(\sysCoord)}{\pdiff[\particlePos{\PidxI}]{t}(\sysCoord, t)}
%  + \tfrac{1}{2} \sumParticles \particleMass{\PidxI} \norm{\pdiff[\particlePos{\PidxI}]{t}(\sysCoord, t)}^2
\end{align}
As this is the structure assumed in \eqref{eq:Functional} we may use the result \eqref{eq:MyEulerLagrange} from the calculus of variations to obtain
\begin{align}\label{eq:ResHamiltonPrinciple}
 \genForceInertiaCoeff{\LidxI} 
 = \diff{t} \pdiff[\kineticEnergy]{\sysVelCoeff{\LidxI}} + \BoltzSym{\LidxIII}{\LidxI}{\LidxII} \sysVelCoeff{\LidxII} \pdiff[\kineticEnergy]{\sysVelCoeff{\LidxIII}} - \dirDiff{\LidxI} \kineticEnergy
 = 0
% = \underbrace{\sProd{\dirDiff{\LidxI} \particlePos{\PidxI}}{\particleForceImpressed{\PidxI}}}_{\genForceImpressedCoeff{\LidxI}},
 \quad \LidxI = 1,\ldots,\dimConfigSpace.
\end{align}
Evaluation and some rearrangement using the identities \eqref{eq:IdentityDifferentials} and \eqref{eq:CommutationParticleVel} yields
\begin{multline}\label{eq:ResHamiltonPrinciple2}
 \genForceInertiaCoeff{\LidxI} = \sumParticles \particleMass{\PidxI} \Big(
 \diff{t} \sProd{\pdiff[\particlePosd{\PidxI}]{\sysVelCoeff{\LidxI}}}{\particlePosd{\PidxI}}
 + \BoltzSym{\LidxIII}{\LidxI}{\LidxII} \sysVelCoeff{\LidxII} \sProd{\pdiff[\particlePosd{\PidxI}]{\sysVelCoeff{\LidxIII}}}{\particlePosd{\PidxI}}
 - \sProd{\dirDiff{\LidxI} \particlePosd{\PidxI}}{\particlePosd{\PidxI}} \Big)
\\
 = \sumParticles \particleMass{\PidxI} \Big(
 \sProd{\dirDiff{\LidxI} \particlePos{\PidxI}}{\particlePosdd{\PidxI}}
 + \sProd{\underbrace{\diff{t} \dirDiff{\LidxI} \particlePos{\PidxI} + \BoltzSym{\LidxIII}{\LidxI}{\LidxII} \sysVelCoeff{\LidxII} \dirDiff{\LidxIII} \particlePos{\PidxI} - \dirDiff{\LidxI} \particlePosd{\PidxI}}_{0}}{\particlePosd{\PidxI}} \Big)
\end{multline}
Which matches the generalized inertial force derived above \eqref{eq:ConstraintReleaseEliminated}.

% \paragraph{Conclusion}
% Evaluation of the principles above with the coordinates leads to 
% \begin{align}
%  \underbrace{\sumParticles \particleMass{\PidxI} \sProd{\dirDiff{\LidxI} \particlePos{\PidxI}}{\dirDiff{\LidxII} \particlePos{\PidxI}}}_{\sysInertiaMatCoeff{\LidxI\LidxII}} \sysVelCoeffd{\LidxII}
%  = \underbrace{\sumParticles \sProd{\particleForceImpressed{\PidxI} - \particleMass{\PidxI} \big( \dirDiff{\LidxI} \particlePos{\PidxI}}{\dirDiff{\LidxIII} \dirDiff{\LidxII} \particlePos{\PidxI} \sysVelCoeff{\LidxIII} \sysVelCoeff{\LidxII} + \particleAccTime{\PidxI} \big)}}_{\sysForceCoeff{\LidxI}}&,&
%  \LidxI &= 1,\ldots, \dimConfigSpace
% \end{align}


\subsection{Inertia}
In \eqref{eq:ConstraintReleaseEliminated} we have introduced the generalized inertia force $\genForceInertia = \sumParticles \sProd{\dirDiff{\LidxI} \particlePos{\PidxI}}{\particleForceInertia{\PidxI}}$ as the projection of the particle inertia forces $\particleForceInertia{\PidxI} = \particleMass{\PidxI} \particlePosdd{\PidxI}$.
With this we will review some established formalisms and extend them for the use of redundant configuration coordinates.

\paragraph{Gibbs-Appell formulation.}
Using the identity of the differentials \eqref{eq:IdentityDifferentials} we may formulate
\begin{align}
 \genForceInertiaCoeff{\LidxI}
 = \sumParticles \sProd{\pdiff[\particlePosdd{\PidxI}]{\sysVelCoeffd{\LidxI}}}{-\particleMass{\PidxI} \particlePosdd{\PidxI}}
 = -\pdiff{\sysVelCoeffd{\LidxI}} \underbrace{\Big(\tfrac{1}{2} \sumParticles \particleMass{\PidxI} \norm{\particlePosdd{\PidxI}}^2\Big)}_{\accEnergy},
 \qquad
 \LidxI = 1,\ldots,\dimConfigSpace.
\label{eq:InertiaForceAccEnergy}
\end{align}
So the generalized inertia force $\genForceInertia$ may be derived from the function $\accEnergy = \accEnergy(\sysCoord, \sysVel, \sysVeld, t)$ which is commonly called the \textit{acceleration energy} though its dimension is \textit{not} that of energy.

This formulation was first proposed by \cite{Gibbs:FundamentalFormulaeOfDynamics} for Cartesian, minimal coordinates and by \cite{Appell:formeGenerale} also using nonholonomic velocity coordinates.
Some historic overview is given in \cite[sec. 1]{Lewis:GaussPrinciple}.

\paragraph{Euler-Lagrange formulation.}
Using the identity of the differentials \eqref{eq:IdentityDifferentials}, the commutation relation \eqref{eq:CommutationParticleVel} and the product rule of differentiation we may formulate
\begin{align}
 \genForceInertiaCoeff{\LidxI}
 &= \sumParticles \sProd{\pdiff[\particlePosd{\PidxI}]{\sysVelCoeff{\LidxI}}}{\diff{t}\big(\particleMass{\PidxI} \particlePosd{\PidxI}\big)}
\nonumber\\
 &= \sumParticles \particleMass{\PidxI}  \Big( \diff{t} \sProd{\pdiff[\particlePosd{\PidxI}]{\sysVelCoeff{\LidxI}}}{\particlePosd{\PidxI}} - \sProd{\diff{t} \pdiff[\particlePosd{\PidxI}]{\sysVelCoeff{\LidxI}}}{\particlePosd{\PidxI}}\Big)
\nonumber\\
 &= \sumParticles \particleMass{\PidxI}  \Big( \diff{t} \sProd{\pdiff[\particlePosd{\PidxI}]{\sysVelCoeff{\LidxI}}}{\particlePosd{\PidxI}} - \sProd{\dirDiff{\LidxI} \particlePosd{\PidxI} - \BoltzSym{\LidxIII}{\LidxI}{\LidxII} \pdiff[\particlePosd{\PidxI}]{\sysVelCoeff{\LidxIII}} \sysVelCoeff{\LidxII}}{\particlePosd{\PidxI}}\Big)
\nonumber\\
 &= \Big(\diff{t} \pdiff{\sysVelCoeff{\LidxI}} + \BoltzSym{\LidxIII}{\LidxI}{\LidxII} \sysVelCoeff{\LidxII} \pdiff{\sysVelCoeff{\LidxIII}} - \dirDiff{\LidxI} \Big)
 \underbrace{\Big( \tfrac{1}{2} \sumParticles \particleMass{\PidxI} \norm{\particlePosd{\PidxI}}^2 \Big)}_{\kineticEnergy},
 \qquad
 \LidxI = 1,\ldots,\dimConfigSpace.
\label{eq:InertiaForceKinEnergy}
\end{align}
So the generalized inertia force $\genForceInertia$ may be derived from the \textit{kinetic energy} $\kineticEnergy = \kineticEnergy(\sysCoord, \sysVel, t)$ formulated in terms of the chosen coordinates.
This has already been shown in \eqref{eq:ResHamiltonPrinciple} and the computation is essentially \eqref{eq:ResHamiltonPrinciple2} backwards.

For the special case of minimal configuration coordinates $\sysCoord=\genCoord$ and the holonomic velocity coordinates $\sysVel=\genCoordd$, which implies $\BoltzSym{}{}{} = 0$, this formulation and its derivation may be found in any graduate textbook on mechanics and is commonly called the \textit{Euler-Lagrange equation}.
A nearly identical form as \eqref{eq:InertiaForceKinEnergy} can be found in \cite[p.\ 17]{Hamel:LagrangeEuler}.
The difference is that, in contrast to this work, the directional derivative $\dirDiff{\LidxI}$ and the commutation coefficients $\BoltzSym{\LidxIII}{\LidxI}{\LidxII}$ are therein restricted to minimal configuration coordinates. 

It is worth noting that the quatitiy $2\kineticEnergy$ which appeared in this context, was called \textit{vis viva} in older publications and translated to \textit{living force} or \textit{lebendige Kraft} \cite{Hamel:LagrangeEuler}.
The contemporary term \textit{kinetic energy} seems to have established in the early 20th century.


%\cite[p.\,480, eq.\,II]{Hamel:HM}.

\paragraph{Levi-Civita formulation.}
Formulation of the particle accelerations $\particlePosdd{\PidxI}$ explicitly in terms of the chosen coordinates \eqref{eq:particleAcc} yields
\begin{align}\label{eq:genInertiaForceLeviCivita}
 \genForceInertiaCoeff{\LidxI} &= \sumParticles \sProd{\dirDiff{\LidxI} \particlePos{\PidxI}}{\particleMass{\PidxI} \overbrace{\big(\dirDiff{\LidxII} \particlePos{\PidxI} \sysVelCoeffd{\LidxII} + \dirDiff{\LidxIII} \dirDiff{\LidxII} \particlePos{\PidxI} \sysVelCoeff{\LidxIII} \sysVelCoeff{\LidxII} + \particleAccTime{\PidxI}\big)}^{\particlePosdd{\PidxI}}}
\\
 &= \underbrace{\sumParticles \particleMass{\PidxI} \sProd{\dirDiff{\LidxI} \particlePos{\PidxI}}{\dirDiff{\LidxII} \particlePos{\PidxI}}}_{\sysInertiaMatCoeff{\LidxI\LidxII}} \sysVelCoeffd{\LidxII}
 + \underbrace{\sumParticles \particleMass{\PidxI} \sProd{\dirDiff{\LidxI} \particlePos{\PidxI}}{\dirDiff{\LidxIII} \dirDiff{\LidxII} \particlePos{\PidxI}}}_{\ConnCoeffL{\LidxI}{\LidxII}{\LidxIII}} \sysVelCoeff{\LidxIII} \sysVelCoeff{\LidxII}
 + \sumParticles \particleMass{\PidxI} \sProd{\dirDiff{\LidxI} \particlePos{\PidxI}}{\particleAccTime{\PidxI}},
 \
 \LidxI = 1,\ldots,\dimConfigSpace.
\nonumber
\end{align}
The inertia matrix $\sysInertiaMat$ was already discussed above.
Here we are interested in the terms denoted by $\ConnCoeffL{\LidxI}{\LidxII}{\LidxIII}$.
Based on their definition in \eqref{eq:genInertiaForceLeviCivita} one may validate the following identities
\begin{subequations}
\begin{align}
 \label{eq:ConnCoeffSymOne}
 \dirDiff{\LidxIII} \sysInertiaMatCoeff{\LidxI\LidxII}
 &= \ConnCoeffL{\LidxI}{\LidxII}{\LidxIII} + \ConnCoeffL{\LidxII}{\LidxI}{\LidxIII},
% &= \underbrace{\sumParticles \particleMass{\PidxI} \sProd{\dirDiff{\LidxI} \particlePos{\PidxI}}{\dirDiff{\LidxIII} \dirDiff{\LidxII} \particlePos{\PidxI}}}_{\ConnCoeffL{\LidxI}{\LidxII}{\LidxIII}}
% + \underbrace{\sumParticles \particleMass{\PidxI} \sProd{\dirDiff{\LidxIII} \dirDiff{\LidxI} \particlePos{\PidxI}}{\dirDiff{\LidxII} \particlePos{\PidxI}}}_{\ConnCoeffL{\LidxII}{\LidxI}{\LidxIII}}
\\ 
 \label{eq:ConnCoeffSymTwo}
 \BoltzSym{\LidxV}{\LidxI}{\LidxII} \sysInertiaMatCoeff{\LidxV\LidxIII}
 &= \ConnCoeffL{\LidxIII}{\LidxII}{\LidxI} - \ConnCoeffL{\LidxIII}{\LidxI}{\LidxII}.
% &= \sumParticles \particleMass{\PidxI} \sProd{\BoltzSym{\LidxV}{\LidxI}{\LidxII} \partial_\LidxV \particlePos{\PidxI}}{\dirDiff{\LidxIII} \particlePos{\PidxI}}
% = \underbrace{\sumParticles \particleMass{\PidxI} \sProd{\dirDiff{\LidxI} \dirDiff{\LidxII} \particlePos{\PidxI}}{\dirDiff{\LidxIII} \particlePos{\PidxI}}}_{\ConnCoeffL{\LidxIII}{\LidxII}{\LidxI}}
% - \underbrace{\sumParticles \particleMass{\PidxI} \sProd{\dirDiff{\LidxII} \dirDiff{\LidxI} \particlePos{\PidxI}}{\dirDiff{\LidxIII} \particlePos{\PidxI}}}_{\ConnCoeffL{\LidxIII}{\LidxI}{\LidxII}}
\end{align} 
\end{subequations}
Plugging these together while permuting the indices, we find
\begin{align}
 \ConnCoeffL{\LidxI}{\LidxII}{\LidxIII} &= \dirDiff{\LidxIII} \sysInertiaMatCoeff{\LidxI\LidxII} - \ConnCoeffL{\LidxII}{\LidxI}{\LidxIII} 
\nonumber\\
 &= \dirDiff{\LidxIII} \sysInertiaMatCoeff{\LidxI\LidxII} + \BoltzSym{\LidxV}{\LidxI}{\LidxIII} \sysInertiaMatCoeff{\LidxV\LidxII} - \ConnCoeffL{\LidxII}{\LidxIII}{\LidxI} 
\nonumber\\
 &= \dirDiff{\LidxIII} \sysInertiaMatCoeff{\LidxI\LidxII} + \BoltzSym{\LidxV}{\LidxI}{\LidxIII} \sysInertiaMatCoeff{\LidxV\LidxII} - \dirDiff{\LidxI} \sysInertiaMatCoeff{\LidxIII\LidxII} + \ConnCoeffL{\LidxIII}{\LidxII}{\LidxI}
\nonumber\\
 &= \dirDiff{\LidxIII} \sysInertiaMatCoeff{\LidxI\LidxII} + \BoltzSym{\LidxV}{\LidxI}{\LidxIII} \sysInertiaMatCoeff{\LidxV\LidxII} - \dirDiff{\LidxI} \sysInertiaMatCoeff{\LidxIII\LidxII} + \BoltzSym{\LidxV}{\LidxI}{\LidxII} \sysInertiaMatCoeff{\LidxV\LidxIII} + \ConnCoeffL{\LidxIII}{\LidxI}{\LidxII}
\nonumber\\
 &= \dirDiff{\LidxIII} \sysInertiaMatCoeff{\LidxI\LidxII} + \BoltzSym{\LidxV}{\LidxI}{\LidxIII} \sysInertiaMatCoeff{\LidxV\LidxII} - \dirDiff{\LidxI} \sysInertiaMatCoeff{\LidxIII\LidxII} + \BoltzSym{\LidxV}{\LidxI}{\LidxII} \sysInertiaMatCoeff{\LidxV\LidxIII} + \dirDiff{\LidxII} \sysInertiaMatCoeff{\LidxI\LidxIII} - \ConnCoeffL{\LidxI}{\LidxIII}{\LidxII}
\nonumber\\
 &= \dirDiff{\LidxIII} \sysInertiaMatCoeff{\LidxI\LidxII} + \BoltzSym{\LidxV}{\LidxI}{\LidxIII} \sysInertiaMatCoeff{\LidxV\LidxII} - \dirDiff{\LidxI} \sysInertiaMatCoeff{\LidxIII\LidxII} + \BoltzSym{\LidxV}{\LidxI}{\LidxII} \sysInertiaMatCoeff{\LidxV\LidxIII} + \dirDiff{\LidxII} \sysInertiaMatCoeff{\LidxI\LidxIII} - \BoltzSym{\LidxV}{\LidxIII}{\LidxII} \sysInertiaMatCoeff{\LidxV\LidxI} - \ConnCoeffL{\LidxI}{\LidxII}{\LidxIII}
\\[1ex]
 \Leftrightarrow \quad
 \ConnCoeffL{\LidxI}{\LidxII}{\LidxIII} &= \tfrac{1}{2}\big( \dirDiff{\LidxIII} \sysInertiaMatCoeff{\LidxI\LidxII} + \dirDiff{\LidxII} \sysInertiaMatCoeff{\LidxI\LidxIII} - \dirDiff{\LidxI} \sysInertiaMatCoeff{\LidxII\LidxIII} + \BoltzSym{\LidxV}{\LidxI}{\LidxII} \sysInertiaMatCoeff{\LidxV\LidxIII} + \BoltzSym{\LidxV}{\LidxI}{\LidxIII} \sysInertiaMatCoeff{\LidxV\LidxII} - \BoltzSym{\LidxV}{\LidxII}{\LidxIII} \sysInertiaMatCoeff{\LidxV\LidxI} \big).
\label{eq:DefConnCoeffL}
\end{align}
This means the coefficients $\ConnCoeffL{\LidxI}{\LidxII}{\LidxIII}$ are completely determined by the inertia matrix $\sysInertiaMat$ and the geometric matrix $\kinMat$ which determines the directional derivative $\dirDiff{\LidxI}$ and the commutation coefficients $\mat{\BoltzSymSym}$.

A similar, coordinate free derivation can be found in \cite[proof of Theorem 2.7.6]{Abraham:FoundationsOfMechanics} for the proof of the fundamental theorem of Riemannian geometry, i.e.\ the existence and uniqueness of the \textit{Levi-Civita connection}.
However, all coordinate versions therein are restricted to minimal holonomic coordinates.
For this \textit{special case}, i.e.\ $\sysCoord = \genCoord$, $\sysVel = \genCoordd$, $\kinMat=\idMat[\dimConfigSpace]$ and consequently $\BoltzSym{}{}{} = 0$, \eqref{eq:DefConnCoeffL} simplifies to the familiar definition of the \textit{Christoffel symbols} $\ConnCoeffL{\LidxI}{\LidxII}{\LidxIII} = \tfrac{1}{2}\big( \pdiff[\sysInertiaMatCoeff{\LidxI\LidxII}]{\genCoordCoeff{\LidxIII}} + \pdiff[\sysInertiaMatCoeff{\LidxI\LidxIII}]{\genCoordCoeff{\LidxII}} - \pdiff[\sysInertiaMatCoeff{\LidxII\LidxIII}]{\genCoordCoeff{\LidxI}} \big)$, see e.g.\ \cite[p.\,145]{Abraham:FoundationsOfMechanics} or \cite[Vol.\,2, p.\,221]{Spivak:DiffGeo}.
%\begin{align} 
% \ConnCoeffL{\LidxI}{\LidxII}{\LidxIII} &= \tfrac{1}{2}\Big( \pdiff[\sysInertiaMatCoeff{\LidxI\LidxII}]{\genCoordCoeff{\LidxIII}} + \pdiff[\sysInertiaMatCoeff{\LidxI\LidxIII}]{\genCoordCoeff{\LidxII}} - \pdiff[\sysInertiaMatCoeff{\LidxII\LidxIII}]{\genCoordCoeff{\LidxI}} \Big).
%\end{align}
In \cite[sec.\,9.2]{Frankel:GeometryOfPhysics} it is pointed out that the name Christoffel symbols is exclusive for the holonomic case, whereas in general $\ConnCoeffL{\LidxI}{\LidxII}{\LidxIII}$ are referred to as the \textit{(Levi-Civita) connection coefficients}.
To the best of the authors knowledge, the only popular source that states the coordinate version \eqref{eq:DefConnCoeffL} explicitly is \cite[eq.\ 8.24]{Misner:Gravitation}, though restricted to minimal coordinates and in a rather different context of relativistic point masses.
Since the directional derivative $\dirDiff{\LidxI}$ and the commutation coefficients $\BoltzSym{\LidxV}{\LidxI}{\LidxII}$ are defined in a setting supporting redundant coordinates, so does \eqref{eq:DefConnCoeffL} as definition of the connection coefficients.

% In \cite[eq.\,4.10.9]{Lurie:AnalyticalMechanics} only the first three terms of \eqref{eq:DefConnCoeffLeviCivita} are introduced as the ``generalized Christoffel symbols'' in the context of minimal coordinates and non-coordinate basis vectors.
% This might be misleading since these quantities do not obey the transformation rule \eqref{eq:TrafoRuleConnCoeff}, so do not define a connection.



\subsection{Gravitation}\label{sec:ParticleSysGravitation}
Earth's gravity acts on a system of particles just as on a single particle \eqref{eq:ParticleGravity}, i.e.\ with a force $\particleForceGravity{\PidxI} = \particleMass{\PidxI} \gravityAcc$ on each particle.
The resulting generalized force on the system is
\begin{align}
 \genForceGravityCoeff{\LidxI}
 = \sumParticles \sProd{\dirDiff{\LidxI} \particlePos{\PidxI}}{\particleForceGravity{\PidxI}}
 = \sumParticles \particleMass{\PidxI} \sProd{\dirDiff{\LidxI}\particlePos{\PidxI}}{\gravityAcc}
 = \dirDiff{\LidxI} \underbrace{\sumParticles \particleMass{\PidxI} \sProd{\particlePos{\PidxI}}{-\gravityAcc}}_{\potentialGravity}.
\label{eq:ParticleSysGravity}
\end{align}
The force may also be derived from the \textit{gravitational potential} $\potentialGravity$ of the system which is simply the sum of the potentials of the individual particles.
%Since it can be derived from a potential $\potentialGravity:\configSpace\rightarrow\RealNum$, it is called a \textit{conservative force}.

The \textit{gravitational mass} $\particleMass{\PidxI}$ in \eqref{eq:ParticleSysGravity} takes the same value as the \textit{inertial mass} from the previous section.
This is sometimes referred to as the \textit{(Galilean) equivalence principle} and is crucial topic for general relativity, see e.g., \cite[chap.\,16]{Misner:Gravitation}.
In this context, there is no \textit{absolute} acceleration $\particlePosdd{\PidxI}$, instead, we are interested in the deviation from the free fall acceleration $\gravityAcc$.
This motivates the following formulation for the sum of generalized inertial and gravitational force
\begin{multline}
  \genForceInertiaCoeff{\LidxI} + \genForceGravityCoeff{\LidxI} 
  = \sumParticles \particleMass{\PidxI} \sProd{\dirDiff{\LidxI} \particlePos{\PidxI}}{\particlePosdd{\PidxI} - \gravityAcc}
  = \sumParticles \particleMass{\PidxI} \sProd{\pdiff[\particlePosdd{\PidxI}]{\sysVelCoeffd{\LidxI}}}{\particlePosdd{\PidxI} - \gravityAcc}
\\
  = \pdiff{\sysVelCoeffd{\LidxI}} \underbrace{\Big(\tfrac{1}{2} \sumParticles \particleMass{\PidxI} \norm{\particlePosdd{\PidxI} - \gravityAcc}^2\Big)}_{\accEnergyInertial},
  \qquad
  \LidxI = 1,\ldots,\dimConfigSpace.
 \label{eq:EinsteinianForce}
\end{multline}
The quantity $\accEnergyInertial$ may be regarded as a metric for the deviation of the system to its natural acceleration, the free fall.

\newcommand{\particleVelGravity}[1]{\particleStyle{v}_{#1\idxText{G}}}
Taking this one step further, we may consider the free fall velocities $\particleVelGravity{\PidxI}(t) = \gravityAcc t + \particleStyle{v}_{\PidxI 0}$ with arbitrary initial velocities $\particleStyle{v}_{\PidxI 0} \in \RealNum^3, \PidxI=1,\ldots,\numParticles$ and compute analog to \eqref{eq:InertiaForceKinEnergy}:
\begin{align}
  \genForceInertiaCoeff{\LidxI} + \genForceGravityCoeff{\LidxI}
  &= \sumParticles \particleMass{\PidxI} \sProd{\pdiff[\particlePosd{\PidxI}]{\sysVelCoeff{\LidxI}}}{\diff{t}\big(\particlePosd{\PidxI} - \particleVelGravity{\PidxI} \big)}
 \nonumber\\
  &= \sumParticles \particleMass{\PidxI}  \Big( \diff{t} \sProd{\pdiff[\particlePosd{\PidxI}]{\sysVelCoeff{\LidxI}}}{\particlePosd{\PidxI} - \particleVelGravity{\PidxI}} - \sProd{\diff{t} \pdiff[\particlePosd{\PidxI}]{\sysVelCoeff{\LidxI}}}{\particlePosd{\PidxI} - \particleVelGravity{\PidxI}}\Big)
 \nonumber\\
  &= \sumParticles \particleMass{\PidxI}  \Big( \diff{t} \sProd{\pdiff[\particlePosd{\PidxI}]{\sysVelCoeff{\LidxI}}}{\particlePosd{\PidxI} - \particleVelGravity{\PidxI}} - \sProd{\dirDiff{\LidxI} \particlePosd{\PidxI} - \BoltzSym{\LidxIII}{\LidxI}{\LidxII} \pdiff[\particlePosd{\PidxI}]{\sysVelCoeff{\LidxIII}} \sysVelCoeff{\LidxII}}{\particlePosd{\PidxI} - \particleVelGravity{\PidxI}}\Big)
 \nonumber\\
  &= \Big(\diff{t} \pdiff{\sysVelCoeff{\LidxI}} + \BoltzSym{\LidxIII}{\LidxI}{\LidxII} \sysVelCoeff{\LidxII} \pdiff{\sysVelCoeff{\LidxIII}} - \dirDiff{\LidxI} \Big)
  \underbrace{\Big( \tfrac{1}{2} \sumParticles \particleMass{\PidxI} \norm{\particlePosd{\PidxI} - \particleVelGravity{\PidxI}}^2 \Big)}_{\kineticEnergyInertial},
  \qquad
  \LidxI = 1,\ldots,\dimConfigSpace.
\end{align}

\subsection{Stiffness}\label{sec:ParticleSysStiffness}
Consider that \textit{one} particle $\particlePos{\PidxI}$ of the system is connected by a linear spring with stiffness $\particleStiffness{\PidxI}$ to a position $\springHubPos{\PidxI}$.
The force on this one particle, as proposed in \eqref{eq:ParticleStiffness}, is $\particleForceStiff{\PidxI} = \particleStiffness{\PidxI} (\springHubPos{\PidxI} - \particlePos{\PidxI})$.
The generalized force on the system is
\begin{align}
 \genForceStiffCoeff{\LidxI} 
 = \sProd{\dirDiff{\LidxI} \particlePos{\PidxI}}{\particleForceStiff{\PidxI}} 
 = \particleStiffness{\PidxI} \sProd{\dirDiff{\LidxI} \particlePos{\PidxI}}{\springHubPos{\PidxI} - \particlePos{\PidxI}}
 = -\dirDiff{\LidxI} \underbrace{\big( \tfrac{1}{2} \particleStiffness{\PidxI} \norm{\particlePos{\PidxI} - \springHubPos{\PidxI}}^2 \big)}_{\potentialStiff_{\PidxI}}.
\end{align}

We may also consider a spring with stiffness $\particleStiffness{\PidxI\PidxII}$ between two particles of the system:
Then we have the force $\particleForceStiff{\PidxI} = \particleStiffness{\PidxI\PidxII} (\particlePos{\PidxII} - \particlePos{\PidxI})$ on particle $\particlePos{\PidxI}$ and the opposite force $\particleForceStiff{\PidxII} = \particleStiffness{\PidxI\PidxII} (\particlePos{\PidxI} - \particlePos{\PidxII})$ on particle $\particlePos{\PidxII}$.
The generalized force on the system is
\begin{align}
 \genForceStiffCoeff{\LidxI} &= \sProd{\dirDiff{\LidxI} \particlePos{\PidxI}}{\particleForceStiff{\PidxI}} + \sProd{\dirDiff{\LidxI} \particlePos{\PidxII}}{\particleForceStiff{\PidxII}} 
% &= \particleStiffness{\PidxI\PidxII} \sProd{\dirDiff{\LidxI} \particlePos{\PidxI}}{\particlePos{\PidxII} - \particlePos{\PidxI}} + \particleStiffness{\PidxI\PidxII} \sProd{\dirDiff{\LidxI} \particlePos{\PidxII}}{\particlePos{\PidxI} - \particlePos{\PidxII}}
 = -\particleStiffness{\PidxI\PidxII} \sProd{\dirDiff{\LidxI} (\particlePos{\PidxI} - \particlePos{\PidxII})}{\particlePos{\PidxI} - \particlePos{\PidxII}}
 = -\dirDiff{\LidxI} \underbrace{\big( \tfrac{1}{2} \particleStiffness{\PidxI\PidxII} \norm{\particlePos{\PidxI} - \particlePos{\PidxII}}^2 \big)}_{\potentialStiff_{\PidxI\PidxII}}.
\end{align}
In both cases the generalized force can be derived from a potential $\potentialStiff$.

For a system with an arbitrary number of linear springs, one may simply sum up the individual potentials to obtain the stiffness potential $\potentialStiff$ and derive the corresponding generalized force $\genForceStiff = \differential \potentialStiff$. 
Note that non-negativity of the spring constants $\particleStiffness{} \geq 0$ implies non-negativity of the potential $\potentialStiff \geq 0$.



%\cite[p.\ 24]{Goldstein:ClassicalMechanics} ''Frictional forces of this type may be derived in terms of a function, known as Rayleigh's dissipation function``
%\cite[p.\ 519]{Papastavridis:AnalyticalMechanics} same as above
%\cite[§81]{Rayleigh:TheoryOfSound}: ''Suppose that each particle of the system is retarded by forces proportional to its component velocities``.
\subsection{Dissipation}
Let the system of particles move within a viscous fluid with velocities $\fluidVel{\PidxI}$ at the positions $\particlePos{\PidxI}$ of the particles.
Then, as proposed in \eqref{eq:ParticleDamping}, each particle is subject to a friction force $\particleForceDiss{\PidxI} = -\particleDamping{\PidxI} (\particlePosd{\PidxI} - \fluidVel{\PidxI})$.
The generalized force on the the system is
\begin{align}
 \genForceDissCoeff{\LidxI}
 = \sumParticles \sProd{\dirDiff{\LidxI} \particlePos{\PidxI}}{\particleForceDiss{\PidxI}}
 = -\sumParticles \particleDamping{\PidxI} \sProd{\pdiff[\particlePosd{\PidxI}]{\sysVelCoeff{\LidxI}}}{\particlePosd{\PidxI} - \fluidVel{\PidxI}}
 = -\pdiff{\sysVelCoeff{\LidxI}} \underbrace{\Big( \tfrac{1}{2} \sumParticles \particleDamping{\PidxI} \norm{\particlePosd{\PidxI} - \fluidVel{\PidxI}}^2 \Big)}_{\dissFkt}.
\end{align}
This dissipative force may be derived from $\dissFkt$, which is commonly called \textit{Rayleigh dissipation function}, e.g.\ \cite[p.\ 24]{Goldstein:ClassicalMechanics}.
Its dimension is that of power, i.e.\ watts.
Note that non-negativity of the damping parameters $\particleStiffness{} \geq 0$ implies non-negativity of the dissipation function $\potentialStiff \geq 0$.


\subsection{Energy}
\paragraph{Total energy.}
The time derivative of the kinetic energy may be formulated as
\begin{align}
 \dot{\kineticEnergy} = \sumParticles \particleMass{\PidxI} \sProd{\particlePosd{\PidxI}}{\particlePosdd{\PidxI}} 
 = \sysVelCoeff{\LidxI} \underbrace{\sumParticles \particleMass{\PidxI} \sProd{\dirDiff{\LidxI}\particlePos{\PidxI}}{\particlePosdd{\PidxI}}}_{\genForceInertiaCoeff{\LidxI}}
 + \sumParticles \particleMass{\PidxI} \sProd{\pdiff[\particlePos{\PidxI}]{t}}{\particlePosdd{\PidxI}}.
\end{align}	
In \autoref{sec:ParticleSysGravitation} and \autoref{sec:ParticleSysGravitation} we have seen potentials of the form $\potentialEnergy(\sysCoord, t)$ and their associated generalized force $\genForcePotentialCoeff{\LidxI} = \dirDiff{\LidxI} \potentialEnergy$.
The time derivative of this potential is
\begin{align}
 \dot{\potentialEnergy} &= \sysVelCoeff{\LidxI} \underbrace{\dirDiff{\LidxI} \potentialEnergy}_{\genForcePotentialCoeff{\LidxI}} + \pdiff[\potentialEnergy]{t}.
\end{align}

The sum $\totalEnergy = \kineticEnergy + \potentialEnergy$ is commonly called the \textit{total energy}.
Its time derivative is
\begin{align}
 \dot{\totalEnergy} = \sysVel^\top \big( \genForceInertia + \genForcePotential \big) + \sumParticles \particleMass{\PidxI} \sProd{\pdiff[\particlePos{\PidxI}]{t}}{\particlePosdd{\PidxI}} + \pdiff[\potentialEnergy]{t}
 .
\end{align}
Note that the equation of motion implies $\genForceInertia + \genForcePotential = \genForceEx - \genForceDiss$.

A mechanical system is called skleronomic (otherwise rheonomic) if it does not contain explicit time dependency. 
For this important case the change of total energy may be exressed by the external and dissipative forces alone
\begin{align}
 \dot{\totalEnergy} = \sysVel^\top \big( \genForceEx - \genForceDiss \big) %+ \sumParticles \particleMass{\PidxI} \sProd{\pdiff[\particlePos{\PidxI}]{t}}{\particlePosdd{\PidxI}} + \pdiff[\potentialEnergy]{t}
 .
\end{align}

\paragraph{Lagrangian and Hamiltonian.}
Define the \textit{Lagrangian} as $\Lagrangian = \kineticEnergy - \potentialEnergy$ which according to \autoref{sec:HamiltonsEquations} implies the generalized momentum and \textit{Hamiltonian} as
\begin{align}
 \genMomentumCoeff{\LidxI} &= \pdiff[\Lagrangian]{\sysVelCoeff{\LidxI}} = \sumParticles \particleMass{\PidxI} \sProd{\dirDiff{\LidxI} \particlePos{\PidxI}}{\particlePosd{\PidxI}}, \quad \LidxI = 1,\ldots,\dimConfigSpace,
\\
 \Hamiltonian &= \genMomentumCoeff{\LidxI} \sysVelCoeff{\LidxI} - \Lagrangian = \underbrace{\tfrac{1}{2} \sumParticles \particleMass{\PidxI} \norm{\particlePosd{\PidxI}}^2 + \potentialEnergy}_{\totalEnergy} - \sumParticles \particleMass{\PidxI} \sProd{\pdiff[\particlePos{\PidxI}]{t}}{\particlePosd{\PidxI}} 
 .
\end{align}
Extending the derivation from \autoref{sec:HamiltonsEquations} with the nonconservative forces $\genForceDiss$ and $\genForceEx$ we find the change of the Hamiltonian along the solutions of the equations of motion as
\begin{align}
 \dot{\Hamiltonian} &=  \sysVel^\top \big( \genForceEx - \genForceDiss \big) - \pdiff[\Lagrangian]{t}
 = \sysVel^\top \big( \genForceEx - \genForceDiss \big) - \sumParticles \particleMass{\PidxI} \sProd{\pdiff[\particlePosd{\PidxI}]{t}}{\particlePosd{\PidxI}} + \pdiff[\potentialEnergy]{t}
\end{align}
Note that for a skleronomic system, the Hamiltonian $\Hamiltonian$ coincides with the total energy $\totalEnergy$. 

%For the important case that the particle positions have no no explicit time dependency, i.e.\ $\spdiff[\particlePos{\PidxI}]{t} = \tuple{0}$, and consequently $\spdiff[\kineticEnergy]{t} = 0$, the Hamiltonian $\Hamiltonian$ coincides with the total energy $\totalEnergy$.

% several forms:
% \begin{subequations}
% \begin{align}
%  \genForceInertiaCoeff{\LidxI}  &= \sumParticles \sProd{\dirDiff{\LidxI} \particlePos{\PidxI}}{\particleMass{\PidxI} \particlePosdd{\PidxI}},&
%  \LidxI &= 1,\ldots,\dimConfigSpace
% \\ 
%  &= \sumParticles \particleMass{\PidxI} \sProd{\dirDiff{\LidxI} \particlePos{\PidxI}}{\dirDiff{\LidxII} \particlePos{\PidxI} \sysVelCoeffd{\LidxII} + \dirDiff{\LidxIII} \dirDiff{\LidxII} \particlePos{\PidxI} \sysVelCoeff{\LidxIII} \sysVelCoeff{\LidxII} + \particleAccTime{\PidxI}}
% \\
%  \label{eq:InertiaForceConnCoeff}
%  &= \sysInertiaMatCoeff{\LidxI\LidxII} \sysVelCoeffd{\LidxII} + \ConnCoeffL{\LidxI}{\LidxII}{\LidxIII} \sysVelCoeff{\LidxII}\sysVelCoeff{\LidxIII} + \sumParticles \particleMass{\PidxI} \sProd{\dirDiff{\LidxI} \particlePos{\PidxI}}{\particleAccTime{\PidxI}},&
%  \sysInertiaMatCoeff{\LidxI\LidxII} &= \sumParticles \particleMass{\PidxI} \sProd{\dirDiff{\LidxI} \particlePos{\PidxI}}{\dirDiff{\LidxII} \particlePos{\PidxI}}
% \\
%  \label{eq:InertiaForceKinEnergy}
%  &= \diff{t} \pdiff[\kineticEnergy]{\sysVelCoeff{\LidxI}} + \BoltzSym{\LidxIII}{\LidxI}{\LidxII} \sysVelCoeff{\LidxII} \pdiff[\kineticEnergy]{\sysVelCoeff{\LidxIII}} - \dirDiff{\LidxI} \kineticEnergy,& 
%  \kineticEnergy &= \tfrac{1}{2} \sumParticles \particleMass{\PidxI} \norm{\particlePosd{\PidxI}}^2
% \\
%  \label{eq:InertiaForceAccEnergy}
%  &= \pdiff[\accEnergy]{\sysVelCoeffd{\LidxI}},&
%  \accEnergy &= \tfrac{1}{2} \sumParticles \particleMass{\PidxI} \norm{\particlePosdd{\PidxI}}^2
% \end{align} 
% \end{subequations}
% 
% \begin{subequations}
% \begin{align}
%  \particleForceInertia{\PidxI} &= \particleMass{\PidxI} \particlePosdd{\PidxI},&
%  \genForceInertiaCoeff{\LidxI} &= \sumParticles \sProd{\dirDiff{\LidxI} \particlePos{\PidxI}}{\particleMass{\PidxI} \particlePosdd{\PidxI}} = \pdiff[\accEnergy]{\sysVelCoeffd{\LidxI}},&
%  \accEnergy &= \tfrac{1}{2} \sumParticles \particleMass{\PidxI} \norm{\particlePosdd{\PidxI}}^2
% \\
%  \particleForceGravity{\PidxI} &= \particleMass{\PidxI} \gravityAcc,&
%  \genForceGravityCoeff{\LidxI} &= \sumParticles \sProd{\dirDiff{\LidxI} \particlePos{\PidxI}}{\particleMass{\PidxI} \gravityAcc} = \dirDiff{\LidxI} \potentialGravity,&
%  \potentialGravity &= \sumParticles \particleMass{\PidxI} \sProd{\particlePos{\PidxI}}{\gravityAcc}
% \\
%  \particleForceStiff{\PidxI} &= \particleStiffness{\PidxI} (\particlePos{\PidxI} - \springHubPos{\PidxI}),&
%  \genForceStiffCoeff{\LidxI} &= \sumParticles \particleStiffness{\PidxI} \sProd{\dirDiff{\LidxI} \particlePos{\PidxI}}{\particlePos{\PidxI} - \springHubPos{\PidxI}} = \dirDiff{\LidxI} \potentialStiff,&
%  \potentialStiff &= \tfrac{1}{2} \sumParticles \particleStiffness{\PidxI} \norm{\particlePos{\PidxI} - \springHubPos{\PidxI}}^2
% \\
%  \particleForceDiss{\PidxI} &= \particleDamping{\PidxI} (\particlePosd{\PidxI} - \tuple{v}_{\mathsf{D}\PidxI})
% \end{align} 
% \end{subequations}