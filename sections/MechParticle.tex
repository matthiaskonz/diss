\section{A single free particle}
\cite[§1]{Landau:Mechanics}: \textit{One of the fundamental concepts of mechanics is that of a \textit{particle}, also called material point.}
This abstracts a body whose dimensions may be neglected and all its mass $\particleMass{}$ is located at a point with the Cartesian coordinates $\particlePos{}(t)\in\RealNum^3$ at time $t$.
Its motion obeys Newton's second law \cite[p.\ 13, lex II]{Newton:Principia}, english translation \cite[p.\ 83]{Newton:PrincipiaEnglish}: \textit{The alternation of motion is ever proportional to the motive force impressed; and is made in the direction of the right line in which that force is impressed}. 
The contemporary version reads (e.g.\ \cite[eq.\ 6.1.1]{Lurie:AnalyticalMechanics} or \cite[eq.\ 1.3]{Goldstein:ClassicalMechanics})
\begin{align}\label{eq:NewtonsSecondLaw}
 \particleMass{} \particlePosdd{} &= \particleForceImpressed{}.
\end{align}
where $\particlePosdd{} \equiv \sfrac{\d^2 \particlePos{}}{\d t^2}$ is Newton's notation of differentiation and the applied force $\particleForceImpressed{}$ collects all other (non inertial) influences on the particle.
In this work we will investigate three sources of applied forces: gravity, linear springs and viscous friction.

\paragraph{Gravity.}
For far most engineering applications we are dealing with systems that move close to the surface of the earth and where Galilei's gravitation principle \cite[Day 3]{Galileo:TwoNewSciences} holds.
In a contemporary formulation it states that a particle with mass $\particleMass{\PidxI}$ is subject to the gravitational force
\begin{align}\label{eq:ParticleGravity}
 \particleForceGravity{} = \particleMass{} \gravityAcc
\end{align}
where $\gravityAcc$ are the coefficients of the gravitational acceleration of the earth w.r.t.\ the chosen inertial frame.
Commonly the inertial frame is chosen such that the $\ez$ axis is opposing gravity and we have $\gravityAcc = [0,0,-\gravityAccConst]^\top$ with the \textit{gravity of earth} $\gravityAccConst = 9.8\,\tfrac{\unit{m}}{\unit{s}^2}$.

\paragraph{Linear spring.}
Let the particle be connected with a spring to a point $\particlePos{0}$.
The simplest model of a spring is that of Hooke's law \cite{Hooke:OfSprings}: The force $\particleForceStiff{}$ on the particle is opposite and proportional by a factor $\particleStiffness{} \in \RealNum^+$ to the spring displacement $\particlePos{} - \particlePos{0}$, i.e.\
\begin{align}\label{eq:ParticleStiffness}
 \particleForceStiff{} &= -\particleStiffness{} (\particlePos{} - \particlePos{0}).
\end{align}

\paragraph{Viscous friction.}
\cite[§81]{Rayleigh:TheoryOfSound}: \textit{There is another group of forces whose existence is often advantageous to recognize specially, namely those arising from friction or viscosity. [..] we suppose that each particle is retarded by forces proportional to its component velocities.}
We may think of the particle to be immersed in a viscous fluid which, at the particle position, has the velocity $\tuple{\mathfrak{v}}_0$.
The force on the particle is
\begin{align}\label{eq:ParticleDamping}
 \particleForceDiss{} = -\particleDamping{} (\particlePosd{} - \tuple{\mathfrak{v}}_0)
\end{align}
with the damping parameter $\particleDamping{} \in \RealNum^+$.

\paragraph{Equation of motion.}
A single free particle that is subject to all the aforementioned forces and a general, not further specified external force $\particleForceEx{}$, i.e.\ $\particleForceImpressed{} = \particleForceGravity{} + \particleForceStiff{} + \particleForceDiss{} + \particleForceEx{}$, has the equation of motion 
\begin{align}\label{eq:ParticleEoM}
 \particleMass{} (\particlePosdd{} - \gravityAcc) + \particleDamping{} (\particlePosd{} - \tuple{\mathfrak{v}}_0) + \particleStiffness{} (\particlePos{} - \particlePos{0}) &= \particleForceEx{}.
\end{align}

\paragraph{Control engineering.}
From a control engineering perspective the structure of this system is already nice enough to consider it as a desired closed loop dynamics:
If we want the particle to track a sufficiently smooth reference trajectory $t\mapsto\particlePosR{}(t)$ a reasonable desired closed loop dynamics is
\begin{align}\label{eq:ParticleDesiredClosedLoop}
 \particlePosE{} = \particlePos{} - \particlePos{0},
\qquad
 \particleMass{} \particlePosEdd{} + \particleDampingC{} \particlePosEd{} + \particleStiffnessC{} \particlePosE{} &= \tuple{0}.
% \particleMass{} (\particlePosdd{} - \particlePosRdd{}) + \particleDampingC{} (\particlePosd{} - \particlePosRd{}) + \particleStiffnessC{} (\particlePos{} - \particlePosR{}) &= \tuple{0}.
\end{align}
This is essentially the same as above \eqref{eq:ParticleEoM}, but we replaced the spring origin $\particlePos{0}$ with the reference position $\particlePosR{}$, the fluid velocity $\tuple{\mathfrak{v}}_0$ with the reference velocity $\particlePosRd{}$ and the free-fall acceleration $\gravityAcc$ with the reference acceleration $\particlePosRdd{}$.
Furthermore, we replaced the spring stiffness $\particleStiffness{}$ and viscosity $\particleDamping{}$ by analog tuning parameters $\particleStiffnessC{}, \particleDampingC{} \in \RealNum^+$.
Plugging the desired dynamics \eqref{eq:ParticleDesiredClosedLoop} into the plant dynamics \eqref{eq:ParticleEoM} yields the required control law
\begin{align}
 \particleForceEx{} &= \particleMass{} (\particlePosRdd{} - \gravityAcc) + \particleDamping{} (\particlePosd{} - \tuple{\mathfrak{v}}_0) + \particleStiffness{} (\particlePos{} - \particlePos{0}) - \particleDampingC{} (\particlePosd{} - \particlePosRd{}) - \particleStiffnessC{} (\particlePos{} - \particlePosR{}) -\particleMass{}\gravityAcc.
\end{align}
As this control approach is so closely related to basic mechanics, it could be more intuitive for an engineer than other generic mathematical approaches.

% \subsection{constitutive relations}
% \begin{itemize}
%  \item inertia (Newton): $\particleForceInertia{} = \particleMass{} \particlePosdd{}$
%  \item damping (Rayleigh): $\particleForceDiss{} = \particleDamping{} (\particlePosd{} - \tuple{v}_0)$
%  \item stiffness (Hooke): $\particleForceStiff{} = \particleStiffness{} (\particlePos{} - \particlePos{0})$
%  \item Gravity (Galileo): $\particleForceGravity{} = \particleMass{} \gravityAcc$
% \end{itemize}
