% A \textit{rigid body} can be regarded as a special case of a particle system whose particles have constant distance to each other.
% Several rigid bodies that may be constrained to each other and/or to the surrounding space constitute a \textit{rigid body system}.
% This chapter applies the results from the previous section to them.
% As the configuration space of a free rigid body is nonlinear, the use of redundant coordinates is very appropriate.
% 
% \paragraph{Goal.}
% The goal for this section is to motivate a recipe or formalism for the derivation of the equations of motion for rigid body systems.
% In contrast to established formalisms, the approach here should be most flexible w.r.t.\ the parameterization of the configuration and velocity.
% Furthermore, it should not only apply to the inertia of a system but also to damping and stiffness, as motivated in the previous chapter.
% Overall, this should give deeper insight into the structure of rigid body systems which will be exploited in the next chapter on control of rigid body systems.
% 
% 
% \fixme{
% \cite[chap. 4]{Goldstein:ClassicalMechanics}: A rigid body is defined as a system of mass points subject top the holonomic constraints that the distances between all pairs of points remain constant throughout the motion.
% \\
% \cite[§31]{Landau:Mechanics}: A rigid body may be defined in mechanics as a system of particles such that the distances between the particles do not vary.
% \\
% \cite[§44]{Boltzmann:PrincipeDerMechanik}: \ldots starrer Körper, d.h.\ für ein System materieller Punkte, welche so verbunden sind, dass sich während ihrer ganzen Bewegung ihre relative Lage nicht ändern kann.
% \\
% \cite[§28]{Arnold:MathematicalMethodsOfClassicalMechanics}: A rigid body is a system of point masses, constrained by holonomic relations expressed by the fact that the distance between points is constant.
% \\
% \cite[sec.\ 4.1]{Bremer:ElasticMultibodyDynamics}: A body can be defined as a number N of particles or mass points with N going to infinity.
% \\
% \cite[sec.\ 8, § 1]{Hamel:TheoretischeMechanik}
% \\
% \cite[sec.\ 3.1.1]{Schwertassek:MultibodySystems}: definition by frames
% \\
% \cite{Kane:Dynamics} no direct definition
% \\
% \cite{Shabana:MultibodySystems} no direct definition
% \\
% \cite[p.\ 229]{Abraham:FoundationsOfMechanics} no direct definition, but ``The constrained system par excellence is the rigid body''
% }

\clearpage
\section{A single free rigid body}\label{sec:RB}
Established textbooks on physics (e.g.\ \cite[chap. 4]{Goldstein:ClassicalMechanics}, \cite[§31]{Landau:Mechanics} or \cite[§44]{Boltzmann:PrincipeDerMechanik}) define a \textit{rigid body} as a a system of a \textit{finite} number $\numParticles$ particles such that the distances $d_{\PidxI\PidxII} = \norm{\particlePos{\PidxI} - \particlePos{\PidxII}}$ between their positions $\particlePos{\PidxI}$ are constant.
Textbooks that are more focused on engineering like \cite[sec.\ 8, § 1]{Hamel:TheoretischeMechanik}, \cite[sec.\ 4.1]{Bremer:ElasticMultibodyDynamics} or \cite[sec.\ 6.1.1]{Schwertassek:MultibodySystems} rather define a rigid body as a rigid volume over which mass is continuously distributed.
Both modeling assumptions eventually lead to the same equations of motion when using the same generalized coordinates.
They differ in the computation of the inertial parameters of total mass, center of mass and moment of inertia:
The physics perspective uses a finite sum over the particles, whereas the engineering point of view requires an integral over the body volume.
This work will consider a finite number of particles.

In contrast to the sources mentioned above, this section will investigate apart from inertia and gravitation, also stiffness and damping for a rigid body.
In particular for the latter two parts, the model of concentrated particles is more intuitive in the authors humble opinion.


\subsection{Coordinates}
\begin{figure}[ht]
 \centering
 \input{graphics/RBCoordinates.pdf_tex}
 \caption{body fixed frame and particle positions}
 \label{fig:RigidBodyIllustration}
\end{figure}

\paragraph*{Body fixed frame.}
A common approach for modeling the rigid body (used in all sources above) is the use of a \textit{body fixed frame}.
This is the choice of a position $\r\in \RealNum^3$ and a triple of orthonormal, right handed vectors $[\Rx,\Ry,\Rz] = \R \in \SpecialOrthogonalGroup(3)$ which are rigidly attached to the body, i.e.\ move with it, see \autoref{fig:RigidBodyIllustration}.
With this, the position of any particle of the body may then be written as
\begin{align}\label{eq:RBParticlePos}
 \particlePos{\PidxI} = \r + \R \particleBodyPos{\PidxI}, \quad \PidxI=1,\ldots,\numParticles
\end{align}
where the relative particle positions $\particleBodyPos{\PidxI} \in \RealNum^3$ to the body fixed frame are \textit{constant}.
Consequently, the motion of the rigid body is completely captured by the position $\r(t) \in \RealNum^3$ and orientation $\R(t) \in \SpecialOrthogonalGroup(3)$ of the body fixed frame.

\paragraph*{Usefulness of these particular coordinates.}
This is an example par excellence for the use of redundant parameters as discussed in the previous chapter:
The system of particles has $\numParticles$ has $\numCoord = 3\numParticles$ coordinates, the coefficients of the particle positions $\particlePos{\PidxI}, \PidxI=1,\ldots,\numParticles$, and $\numGeoConst = \tfrac{1}{2} \numParticles(\numParticles-1)$ fixing their distances to each other.
Actually, just fixing the distances between the particles still allows a mirroring of the rigid body, which from a physical perspective is not permitted. 
To resolve this there should be additional constraints on the ``handedness'' of the particles to each other.
Even without the ``handedness'' constraints, the number of distance constraints $\numGeoConst$ surpasses the number of coordinates $\numCoord$ for larger number of particles.
Consequently, the distance constraints cannot be independent.
All these issues discourage us from working with the particle positions as configuration parameters.

On the other hand, since the configuration space of the rigid body $\configSpace \cong \RealNum^3 \times \SpecialOrthogonalGroup(3)$ contains $\SpecialOrthogonalGroup(3)$, any set of $\dimConfigSpace=\dim \configSpace = 6$ minimal generalized coordinates will lead to singularities as discussed in \autoref{sec:MotivationRigidBodyAttitude}.

This particular choice of coordinates $(\r, \R)$ may be regarded as a trade-off between these two extremes:
It uses a fixed number of $\numCoord=12$ coordinates and constraints and respects the topology of the configuration space.
Furthermore, the interpretation of $\r$ and the columns of $\R$ as a body fixed frame are quite intuitive for practical applications.

\paragraph*{Velocity.}
For parameterization of the velocity of the rigid body, we will use the body fixed velocity $\v(t)\in \RealNum^3$ and the angular velocity $\w(t)\in \RealNum^3$ which are related to the configuration by
\begin{align}\label{eq:RBKinematicEq1}
 \rd = \R \v, \qquad \Rd = \R \wedOp(\w).
\end{align}
With this we may express the velocity and accelerations of the body particles as
\begin{subequations}
\begin{align}
% \label{eq:RBParticlePos}
% \particlePos{\PidxI} &= \r + \R \particleBodyPos{\PidxI},
%\\
 \label{eq:RBParticleVel}
 \particlePosd{\PidxI} &= \R \big( \v - \wedOp(\particleBodyPos{\PidxI}) \w \big),
\\
 \label{eq:RBParticleAcc}
 \particlePosdd{\PidxI} &= \R \big( \vd - \wedOp(\particleBodyPos{\PidxI})\wdot + \wedOp(\w)(\v - \wedOp(\particleBodyPos{\PidxI})\w) \big), \quad \PidxI=1,\ldots,\numParticles
\end{align}
\end{subequations}

\paragraph*{Compliance with the framework.}
In order to comply to the framework from the previous chapter we may group the configuration coordinates as $\sysCoord = [\r^\top, \Rx^\top, \Ry^\top, \Rz^\top]^\top \in \configSpace \subset\RealNum^{12}$ and their geometric constraint as $\geoConstraint(\sysCoord) = \tuple{0}$ are the constraints of $\SpecialOrthogonalGroup(3)$ as given in \eqref{eq:constraintSO3}.
The vector form of the kinematic relation \eqref{eq:RBKinematicEq1} is 
\begin{align}\label{eq:RBKinematicEq2}
 \underbrace{\begin{bmatrix} \rd \\ \Rxd \\ \Ryd \\ \Rzd  \end{bmatrix}}_{\sysCoordd}
 = 
 \underbrace{\begin{bmatrix}
  \Rx & \Ry & \Rz & \tuple{0} & \tuple{0} & \tuple{0} \\
  \tuple{0} & \tuple{0} & \tuple{0} & \tuple{0} & -\Rz & \Ry \\
  \tuple{0} & \tuple{0} & \tuple{0} & \Rz & \tuple{0} & -\Rx \\
  \tuple{0} & \tuple{0} & \tuple{0} & -\Ry & \Rx & \tuple{0}
 \end{bmatrix}}_{\kinMat}
 \underbrace{\begin{bmatrix} \vx \\ \vy \\ \vz \\ \wx \\ \wy \\ \wz \end{bmatrix}}_{\sysVel}
 .
\end{align}

\paragraph*{Commutation coefficients.}
Plugging the kinematic matrix $\kinMat$ from \eqref{eq:RBKinematicEq2} into the definition \eqref{eq:DefCommutationCoeff} of the commutation symbols $\bodyBoltzSym{}{}{}$ yields
\begin{align}\label{eq:RBCommutationCoeff}
 \bodyBoltzSym{1}{2}{6} = \bodyBoltzSym{1}{5}{3} = 
 \bodyBoltzSym{2}{3}{4} = \bodyBoltzSym{2}{6}{1} =
 \bodyBoltzSym{3}{1}{5} = \bodyBoltzSym{3}{4}{2} =
 \bodyBoltzSym{4}{5}{6} =
 \bodyBoltzSym{5}{6}{4} = 
 \bodyBoltzSym{6}{4}{5} &= 1 \ ,
\nonumber\\
 \bodyBoltzSym{1}{6}{2} = \bodyBoltzSym{1}{3}{5} = 
 \bodyBoltzSym{2}{4}{3} = \bodyBoltzSym{2}{1}{6} =
 \bodyBoltzSym{3}{5}{1} = \bodyBoltzSym{3}{2}{4} =
 \bodyBoltzSym{4}{6}{5} =
 \bodyBoltzSym{5}{4}{6} = 
 \bodyBoltzSym{6}{5}{4} &= -1 \ 
\end{align}
and the remaining coefficients vanish.
With this we have
\begin{align}\label{eq:RBCommutationCoeffAd}
 \big[ \BoltzSym{\LidxIII}{\LidxI}{\LidxII} \sysVelCoeff{\LidxII} \big]^{\LidxIII=1\ldots6}_{\LidxI=1\ldots6} =
 \begin{bmatrix} \wedOp(\bodyAngVel{}{}) & 0 \\ \wedOp(\bodyLinVel{}{}) & \wedOp(\bodyAngVel{}{}) \end{bmatrix}
 = -\ad{\sysVel}^\top
\end{align}
whose naming will be discussed later.


\subsection{Inertia}
The previous section derived several formulations for the generalized inertial force $\bodyGenForceInertia{}{}$.
These will now be applied to the the rigid body with the chosen coordinates.

\paragraph{Kinetic energy.}
With the particle velocities $\particlePosd{\PidxI}$ in terms of the chosen coordinates \eqref{eq:RBParticleVel} we obtain the kinetic energy $\kineticEnergy$ of a free rigid body as 
\begin{align}\label{eq:RigidBodyKineticEnergy}
 \kineticEnergy &= \tfrac{1}{2} \sumParticles \particleMass{\PidxI} \norm{\overbrace{\R \big( \v - \wedOp(\particleBodyPos{\PidxI}) \w \big)}^{\particlePosd{\PidxI}}}^2
\nonumber\\[1ex]
 &= \tfrac{1}{2} \underbrace{\sumParticles \particleMass{\PidxI}}_{\bodyMass{}{}} \norm{\v}^2
  - \v^\top \underbrace{\sumParticles \particleMass{\PidxI} \wedOp(\particleBodyPos{\PidxI})}_{\bodyMass{}{} \wedOp(\bodyCOM{}{})} \w
  + \tfrac{1}{2} \w^\top \underbrace{\sumParticles \particleMass{\PidxI} \wedOp(\particleBodyPos{\PidxI})^\top \wedOp(\particleBodyPos{\PidxI})}_{\bodyMOI{}{}} \w
\nonumber\\
 &= \tfrac{1}{2} \underbrace{\big[ \v^\top \ \w^\top \big]}_{\bodyVel{}{}^\top}
 \underbrace{\begin{bmatrix} \bodyMass{}{}\idMat[3] & \bodyMass{}{} \wedOp(\bodyCOM{}{})^\top \\ \bodyMass{}{} \wedOp(\bodyCOM{}{}) & \bodyMOI{}{} \end{bmatrix}}_{\bodyInertiaMat{}{}}
 \underbrace{\begin{bmatrix} \v \\ \w \end{bmatrix}}_{\bodyVel{}{}}
 .
\end{align}
Here we have substituted some well established inertia parameters: the total mass $\bodyMass{}{}$, the center of mass $\bodyCOM{}{} = \bodyMass{}{}^{-1} \sumParticles \particleMass{\PidxI} \particleBodyPos{\PidxI}$ and the moment of inertia $\bodyMOI{}{} = \bodyMOI{}{}^\top$.
Assuming that the particle masses are positive $\particleMass{\PidxI} > 0, \PidxI=1,\ldots,\numParticles$ implies that the total mass is positive $\bodyMass{}{} > 0$.
Furthermore, if the rigid body has at least three particles that do not lie on a line, the inertia matrix is positive definite $\bodyMOI{}{} > 0$.
It is important to notice that the inertia matrix $\bodyInertiaMat{}{}$ for the chosen coordinates is \textit{constant}%
\footnote{
One reason for the choice of $\v$ as velocity coordinates it the fact that the inertia matrix $\bodyInertiaMat{}{}$ is \textit{constant}.
If we choose instead $\rd$ as velocity coordinates we have
\begin{align*}
 \kineticEnergy = \tfrac{1}{2} [\rd^\top, \w^\top] \begin{bmatrix} \bodyMass{}{} \idMat[3] & \bodyMass{}{} \R \wedOp(\bodyCOM{}{})^\top \\ \bodyMass{}{} \wedOp(\bodyCOM{}{}) \R^\top & \bodyMOI{}{} \end{bmatrix} \begin{bmatrix} \rd \\ \w \end{bmatrix}.
\end{align*}
Obviously the body inertia matrix depends on the orientation $\R$ of the body and is not constant unless the reference position $\r$ coincides with the \textit{center of mass}, i.e.\ $\bodyCOM{}{}{} = 0$.
Actually many textbooks, e.g.\ \cite[p.\,167]{Murray:Robotic} or \cite[p.\,153]{Shabana:MultibodySystems}, restrict to this case for their expressions of the kinetic energy.
In the next section on rigid body systems we will see that it can be quite useful to use \textit{geometrically} meaningful body fixed frames rather than restricting to the center of mass.
}.

Plugging the the kinetic energy \eqref{eq:RigidBodyKineticEnergy} into the corresponding formulation \eqref{eq:InertiaForceKinEnergy} of the generalized inertia force and using the commutation symbols from \eqref{eq:RBCommutationCoeffAd} yields
\begin{align}\label{eq:InertiaForceRB}
 \genForceInertia &=
 \underbrace{\begin{bmatrix} \bodyMass{}{}\idMat[3] & \bodyMass{}{} \wedOp(\bodyCOM{}{})^\top \\ \bodyMass{}{} \wedOp(\bodyCOM{}{}) & \bodyMOI{}{} \end{bmatrix}}_{\bodyInertiaMat{}{}}
 \underbrace{\begin{bmatrix} \vd \\ \dot{\w} \end{bmatrix}}_{\bodyVeld{}{}}
 +
 \underbrace{\begin{bmatrix} \wedOp(\bodyAngVel{}{}) & 0 \\ \wedOp(\bodyLinVel{}{}) & \wedOp(\bodyAngVel{}{}) \end{bmatrix}}_{-\ad{\sysVel}^\top}
 \underbrace{\begin{bmatrix} \bodyMass{}{}\idMat[3] & \bodyMass{}{} \wedOp(\bodyCOM{}{})^\top \\ \bodyMass{}{} \wedOp(\bodyCOM{}{}) & \bodyMOI{}{} \end{bmatrix}}_{\bodyInertiaMat{}{}}
 \underbrace{\begin{bmatrix} \v \\ \w \end{bmatrix}}_{\bodyVel{}{}}
\end{align}

\paragraph{Acceleration energy.}
With the particle accelerations $\particlePosdd{\PidxI}$ in terms of the coordinates \eqref{eq:RBParticleAcc} and using the Jacobi identity \eqref{eq:JacobiIdentitiySO3} we find the acceleration energy $\accEnergy$ for the free rigid body as
%\footnote{using the Jacobi identity $\tuple{a},\tuple{b},\tuple{c} \in \RealNum^3$: $\wedOp(\tuple{a}) \wedOp(\tuple{b}) \tuple{c} + \wedOp(\tuple{b}) \wedOp(\tuple{c}) \tuple{a} + \wedOp(\tuple{c}) \wedOp(\tuple{a}) \tuple{b} = \tuple{0}$}
\begin{align}\label{eq:RigidBodyAccEnergy}
 \accEnergy &= \tfrac{1}{2} \sumParticles \particleMass{\PidxI} \norm{\overbrace{\R \big( \vd - \wedOp(\particleBodyPos{\PidxI})\wdot + \wedOp(\w)(\v - \wedOp(\particleBodyPos{\PidxI})\w) \big)}^{\particlePosdd{\PidxI}}}^2
\nonumber\\[1ex]
 &= \tfrac{1}{2} \underbrace{\sumParticles \particleMass{\PidxI}}_{\bodyMass{}{}} \norm{\vd}^2
  - \vd^\top \underbrace{\sumParticles \particleMass{\PidxI} \wedOp(\particleBodyPos{\PidxI})}_{\bodyMass{}{} \wedOp(\bodyCOM{}{})} \wdot
  + \tfrac{1}{2} \wdot^\top \underbrace{\sumParticles \particleMass{\PidxI} \wedOp(\particleBodyPos{\PidxI})^\top \wedOp(\particleBodyPos{\PidxI})}_{\bodyMOI{}{}} \wdot
\nonumber\\
 &\qquad+ \vd^\top \wedOp(\w) \Big( \underbrace{\sumParticles \particleMass{\PidxI}}_{\bodyMass{}{}}\v - \underbrace{\sumParticles \particleMass{\PidxI} \wedOp(\particleBodyPos{\PidxI})}_{\bodyMass{}{} \wedOp(\bodyCOM{}{})} \w \Big)
\nonumber\\
 &\qquad+ \wdot^\top \Big( \underbrace{\sumParticles \particleMass{\PidxI} \wedOp(\particleBodyPos{\PidxI})}_{\bodyMass{}{} \wedOp(\bodyCOM{}{})} \wedOp(\w) \v +  \wedOp(\w) \underbrace{\sumParticles \particleMass{\PidxI} \wedOp(\particleBodyPos{\PidxI})^\top \wedOp(\particleBodyPos{\PidxI})}_{\bodyMOI{}{}} \w \Big)
\nonumber\\
 &\qquad+ \tfrac{1}{2} \underbrace{\sumParticles \particleMass{\PidxI}}_{\bodyMass{}{}} \norm{\wedOp(\w)\v}^2
 + \v^\top \wedOp(\w)^2 \underbrace{\sumParticles \particleMass{\PidxI} \wedOp(\particleBodyPos{\PidxI})}_{\bodyMass{}{} \wedOp(\bodyCOM{}{})} \w
\nonumber\\
 &\qquad+ \tfrac{1}{2} \tr \Big( \underbrace{\sumParticles \particleMass{\PidxI} \particleBodyPos{\PidxI} \particleBodyPos{\PidxI}^\top}_{\bodyMOIp{}{} = \wedMatOp(\bodyMOI{}{})} \wedOp(\w)^4 \Big)
\end{align}
Not at all surprisingly, we found the same inertia parameters $\bodyMass{}{}$, $\bodyCOM{}{}$ and $\bodyMOI{}{}$ as for the kinetic energy $\kineticEnergy$ in \eqref{eq:RigidBodyKineticEnergy}.
Collecting these further in the inertia matrix $\bodyInertiaMat{}{}$ we have
\begin{multline}\label{eq:RigidBodyAccEnergyMatrix}
 \accEnergy = \tfrac{1}{2} \underbrace{\big[ \vd^\top \ \wdot^\top \big]}_{\bodyVeld{}{}^\top}
 \underbrace{\begin{bmatrix} \bodyMass{}{}\idMat[3] & \bodyMass{}{} \wedOp(\bodyCOM{}{})^\top \\ \bodyMass{}{} \wedOp(\bodyCOM{}{}) & \bodyMOI{}{} \end{bmatrix}}_{\bodyInertiaMat{}{}}
 \underbrace{\begin{bmatrix} \vd \\ \wdot \end{bmatrix}}_{\bodyVeld{}{}}
\\
 + \underbrace{\big[ \vd^\top \ \wdot^\top \big]}_{\bodyVeld{}{}^\top} 
% \underbrace{\begin{bmatrix} \wedOp(\w) \bodyMass{}{}(\v - \wedOp(\bodyCOM{}{})\w) \\ \bodyMass{}{} \wedOp(\bodyCOM{}{}) \wedOp(\w) \v + \wedOp(\w)\bodyMOI{}{} \w \end{bmatrix}}_{\gyroForce}
 \underbrace{\begin{bmatrix} \wedOp(\w) & \mat{0} \\ \wedOp(\v) & \wedOp(\w) \end{bmatrix}}_{-\ad{\bodyVel{}{}}^\top} 
 \underbrace{\begin{bmatrix} \bodyMass{}{}\idMat[3] & \bodyMass{}{} \wedOp(\bodyCOM{}{})^\top \\ \bodyMass{}{} \wedOp(\bodyCOM{}{}) & \bodyMOI{}{} \end{bmatrix}}_{\bodyInertiaMat{}{}} 
 \underbrace{\begin{bmatrix} \v \\ \w \end{bmatrix}}_{\bodyVel{}{}}
\\
 + \underbrace{\tfrac{1}{2} \bodyMass{}{} \norm{\wedOp(\w)\v}^2 + \v^\top \wedOp(\w)^2 \bodyMass{}{} \wedOp(\bodyCOM{}{}) \w + \tfrac{1}{2} \tr \big( \bodyMOIp{}{} \wedOp(\w)^4 \big)}_{\accEnergy_0}
 .
\end{multline}
Plugging this into the corresponding formulation \eqref{eq:InertiaForceAccEnergy} of the generalized inertia force, i.e.\ $\bodyGenForceInertia{}{} = \spdiff[\accEnergy]{\bodyVeld{}{}}$, we obviously find the same expression as above \eqref{eq:InertiaForceRB}.
Note that $\accEnergy_0$ is independent of the generalized acceleration $\bodyVeld{}{}$ and consequently does not contribute to the inertia force.


\paragraph{Inertia matrix and connection coefficients.}
We can compute the Jacobian of the particle positions from
\begin{align}
 \differential \particlePos{\PidxI}
 =\! \Big[ \dirDiff{\LidxI} \particlePos{\PidxI} \Big]_{\LidxI=1,\ldots,6}\!\!
 =\! \Big[ \pdiff[\particlePosd{\PidxI}]{\sysVelCoeff{\LidxI}} \Big]_{\LidxI=1,\ldots,6}\!\!
% = \pdiff[\particlePosd{\PidxI}]{\sysVel}
% = \begin{bmatrix} \pdiff[\particlePosd{\PidxI}]{\v} \\ \pdiff[\particlePosd{\PidxI}]{\w} \end{bmatrix}
 = \begin{bmatrix} \pdiff{\v} \\ \pdiff{\w} \end{bmatrix}^{\!\top} \!\overbrace{\R \big( \v - \wedOp(\particleBodyPos{\PidxI}) \w \big)}^{\particlePosd{\PidxI}}
% = \begin{bmatrix} \R \\ -\R \wedOp(\particleBodyPos{\PidxI}) \end{bmatrix}^\top
 = \R \begin{bmatrix} \idMat[3] \ \wedOp(\particleBodyPos{\PidxI})^{\!\top} \end{bmatrix}\!.
\end{align}
With this, the rigid body inertia matrix may be written as
\begin{multline}
 \sysInertiaMat = \Big[ \sumParticles \particleMass{\PidxI} \sProd{\dirDiff{\LidxI} \particlePos{\PidxI}}{\dirDiff{\LidxII} \particlePos{\PidxI}} \Big]_{\LidxI,\LidxII=1,\ldots,6}
 = \sumParticles \particleMass{\PidxI} (\differential \particlePos{\PidxI})^\top \differential \particlePos{\PidxI}
\\
 = \sumParticles \particleMass{\PidxI} \begin{bmatrix} \idMat[3] & \wedOp(\particleBodyPos{\PidxI})^\top \\ \wedOp(\particleBodyPos{\PidxI}) & \wedOp(\particleBodyPos{\PidxI})\wedOp(\particleBodyPos{\PidxI})^\top \end{bmatrix}
 = \begin{bmatrix} \bodyMass{}{}\idMat[3] & \bodyMass{}{} \wedOp(\bodyCOM{}{})^\top \\ \bodyMass{}{} \wedOp(\bodyCOM{}{}) & \bodyMOI{}{} \end{bmatrix}
\end{multline}
which obviously coincides with what we found from the kinetic energy \eqref{eq:RigidBodyKineticEnergy} and from the acceleration energy \eqref{eq:RigidBodyAccEnergyMatrix}.

As already pointed out above, for the chosen velocity coordinates $\sysVel = (\v,\w)$, the coefficients of the rigid body inertia matrix $\sysInertiaMatCoeff{\LidxI\LidxII}$ are constant.  
Consequently, since $\dirDiff{\LidxIII}\sysInertiaMatCoeff{\LidxI\LidxII} \equiv 0$, the corresponding connection coefficients $\bodyConnCoeffL{}{}{\LidxI}{\LidxII}{\LidxIII}$ only consist of the terms with the commutation coefficients $\bodyBoltzSym{}{}{}$:
\begin{align}\label{eq:RBConnCoeff}
 \bodyConnCoeffL{}{}{\LidxI}{\LidxII}{\LidxIII} &= \tfrac{1}{2}\big(\bodyBoltzSym{\LidxV}{\LidxI}{\LidxII} \bodyInertiaMatCoeff{}{}{\LidxV\LidxIII} + \bodyBoltzSym{\LidxV}{\LidxI}{\LidxIII} \bodyInertiaMatCoeff{}{}{\LidxV\LidxII} - \bodyBoltzSym{\LidxV}{\LidxII}{\LidxIII} \bodyInertiaMatCoeff{}{}{\LidxV\LidxI} \big)
 = -\bodyConnCoeffL{}{}{\LidxII}{\LidxI}{\LidxIII}.
\end{align}
Using the commutation coefficients $\bodyBoltzSym{}{}{}$ given in \eqref{eq:RBCommutationCoeff} and taking into account the skew symmetry above, the non-zero connection coefficients are
\begin{subequations}\label{eq:RBConnCoeffRes}
\begin{align}
 \bodyConnCoeffL{}{}{3}{2}{4} = \bodyConnCoeffL{}{}{1}{3}{5} = \bodyConnCoeffL{}{}{2}{1}{6} \ &= \ \bodyMass{}{},
\\
 \bodyConnCoeffL{}{}{2}{5}{4} = \bodyConnCoeffL{}{}{3}{6}{4} = \bodyConnCoeffL{}{}{5}{1}{5} = \bodyConnCoeffL{}{}{6}{1}{6} \ &= \ \bodyMass{}{} \bodyCOMCoeff{}{}{\idxX},
\\
 \bodyConnCoeffL{}{}{4}{2}{4} = \bodyConnCoeffL{}{}{1}{4}{5} = \bodyConnCoeffL{}{}{3}{6}{5} = \bodyConnCoeffL{}{}{6}{2}{6} \ &= \ \bodyMass{}{} \bodyCOMCoeff{}{}{\idxY},
\\
 \bodyConnCoeffL{}{}{4}{3}{4} = \bodyConnCoeffL{}{}{5}{3}{5} = \bodyConnCoeffL{}{}{1}{4}{6} = \bodyConnCoeffL{}{}{2}{5}{6} \ &= \ \bodyMass{}{} \bodyCOMCoeff{}{}{\idxZ},
\\
 \bodyConnCoeffL{}{}{6}{5}{4} \ &= \ \bodyMOICoeffp{}{}{\idxX\idxX} = \tfrac{1}{2}\big( \bodyMOICoeff{}{}{\idxY\idxY} + \bodyMOICoeff{}{}{\idxZ\idxZ} - \bodyMOICoeff{}{}{\idxX\idxX} \big),
\\
 \bodyConnCoeffL{}{}{4}{6}{5} \ &= \ \bodyMOICoeffp{}{}{\idxY\idxY} = \tfrac{1}{2}\big( \bodyMOICoeff{}{}{\idxX\idxX} + \bodyMOICoeff{}{}{\idxZ\idxZ} - \bodyMOICoeff{}{}{\idxY\idxY} \big),
\\
 \bodyConnCoeffL{}{}{5}{4}{6} \ &= \ \bodyMOICoeffp{}{}{\idxZ\idxZ} = \tfrac{1}{2}\big( \bodyMOICoeff{}{}{\idxX\idxX} + \bodyMOICoeff{}{}{\idxY\idxY} - \bodyMOICoeff{}{}{\idxZ\idxZ} \big),
\\
 \bodyConnCoeffL{}{}{4}{6}{4} = \bodyConnCoeffL{}{}{6}{5}{5} \ &= \ \bodyMOICoeffp{}{}{\idxX\idxY} = -\bodyMOICoeff{}{}{\idxX\idxY},
\\
 \bodyConnCoeffL{}{}{5}{4}{4} = \bodyConnCoeffL{}{}{6}{5}{6} \ &= \ \bodyMOICoeffp{}{}{\idxX\idxZ} = -\bodyMOICoeff{}{}{\idxX\idxZ},
\\
 \bodyConnCoeffL{}{}{5}{4}{5} = \bodyConnCoeffL{}{}{4}{6}{6} \ &= \ \bodyMOICoeffp{}{}{\idxY\idxZ} = -\bodyMOICoeff{}{}{\idxY\idxZ}.
\end{align}
\end{subequations}
Note that the quantity $\bodyMOIp{}{} = \wedMatOp(\bodyMOI{}{}) = \tfrac{1}{2} \tr(\bodyMOI{}{})\idMat[3] - \bodyMOI{}{}$ also appeared above in the formulation of the acceleration energy \eqref{eq:RigidBodyAccEnergy}.

Finally, assembling the terms $\genForceInertiaCoeff{\LidxI} = \sysInertiaMatCoeff{\LidxI\LidxII} \sysVelCoeffd{\LidxII} + \ConnCoeffL{\LidxI}{\LidxII}{\LidxIII} \sysVelCoeff{\LidxII}\sysVelCoeff{\LidxIII}$ we may check that this is indeed identical to \eqref{eq:InertiaForceRB}.

% In addition to the symmetry of the connection coefficients for the rigid body we have
% \begin{align}
%  \gyroForceCoeff{\LidxI}
%  = \bodyConnCoeffL{}{}{\LidxI}{\LidxII}{\LidxIII} \sysVelCoeff{\LidxII} \sysVelCoeff{\LidxIII}
%  = \tfrac{1}{2}\big(\bodyBoltzSym{\LidxV}{\LidxI}{\LidxII} \bodyInertiaMatCoeff{}{}{\LidxV\LidxIII} + \bodyBoltzSym{\LidxV}{\LidxI}{\LidxIII} \bodyInertiaMatCoeff{}{}{\LidxV\LidxII} - \bodyBoltzSym{\LidxV}{\LidxII}{\LidxIII} \bodyInertiaMatCoeff{}{}{\LidxV\LidxI} \big) \sysVelCoeff{\LidxII} \sysVelCoeff{\LidxIII}
%  = \bodyBoltzSym{\LidxV}{\LidxI}{\LidxII} \bodyInertiaMatCoeff{}{}{\LidxV\LidxIII} \sysVelCoeff{\LidxII} \sysVelCoeff{\LidxIII}
% \end{align}

\subsection{Gravitation}
The potential energy $\potentialGravity$ of a rigid body due to a gravitational acceleration $\gravityAcc$ according to \eqref{eq:ParticleSysGravity} in terms of the chosen coordinates is
\begin{align}\label{eq:RBPotentialEnergyGravity}
 \potentialGravity 
 = \sumParticles \sProd{\overbrace{\r + \R \particleBodyPos{\PidxI}}^{\particlePos{\PidxI}}}{-\particleMass{\PidxI} \gravityAcc}
 = -\sProd{\underbrace{\sumParticles\! \particleMass{\PidxI}}_{\bodyMass{}{}} \r + \R \underbrace{\sumParticles\! \particleMass{\PidxI} \particleBodyPos{\PidxI}}_{\bodyMass{}{}\bodyCOM{}{}}}{\gravityAcc}
 = -\bodyMass{}{} \sProd{\r + \R \bodyCOM{}{}}{\gravityAcc}.
\end{align}
Note that the parameters of total mass $\bodyMass{}{}$ and center of mass $\bodyCOM{}{}$ are the same as found above for the inertia matrix.
The resulting generalized force is
\begin{align}
 \bodyGenForceGravity{}{} = \differential \potentialGravity = -\bodyMass{}{} \begin{bmatrix} \R^\top \gravityAcc \\ \wedOp(\bodyCOM{}{}) \R^\top \gravityAcc \end{bmatrix}.
\end{align}


\subsection{Stiffness}\label{sec:RBStiffness}
\begin{figure}[ht]
 \centering
 \input{graphics/RBStiffness.pdf_tex}
 \caption{springs attached to a rigid body}
 \label{fig:RBStiffness}
\end{figure}
Assume that every particle of the rigid body with position $\particlePos{\PidxI}$ is connected to a position $\springHubPos{\PidxI} \in \RealNum^3$ by a linear spring with stiffness $\particleStiffness{\PidxI} \in\RealNum^{(+)}$, see \autoref{fig:RBStiffness}.
The resulting potential energy in terms of the rigid body coordinates $\sysCoord \cong (\r, \R)$ is
\begin{align}\label{eq:RBStiffnessPotential}
 \potentialStiff(\sysCoord) &= \tfrac{1}{2} \sumParticles \particleStiffness{\PidxI} \norm{\r + \R \particleBodyPos{\PidxI} - \springHubPos{\PidxI}}^2.
\end{align}

% \paragraph{Some identities.}
% In the following we use some basic identities
% \begin{subequations}
% \begin{align}
%  \tuple{a}, \tuple{b} \in \RealNum^{n}  \ : \ &\tuple{a}^\top \tuple{b} = \tr(\tuple{a}\tuple{b}^\top)
% \\
%  \tuple{a}, \tuple{b} \in \RealNum^{3}  \ : \ &\wedOp(\tuple{a}) \tuple{b} = \veeTwoOp(\tuple{b} \tuple{a}^\top)
% \\
%  &\wedOp(\tuple{a}) \wedOp(\tuple{b}) = \tuple{b} \tuple{a}^\top - (\tuple{b}^\top \tuple{a}) \idMat[3]
% \\
%  \mat{A}\in\RealNum^{3\times3}, \tuple{b}\in\RealNum^3 \ : \ &\tr(\mat{A} \wedOp(\tuple{b})) = -\tuple{b}^\top \veeTwoOp(\mat{A})
% \\
%  &\veeTwoOp(\wedOp(\tuple{b}) \mat{A}) = (\tr(\mat{A})\idMat[3] - \mat{A}) \tuple{b}
% \end{align} 
% \end{subequations}
% Each may be checked by direct computation.

\paragraph{Stiffness parameters.}
Using the identities above we may rearrange \eqref{eq:RBStiffnessPotential} to
\begin{align}\label{eq:RBStiffnessPotential2}
 \potentialStiff(\sysCoord) &= \tfrac{1}{2} \sumParticles \particleStiffness{\PidxI} \norm{\r + \R \particleBodyPos{\PidxI} - \springHubPos{\PidxI}}^2
\nonumber\\
 &= \tfrac{1}{2} \sumParticles \particleStiffness{\PidxI} \big( \norm{\r}^2 + \underbrace{\norm{\R \particleBodyPos{\PidxI}}^2}_{\const} + \underbrace{\norm{\springHubPos{\PidxI}}^2}_{\const} + 2\sProd{\r}{\R \particleBodyPos{\PidxI}} - 2\sProd{\r}{\springHubPos{\PidxI}} - 2\sProd{\R \particleBodyPos{\PidxI}}{\springHubPos{\PidxI}}\big)
\nonumber\\
 &= \tfrac{1}{2} \bodyStiffness{}{}\norm{\r}^2 + \bodyStiffness{}{} \sProd{\r}{\R \bodyCOS{}{}} - \bodyStiffness{}{} \sProd{\r}{\tuple{p}} - \tr(\mat{P} \R)
  + \underbrace{\tfrac{1}{2} \sumParticles \particleStiffness{\PidxI} \big( \norm{\particleBodyPos{\PidxI}}^2 + \norm{\springHubPos{\PidxI}}^2)}_{\potentialStiff_c\,=\,\const}
\end{align}
with substitution of the constant parameters
\begin{align}
 \bodyStiffness{}{} &= \sumParticles \particleStiffness{\PidxI},&
 \bodyCOS{}{} &= \bodyStiffness{}{}^{-1} \sumParticles \particleStiffness{\PidxI} \particleBodyPos{\PidxI},&
 \tuple{p} &= \bodyStiffness{}{}^{-1} \sumParticles \particleStiffness{\PidxI} \springHubPos{\PidxI},&
 \mat{P} &= \sumParticles \particleStiffness{\PidxI} \particleBodyPos{\PidxI} \springHubPos{\PidxI}^\top.
\end{align}
Note that there was no specific assumption for the particle and spring distribution, i.e.\ on the values of $\particleBodyPos{\PidxI}$, $\springHubPos{\PidxI}$ and $\particleStiffness{\PidxI}$.
Consequently, any constellation may be captured by the $1+3+3+9+1=17$ parameters within $(\bodyStiffness{}{}, \bodyCOS{}{}, \tuple{p}, \mat{P}, \potentialStiff_c)$.

\paragraph{Critical points.}
% The time derivatives of the potential may be written as
% \begin{align}\label{eq:RBStiffnessPotentialDifferential}
%  \tdiff{t} \potentialStiff &= \bodyStiffness{}{} \sProd{\r}{\R\v} + \bodyStiffness{}{} \sProd{\R\v}{\R \bodyCOS{}{}} + \bodyStiffness{}{} \sProd{\r}{\R\wedOp(\w) \bodyCOS{}{}} - \bodyStiffness{}{} \sProd{\R\v}{\tuple{p}} - \tr(\mat{P} \R \wedOp(\w))
% \nonumber\\
%  &= \sysVel^\top \underbrace{\begin{bmatrix} \bodyStiffness{}{} (\R^\top (\r - \tuple{p}) + \bodyCOS{}{}) \\ \bodyStiffness{}{} \wedOp(\bodyCOS{}{}) \R^\top \r + \veeTwoOp(\mat{P} \R) \end{bmatrix}}_{\differential \potentialStiff}
% \\
%  \tfrac{\d^2}{\d t^2} \potentialStiff 
%  &= \sysVeld^\top \differential \potentialStiff + \sysVel^\top \underbrace{\begin{bmatrix} \bodyStiffness{}{} \idMat[3] & \bodyStiffness{}{}\wedOp\big(\R^\top (\r - \tuple{p})\big) \\ \bodyStiffness{}{}\wedOp(\bodyCOS{}{}) & \bodyStiffness{}{} \wedOp(\bodyCOS{}{}) \wedOp(\R^\top \r) + \tr(\mat{P} \R)\idMat[3] - (\mat{P} \R)^\top \end{bmatrix}}_{\differential^2 \potentialStiff} \sysVel
% \end{align}
We are interested in configurations $\sysCoordR\cong (\rR, \RR)$ at which the potential is stationary, i.e.
\begin{align}\label{eq:RBStiffnessPotentialDifferential}
\differential \potentialStiff
 = \begin{bmatrix} \bodyStiffness{}{} (\R^\top (\r - \tuple{p}) + \bodyCOS{}{}) \\ \bodyStiffness{}{} \wedOp(\bodyCOS{}{}) \R^\top \r + \veeTwoOp(\mat{P} \R) \end{bmatrix} = \tuple{0}
 .
\end{align}
From the upper part of \eqref{eq:RBStiffnessPotentialDifferential} we get the condition
\begin{align}
 \rR = \tuple{p} - \RR \bodyCOS{}{}.
\end{align}
Plugging this into the lower part of \eqref{eq:RBStiffnessPotentialDifferential} we obtain
\begin{align}
 \bodyStiffness{}{} \wedOp(\bodyCOS{}{}) \RR^\top (\tuple{p} - \RR \bodyCOS{}{}) + \veeTwoOp(\mat{P} \RR)
% = \bodyStiffness{}{} \wedOp(\bodyCOS{}{}) \RR^\top \tuple{p} + 2\veeOp(\mat{P} \RR)
 = \veeTwoOp(\underbrace{(\mat{P} - \bodyStiffness{}{}\bodyCOS{}{}\tuple{p}^\top)}_{\mat{P}_{\!s}} \RR ) = \tuple{0}.
\end{align}
The solution to this subproblem $\veeTwoOp(\mat{P}_{\!s} \RR) = \tuple{0}, \RR\in\SpecialOrthogonalGroup(3)$ is discussed in great detail in \autoref{sec:AppendixAttitudePotential}:
Let $\mat{P}_{\!s}^\top = \mat{X} \wedMatOp(\bodyMOSs{}{})$ with $\mat{X}\in\SpecialOrthogonalGroup(3)$, $\bodyMOSs{}{}\in\SymMatSP(3)$ be a \textit{special polar decomposition}.
Then $\RR = \mat{X}$ is clearly a critical point.
Plugging $\mat{P} = \wedMatOp(\bodyMOSs{}{}) \RR^\top + \bodyStiffness{}{}\bodyCOS{}{}\tuple{p}^\top$ into the Hessian matrix, we have
\begin{align}
 \differential^2 \potentialStiff(\sysCoordR) 
 &= \begin{bmatrix} \bodyStiffness{}{} \idMat[3] & \bodyStiffness{}{}\wedOp(\bodyCOS{}{})^\top \\ \bodyStiffness{}{}\wedOp(\bodyCOS{}{}) & \bodyMOSs{}{} - \bodyStiffness{}{} \wedOp(\bodyCOS{}{})^2 \end{bmatrix}
\nonumber\\
 &= \begin{bmatrix} \idMat[3] & \mat{0} \\ \wedOp(\bodyCOS{}{}) & \idMat[3] \end{bmatrix}
 \begin{bmatrix} 
  \bodyStiffness{}{} \idMat[3] & \mat{0} \\
  \mat{0} & \bodyMOSs{}{}
 \end{bmatrix}
 \begin{bmatrix} \idMat[3] & \wedOp(\bodyCOS{}{})^\top \\ \mat{0} & \idMat[3] \end{bmatrix}
\end{align}
By Sylvester's law of inertia, the definiteness of $\differential^2 \potentialStiff(\sysCoordR)$ coincides with the definiteness of $\bodyStiffness{}{} \geq 0$ and $\bodyMOSs{}{} \geq 0$.
Using the results from \autoref{sec:AppendixAttitudePotential} we may conclude that $\sysCoordR$ is a minimum, and it is strict and global, if, and only if, $\bodyStiffness{}{} > 0$ and $\bodyMOSs{}{} > 0$.

\paragraph{Stiffness parameters cont'd.}
We may express the parameters $\tuple{p}$ and $\mat{P}$ in terms of the minimum configuration $(\rR, \RR)$ and the matrix $\bodyMOSs{}{}$ as
\begin{align}
 \tuple{p} &= \rR + \RR \bodyCOS{}{},&
 \mat{P} &= \wedMatOp(\bodyMOSs{}{}) \RR^\top + \bodyStiffness{}{}\bodyCOS{}{}(\rR + \RR \bodyCOS{}{})^\top
\end{align}
Plugging this into \eqref{eq:RBStiffnessPotential2} we may reformulate the potential energy as
\begin{align}\label{eq:RBStiffnessPotential3}
 \potentialStiff(\sysCoord) &= \tfrac{1}{2} \bodyStiffness{}{}\norm{\r}^2 + \bodyStiffness{}{} \sProd{\r}{\R \bodyCOS{}{}} - \bodyStiffness{}{} \sProd{\r}{\rR + \RR \bodyCOS{}{}} - \bodyStiffness{}{}\sProd{\rR + \RR \bodyCOS{}{}}{\R\bodyCOS{}{}}
\nonumber\\
 &\qquad -\tr(\wedMatOp(\bodyMOSs{}{}) \RR^\top \R) + \tfrac{1}{2} \sumParticles \particleStiffness{\PidxI} \big( \norm{\particleBodyPos{\PidxI}}^2 + \norm{\springHubPos{\PidxI}}^2)
\nonumber\\
 &= \tfrac{1}{2} \bodyStiffness{}{}\norm{\r+\R\bodyCOS{}{} - (\rR+\RR\bodyCOS{}{})}^2 + \tr\big(\wedMatOp(\bodyMOSs{}{}) (\idMat[3] - \RR^\top \R)\big)
\nonumber\\
 &\qquad \underbrace{-\tfrac{1}{2} \bodyStiffness{}{}\norm{\rR+\RR\bodyCOS{}{}}^2 - \tfrac{1}{2} \bodyStiffness{}{}\norm{\bodyCOS{}{}}^2 - \tr\big(\wedMatOp(\bodyMOSs{}{})\big)+ \tfrac{1}{2} \sumParticles \particleStiffness{\PidxI} \big( \norm{\particleBodyPos{\PidxI}}^2 + \norm{\springHubPos{\PidxI}}^2)}_{\potentialStiff_0=\potentialStiff(\sysCoordR)}
\nonumber\\
%  &= \tfrac{1}{2} \bodyStiffness{}{} \norm{\r - \rR}^2 
%  + \bodyStiffness{}{} \sProd{\r-\rR}{(\R-\RR)\bodyCOS{}{}}
%  + \tr{\bodyStiffness{}{} \bodyCOS{}{} \bodyCOS{}{}^\top (\idMat[3] - \RR^\top\R)}
%  + \tr\big(\wedMatOp(\bodyMOSs{}{}) (\idMat[3] - \RR^\top \R)\big)
%  + \potentialStiff_0
% \\
 &= \tfrac{1}{2} \bodyStiffness{}{} \norm{\r - \rR}^2 
 + \bodyStiffness{}{} \sProd{\r-\rR}{(\R-\RR)\bodyCOS{}{}}
 + \tr\big(\wedMatOp(\bodyMOS{}{}) (\idMat[3] - \RR^\top \R)\big)
 + \potentialStiff_0
\end{align}
where $\bodyMOS{}{} = \bodyMOSs{}{} + \bodyStiffness{}{} \wedOp(\bodyCOS{}{}) \wedOp(\bodyCOS{}{})^\top$ and  $\potentialStiff_0$ is the minimal potential, i.e.\ the potential of the residual displacement of the springs at the minimum $\sysCoordR$.
The differential may be written as
\begin{align}\label{eq:RBStiffnessForce}
 \differential\potentialStiff(\sysCoord) = \begin{bmatrix} \bodyStiffness{}{} \R^\top (\r - \rR) + (\idMat[3] - \R^\top\RR) \bodyStiffness{}{}\bodyCOS{}{} \\ \bodyStiffness{}{}\wedOp(\bodyCOS{}{}) \R^\top (\r-\rR) + \veeTwoOp\big(\wedMatOp(\bodyMOS{}{}) \RR^\top\R\big) \end{bmatrix}.
\end{align}
The Hessian at the minimum is
\begin{align} 
 \differential^2\potentialStiff(\sysCoordR) &= \begin{bmatrix} \bodyStiffness{}{}\idMat[3] & \bodyStiffness{}{} \wedOp(\bodyCOS{}{})^\top \\ \bodyStiffness{}{}\wedOp(\bodyCOS{}{}) & \bodyMOS{}{} \end{bmatrix} \geq 0
\end{align}
%where $\bodyMOS{}{} = \tr(\bodyMOSp{}{})\idMat[3]-\bodyMOSp{}{} = \bodyMOSs{}{} + \bodyStiffness{}{} \wedOp(\bodyCOS{}{}) \wedOp(\bodyCOS{}{})^\top \in \SymMatSP(3)$.


\paragraph{Conclusion.}
The conclusion of this subsection is that any constellation of linear springs attached to a rigid body may be captured by the potential $\potentialStiff$ from \eqref{eq:RBStiffnessPotential3} and the resulting force $\genForceStiff = \differential\potentialStiff$ from \eqref{eq:RBStiffnessForce}.
It is parameterized by 6 parameters within $(\rR,\RR) \in \RealNum^3 \times \SpecialOrthogonalGroup(3)$ which describe the configuration at the minimum, the $1+3+6=10$ parameters within $\bodyStiffness{}{}, \in\RealNum^+_0$, $\bodyCOS{}{} \in \RealNum^3$ and $\bodyMOS{}{} \in \SymMatSP(3)$, and the minimum $\potentialStiff_0$: $\potentialStiff(\sysCoord) \geq \potentialStiff_0 \geq 0$.
The minimum is strict and global if, and only if, $\bodyStiffness{}{} > 0$ and $\bodyMOSs{}{} = \bodyMOS{}{} - \bodyStiffness{}{} \wedOp(\bodyCOS{}{}) \wedOp(\bodyCOS{}{})^\top > 0$.

The rigid body stiffness matrix $\bodyStiffMat{}{} = \differential^2\potentialStiff(\sysCoordR)$ has the same structure as the inertia matrix $\bodyInertiaMat{}{} = \sfrac{\partial^2 \kineticEnergy}{\partial \sysVel \partial \sysVel}$ for the chosen coordinates.
Due to these analogies to the established inertia parameters, we refer to the rigid body stiffness parameters in the following as: total stiffness $\bodyStiffness{}{}$, center of stiffness $\bodyCOS{}{}$, moment of stiffness $\bodyMOS{}{}$, and moment of stiffness at the center of stiffness $\bodyMOSs{}{}$.

\subsection{Dissipation}\label{sec:DampingSE3}
As motivated in the previous section we may motivate damping as particles moving within a viscous fluid which produce a drag force proportional to the particles velocity $\particlePosd{\PidxI}$.
Different volumes of the particles may motivate different drag coefficients $\particleDamping{\PidxI}$, see \autoref{fig:RBdamping}.
\begin{figure}[ht]
 \centering
 \input{graphics/RBDamping.pdf_tex}
 \caption{rigid body within viscous fluid}
 \label{fig:RBdamping}
\end{figure}

\paragraph{General fluid motion.}
Let the fluid at the position of $\PidxI$-th particle position have the velocity $\tuple{v}_{\mathsf{D}\PidxI}(t)\in\RealNum^3$ and let its drag coefficient be $\particleDamping{\PidxI}\in\RealNumSP$.
Overall, the dissipation function is
\begin{align}
 \dissFkt &= \tfrac{1}{2} \sumParticles \!\particleDamping{\PidxI} \norm{\overbrace{\R(\v - \wedOp(\particleBodyPos{\PidxI})\w)}^{\particlePosd{\PidxI}} - \, \tuple{v}_{\mathsf{D}\PidxI})}^2
\nonumber\\
 &= \tfrac{1}{2} \underbrace{\sumParticles \!\particleDamping{\PidxI}}_{\bodyDamping{}{}} \norm{\v}^2
  - \v^\top \underbrace{\sumParticles \!\particleDamping{\PidxI} \wedOp(\particleBodyPos{\PidxI})}_{\bodyDamping{}{} \wedOp(\bodyCOD{}{})} \w
  + \tfrac{1}{2} \w^\top \underbrace{\sumParticles \!\particleDamping{\PidxI} \wedOp(\particleBodyPos{\PidxI})^\top \wedOp(\particleBodyPos{\PidxI})}_{\bodyMOD{}{}} \w
\nonumber\\
 &\quad - \v^\top \sumParticles \!\particleDamping{\PidxI} \R^\top \tuple{v}_{\mathsf{D}\PidxI}
  + \w^\top \sumParticles \!\particleDamping{\PidxI} \wedOp(\particleBodyPos{\PidxI}) \R^\top \tuple{v}_{\mathsf{D}\PidxI}
 + \tfrac{1}{2} \sumParticles \!\particleDamping{\PidxI} \norm{\tuple{v}_{\mathsf{D}\PidxI}}^2
\end{align}
The resulting generalized force is
\begin{align}
 \bodyGenForceDiss{}{} &= \pdiff[\dissFkt]{\bodyVel{}{}} =
 \underbrace{\begin{bmatrix} \bodyDamping{}{}\idMat[3] & \bodyDamping{}{} \wedOp(\bodyCOD{}{})^\top \\ \bodyDamping{}{} \wedOp(\bodyCOD{}{}) & \bodyMOD{}{} \end{bmatrix}}_{\bodyDissMat{}{}}
 \underbrace{\begin{bmatrix} \v \\ \w \end{bmatrix}}_{\bodyVel{}{}}
 +
 \sumParticles \!\particleDamping{\PidxI} \begin{bmatrix} -\idMat[3] \\ \wedOp(\particleBodyPos{\PidxI}) \end{bmatrix} \R^\top \tuple{v}_{\mathsf{D}\PidxI}.
\end{align}
Again we found parameters similar to the established inertia parameters.
In analogy to them we call $\bodyDamping{}{}\in\RealNumSP$ the total damping, $\bodyCOD{}{}\in\RealNum^3$ the center of damping, and $\bodyMOD{}{}\in\SymMatSP(3)$ the moment of damping.

\paragraph{Rigid fluid motion.}
Let us consider a special case in which the fluid velocity also obeys a rigid body motion parameterized by $(\rR,\RR)$ and the velocity $(\vR,\wR)$, see \autoref{fig:RBdamping}.
The absolute fluid velocity at the particle position is then $\tuple{v}_{\mathsf{D}\PidxI} = \RR(\vR + \wedOp(\wR)\RR^\top(\particlePos{\PidxI} - \rR))$.
The we have the dissipation function
\begin{align}\label{eq:RBDissFkt}
 \dissFkt &= \tfrac{1}{2} \sumParticles \!\particleDamping{\PidxI} \norm{\overbrace{\R(\v - \wedOp(\particleBodyPos{\PidxI})\w)}^{\particlePosd{\PidxI}} - \overbrace{\RR(\vR + \wedOp(\wR)\RR^\top(\r + \R\particleBodyPos{\PidxI} - \rR))}^{\tuple{v}_{\mathsf{D}\PidxI}}}^2
\nonumber\\
 &= \tfrac{1}{2} \sumParticles \!\particleDamping{\PidxI} \norm{\underbrace{\v - \R^\top(\RR\vR - \wedOp(\r \!-\! \rR)\RR\wR)}_{\vE} - \wedOp(\particleBodyPos{\PidxI})\underbrace{(\w - \R^\top\RR\wR)}_{\wE}}^2
\nonumber\\
 &= \tfrac{1}{2} 
 \underbrace{\big[ \vE^\top, \wE^\top \big]}_{\bodyVelE{}{}^\top }
 \underbrace{\begin{bmatrix} \bodyDamping{}{}\idMat[3] & \bodyDamping{}{} \wedOp(\bodyCOD{}{})^\top \\ \bodyDamping{}{} \wedOp(\bodyCOD{}{}) & \bodyMOD{}{} \end{bmatrix}}_{\bodyDissMat{}{}}
 \underbrace{\begin{bmatrix} \vE \\ \wE \end{bmatrix}}_{\bodyVelE{}{}}
\end{align}
The resulting generalized force is 
\begin{align}
 \bodyGenForceDiss{}{} = \pdiff[\dissFkt]{\bodyVel{}{}} = \pdiff[\dissFkt]{\bodyVelE{}{}} = \bodyDissMat{}{} \bodyVelE{}{}
\end{align}


\subsection{Summary and the special Euclidean group}\label{sec:RBSummary}

\paragraph*{Special Euclidean group.}
Instead of collecting the configuration coordinates in a tuple $\sysCoord = [\r^\top, \Rx^\top, \Ry^\top, \Rz^\top]^\top \in \configSpace$ as proposed in the previous chapter, it can also be useful to arrange them within a matrix: 
\begin{align}
 \G = \begin{bmatrix} \R & \r \\ \mat{0} & 1 \end{bmatrix} \ \in \
 \SpecialEuclideanGroup(3) = \left\{ \begin{bmatrix} \R & \r \\ \mat{0} & 1 \end{bmatrix} \, \bigg| \, \r \in \RealNum^3, \R \in \SpecialOrthogonalGroup(3) \right\}
\end{align}
which is commonly referred to as the \textit{homogeneous representation}, e.g.\ \cite[sec.\ 2.3.1]{Murray:Robotic}.
The set $\SpecialEuclideanGroup(3)$ combined with matrix multiplication forms a Lie group which is called the \textit{special Euclidean group}.
Euclidean denotes to the fact that its transformations preserve the Euclidean distance, while special denotes to the fact that it does not permit reflections (analog to the special orthogonal group $\SpecialOrthogonalGroup(3)$).

\paragraph*{Operators.}
This section already used the $\wedOp$ on $\RealNum^3$ quite extensively.
On $\RealNum^6$ we define it as
\begin{subequations}\label{eq:DefWedgeOp}
\begin{align}
 \wedOp : \, \RealNum^3 \rightarrow \SpecialOrthogonalAlgebra(3) \, : \, \begin{bmatrix} \omega_1 \\ \omega_2 \\ \omega_3 \end{bmatrix} &\mapsto \begin{bmatrix} 0 & -\omega_3 & \omega_2 \\ \omega_3 & 0 & -\omega_1 \\ -\omega_2 & \omega_1 & 0 \end{bmatrix}
 %\SpecialOrthogonalAlgebra(3) &= \{ \wedOp\w \, | \, \w \in \RealNum^3 \},
\\
 \wedOp: \, \RealNum^6 \rightarrow \SpecialEuclideanAlgebra(3) \, : \ \ \begin{bmatrix} \v \\ \w \end{bmatrix} &\mapsto \begin{bmatrix} \wedOp\w & \v \\ \mat{0} & 0 \end{bmatrix}
 %\SpecialEuclideanAlgebra(3) &= \{ \wedOp\bodyVel{}{} \, | \, \bodyVel{}{} \in \RealNum^6 \}.
\end{align}
\end{subequations}
Its inverse is denoted $\veeOp(\cdot)$, i.e.\ $\veeOp(\wedOp(\sysVel))=\sysVel$.
The $\wedOp$ and $\veeOp$ operators are well established in the literature, see e.g.\ \cite[sec.\ 2.3.2]{Murray:Robotic}.

Using the $\wedOp$ operator may rewrite the rigid body kinematics from \eqref{eq:RBKinematicEq1} in matrix form as:
\begin{align}
 \underbrace{\begin{bmatrix} \Rd & \rd \\ \mat{0} & 0 \end{bmatrix}}_{\dot{\G}} &= \underbrace{\begin{bmatrix} \R & \r \\ \mat{0} & 1 \end{bmatrix}}_{\G} \underbrace{\begin{bmatrix} \wedOp(\w) & \v \\ \mat{0} & 0 \end{bmatrix}}_{\wedOp(\bodyVel{}{})}
\end{align}

\paragraph{More operators.}
The following operators are not established in the literature, but will prove quite useful for this work.
Define the $\veeTwoOp$ operator through
\begin{align}\label{eq:EqVeeTwoOp}
 \tr\big( \mat{A} (\wedOp\tuple{\xi})^\top \big) = \tuple{\xi}^\top \veeTwoOp(\mat{A}),
\end{align}
this is
\begin{subequations}
\begin{align}
 \veeTwoOp &: \RealNum^{3\times 3} \rightarrow \RealNum^3 \, : \, \mat{A} \mapsto \veeOp(\mat{A}-\mat{A}^\top)
\\
 \veeTwoOp &: \RealNum^{4\times 4} \rightarrow \RealNum^6 \, : \, \begin{bmatrix} \mat{A} & \tuple{b} \\ \ast & \ast \end{bmatrix} \mapsto \begin{bmatrix} \tuple{b} \\ \veeTwoOp\mat{A} \end{bmatrix}.
\end{align}
\end{subequations}
Note that for $\mat{\Omega} \in \SpecialOrthogonalAlgebra(3) \subset \RealNum^{3\times3}$ we have $\veeTwoOp(\mat{\Omega}) = 2 \veeOp(\mat{\Omega})$, thus giving the motivation for the name.
Define the $\veeMatOp$ operator through
\begin{align}\label{eq:EqVeeMatOp}
 \tr\big( \wedOp\tuple{\xi}\, \mat{A} (\wedOp\tuple{\eta})^\top \big) = \tuple{\eta}^\top (\veeMatOp\mat{A}) \tuple{\xi},
\end{align}
this is
\begin{subequations}
\begin{align}
 \veeMatOp &: \RealNum^{3\times 3} \!\rightarrow\! \RealNum^{3\times 3} \, : \mat{A} \mapsto \tr(\mat{A}) \idMat[3] - \mat{A}
\\
 \veeMatOp &: \RealNum^{4\times 4} \!\rightarrow\! \RealNum^{6\times 6} \, : \!\begin{bmatrix} \mat{A} & \tuple{b} \\ \tuple{c}^\top & d \end{bmatrix} \!\mapsto\!\! \begin{bmatrix} d \idMat[3] & \!(\wedOp \tuple{b})^\top\! \\ \wedOp\tuple{c} & \veeMatOp\mat{A} \end{bmatrix}\!.
\end{align} 
\end{subequations}
Let $\wedMatOp(\cdot)$ denote its inverse.
Combining the definitions \eqref{eq:EqVeeMatOp} and \eqref{eq:EqVeeTwoOp} also yields
\begin{align}
 \veeTwoOp\big(\wedOp\tuple{\xi} \, \mat{A} \big) = \veeMatOp(\mat{A}) \tuple{\xi}
\end{align}

\paragraph{Adjoint representation.}
Define
\begin{subequations}
\begin{align}
 \G &= \begin{bmatrix} \R & \r \\ \mat{0} & 1 \end{bmatrix} \, :&
 \Ad{\G} &= \begin{bmatrix} \R & \wedOp(\r) \R \\ \mat{0} & \R \end{bmatrix}&
 \label{eq:defAdjointSE3}
\\
%\intertext{and}
 \bodyVel{}{} &= \begin{bmatrix} \v \\ \w \end{bmatrix} \, :&
 \ad{\bodyVel{}{}} &= \begin{bmatrix} \wedOp(\w) & \wedOp(\v) \\ \mat{0} & \wedOp(\w) \end{bmatrix}&
\end{align}
\end{subequations}
The notation is due to the established notation for the \textit{adjoint representation} for Lie groups and their associated Lie algebras, see e.g.\ \cite[Def.\ 3.32 \& 3.7]{Hall:LieGroups}.
Using this background we have the obvious relations:
\begin{subequations}\label{eq:IdentitiesAdjointRepresentation}
\begin{align}
 \Ad{\G_1 \G_2} &= \Ad{\G_1} \Ad{\G_1},&
 \Ad{\G}^{-1} &= \Ad{\G^{-1}},&
 \Ad{\idMat[4]} &= \idMat[6],
\\
 &&
 \ad{\tuple{\xi}_1}\tuple{\xi}_2 &= -\ad{\tuple{\xi}_2}\tuple{\xi}_1,&
 \ad{\tuple{\xi}}\tuple{\xi} &= \tuple{0}.
\end{align}
\end{subequations}
Furthermore, for $\tdiff{t} \G = \G \wedOp(\bodyVel{}{})$ we have
\begin{align}
 \tdiff{t}\Ad{\G} = \Ad{\G} \ad{\bodyVel{}{}}.
\end{align}
Though the Lie group theory can be extremely useful for rigid body mechanics, for this work it is sufficient to regard $\Ad{(\cdot)}$ and $\ad{(\cdot)}$ as simple algebraic operators with the identities \eqref{eq:IdentitiesAdjointRepresentation}.



% \paragraph*{Inner product.}
% For matrices $\mat{A}, \mat{B} \in \RealNum^{n\times m}$ and a symmetric, positive definite matrix $ \mat{K} \in \SymMatP(n)$, define the \textit{weighted} Frobenius inner product as
% \begin{align}%\label{eq:DefFrobeniusInnerProduct}
%  \sProd[\mat{K}]{\mat{A}}{\mat{B}} = \tr(\mat{A}^\top \mat{K} \mat{B}).
% \end{align}
% Its induced norm and induced metric are
% \begin{align}
%  \norm[\mat{K}]{\mat{A}} = \sqrt{\sProd[\mat{K}]{\mat{A}}{\mat{A}}}
% \qquad
%  \metric[\mat{K}]{\mat{A}}{\mat{B}} = \norm[\mat{K}]{\mat{A}-\mat{B}}.
% \end{align}
% Note that this contains the common Frobenius inner product/norm for $\mat{K}=\idMat[n]$.
% Furthermore, it contains the common dot product and Euclidean norm/metric for $\mat{A}, \mat{B} \in \RealNum^{n}$.
% Thus, it is compatible with the notation used so far if we identify: $\sProd[{\idMat[n]}]{\cdot}{\cdot} \equiv \sProd[]{\cdot}{\cdot}$ and $\norm[{\idMat[n]}]{\cdot} \equiv \norm[]{\cdot}$.

% For $\protoX_i \in \RealNum^{n\times n}$ and $\R \in \SpecialOrthogonalGroup(n)$ we have
% \begin{subequations}
% \begin{align}
%  \sProd[\mat{K}]{(\R \protoX_1)^\top}{(\R \protoX_2)^\top}
%  &= \sProd[\mat{K}]{\protoX_1^\top}{\protoX_2^\top},
% \\
%  \sProd[\mat{K}]{(\protoX_1 \R)^\top}{(\protoX_2 \R)^\top} 
%  &= \sProd[\R \mat{K} \R^\top]{\protoX_1^\top}{\protoX_2^\top}.
% \end{align} 
% \end{subequations}
% For arguments of the shape $\protoXi_i = \left[\begin{smallmatrix} \protoX_i & \protox_i \\ \mat{0} & 0 \end{smallmatrix}\right]$ with $\protox_i \in \RealNum^{n}$, $\protoX_i \in \RealNum^{n\times n}$ and $\G \in \SpecialEuclideanGroup(n)$ we have
% \begin{subequations}\label{eq:InnerProductSE3Translation}
% \begin{align}
%  \label{eq:InnerProductSE3LeftTranslation}
%  \sProd[\mat{K}]{(\G \protoXi_1)^\top}{(\G \protoXi_2)^\top}
%  &= \sProd[\mat{K}]{\protoXi_1^\top}{\protoXi_2^\top},
% \\
%  \label{eq:InnerProductSE3RightTranslation}
%  \sProd[\mat{K}]{(\protoXi_1 \G)^\top}{(\protoXi_2 \G)^\top} 
% % &= \tr(\protoXi_1 \G \mat{K} \G^\top \protoXi_2^\top)
%  &= \sProd[\G \mat{K} \G^\top]{\protoXi_1^\top}{\protoXi_2^\top}.
% \end{align} 
% \end{subequations}
% Note that this case includes in particular $\protoXi_1, \protoXi_2 \in \SpecialEuclideanAlgebra(n)$.


\paragraph*{Rigid body energies.}
Notice that for $\protox, \particleBodyPos{\PidxI} \in \RealNum^{n}$, $\protoX \in \RealNum^{n\times n}$ we have
\begin{align}\label{eq:RBProtoEnergy}
 \sumParticles \particleMass{\PidxI} \norm{\protox + \protoX \particleBodyPos{\PidxI}}^2
 %= \sumParticles \particleMass{\PidxI} (\protox + \protoX \particleBodyPos{\PidxI})^\top (\protox + \protoX \particleBodyPos{\PidxI})
 &= \sumParticles \particleMass{\PidxI} \tr\big( (\protox + \protoX \particleBodyPos{\PidxI})(\protox + \protoX \particleBodyPos{\PidxI})^\top \big)
\nonumber\\
 &= \sumParticles \particleMass{\PidxI} \tr\left( \left(\begin{bmatrix} \protoX & \protox \\ \mat{0} & 0 \end{bmatrix}\begin{bmatrix} \particleBodyPos{\PidxI}\\ 1 \end{bmatrix} \right) \left(\begin{bmatrix} \protoX & \protox \\ \mat{0} & 0 \end{bmatrix}\begin{bmatrix} \particleBodyPos{\PidxI}\\ 1 \end{bmatrix} \right)^\top \right)
\nonumber\\
 &= \tr\bigg( \underbrace{\begin{bmatrix} \protoX & \protox \\ \mat{0} & 0 \end{bmatrix}}_{\protoXi} \underbrace{\left( \sumParticles \particleMass{\PidxI} \begin{bmatrix} \particleBodyPos{\PidxI} \particleBodyPos{\PidxI}^\top & \particleBodyPos{\PidxI}^\top \\ \particleBodyPos{\PidxI} & 1 \end{bmatrix} \right)}_{\bodyInertiaMatp{}{}} \underbrace{\begin{bmatrix} \protoX^\top & \mat{0} \\ \protox^\top & 0 \end{bmatrix}}_{\protoXi^\top} \bigg)
\nonumber\\
 &= \norm[\bodyInertiaMatp{}{}]{\protoXi^\top}^2
\end{align}
with the weighted Frobenius norm motivated in \eqref{eq:DefMatrixNorm}.
Furthermore, for the special case $\protoXi = \wedOp \bodyVel{}{}$ we have due to \eqref{eq:EqVeeMatOp}:
\begin{align}
 \norm[\bodyInertiaMatp{}{}]{\wedOp(\bodyVel{}{})^\top}^2 = \norm[\veeMatOp\bodyInertiaMatp{}{}]{\bodyVel{}{}}^2,
\end{align}
Using this, we may rewrite the kinetic energy $\kineticEnergy$ of a free rigid body \eqref{eq:RigidBodyKineticEnergy}, the acceleration energy $\accEnergy$ from \eqref{eq:RigidBodyAccEnergy}, the dissipation function $\dissFkt$ in \eqref{eq:RBDissFkt}, the potential energy $\potentialEnergy$ due to linear springs \eqref{eq:RBStiffnessPotential3} and the potential energy $\potentialGravity$ due to earth's gravitation from \eqref{eq:RBPotentialEnergyGravity} as
\begin{subequations}
\begin{align}
 \kineticEnergy
 &= \tfrac{1}{2} \sumParticles \particleMass{\PidxI} \norm{\overbrace{\rd + \bodyRotd{}{}{}{} \particleBodyPos{\PidxI}}^{\particlePosd{\PidxI}}}^2
 = \tfrac{1}{2} \norm[\bodyInertiaMatp{}{}]{\bodyHomoCoordd{}{}^\top}^2
 = \tfrac{1}{2} \norm[\bodyInertiaMat{}{}]{\bodyVel{}{}}^2
\\
 \kineticEnergyInertial
 &= \tfrac{1}{2} \sumParticles \particleMass{\PidxI} \norm{\overbrace{\rd + \bodyRotd{}{}{}{} \particleBodyPos{\PidxI}}^{\particlePosd{\PidxI}} - \gravityAcc t}^2
 = \tfrac{1}{2} \norm[\bodyInertiaMatp{}{}]{(\bodyHomoCoordd{}{} - \wedOp(\gravityAccWrench t))^\top}^2
\qquad
 \gravityAccWrench = \begin{bmatrix} \gravityAcc \\ \tuple{0}_{3\times1} \end{bmatrix}
\\
 \label{eq:RBAccEnergyHomo}
 \accEnergy 
 &= \tfrac{1}{2} \sumParticles \particleMass{\PidxI} \norm{\overbrace{\bodyPosdd{}{}{} + \bodyRotdd{}{}{}{} \particleBodyPos{\PidxI}}^{\particlePosdd{\PidxI}}}^2
 = \tfrac{1}{2} \norm[\bodyInertiaMatp{}{}]{\bodyHomoCoorddd{}{}^\top}^2
% = \tfrac{1}{2} \norm[\bodyInertiaMatp{}{}]{(\wedOp(\bodyVeld{}{}) + \wedOp(\bodyVel{}{})^2)^\top}^2
\\
 \accEnergyInertial 
 &= \tfrac{1}{2} \sumParticles \particleMass{\PidxI} \norm{\overbrace{\bodyPosdd{}{}{} + \bodyRotdd{}{}{}{} \particleBodyPos{\PidxI}}^{\particlePosdd{\PidxI}} - \gravityAcc}^2
 = \tfrac{1}{2} \norm[\bodyInertiaMatp{}{}]{(\bodyHomoCoorddd{}{} - \wedOp(\gravityAccWrench))^\top}^2
\\
 \dissFkt
 &= \tfrac{1}{2} \sumParticles \particleDamping{\PidxI} \norm{\overbrace{\rd + \bodyRotd{}{}{}{} \particleBodyPos{\PidxI}}^{\particlePosd{\PidxI}}}^2
 = \tfrac{1}{2} \norm[\bodyDissMatp{}{}]{\bodyHomoCoordd{}{}^\top}^2
 = \tfrac{1}{2} \norm[\bodyDissMat{}{}]{\bodyVel{}{}}^2
% = \tfrac{1}{2} \norm[\bodyDissMatp{}{}]{\wedOp(\bodyVel{}{})^\top}^2
\\
 \potentialStiff
 &= \tfrac{1}{2} \sumParticles \particleStiffness{\PidxI} \norm{\overbrace{\r + \R\particleBodyPos{\PidxI}}^{\particlePos{\PidxI}} - (\overbrace{\rR + \RR\particleBodyPos{\PidxI}}^{\particlePosR{\PidxI}})}^2
 = \tfrac{1}{2} \norm[\bodyStiffMatp{}{}]{(\bodyHomoCoord{}{} - \bodyHomoCoordR{}{})^\top}^2
% = \tfrac{1}{2} \norm[\bodyStiffMatp{}{}]{(\idMat[4] - \bodyHomoCoord{}{}^{-1}\bodyHomoCoordR{}{})^\top}^2
\\
 \label{eq:RBGravityEnergyHomo}
 \potentialGravity 
 &= \sumParticles \sProd{\overbrace{\r + \R \particleBodyPos{\PidxI}}^{\particlePos{\PidxI}}}{-\particleMass{\PidxI} \gravityAcc}
 = \sProd[\bodyInertiaMatp{}{}]{\G^\top}{\wedOp(-\gravityAccWrench)^\top},
%\qquad
% \gravityAccWrench^\top = [ \gravityAcc^\top, \tuple{0}_{1\times3} ].
\end{align}
\end{subequations}
where
\begin{subequations}
\begin{align}
 \bodyInertiaMatp{}{} = \sumParticles \particleMass{\PidxI} \begin{bmatrix} \particleBodyPos{\PidxI} \particleBodyPos{\PidxI}^\top & \particleBodyPos{\PidxI}^\top \\ \particleBodyPos{\PidxI} & 1 \end{bmatrix} 
 &= \begin{bmatrix} \bodyMOIp{}{}{} & \bodyMass{}{}\bodyCOM{}{}{} \\ \bodyMass{}{}\bodyCOM{}{}{}^\top & \bodyMass{}{} \end{bmatrix}
 = \wedMatOp(\bodyInertiaMat{}{})
\\
 \bodyDissMatp{}{} = \sumParticles \particleDamping{\PidxI} \begin{bmatrix} \particleBodyPos{\PidxI} \particleBodyPos{\PidxI}^\top & \particleBodyPos{\PidxI}^\top \\ \particleBodyPos{\PidxI} & 1 \end{bmatrix} 
 &= \begin{bmatrix} \bodyMODp{}{}{} & \bodyDamping{}{}\bodyCOD{}{}{} \\ \bodyDamping{}{}\bodyCOD{}{}{}^\top & \bodyDamping{}{} \end{bmatrix}
 = \wedMatOp(\bodyDissMat{}{})
\\
 \bodyStiffMatp{}{} = \sumParticles \particleStiffness{\PidxI} \begin{bmatrix} \particleBodyPos{\PidxI} \particleBodyPos{\PidxI}^\top & \particleBodyPos{\PidxI}^\top \\ \particleBodyPos{\PidxI} & 1 \end{bmatrix} 
 &= \begin{bmatrix} \bodyMOSp{}{}{} & \bodyStiffness{}{}\bodyCOS{}{}{} \\ \bodyStiffness{}{}\bodyCOS{}{}{}^\top & \bodyStiffness{}{} \end{bmatrix}
 = \wedMatOp(\bodyStiffMat{}{})
\end{align}
\end{subequations}
Note that $\bodyMass{}{}$, $\bodyMass{}{}\bodyCOM{}{}{}$ and $\bodyMOIp{}{}{}$ correspond to the zeroth, first and second \textit{mathematical moments} of the distribution $\particleMass{\PidxI} \particleBodyPos{\PidxI}, \PidxI=1,\ldots,\numParticles$.
Their collection in the matrix $\bodyInertiaMatp{}{} \in \SymMat(4)$ is bijective to the previously encountered inertia matrix $\bodyInertiaMatp{}{} = \wedMatOp(\bodyInertiaMat{}{}) \Leftrightarrow \bodyInertiaMat{}{} = \veeMatOp(\bodyInertiaMatp{}{})$.
Obviously, the same holds for the damping $\bodyStiffMatp{}{}$ and stiffness matrix $\bodyDissMatp{}{}$.

\paragraph{Rigid body forces.}
The corresponding forces forces which were already derived in the previous subsections can be written in a more compact form:
\begin{subequations}
\begin{align}
 \genForceInertia = \pdiff[\accEnergy]{\bodyVeld{}{}}
 &= \veeTwoOp \big( (\wedOp(\bodyVeld{}{}) + \wedOp(\bodyVel{}{})^2) \bodyInertiaMatp{}{} \big)
 = \bodyInertiaMat{}{}\bodyVeld{}{} - \ad{\bodyVel{}{}}^\top \bodyInertiaMat{}{} \bodyVel{}{}
% = \bodyInertiaMat{}{} \bodyVeld{}{} + \veeMatOp( \wedOp(\bodyVel{}{}) \bodyInertiaMatp{}{}) \bodyVel{}{}}_{\gyroForce}
\\
 \genForceDiss = \pdiff[\dissFkt]{\bodyVel{}{}}
 &= \veeTwoOp \big( \wedOp(\bodyVel{}{}) \bodyDissMatp{}{} \big)
 = \bodyDissMat{}{} \bodyVel{}{}
% = \underbrace{\veeMatOp(\bodyDissMatp{}{})}_{\bodyDissMat{}{}} \bodyVel{}{}
\\
 \genForceStiff = \differential \potentialStiff
% = \pdiff[\potentialStiffd]{\bodyVel{}{}} 
% = \pdiff{\bodyVel{}{}} \tr \big( (\bodyHomoCoord{}{}\wedOp(\bodyVel{}{}) \!-\! \bodyHomoCoordR{}{}\wedOp(\bodyVelR{}{})) \bodyStiffMatp{}{} (\bodyHomoCoord{}{} \!-\! \bodyHomoCoordR{}{})^\top \big)
% = \pdiff{\bodyVel{}{}} \tr \big( \wedOp(\bodyVel{}{}) \bodyStiffMatp{}{} (\idMat[4] \!-\! \bodyHomoCoord{}{}^{-1} \bodyHomoCoordR{}{})^\top \big)
 &= \veeTwoOp \big( (\idMat[4] - \bodyHomoCoord{}{}^{-1} \bodyHomoCoordR{}{}) \bodyStiffMatp{}{} \big)
\\
 \genForceGravity = \differential\potentialGravity
 &= \veeTwoOp \big( \bodyHomoCoord{}{}^{\top} \wedOp(-\gravityAccWrench) \bodyInertiaMatp{}{} \big)
 = -\bodyInertiaMat{}{} \Ad{\G}^{-1} \gravityAccWrench.
\end{align} 
\end{subequations}

\paragraph{Equation of motion.}
Combining the results above, the equations of motion of a free rigid body subject to inertia, gravity, viscous friction, linear springs and a generalized force $\genForceEx$ may be written as
\begin{subequations}
\begin{align}
 \dot{\G} &= \G \wedOp(\bodyVel{}{}),
\\
 \bodyVeld{}{} &= \Ad{\G}^{-1} \gravityAccWrench + \bodyInertiaMat{}{}^{-1}\big(\genForceEx + (\ad{\bodyVel{}{}}^\top \bodyInertiaMat{}{} - \bodyDissMat{}{}) \bodyVel{}{} - \veeTwoOp \big( (\idMat[4] - \bodyHomoCoord{}{}^{-1} \bodyHomoCoordR{}{}) \bodyStiffMatp{}{} \big) \big).
% \bodyInertiaMat{}{}(\bodyVeld{}{}-\Ad{\G}^{-1} \gravityAccWrench) - \ad{\bodyVel{}{}}^\top \bodyInertiaMat{}{} \bodyVel{}{} + \bodyDissMat{}{} \bodyVel{}{} + \veeTwoOp \big( (\idMat[4] - \bodyHomoCoord{}{}^{-1} \bodyHomoCoordR{}{}) \bodyStiffMatp{}{} \big) &= \genForceEx.
\end{align} 
\end{subequations}
Thee are three sets of ingredients:
\begin{itemize}
\item The chosen coordinates are collected within $\bodyHomoCoord{}{}(t)\in\SpecialEuclideanGroup(3)$ and $\bodyVel{}{}(t)\in\RealNum^6$.
\item The matrices $\bodyInertiaMatp{}{}, \bodyDissMatp{}{}, \bodyStiffMatp{}{} \in \SymMat(4)$ capture the distribution of mass, damping and stiffness.
\item External influences are collected within the gravity wrench $\gravityAccWrench^\top = [ \gravityAcc^\top, \tuple{0}_{1\times3} ]$, the equilibrium configuration of the springs $\bodyHomoCoordR{}{}(t) \in \SpecialEuclideanGroup(3)$ and the generalized external force $\genForceEx(t)\in\RealNum^6$.
\end{itemize}
