\section{Rigid body systems}\label{sec:RBSRigidBodySys}
A rigid body system is a system of $\numRigidBodies \geq 1$ rigid bodies which may be constrained to each other and/or to the surrounding space.
As before, this section restricts to geometric constraints.

There are many established textbooks on this subject \eg \cite{Schwertassek:MultibodySystems}, \cite{Murray:Robotic}, \cite{Kane:Dynamics}.
However, all these excellent texts restrict to \textit{minimal} generalized coordinates or are even more restrictive by requiring Denavit-Hartenberg parameters \cite{DenavitHartenbergParam}.
While this is just fine when dealing only with one-dimensional joints, it may be too restrictive when dealing with multidimensional joints as e.g.\ mobile robots.
Furthermore, the texts mentioned above mostly focus on inertia but not dissipation and stiffness.

This section deals with the derivation of equations of motion for rigid body systems subject to inertia, gravity, linear springs and viscous friction.
It allows for a quite general parameterization as motivated in the previous sections.
%The goal of this section is to present an algorithm for the computation of the equations of motion for rigid body systems that allows for a rather flexible parameterization.

\subsection{Parameterization}\label{sec:RBSParameterization}
\begin{figure}[ht]
 \centering
% \def\svgwidth{.7\linewidth}
 \input{graphics/RBSConfigurations.pdf_tex}
 \caption{inertial frame and body fixed frames}
 \label{fig:RBSConfigurations}
\end{figure}

\paragraph{Configuration coordinates.}
As motivated for the single rigid body, let there be a body fixed frame for each body of the system as illustrated in \autoref{fig:RBSConfigurations}.
The components of the position of the $\BidxII\text{-th}$ body w.r.t.\ the inertial frame are $\bodyPos{0}{\BidxII} \in \RealNum^3$ and the components of its attitude are $\bodyRot{0}{\BidxII} = [\bodyRotx{0}{\BidxII}, \bodyRoty{0}{\BidxII}, \bodyRotz{0}{\BidxII}] \in \SpecialOrthogonalGroup(3)$.

The configuration can also be expressed w.r.t.\ any other body: $\bodyPos{\BidxI}{\BidxII}$ is the position of the $\BidxII\text{-th}$ frame w.r.t.\ the frame of the $\BidxI\text{-th}$ body and analog of the attitude $\bodyRot{\BidxI}{\BidxII}$.
The left side indices are used for readability but also to emphasize their different nature compared to the right side indices.
The sum convention does not apply to these indices.
For the positions and attitudes we have the following relations
\begin{subequations}\label{eq:rigidBodyRelCoordRules}
\begin{align}
 \bodyPos{\BidxI}{\BidxIII} &= \bodyPos{\BidxI}{\BidxII} + \bodyRot{\BidxI}{\BidxII} \bodyPos{\BidxII}{\BidxIII},&
 \bodyRot{\BidxI}{\BidxIII} &= \bodyRot{\BidxI}{\BidxII} \bodyRot{\BidxII}{\BidxIII},
\\
 \bodyPos{\BidxII}{\BidxI} &= -\bodyRot{\BidxI}{\BidxII}^\top \bodyPos{\BidxI}{\BidxII},&
 \bodyRot{\BidxII}{\BidxI} &= \bodyRot{\BidxI}{\BidxII}^\top,&
\\
 \bodyPos{\BidxI}{\BidxI} &= \tuple{0},&
 \bodyRot{\BidxI}{\BidxI} &= \idMat[3],&
 \BidxI, \BidxII, \BidxIII &= 0,\ldots,\numRigidBodies.
\end{align}
\end{subequations}

As motivated in the previous section, it will be convenient to merge position $\bodyPos{0}{\BidxII} \in \RealNum^3$ and rotation matrix $\bodyRot{0}{\BidxII} \in \SpecialOrthogonalGroup(3)$ into the (rigid body) configuration matrix
\begin{align}
 \bodyHomoCoord{\BidxI}{\BidxII}{}{} &= \begin{bmatrix} \bodyRot{\BidxI}{\BidxII} & \bodyPos{\BidxI}{\BidxII}{} \\ 0 & 1 \end{bmatrix} \in \SpecialEuclideanGroup(3).
\end{align}
Then \eqref{eq:rigidBodyRelCoordRules} is equivalent to
\begin{align}\label{eq:rigidBodyRelCoordRules2}
 \bodyHomoCoord{\BidxI}{\BidxIII} &= \bodyHomoCoord{\BidxI}{\BidxII} \bodyHomoCoord{\BidxII}{\BidxIII},&
 \bodyHomoCoord{\BidxII}{\BidxI} &= \bodyHomoCoord{\BidxI}{\BidxII}^{-1},&
 \bodyHomoCoord{\BidxI}{\BidxI} &= \idMat[4],&
 \BidxI, \BidxII, \BidxIII &= 0,\ldots,\numRigidBodies.
\end{align}
For a system of $\numRigidBodies$ body fixed frames and a inertial frame there are $(\numRigidBodies+1)^2$ transformations, but due to the rules \eqref{eq:rigidBodyRelCoordRules2}, only $\numRigidBodies$ can be independent.
So a \RBS can have at most $6\numRigidBodies$ degrees of freedom which is the case if there are no constraints (like joints) between the bodies.
Constraints of a joint between body $a$ and $b$ can be captured inside the corresponding transformation $\bodyHomoCoord{\BidxI}{\BidxII}{}{}$.
%The description of a rigid body system by a set of transformations is shown in the following example.
We will discuss this in the following example.

\begin{figure}[p]
 \centering
% \def\svgwidth{0.95\textwidth}
 \input{graphics/TriV3ConfigGraph.pdf_tex}
 \caption{Frames attached to the Tricopter bodies (top) and the configuration graph (bottom)}
 \label{fig:TriV3ConfigGraph}
\end{figure}

\begin{Example}\label{exp:TricopterWithLoadConfiguration}
\textbf{Tricopter with suspended load: configuration.}
Consider the Tricopter with a suspended load as shown in \autoref{fig:TriV3ConfigGraph}.
The top part of the figure shows the body fixed frames which are attached to geometrically meaningful points.
The numbering of the bodies is rather arbitrary.

The Tricopter flies freely in space, i.e.\ there are no constraints between the inertial frame and any body of the system.
So we chose to describe the configuration of the central body w.r.t. the inertial frame as
\begin{subequations}\label{eq:configurationMatricesTricopterExample}
\begin{align}
 \bodyHomoCoord{0}{1} = 
 \begin{bmatrix}
  \bodyRotxx{0}{1} & \bodyRotxy{0}{1} & \bodyRotxz{0}{1} & \bodyPosx{0}{1} \\
  \bodyRotyx{0}{1} & \bodyRotyy{0}{1} & \bodyRotyz{0}{1} & \bodyPosy{0}{1} \\
  \bodyRotzx{0}{1} & \bodyRotzy{0}{1} & \bodyRotzz{0}{1} & \bodyPosz{0}{1} \\
  0 & 0 & 0 & 1
 \end{bmatrix}.
\end{align}
The suspended load is a rigid body that is attached by a spherical joint to the central body.
The body fixed frame of the load was placed in the center of this spherical joint.
As a consequence, the position of the load (the position of its body fixed frame not the position of its center of mass) w.r.t.\ the central body is constant.
This is reflected by the configuration
\begin{align}
 \bodyHomoCoord{1}{2} =
 \begin{bmatrix}
  \bodyRotxx{1}{2} & \bodyRotxy{1}{2} & \bodyRotxz{1}{2} & 0 \\
  \bodyRotyx{1}{2} & \bodyRotyy{1}{2} & \bodyRotyz{1}{2} & 0 \\
  \bodyRotzx{1}{2} & \bodyRotzz{1}{2} & \bodyRotzz{1}{2} & \LoadHeight \\
  0 & 0 & 0 & 1
 \end{bmatrix}.
\end{align}
The three arms are connected to the central body each by revolute joints with tilt angles $\aServo[k],\,k=1,2,3$.
The joint axis lie in the plane spanned by $\bodyRotx{0}{1}$ and $\bodyRoty{0}{1}$ and their angles to $\bodyRotx{0}{1}$ are $\ArmAngle[1] = \tfrac{\pi}{3}, \ArmAngle[2] = \pi, \ArmAngle[3] = -\tfrac{\pi}{3}$.
The body fixed axes are placed such that $\bodyRotx{0}{2k+1}$ coincide with the tilt axis and $\bodyRotz{0}{2k+1},\,k=1,2,3$ coincide with the propeller spinning axis.
The configuration of the $k\text{-th}$ arm w.r.t.\ the central body is
\begin{align}
 \bodyHomoCoord{1}{2k+1} &= 
 \begin{bmatrix}
  \cos\ArmAngle[k] & -\sin\ArmAngle[k] \cos\aServo[k] & \sin\ArmAngle[k] \sin\aServo[k] & \ArmRadius \cos\ArmAngle[k] \\
  \sin\ArmAngle[k] & \cos\ArmAngle[k] \cos\aServo[k] & -\cos\ArmAngle[k] \sin\aServo[k] & \ArmRadius \sin\ArmAngle[k] \\ 
  0 & \sin\aServo[k] & \cos\aServo[k] & 0 \\
  0 & 0 & 0 & 1
 \end{bmatrix},
 \quad 
 k = 1,\ldots,3.
\end{align}
The propellers are connected by revolute joints to the arms.
The body fixed frame is attached to the geometric center of the propeller (which will be an important point for its aerodynamic model).
The configuration w.r.t.\ the corresponding arm is
\begin{align}
 \bodyHomoCoord{2k+1}{2k+2} &= 
 \begin{bmatrix}
  \cPropAngle[k] & -\sPropAngle[k] & 0 & 0 \\
  \sPropAngle[k] & \cPropAngle[k] & 0 & 0 \\ 
  0 & 0 & 1 & \PropHeight \\
  0 & 0 & 0 & 1
 \end{bmatrix},
 \qquad
 k = 1,\ldots,3.
\end{align}
\end{subequations}
The set of configurations $\HomoCoordGraph_0 = \{ \bodyHomoCoord{0}{1}, \bodyHomoCoord{1}{2}, \bodyHomoCoord{1}{3}, \bodyHomoCoord{3}{4}, \bodyHomoCoord{1}{5}, \bodyHomoCoord{5}{6}, \bodyHomoCoord{1}{7}, \bodyHomoCoord{7}{8} \}$ form a directed graph as shown at the bottom of \autoref{fig:TriV3ConfigGraph}.
With them and the rules from \eqref{eq:rigidBodyRelCoordRules2} we can compute the configuration $\bodyHomoCoord{\BidxI}{\BidxII}$ of any body w.r.t.\ any other body or the inertial frame.

The configurations can be seen as functions $\bodyHomoCoord{\BidxI}{\BidxII}(\sysCoord)$ of the system coordinates
\begin{align}
 \sysCoord = \big[
  \bodyPosx{0}{1}, \bodyPosy{0}{1}, \bodyPosz{0}{1},
  \bodyRotxx{0}{1}, \ldots, \bodyRotzz{0}{1},
  \bodyRotxx{1}{2}, \ldots, \bodyRotzz{1}{2},
  \PropTilt[1], \PropTilt[2], \PropTilt[3],
  \cPropAngle[1], \sPropAngle[1],
  \cPropAngle[2], \sPropAngle[2],
  \cPropAngle[3], \sPropAngle[3]
 \big]^\top \in \RealNum^{30}
\end{align}
and the constant parameters $\LoadHeight, \ArmRadius, \ArmAngle[1], \ArmAngle[2], \ArmAngle[3], \PropHeight$.
From the rules \eqref{eq:rigidBodyRelCoordRules2} emerge the geometric constraints
\begin{align}\label{eq:TricopterExampleConstraint}
 \geoConstraint(\sysCoord) = \tuple{0} 
\qquad \cong \qquad 
 \begin{cases}
  \bodyRot{0}{1}^\top \bodyRot{0}{1} = \idMat[3], \, \det\bodyRot{0}{1} = +1, \\
  \bodyRot{1}{2}^\top \bodyRot{1}{2} = \idMat[3], \, \det\bodyRot{1}{2} = +1, \\
  (\cPropAngle[k])^2 + (\sPropAngle[k])^2 = 1, \ k=1,2,3
 \end{cases}
\end{align}
The configuration space of the rigid body system is
\begin{align}
 \configSpace = \{ \sysCoord \in \RealNum^{30} \, | \, \geoConstraint(\sysCoord) = 0 \} \ \cong \ \SpecialEuclideanGroup(3) \times \SpecialOrthogonalGroup(3) \times \RealNum^3 \times (\Sphere^1)^3.
\end{align}

This example was mainly chosen as the Tricopter will be discussed in the following chapters.
However, it is also an example of a system that is complex enough that one probably does not want to derive the equations of motion without a formalism.
It also covers the most common manifolds encountered in rigid body mechanics.
Even though the revolute joints for the propeller tilt and the propeller spinning axes both imply a $\Sphere^1$ manifold, the local parameterization by the angle $\aServo[k],k=1,2,3$ is chosen.
This has a practical motivation: The tilt mechanism also twists the cables to the propeller motor and so $\aServo[k] = 0$ and $\aServo[k] = 2\pi$ are really different situations in practice.
On the other hand it should also show that the following algorithm handles minimal coordinates just as fine.
\end{Example}

A generalization of the example states:
A rigid body system can be parameterized by a set of $\numCoord$ (possibly redundant) coordinates $\sysCoord$ which again parameterize a set of configurations $\bodyHomoCoord{\BidxI}{\BidxII}(\sysCoord)$ which form a \textit{connected} graph.
The property connected is essential: it ensures that, with the rules \eqref{eq:rigidBodyRelCoordRules2}, all remaining configurations of the graph can be computed \ie the corresponding \textit{complete} graph.
Loops in the graph and the property $\bodyHomoCoord{\BidxI}{\BidxII} \in \SpecialEuclideanGroup(3)$ may imply geometric constraints.

The use of graph theory in the context of algorithms for rigid body systems is quite common, see \eg \cite[sec.\,8.2]{Schwertassek:MultibodySystems} or \cite[sec.\,5.3]{Wittenburg:DynamicsOfMultibodySystems}.
However, we will not go any deeper into this.
All we need for the following is that any configuration $\bodyHomoCoord{\BidxI}{\BidxII}(\sysCoord), \BidxI,\BidxII = 0\ldots\numRigidBodies$ can be expressed in terms of the configuration coordinates $\sysCoord$.

\paragraph{Body velocity.}
The previous section motivated particular velocity coordinates $\bodyVel{}{} = [\v^\top, \w^\top]^\top$ for the free rigid body, which did lead to a convenient mathematical expressions.
In the context of rigid body systems we may associate a \textit{body velocity} $\bodyVel{\BidxI}{\BidxII} = [\bodyLinVel{\BidxI}{\BidxII}^\top, \bodyAngVel{\BidxI}{\BidxII}^\top]^\top$ with any configuration $\bodyHomoCoord{\BidxI}{\BidxII}, \BidxI,\BidxII = 0,\ldots,\numRigidBodies$ defined by 
\begin{align}\label{eq:DefBodyVelocityMulti}
 \bodyVel{\BidxI}{\BidxII} = \veeOp\big(\bodyHomoCoord{\BidxII}{\BidxI} \bodyHomoCoordd{\BidxI}{\BidxII}{}{} \big), \quad \BidxI,\BidxII = 0,\ldots,\numRigidBodies.
\end{align}
From the rules \eqref{eq:rigidBodyRelCoordRules2} for the configurations we can conclude similar rules for their velocities:
For the composition $\bodyHomoCoord{\BidxI}{\BidxIII}{}{} = \bodyHomoCoord{\BidxI}{\BidxII}{}{} \bodyHomoCoord{\BidxII}{\BidxIII}{}{}$ we get
\begin{subequations}\label{eq:rulesBodyVelocity}
\begin{align}
 \bodyVel{\BidxI}{\BidxIII} = \veeOp\big(\bodyHomoCoord{\BidxIII}{\BidxII} \bodyHomoCoord{\BidxII}{\BidxI}(\bodyHomoCoordd{\BidxI}{\BidxII} \bodyHomoCoord{\BidxII}{\BidxIII} + \bodyHomoCoord{\BidxI}{\BidxII} \bodyHomoCoordd{\BidxII}{\BidxIII})\big)
 = \veeOp\big(\bodyHomoCoord{\BidxIII}{\BidxII} \wedOp(\bodyVel{\BidxI}{\BidxII}) \bodyHomoCoord{\BidxII}{\BidxIII} \big) + \bodyVel{\BidxII}{\BidxIII}
 = \Ad{\bodyHomoCoord{\BidxIII}{\BidxII}} \bodyVel{\BidxI}{\BidxII} + \bodyVel{\BidxII}{\BidxIII}
\end{align}
% where
% \begin{align*}
%  \Ad{\bodyHomoCoord{}{}} = \begin{bmatrix} R & \widehat{r} R \\ 0 & R \end{bmatrix},
% \qquad
%  \Ad{\bodyHomoCoord{}{}^{-1}} = \begin{bmatrix} R^\top & -R^\top \widehat{r} \\ 0 & R^\top \end{bmatrix} = \Ad{\bodyHomoCoord{}{}}^{-1}
%  .
% \end{align*}
Differentiation of the rule for the inverse yields 
\begin{align}
 \tdiff{t} \big(\bodyHomoCoord{\BidxI}{\BidxII} \bodyHomoCoord{\BidxII}{\BidxI}\big)
 &= \bodyHomoCoordd{\BidxI}{\BidxII} \bodyHomoCoord{\BidxII}{\BidxI} + \bodyHomoCoord{\BidxI}{\BidxII} \bodyHomoCoordd{\BidxII}{\BidxI}
 = \bodyHomoCoord{\BidxI}{\BidxII} \wedOp(\bodyVel{\BidxI}{\BidxII}) \bodyHomoCoord{\BidxII}{\BidxI} + \wedOp(\bodyVel{\BidxII}{\BidxI})
 = \mat{0}&
 &\Leftrightarrow&
 \bodyVel{\BidxII}{\BidxI}{} &= -\Ad{\bodyHomoCoord{\BidxI}{\BidxII}{}{}} \bodyVel{\BidxI}{\BidxII}{}
\end{align}
and obviously 
\begin{align}
 \bodyVel{\BidxI}{\BidxI}{} = \tuple{0}.
\end{align}
\end{subequations}

\paragraph{System velocity and body Jacobians.}
Based on their definition \eqref{eq:DefBodyVelocityMulti}, the body velocities $\bodyVel{\BidxI}{\BidxII}$ can be seen as a function of the system coordinates $\sysCoord$ and their derivatives $\sysCoordd = \kinMat(\sysCoord)\sysVel$.
Crucially the velocity is linear in $\sysCoordd$ and consequently linear in the system velocity $\sysVel$ and we can write
\begin{align}\label{eq:DefBodyJacobian}
 \bodyVel{\BidxI}{\BidxII}(\sysCoord, \sysVel) &= \bodyJac{\BidxI}{\BidxII}(\sysCoord) \sysVel,&
 \bodyJac{\BidxI}{\BidxII}(\sysCoord) &= \pdiff[\,\bodyVel{\BidxI}{\BidxII}]{\sysVel}(\sysCoord) = \pdiff{\sysCoordd} \veeOp\Big( \bodyHomoCoord{\BidxII}{\BidxI}(\sysCoord) \diff{t} \big(\bodyHomoCoord{\BidxI}{\BidxII}(\sysCoord) \big) \Big) \kinMat.
\end{align}
The matrix $\bodyJac{\BidxI}{\BidxII}(\sysCoord) \in \RealNum^{6\times\dimConfigSpace}$ that maps the system velocity $\sysVel$ to the body velocity $\bodyVel{\BidxI}{\BidxII}$ is commonly called the \textit{body Jacobian}.
\fixme{An alternative formula for the body Jacobian, which might give additional geometric insight, is given in [eq:AppendixDefBodyJac].} 
The following rules emerge directly from \eqref{eq:rulesBodyVelocity}:
\begin{align}\label{eq:rulesBodyJacobian}
 \bodyJac{\BidxI}{\BidxIII} &= \Ad{\bodyHomoCoord{\BidxIII}{\BidxII}} \bodyJac{\BidxI}{\BidxII} + \bodyJac{\BidxII}{\BidxIII},&
 \bodyJac{\BidxII}{\BidxI} &= -\Ad{\bodyHomoCoord{\BidxI}{\BidxII}} \bodyJac{\BidxI}{\BidxII},&
 \bodyJac{\BidxI}{\BidxI} &= \mat{0}.
\end{align}

\begin{Example}
\textbf{Tricopter with suspended load: kinematics.}
For the tricopter with load from Example \ref{exp:TricopterWithLoadConfiguration} we chose the following velocity coordinates:
The components of the body velocity $\bodyVel{0}{1}$ of the central body w.r.t.\ the inertial frame, the components of the angular velocity $\bodyAngVel{0}{2}$ of the load w.r.t.\ the inertial frame, the angular velocities $\aServod[k], k=1,2,3$ of the arm tilt mechanism and the angular velocities $\PropVel[k], k=1,2,3$ of the propellers w.r.t.\ the arms.
These velocity coordinates $\sysVel = [\bodyVel{0}{1}^\top, \bodyAngVel{0}{2}^\top, \aServod[1], \aServod[2], \aServod[3], \PropVel[1], \PropVel[2], \PropVel[3]]^\top$ are related to the configuration coordinates $\sysVel$ by the kinematic equation
\begin{align}
 \sysCoordd &= \kinMat \sysVel
\qquad\cong\qquad
 \begin{cases}
  \bodyHomoCoordd{0}{1} = \bodyHomoCoord{0}{1} \wedOp(\bodyVel{0}{1}), \\
  \bodyRotd{1}{2} = \bodyRot{1}{2} \wedOp(\bodyAngVel{0}{2}) - \wedOp(\bodyAngVel{0}{1}) \bodyRot{1}{2}, \\
  \aServod[k] = \aServod[k], \ k=1,2,3 \\
  \cPropAngled[k] = -\sPropAngle[k]\PropVel[k], \ k=1,2,3 \\
  \sPropAngled[k] =  \cPropAngle[k]\PropVel[k], \ k=1,2,3
 \end{cases}
 .
\end{align}
The relative velocity $\bodyAngVel{1}{2} = \veeOp(\bodyRot{1}{2}^\top \bodyRotd{1}{2})$ of the load would be another possible and probably more obvious choice.
The absolute velocity $\bodyAngVel{0}{2}$ is mainly chosen to demonstrate the flexibility of the presented approach but the use of absolute velocities also leads to less cumbersome terms in the system inertia matrix.

The body velocities associated with the configuration matrices from \eqref{eq:configurationMatricesTricopterExample} are
\begin{subequations}
\begin{align}
 \bodyVel{0}{1} &= \begin{bmatrix} \bodyLinVelx{0}{1} \\ \bodyLinVely{0}{1} \\ \bodyLinVelz{0}{1} \\ \bodyAngVelx{0}{1} \\ \bodyAngVely{0}{1} \\ \bodyAngVelz{0}{1} \end{bmatrix},&
 \bodyVel{1}{2} &= 
 \begin{bmatrix}
  0 \\ 0 \\ 0 \\
  \bodyAngVelx{0}{2} - \bodyRotxx{1}{2} \bodyAngVelx{0}{2} - \bodyRotyx{1}{2} \bodyAngVely{0}{2} - \bodyRotzx{1}{2} \bodyAngVelz{0}{2} \\
  \bodyAngVely{0}{2} - \bodyRotxy{1}{2} \bodyAngVelx{0}{2} - \bodyRotyy{1}{2} \bodyAngVely{0}{2} - \bodyRotzy{1}{2} \bodyAngVelz{0}{2} \\
  \bodyAngVelz{0}{2} - \bodyRotxz{1}{2} \bodyAngVelx{0}{2} - \bodyRotyz{1}{2} \bodyAngVely{0}{2} - \bodyRotzz{1}{2} \bodyAngVelz{0}{2} 
 \end{bmatrix},&
\\
 \bodyVel{1}{2k+1} &= \begin{bmatrix} 0 \\ 0 \\ 0 \\ \aServod[k] \\ 0 \\ 0 \end{bmatrix},&
 \bodyVel{2k+1}{2k+2} &= \begin{bmatrix} 0 \\ 0 \\ 0 \\ 0 \\ 0 \\ \PropVel[k] \end{bmatrix},
\quad k=1,2,3.
\end{align}
\end{subequations}
From this it should clear how the corresponding body Jacobians look like, \eg $\bodyJac{0}{1} = [\idMat[6] \ 0]$.

For the formulation of the kinetic energy in the following subsection we will need the body Jacobians $\bodyJac{0}{\BidxII}, \BidxII=1,\ldots,\numRigidBodies$.
From the graph structure and the rules \eqref{eq:rulesBodyJacobian} we can compute them iteratively as
\begin{subequations}
\begin{align}
 \bodyJac{0}{2} &= \Ad{\bodyHomoCoord{2}{1}} \ \bodyJac{0}{1} + \bodyJac{1}{2},
\\
 \bodyJac{0}{2k+1} &= \Ad{\bodyHomoCoord{2k+1}{1}} \ \bodyJac{0}{1} + \bodyJac{1}{2k+1},
\\
 \bodyJac{0}{2k+2} &= \Ad{\bodyHomoCoord{2k+2}{2k+1}} \ \bodyJac{0}{2k+1} + \bodyJac{2k+1}{2k+2}.
\end{align}
\end{subequations}
These terms are significantly more cumbersome, so they are not displayed explicitly.
\end{Example}



\subsection{Inertia}\label{sec:RBSInertia}
\paragraph{Kinetic energy and inertia matrix.}
The kinetic energy $\kineticEnergy$ of a rigid body system is simply the sum of the kinetic energies of its bodies.
Combining this with the kinetic energy \eqref{eq:RigidBodyKineticEnergy} of a single free rigid body and the formulation of the absolute body velocities $\bodyVel{0}{\BidxII}$ in terms of the chosen coordinates using the body Jacobian $\bodyJac{0}{\BidxII}$ from \eqref{eq:DefBodyJacobian}, yields
\begin{align}\label{eq:RBSKineticEnergy}
 \kineticEnergy
 &= \sumBodies \tfrac{1}{2} \bodyVel{0}{\BidxII}^\top \bodyInertiaMat{0}{\BidxII}\bodyVel{0}{\BidxII}
 = \tfrac{1}{2} \sysVel^\top \underbrace{\sumBodies \bodyJac{0}{\BidxII}^\top \bodyInertiaMat{0}{\BidxII} \bodyJac{0}{\BidxII}}_{\sysInertiaMat} \sysVel.
\end{align}
Recall from the previous section, that the constant \textit{body} inertia matrix $\bodyInertiaMat{0}{\BidxII} \in \RealNum^{6\times6}$ collects the inertia parameters of the rigid body with index $\BidxII$ and w.r.t.\ its body fixed frame. 
The matrix $\sysInertiaMat(\sysCoord) \in \RealNum^{\dimConfigSpace\times\dimConfigSpace}$ is the \textit{system} inertia matrix.

\paragraph{Connection coefficients.}
\fixme{
With a rather cumbersome computation (see \fixme{[eq:AppendixSysConnCoeffL}]), it can be shown that the connection coefficients $\ConnCoeffL{\LidxI}{\LidxII}{\LidxIII}$ associated to the system inertia matrix $\sysInertiaMat$ from \eqref{eq:RBSKineticEnergy} can be expressed in terms of the body Jacobians $\bodyJac{0}{\BidxII}$, the body inertia matrices $\bodyInertiaMat{0}{\BidxII}$ and the body connection coefficients $\bodyConnCoeffL{0}{\BidxII}{\LBidxI}{\LBidxII}{\LBidxIII}$ from \eqref{eq:RBConnCoeffRes} or the body commutation coefficients $\bodyBoltzSym{\LBidxIV}{\LBidxI}{\LBidxII}$ from \eqref{eq:RBCommutationCoeff} as
\begin{align}\label{eq:RBSConnCoeff}
 \ConnCoeffL{\LidxI}{\LidxII}{\LidxIII}
 &= \sumBodies \bodyJacCoeff{0}{\BidxII}{\LBidxI}{\LidxI} \big( \bodyInertiaMatCoeff{0}{\BidxII}{\LBidxI\LBidxII} \dirDiff{\LidxIII} \bodyJacCoeff{0}{\BidxII}{\LBidxII}{\LidxII} + \underbrace{\tfrac{1}{2} \big( \bodyBoltzSym{\LBidxIV}{\LBidxI}{\LBidxII} \bodyInertiaMatCoeff{0}{\BidxII}{\LBidxIV\LBidxIII} + \bodyBoltzSym{\LBidxIV}{\LBidxI}{\LBidxIII} \bodyInertiaMatCoeff{0}{\BidxII}{\LBidxIV\LBidxII} - \bodyBoltzSym{\LBidxIV}{\LBidxII}{\LBidxIII} \bodyInertiaMatCoeff{0}{\BidxII}{\LBidxIV\LBidxI} \big)}_{\bodyConnCoeffL{0}{\BidxII}{\LBidxI}{\LBidxII}{\LBidxIII}} \bodyJacCoeff{0}{\BidxII}{\LBidxII}{\LidxII} \bodyJacCoeff{0}{\BidxII}{\LBidxIII}{\LidxIII} \big)
\end{align}
% With this we can state the gyroscopic terms as 
% \begin{align}\label{eq:RBSGyroCoeff}
%  \gyroForceCoeff{\LidxI}
%  &= \ConnCoeffL{\LidxI}{\LidxII}{\LidxIII} \sysVelCoeff{\LidxII} \sysVelCoeff{\LidxIII}
%   = \sumBodies \bodyJacCoeff{0}{\BidxII}{\LBidxI}{\LidxI} \big( \bodyInertiaMatCoeff{0}{\BidxII}{\LBidxI\LBidxII} \dirDiff{\LidxIII} \bodyJacCoeff{0}{\BidxII}{\LBidxII}{\LidxII} + \bodyJacCoeff{0}{\BidxII}{\LBidxII}{\LidxII} \bodyBoltzSym{\LBidxIV}{\LBidxI}{\LBidxII} \bodyInertiaMatCoeff{0}{\BidxII}{\LBidxIV\LBidxIII} \bodyJacCoeff{0}{\BidxII}{\LBidxIII}{\LidxIII} \big) \sysVelCoeff{\LidxII} \sysVelCoeff{\LidxIII}
% \end{align}
}

\paragraph{Acceleration energy.}
As with the kinetic energy, the acceleration energy of a rigid body system is simply the sum of the acceleration energies of its bodies.
Summing up the body acceleration energies from \eqref{eq:RBAccEnergyHomo} and plugging in the body velocities $\bodyVel{0}{\BidxII} = \bodyJac{0}{\BidxII} \sysVel$ yields
\begin{align}\label{eq:RBSAccEnergy}
 \accEnergy
 &= \sumBodies \tfrac{1}{2} \norm[\bodyInertiaMatp{0}{\BidxII}]{(\bodyHomoCoorddd{0}{\BidxII})^\top}^2
%\nonumber\\
% &= \sumBodies \tfrac{1}{2} \norm[\bodyInertiaMatp{0}{\BidxII}]{(\wedOp(\bodyVeld{0}{\BidxII}) + \wedOp(\bodyVel{0}{\BidxII})^2)^\top}^2
\nonumber\\
 &= \sumBodies \tfrac{1}{2} \norm[\bodyInertiaMatp{0}{\BidxII}]{(\wedOp(\bodyJac{0}{\BidxII} \sysVeld + \bodyJacd{0}{\BidxII} \sysVel) + \wedOp(\bodyJac{0}{\BidxII} \sysVel)^2)^\top}^2
\nonumber\\
 &= \sumBodies \tfrac{1}{2} \norm[\bodyInertiaMatp{0}{\BidxII}]{(\wedOp(\bodyJac{0}{\BidxII} \sysVeld))^\top}^2
 + \sumBodies \tr\big( \wedOp(\bodyJac{0}{\BidxII} \sysVeld) \bodyInertiaMatp{0}{\BidxII} (\wedOp(\bodyJacd{0}{\BidxII} \sysVel) + \wedOp(\bodyJac{0}{\BidxII} \sysVel)^2)^\top \big)
\nonumber\\
 &\quad + \underbrace{\sumBodies \tfrac{1}{2} \norm[\bodyInertiaMatp{0}{\BidxII}]{(\wedOp(\bodyJacd{0}{\BidxII} \sysVel) + \wedOp(\bodyJac{0}{\BidxII} \sysVel)^2)^\top}^2}_{\accEnergy_0}
\nonumber\\
 &= \tfrac{1}{2} \sysVeld^\top \underbrace{\sumBodies \bodyJac{0}{\BidxII}^\top \bodyInertiaMat{0}{\BidxII} \bodyJac{0}{\BidxII}}_{\sysInertiaMat} \sysVeld
 + \sysVeld^\top \underbrace{\sumBodies \bodyJac{0}{\BidxII}^\top \big(\bodyInertiaMat{0}{\BidxII} \bodyJacd{0}{\BidxII} - \ad{\bodyJac{0}{\BidxII}\sysVel}^\top \bodyInertiaMat{0}{\BidxII}\bodyJac{0}{\BidxII} \big) \sysVel}_{\gyroForce}
 + \, \accEnergy_0.
\end{align}
Obviously, we found again the system inertia matrix $\sysInertiaMat$ and one may check that indeed $\gyroForceCoeff{\LidxI} = \ConnCoeffL{\LidxI}{\LidxII}{\LidxIII} \sysVelCoeff{\LidxII} \sysVelCoeff{\LidxIII}$ with the connection coefficients $\ConnCoeffL{\LidxI}{\LidxII}{\LidxIII}$ from \eqref{eq:RBSConnCoeff}.
Note that $\accEnergy_0$ is independent of $\sysVeld$, so it does not contribute to the generalized inertia force.

\paragraph{Inertia force.}
As before, there are several equivalent ways for computing the inertia force $\genForceInertia$ of a rigid body system:
We may use the Lagrange operator on the kinetic energy from \eqref{eq:RBSKineticEnergy}, use the inertia matrix and connection coefficients from \eqref{eq:RBSConnCoeff}, or taking the differential of the acceleration energy from \eqref{eq:RBSAccEnergy}.
% \begin{itemize}
%  \item $\genForceInertiaCoeff{\LidxI} = \sysInertiaMatCoeff{\LidxI\LidxII} \sysVelCoeffd{\LidxII} + \ConnCoeffL{\LidxI}{\LidxII}{\LidxIII} \sysVelCoeff{\LidxIII} \sysVelCoeff{\LidxII}$ using the inertia matrix $\sysInertiaMatCoeff{\LidxI\LidxII}$ from \eqref{eq:RBSKineticEnergy} and the connection coefficients $\ConnCoeffL{\LidxI}{\LidxII}{\LidxIII}$ from \eqref{eq:RBSConnCoeff}.
%  \item $\genForceInertiaCoeff{\LidxI} = \diff{t} \pdiff[\kineticEnergy]{\sysVelCoeff{\LidxI}} + \BoltzSym{\LidxIII}{\LidxI}{\LidxII} \sysVelCoeff{\LidxII} \pdiff[\kineticEnergy]{\sysVelCoeff{\LidxIII}} - \dirDiff{\LidxI} \kineticEnergy$ with the kinetic energy $\kineticEnergy$ from \eqref{eq:RBSKineticEnergy}.
%  \item $\genForceInertiaCoeff{\LidxI} = \pdiff[\accEnergy]{\sysVelCoeffd{\LidxI}}$ using the acceleration energy $\accEnergy$ from \eqref{eq:RBSAccEnergy}.
% \end{itemize}
Each of these approaches will yield
\begin{align}\label{eq:RBSInertiaForce}
% \genForceInertia(\sysCoord, \sysVel, \sysVeld) = \sysInertiaMat(\sysCoord) \sysVeld + \gyroForce(\sysCoord, \sysVel)
 \genForceInertia 
 = \underbrace{\sumBodies \bodyJac{0}{\BidxII}^\top \bodyInertiaMat{0}{\BidxII} \bodyJac{0}{\BidxII}}_{\sysInertiaMat} \sysVeld
 + \underbrace{\sumBodies \bodyJac{0}{\BidxII}^\top \big(\bodyInertiaMat{0}{\BidxII} \bodyJacd{0}{\BidxII} - \ad{\bodyJac{0}{\BidxII}\sysVel}^\top \bodyInertiaMat{0}{\BidxII}\bodyJac{0}{\BidxII} \big) \sysVel}_{\gyroForce}.
\end{align}
Similar results are called \textit{the projection equation} in \cite[sec.\ 4.2.5]{Bremer:ElasticMultibodyDynamics} and \textit{the Kane equations} \cite[chap.\ 6]{Kane:Dynamics}.
There is some controversy (starting in \cite{Desloge:KaneAppell}) about the naming, since the equations result rather directly (as shown above) from the Gibbs-Appell formulation.
See \cite{Lesser:GeometricInterpretationOfKanesEquations} or \cite[p.\ 714]{Papastavridis:AnalyticalMechanics} for an overview.

\fixme{
In contrast to the derivations in the sources above, the formulation \eqref{eq:RBSInertiaForce} poses no restrictions on the body fixed frames and allows redundant configuration coordinates.
}

\subsection{Gravitation}
The potential energy of gravitation of a rigid body system is the sum of the potentials of the individual bodies \eqref{eq:RBGravityEnergyHomo}.
This is
\begin{align}\label{eq:RBSPotentialGravity}
 \potentialGravity
% &= \sumBodies \bodyMass{0}{\BidxII} \sProd{ \bodyPos{0}{\BidxII}(\sysCoord) + \bodyRot{0}{\BidxII}(\sysCoord) \, \bodyCOM{0}{\BidxII}}{ \gravityAcc}
%\nonumber\\
 &= \sumBodies \sProd[\bodyInertiaMatp{0}{\BidxII}]{(\bodyHomoCoord{0}{\BidxII})^\top}{\wedOp(\gravityAccWrench)^\top},
\qquad 
 \gravityAccWrench^\top = [\gravityAcc^\top, \tuple{0}_{1\times3} ],
\end{align}
where $\bodyInertiaMatp{0}{\BidxII} = \veeMatOp(\bodyInertiaMat{0}{\BidxII})$ is the body inertia matrix and $\gravityAccWrench$ is earth's gravity wrench.

Finally, the generalized force of gravity on a rigid body system may be formulated as
\begin{multline}\label{eq:RBSGenForceGravity}
 \genForceGravity = \differential\potentialGravity
 = \frac{\partial \potentialGravityd}{\partial \sysVel}
 = \frac{\partial}{\partial \sysVel} \sumBodies \tr\big( \bodyHomoCoord{0}{\BidxII} \wedOp(\bodyVel{0}{\BidxII}) \bodyInertiaMatp{0}{\BidxII} \wedOp(\gravityAccWrench)^\top \big)
\\ 
 = \sumBodies \Big(\frac{\partial \bodyVel{0}{\BidxII}}{\partial \sysVel}\Big)^\top \veeTwoOp\big( \bodyHomoCoord{0}{\BidxII}^\top \wedOp(\gravityAccWrench) \bodyInertiaMatp{0}{\BidxII} \big) 
 = \sumBodies \bodyJac{0}{\BidxII}^\top \bodyInertiaMat{0}{\BidxII} \, \Ad{\bodyHomoCoord{0}{\BidxII}}^{-1} \gravityAccWrench.
\end{multline}

\subsection{Stiffness}
In \autoref{sec:RBStiffness} we considered linear springs between arbitrary points of the body and the inertial frame.
For a system of rigid bodies we may consider the same for each body, but additionally we may also consider springs connecting the bodies to each other, see \autoref{fig:MultibodyStiffnessIllustration}.

\begin{figure}[ht]
 \centering
 \input{graphics/MultibodyPotentialIllustration.pdf_tex}
 \caption{MultibodyStiffnessIllustration}
 \label{fig:MultibodyStiffnessIllustration}
\end{figure}

The total potential energy $\potentialStiff$ is the sum of the potentials of the individual springs.
Here it will make sense to group them differently:
Let $\bodyPotentialStiff{\BidxI}{\BidxII}$ with $0 \leq \BidxI < \BidxII \leq \numRigidBodies$ denote the combined potential of all springs connecting body $\BidxI$ and $\BidxII$.
The total energy is
\begin{align}
 \potentialStiff
 = \sum_{\BidxI=0}^{\numRigidBodies-1} \sum_{\BidxII=\BidxI+1}^{\numRigidBodies} \bodyPotentialStiff{\BidxI}{\BidxII}
\end{align}

Using the results from \autoref{sec:RBStiffness} it may be shown that each potential can be formulated as
\begin{align}
 \bodyPotentialStiff{\BidxI}{\BidxII} &= \tfrac{1}{2} \norm[\bodyStiffMatp{\BidxI}{\BidxII}]{(\bodyHomoCoord{\BidxI}{\BidxII} - \bodyHomoCoordR{\BidxI}{\BidxII})^\top}^2
 = \tfrac{1}{2} \norm[\bodyStiffMatp{\BidxI}{\BidxII}]{(\bodyHomoCoordR{\BidxI}{\BidxII}^{-1} \bodyHomoCoord{\BidxI}{\BidxII} - \idMat[4])^\top}^2
\end{align}
with the constant parameters $\bodyStiffMatp{\BidxI}{\BidxII} \in \SymMat(4)$ and $\bodyHomoCoordR{\BidxI}{\BidxII}\in \SpecialEuclideanGroup(3)$ resulting from the particular spring distribution between body $\BidxI$ and $\BidxII$.

Finally, the generalized force due to an arbitrary constellation of linear springs on a rigid body system may be formulated as
\begin{multline}\label{eq:RBSGenForceStiff}
 \genForceStiff = \differential\potentialStiff
 = \frac{\partial \potentialStiffd}{\partial \sysVel}
 = \frac{\partial}{\partial \sysVel} \sumBodiesAB \tr\big((\bodyHomoCoord{\BidxI}{\BidxII} - \bodyHomoCoordR{\BidxI}{\BidxII}) \bodyStiffMatp{\BidxI}{\BidxII} (\bodyHomoCoord{\BidxI}{\BidxII} \wedOp(\bodyVel{\BidxI}{\BidxII}))^\top \big)
% = \frac{\partial}{\partial \sysVel} \sumBodiesAB \tr\big((\bodyHomoCoordR{\BidxI}{\BidxII}^{-1} \bodyHomoCoord{\BidxI}{\BidxII} - \idMat[4]) \bodyStiffMatp{\BidxI}{\BidxII} (\bodyHomoCoordR{\BidxI}{\BidxII}^{-1} \bodyHomoCoord{\BidxI}{\BidxII} \wedOp(\bodyVel{\BidxI}{\BidxII}))^\top \big)
\\ 
 = \sumBodiesAB \Big(\frac{\partial \bodyVel{\BidxI}{\BidxII}}{\partial \sysVel}\Big)^\top \veeTwoOp\big(\bodyHomoCoord{\BidxI}{\BidxII}^\top (\bodyHomoCoord{\BidxI}{\BidxII} - \bodyHomoCoordR{\BidxI}{\BidxII}) \bodyStiffMatp{\BidxI}{\BidxII} \big)
\\
 = \sumBodiesAB \bodyJac{\BidxI}{\BidxII}^\top \veeTwoOp\big((\idMat[4] - \bodyHomoCoord{\BidxI}{\BidxII}^{-1} \bodyHomoCoordR{\BidxI}{\BidxII}) \bodyStiffMatp{\BidxI}{\BidxII} \big)
\end{multline}


\subsection{Dissipation}
Similar to the previous subsection we may consider viscous friction of the bodies to each other and to the inertial frame.
Let the body with index $\BidxII$ move through a viscous fluid that is attached to the body with index $\BidxI$.
The corresponding dissipation function $\bodyDissFkt{\BidxI}{\BidxII}$ was derived in \eqref{eq:RBDissFkt}:
\begin{align}
 \bodyDissFkt{\BidxI}{\BidxII} = \tfrac{1}{2} \bodyVel{\BidxI}{\BidxII}^\top \bodyDissMat{\BidxI}{\BidxII} \bodyVel{\BidxI}{\BidxII}
\end{align}
with the body dissipation matrix $\bodyDissMat{\BidxI}{\BidxII}$.
Notice that in contrast to stiffness, the dissipation is, in general, not symmetric in the sense $\bodyDissFkt{\BidxI}{\BidxII} \neq \bodyDissFkt{\BidxII}{\BidxI}$.
But due to $\bodyVel{\BidxI}{\BidxI} = \tuple{0}$ we have $\bodyDissFkt{\BidxI}{\BidxII}=0$.
For system of rigid bodies we have the dissipation function
\begin{align}
 \dissFkt = \sum_{\BidxI=0}^{\numRigidBodies} \sum_{\BidxII=0, \BidxII\neq\BidxI}^{\numRigidBodies} \bodyDissFkt{\BidxI}{\BidxII}
 = \tfrac{1}{2} \sysVel^\top \underbrace{\sum_{\BidxI=0}^{\numRigidBodies} \sum_{\BidxII=0, \BidxII\neq\BidxI}^{\numRigidBodies} \bodyJac{\BidxI}{\BidxII}^\top \bodyDissMat{\BidxI}{\BidxII} \bodyJac{\BidxI}{\BidxII}}_{\sysDissMat} \sysVel
\end{align}
where $\sysDissMat$ is called the system dissipation matrix.

Finally, the generalized force due to viscous friction on a rigid body system may be formulated as
\begin{align}
 \genForceDiss = \pdiff[\dissFkt]{\sysVel} = \sysDissMat \sysVel
\end{align}

\subsection{Summary}
There are three ingredients:
\begin{enumerate}
 \item the chosen parameterization in the configuration coordinates $\sysCoord$, the velocity coordinates $\sysVel$ and their relation captured by the kinematics matrix $\kinMat$
 \item the rigid body graph $\bodyHomoCoord{\BidxI}{\BidxII}$ that maps the coordinates to the rigid body configuration.
 \item the constitutive parameters merged into the inertia matrices $\bodyInertiaMatp{0}{\BidxII}$, dissipation matrices $\bodyDissMatp{\BidxI}{\BidxII}$, stiffness matrices $\bodyStiffMatp{\BidxI}{\BidxII}$, their corresponding minimum $\bodyHomoCoordR{\BidxI}{\BidxII}$ and the gravity vector $\gravityAcc$.
 It should be stressed that these are independent of the chosen coordinates.
\end{enumerate}
% There is an algorithm that takes the system parameterization $(\sysCoord, \sysVel, \kinMat)$, the 
% \begin{align}
%  \big( \sysCoord, \sysVel, \kinMat, \bodyHomoCoord{\BidxI}{\BidxII}, \bodyInertiaMatp{0}{\BidxII}, \bodyDissMatp{\BidxI}{\BidxII}, \bodyStiffMatp{\BidxI}{\BidxII}, \bodyHomoCoordR{\BidxI}{\BidxII}, \gravityAcc \big) \ \mapsto \ \big( \sysInertiaMat, \gyroForce, \genForceDiss, \genForceStiff, \genForceGravity \big)
% \end{align}
% which is

Based on this input we may compute the equations of motion in three steps:
\begin{enumerate}
\item compute the body Jacobians for the given configurations:
\begin{align}\label{eq:RBSAlogJacobian}
 \bodyJac{\BidxI}{\BidxII}(\sysCoord) = \pdiff{\sysCoordd} \veeOp\Big( \bodyHomoCoord{\BidxII}{\BidxI}(\sysCoord) \diff{t} \big(\bodyHomoCoord{\BidxI}{\BidxII}(\sysCoord) \big) \Big) \kinMat(\sysCoord)
\end{align}
\item use the group rules to compute the missing configurations and Jacobians
\begin{subequations}
\begin{align}
 \bodyHomoCoord{\BidxI}{\BidxIII} &= \bodyHomoCoord{\BidxI}{\BidxII} \bodyHomoCoord{\BidxII}{\BidxIII},&
 \bodyHomoCoord{\BidxII}{\BidxI} &= \bodyHomoCoord{\BidxI}{\BidxII}^{-1},&
\\
 \bodyJac{\BidxI}{\BidxIII} &= \Ad{\bodyHomoCoord{\BidxIII}{\BidxII}} \bodyJac{\BidxI}{\BidxII} + \bodyJac{\BidxII}{\BidxIII},&
 \bodyJac{\BidxII}{\BidxI} &= -\Ad{\bodyHomoCoord{\BidxI}{\BidxII}} \bodyJac{\BidxI}{\BidxII},&
\\
 \bodyJacd{\BidxI}{\BidxIII} &= \Ad{\bodyHomoCoord{\BidxIII}{\BidxII}} \big(\bodyJacd{\BidxI}{\BidxII} + \ad{\bodyJac{\BidxIII}{\BidxII}\sysVel} \bodyJac{\BidxI}{\BidxII} \big) + \bodyJacd{\BidxII}{\BidxIII},&
 \bodyJacd{\BidxII}{\BidxI} &= -\Ad{\bodyHomoCoord{\BidxI}{\BidxII}} \big(\bodyJacd{\BidxI}{\BidxII} + \ad{\bodyJac{\BidxI}{\BidxII}\sysVel} \bodyJac{\BidxI}{\BidxII} \big),&
% \BidxI, \BidxII, \BidxIII &= 0,\ldots,\numRigidBodies.
\end{align}
\end{subequations} 
\item assemble the system matrices
\begin{subequations}
\begin{align}
 \sysInertiaMat &= \sumBodies \bodyJac{0}{\BidxII}^\top \veeMatOp(\bodyInertiaMatp{0}{\BidxII}) \bodyJac{0}{\BidxII},
\\
 \gyroForce &= \sumBodies \bodyJac{0}{\BidxII}^\top \veeTwoOp \big(\big(\wedOp(\bodyJacd{0}{\BidxII} \sysVel) + \wedOp(\bodyJac{0}{\BidxII} \sysVel)^2 \big) \bodyInertiaMatp{0}{\BidxII} \big) 
 %= \sum_{\BidxII} \big( \bodyJac{0}{\BidxII}^\top \bodyInertiaMat{0}{\BidxII} \bodyJacd{0}{\BidxII} - \ad{\bodyJac{0}{\BidxII}\sysVel}^\top \bodyInertiaMat{0}{\BidxII} \bodyJac{0}{\BidxII}\big) \sysVel
\\
 \sysDissMat &= \sumBodiesAB \bodyJac{\BidxI}{\BidxII}^\top \veeMatOp(\bodyDissMatp{\BidxI}{\BidxII}) \bodyJac{\BidxI}{\BidxII}
\\
 \genForceStiff &= \sumBodiesAB \bodyJac{\BidxI}{\BidxII}^\top \veeTwoOp \big( (\idMat[4] - \bodyHomoCoord{\BidxII}{\BidxI} \bodyHomoCoordR{\BidxI}{\BidxII}) \bodyStiffMatp{\BidxI}{\BidxII} \big)
\\
 \genForceGravity &= \sumBodies \bodyJac{0}{\BidxII}^\top \veeMatOp(\bodyInertiaMatp{0}{\BidxII}) \Ad{\bodyHomoCoord{\BidxII}{0}} \gravityAccWrench, \qquad \gravityAccWrench = [\gravityAcc^\top, \tuple{0}_{1\times3}]^\top
%\\
% \sysForce &= \gyroForce + \sysDissMat\sysVel + \genForceStiff + \genForceGravity
%  \kineticEnergy &= \tfrac{1}{2} \norm[\sysInertiaMat]{\sysVel}^2
% \\
%  \dissFkt &= \tfrac{1}{2} \norm[\sysDissMat]{\sysVel}^2
% \\
%  \potentialStiff &= \sumBodiesAB \tfrac{1}{2} \norm[\bodyStiffMatp{\BidxI}{\BidxII}]{\bodyHomoCoord{\BidxI}{\BidxII}^\top - \bodyHomoCoordR{\BidxI}{\BidxII}^\top}^2
% \\
%  \potentialGravity &= \sumBodies \sProd[\bodyInertiaMatp{0}{\BidxII}]{\bodyHomoCoord{0}{\BidxII}^\top}{\wedOp(\gravityAccWrench)},
\end{align}
\end{subequations} 
\item explicit equations of motion
\begin{subequations}
\begin{align}
 \sysCoordd &= \kinMat \sysVel, 
\\
 \sysVeld &= \sysInertiaMat^{-1}(\genForceEx - \gyroForce - \sysDissMat\sysVel - \genForceStiff - \genForceGravity).
\end{align}
\end{subequations} 
\end{enumerate}

The first step \eqref{eq:RBSAlogJacobian} requires differentiation, so must be performed symbolically.
The remaining steps only require basic linear algebra, so can be preformed numerically.
For small systems it might be still reasonable to compute $\sysInertiaMat(\sysCoord)$ symbolically, but for larger systems the explicit expressions can be overwhelming even for contemporary computers.

