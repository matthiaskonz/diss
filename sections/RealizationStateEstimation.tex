\section{State estimation}\label{sec:RealizationStateEstimation}
Rigid body model
\begin{align}
 m (\rdd - \gravityAcc) + d \rd &= \RotMatQuat(\quat) (F + \FB)
\\
 \quatd = \kinMatQuat(\quat) \, \w, \qquad
 \Theta \dot{\w} + \wedOp(\w) \Theta \w &= \tau + \tauB
\end{align}
measurements
\begin{align}
 \rVIC &= \r,&
 \qVIC &= \quat
\\
 \wIMU &= \w + \wIMUN,&
 \aIMU &=  R^\top (\ddot{r} - \gravityAcc) + \aIMUB + \aIMUN
\end{align}

\begin{figure}[ht]
 \centering
 \input{graphics/StateEstimationBlock.pdf_tex}
 \caption{Overall structure of the state estimation}
 \label{fig:StateEstimationBlock}
\end{figure}

sadsadv \autoref{fig:StateEstimationBlock}.


\subsection{IMU model and misalignment correction}
Onboard the multicopters there is an inertial measurement unit (IMU), the model VN100 from \textsc{VectorNav}.
The IMU contains a 3-axis accelerometer, gyroscope and magnetometer connected to microcontroller which implements an extended Kalman-filter to estimate the attitude quaternion and gyroscope biases.

By construction the IMU has a small misalignment $\RIMUBR \in \SpecialOrthogonalGroup(3)$ to the body fixed frame.
Also the reference frame of the IMU is aligned with the gravity direction and the direction to magnetic north.
The lab frame $(\hx, \hy, \hz)$ is also aligned with the gravity $\gravityAcc = [0,0,-9.81]^\top$ but the $\hx$ and $\hy$ axis are aligned with the walls for convenience, yielding a misalignment $\RIMUBL \in \SpecialOrthogonalGroup(3)$.

Overall the relation of the IMU measurements $(\RIMU, \wIMU, \aIMU)$ and the real system variables $(\R, \w, \rdd)$ is assumed as
\begin{subequations}
\begin{align}
 \R &= \RIMUBL \RIMU \RIMUBR^\top
\\
 \omega &= \RIMUBR (\wIMU + \wIMUN)
\\
 \RT (\ddot{r} - \gravityAcc) &= \RIMUBR (\aIMU + \aIMUB + \aIMUN).
\end{align}
\end{subequations}

\paragraph*{Identification of mounting misalignment and accelerometer bias.}
For identification of the misalignment $\RIMUBR$ and the accelerometer bias $\aIMUB$ a multicopter is placed ($\r = \const$) in different random orientations $\R = \const$ while the accelerometer measurements $\aIMU$ are recorded.
Since the robot is not moving the acceleration that \textit{should} be measured is $a = -\R^\top \gravityAcc$.
We record measurements over $10\,\unit{s}$ ($2000$ samples) and take the mean values to eliminate the noise $\aIMUN$.
At the same time the attitude $\R = \RVIC$ captured with the Vicon system.
Between the measurements we have the relation
\begin{align}
 -\RVIC^\top \gravityAcc &= \RIMUBR (\aIMU + \aIMUB).
\label{eq:IdentEquationAIMUNonlin}
\end{align}
We assume that the misalignment $\RIMUBR$ is small, so 
\begin{align}
 \RIMUBR = \exp (\wedOp(\RPYIMUBR)) \approx \idMat[3] + \wedOp(\RPYIMUBR), \quad \RPYIMUBR \in \RealNum^3.
\end{align}
Now can write \eqref{eq:IdentEquationAIMUNonlin} in a form that is \textit{linear} in the parameters 
\begin{align}
 (-\RVIC^\top \gravityAcc)^\wedge \, \RPYIMUBR - \aIMUB &= \aIMU + \RVIC^\top \gravityAcc
\end{align}
and use a least squares approximation to identify values for $\RPYIMUBR$ and $\aIMUB$.

\begin{figure}[htb]
 \centering
 \includegraphics[scale=0.7]{AccBias.pdf}
 \caption{Accelerometer measurements and identification result}
 \label{fig:AccBias}
\end{figure}

For validation \autoref{fig:AccBias} displays the measured gravity coefficients $\gravityAcc$ within the 20 experiments with different orientations.
The red lines are the raw measurements $\gravityAcc = \RVIC \aIMU$, the green lines incorporate the identified values $\gravityAcc = \RVIC \RIMUBR (\aIMU + \aIMUB)$ and the black line is the reference value $\gravityAcc = [0,0,-9.81]^\top$.
Most notably one can see that even the magnitude $\norm{\gravityAcc}$ of the raw acceleration is off by about $5\%$ for some experiments whereas the compensated values are all within a window of about $0.2\%$.

\subsection{Angular velocity and torque bias}
The model for the rotational kinetics and the gyroscope measurement are
\begin{align}
 \Theta \dot{\w} + \wedOp(\w) \Theta \w = \tau + \tauB,
\qquad
 \wIMU = \w + \wIMUN.
\end{align}
Luenberger Observer
\begin{subequations}\label{eq:AngVelObs}
\begin{align}
 \wObs[\kPlusOne] &= \wObs[k] + \Ts \J^{-1} \big( \tau[k] + \tauBObs[k] - \wedOp(\wObs[k]) \Theta \wObs[k] \big) + L_{\w} (\wIMU[][k] - \wObs[k])
\\
 \tauBObs[\kPlusOne] &= \tauBObs[k] + L_\tau (\wIMU[][k] - \wObs[k])
\end{align}
\end{subequations}
% side calculation
% \begin{align}
%  -\wedOp(\w[k]) \Theta \w[k] + \wedOp(\wObs[k]) \Theta \wObs[k]
%  &= -\wedOp(\w[k]) \Theta \w[k] + \wedOp(\w[k] - e_\w[k]) \Theta (\w[k] - e_\w[k])
% \\
%  &= \wedOp(e_\w[k]) \Theta e_\w[k] - \wedOp(\w[k]) \Theta e_\w[k] - \wedOp(e_\w[k]) \Theta \w[k]
% \\
%  &= \wedOp(e_\w[k]) \Theta e_\w[k] + \big(\wedOp(\Theta \w[k]) - \wedOp(\w[k]) \Theta \big) e_\w[k]
% \end{align}
% \begin{align}
%  e_\w[\kPlusOne] &= e_\w[k] + \Ts e_\tau[k] - L_{\w} e_\w[k] - L_{\w} \wIMUN
% \nonumber\\
%  &\qquad \Ts \big( \wedOp(e_\w[k]) \Theta e_\w[k] + \big(\wedOp(\Theta \w[k]) - \wedOp(\w[k]) \Theta \big) e_\w[k] \big)
% \\
%  e_\tau[\kPlusOne] &= e_\tau[k] - L_\tau e_\w[k] - L_\tau \wIMUN
% \end{align}
Introducing the error quantities $e_{\w} = \w - \wObs$ and $e_\tau = \tauB - \tauBObs$ and subtracting the observer \eqref{eq:AngVelObs} from the discretized model we find the \textit{observer error dynamics}:
\begin{multline}\label{eq:AngVelObsErrorDyn}
 \begin{bmatrix} e_{\w}[\kPlusOne] \\ e_{\tau}[\kPlusOne] \end{bmatrix} 
 = \begin{bmatrix} 
  \idMat[3] - L_{\w} - S(\w[k]) & \Ts \J^{-1} \\
  - L_{\tau} & \idMat[3]
 \end{bmatrix}
 \begin{bmatrix} e_{\w}[k] \\ e_{\tau}[k] \end{bmatrix}
\\
 +
 \begin{bmatrix} \Ts \J^{-1} \wedOp(e_{\w}[k]) \Theta e_{\w}[k] \\ 0 \end{bmatrix} 
 -
 \begin{bmatrix} L_{\w} \\ L_{\tau} \end{bmatrix} 
 \wIMUN
\end{multline}
where
\begin{align}
 S(\w[k]) = \Ts \J^{-1} \big( \wedOp(\w[k]) \Theta - \wedOp(\Theta \w[k]) \big).
\end{align}
Note that the term quadratic in $e_{\w}$ is negligible if assuming small errors.
The term $S(\w[k])$ contains the time-variant part of the error dynamics.
However, for a reasonable gain $L_{\w}$ and typical working conditions (\eg $\norm{\w} < 10\,\sfrac{\unit{rad}}{\unit{s}}$) the gain $L_{\w}$ dominates over the time-variant part and the error dynamics are asymptotically stable.
Furthermore, for the linearization about the hover case, $\w \approx 0$, the matrix $S$ drops out completely.
This is used to compute the gains $L_{\w} = \diag(l_{\wx}, l_{\wy}, l_{\wz})$ and $L_{\tau} = \diag(l_{\taux}, l_{\tauy}, l_{\tauz})$ corresponding to given desired closed loop poles at this expansion point.


\subsection{Configuration measurement delay}
the configuration measurement $(\rVIC, \qVIC)$ is done by a \textsc{Vicon} motion capturing system.
This measurement relies on the optical tracking of reflective markers fixed on the multicopters by infrared cameras mounted at the walls of the lab and connected to the ground station PC.
The groundstation software runs a timer at $\TsVIC = 0.1\,\unit{ms}$ which reads the measurement, validates it and forwards it to the xBee radio module.
Finally the multicopter microcontroller receives the measurements through its ratio module.

\textsc{Matlab} is not a real-time system and the radio modules are transmitting and receiving other data at the same time.
Thus the frequency at which the configuration measurements are available to the multicopter is \textit{not} constant, but varies by several main sampling steps $\Ts$, see \autoref{fig:ViconDelayIllustrate} for actual measurements.
Furthermore the process is not performed instantaneously but takes some, again variable, time. 
Overall we assume a \textit{variable time delay} $\TVICD[k']$ for each measurement.

\begin{figure}[htb]
 \centering
 \input{graphics/ViconDelayClocks.pdf_tex}
 \caption{Illustration of the measurement time at the PC and when it is available on the microcontroller (MC)}
 \label{fig:ViconDelayClocks}
\end{figure}

To get an estimate $\TVICDObs[k']$ for this delay we utilize the clocks on the ground station PC, $t'$, and the microcontroller, $t$.
The clocks are assumed to be synchronous but have an unknown offset $\TPC = t - t' = \const$ depending on when each clock was started.
\autoref{fig:ViconDelayClocks} illustrates the time $\tVICp[k']$ at which a measurement was taken and the time $\tRec[k']$ at which it is received/available on the microcontroller.
Taking a large enough sample of these times we have the relation
\begin{align}\label{eq:meanTPC}
 \text{mean} (\tRec[1,\ldots,K'] - \tVICp[1,\ldots,K']) = \TPC + \text{mean}(\TVICD[1,\ldots,K']), \quad K'\gg 1.
\end{align}
Whereas $\TPC$ changes whenever the ground station or the micocontroller restarts, the average delay $\meanTVICD = \text{mean}(\TVICD[1,\ldots,K'])$ remains constant.

\paragraph{Experimental average delay identification.}
The Identification of the average delay $\meanTVICD$ is done in a dedicated experiment:
The vehicle is mounted on an incremental encoder that serves as a reference for the yaw-angle.
The encoder signal and the yaw-angle from the motion capture system are recorded by the multicopter microcontroller.
The experimental data here is about $1\,\unit{min}$ long and captures about $K' = 600$ \textsc{Vicon} frames.
Now we can plot the reference encoder angle $\yaw[k]$ at the sampling points $t = k\Ts$ and the yaw angle $\yawVIC[k']$ received from the Vicon system at the shifted time $\tVICp[k'] + \text{mean}(\tRec[1,\ldots,K'] - \tVICp[1,\ldots,K'])$.
If the previous assumption \eqref{eq:meanTPC} hold theses signals should be similar up to a shift of the mean delay $\meanTVICD$.

\begin{figure}[ht]
 \centering
 \includegraphics{ViconDelay/ViconDelayIdent.pdf}
 \caption{A small sample of the yaw-angle measurements (left) and the correlations (right) of the complete $60\,\unit{s}$ measurements for the identification of the average time delay}
 \label{fig:ViconDelayIdent}
\end{figure}

\autoref{fig:ViconDelayIdent} shows a small sample of the measured yaw angle and the correlations of the two measurements.
The identified average delay is where the cross-correlation has its maximum at $\meanTVICD = 0.034\,\unit{s}$.
In order to achieve this resolution the measurements were up-sampled to $1\,\unit{ms}$ by cubic interpolation.


\paragraph{Clock offset estimation.}
For the realtime implementation, \eqref{eq:meanTPC} is approximated by a slow ($\cTPC = 0.99$) low-pass filter
\begin{align}
 \TPCObs[k'] = \cTPC \TPCObs[k'\!-\!1] + (1-\cTPC) (\tRec[k'] - \tVICp[k'] - \meanTVICD)
\end{align}
yielding the estimate $\TPCObs$ for the constant clock offset $\TPC$.
Finally, the estimated delay of the $k'$-th Vicon measurement is
\begin{align}
 \TVICDObs[k'] = \tRec[k'] - \big( \tVICp[k'] + \TPCObs[k'] \big).
\end{align}
\autoref{fig:ViconDelayIllustrate} shows the estimated clock offset $\TPCObs$ and the resulting estimated delay $\TVICDObs$ for the previous example measurement.

\begin{figure}[ht]
 \centering
 \includegraphics{graphics/ViconDelay/ViconDelayIllustrate}                                                                    
 \caption{Measured timings}
 \label{fig:ViconDelayIllustrate}
\end{figure}

\subsection{Attitude}
The attitude measurement for the Vicon system is given in terms of an attitude quaternion $\qVIC$.
\fixme{delay}

\paragraph{Attitude quaternion.}
A unit quaternion is the quadruple $\quat = [\quatw, \quatx, \quaty, \quatz]^\top \in \RealNum^4$ with the geometric constraint 
\begin{align}
 \geoConstraint(\quat) = \norm{\quat}^2 - 1 = \quatw^2 + \quatx^2 + \quaty^2 + \quatz^2 - 1 = 0.
\end{align}
What makes it an attitude quaternion is the relation to the rotation matrix
\begin{align}
 \RotMatQuat(\quat) =
 \begin{bmatrix} 
  1 - 2(\quaty^{2} + \quatz^{2}) & 2(\quaty\quatx -\quatw\quatz)  & 2(\quatz\quatx + \quatw\quaty) \\
  2(\quaty\quatx + \quatw\quatz) & 1 - 2(\quatx^{2} + \quatz^{2}) & 2(\quaty\quatz - \quatw\quatx) \\
  2(\quatz\quatx - \quatw\quaty) & 2(\quaty\quatz + \quatw\quatx) & 1 - 2(\quatx^{2} + \quaty^{2})
 \end{bmatrix}
 \in \SpecialOrthogonalGroup(3)
 .
\end{align}
% Direct calculation 
% \begin{align}
%  \big( \RotMatQuat(\quat) \big)^\top \RotMatQuat(\quat) \overset{\norm{\quat} = 1}{=} \idMat[3],
% \qquad
%  \det \RotMatQuat(\quat) \overset{\norm{\quat} = 1}{=} +1
%  .
% \end{align}
The kinematic relation between the quaternion $\quat$ and the angular velocity $\w$ arises form
\begin{align}
 \tdiff{t} \big( \RotMatQuat(\quat) \big) &= \RotMatQuat(\quat) \wedOp(\w)&
 &\Leftrightarrow&
 \dot{q} &= \kinMatQuat(\quat) \w.
\end{align}
Adding a stabilization term for the constraint, motivated in \fixme{??}, the explicit relations are
\begin{align}\label{eq:QuaternionKinematics}
 \tdiff{t} \underbrace{\begin{bmatrix} \quatw \\ \quatx \\ \quaty \\ \quatz \end{bmatrix}}_{\quat}
 = \underbrace{\tfrac{1}{2} \begin{bmatrix} -\quatx & -\quaty & -\quatz \\ \quatw & -\quatz & \quaty \\ \quatz & \quatw & -\quatx \\ -\quaty & \quatx & \quatw \end{bmatrix}}_{\kinMatQuat(\quat)}
 \underbrace{\begin{bmatrix} \wx \\ \wy \\ \wz \end{bmatrix}}_{\w}
 -
 \underbrace{\tfrac{1}{2} \begin{bmatrix} \quatw \\ \quatx \\ \quaty \\ \quatz \end{bmatrix}}_{\left(\pdiff[\geoConstraint]{\quat}(\quat)\right)^+}
 \lambda\underbrace{(\norm{\quat}^2 - 1)}_{\geoConstraint(\quat)}
 .
\end{align}
The forward Euler discretization with the stabilization gain chosen as $\lambda=\sfrac{2}{\Ts}$ is
\begin{align}\label{eq:QuaternionKinematicsDiscrete}
 \quat[k\!+\!1] = \quat[k] + \Ts \kinMatQuat(\quat[k]) \w[k] - (\norm{\quat[k]}^2-1) \quat[k].
\end{align}
Note that with this particular gain the stabilization coincides with an approximation of the normalization
\begin{align*}
 \quat \mapsto \sfrac{\quat}{\norm{\quat}} \ \overset{\norm{\quat} \approx 1}{\approx} \ \quat - (\norm{\quat}^2-1) \quat.
\end{align*}
In contrast to the exact normalization, the approximation avoids computation of a square root and division.

\paragraph{Attitude integration.}
From the previous subsection we know the micro controller time $\tVIC[k'] = \tRec[k'] - \TVICDObs[k']$ at which the received attitude measurement $\qVIC[k']$ was taken.
To get the an estimate $\quatObs[k]$ for the current attitude we \textit{integrate} the attitude quaternion kinematics \eqref{eq:QuaternionKinematicsDiscrete} using the measurement $\qVIC[k']$ as initial value and the (ring) buffered angular velocity estimates $\wObs[k-\text{floor}(\sfrac{\TVICDObs[k']}{\Ts})],\ldots, \wObs[k]$.
Since the delay $\TVICDObs[k']$ is usually not a multiple of the micro controller sampling time $\Ts$ a linear interpolation of the angular velocity $\wObs(\tVIC[k'])$ is used.


\subsection{Position, velocity and accelerometer bias}
For the translational dynamics we have a similar situation as for the attitude kinematics:
We have a down-sampled and delayed position measurement $\rVIC$ and a biased and noisy measurement of the acceleration $\aIMU$.
From this we want to estimate the current position $\r$ and velocity $\rd$ required for the controller implementation.

The relation between position and accelerometer measurement and the differential equation for a constant accelerometer bias are
% \begin{align}
%  \rdd = \gravityAcc + \R (\aIMU - \aIMUB - \aIMUN),
% \qquad
%  \aIMUBd = 0
% \end{align}
% A discrete approximation for this is
% \begin{multline}\label{eq:AccelerometerDynamicsDiscrete}
%  \underbrace{\begin{bmatrix} \r[\kPlusOne] \\ \rd[\kPlusOne] \\ \aIMUB[\kPlusOne] \end{bmatrix}}_{z[\kPlusOne]}
%  =
%  \underbrace{\begin{bmatrix} \idMat[3] & \Ts\idMat[3] & -\tfrac{\Ts^2}{2}\R[k] \\ 0 & \idMat[3] & -\Ts\R[k] \\ 0 & 0 & \idMat[3] \end{bmatrix}}_{A[k]}
%  \underbrace{\begin{bmatrix} \r[k] \\ \rd[k] \\ \aIMUB[k] \end{bmatrix}}_{z[k]}
%  +
%  \underbrace{\begin{bmatrix} \tfrac{\Ts^2}{2}\idMat[3] \\ \Ts \idMat[3] \\ 0 \end{bmatrix}}_{B}
%  \big(\underbrace{\gravityAcc + \R[k] \aIMU[k]}_{u[k]} - \underbrace{R[k] \aIMUN[k]}_{u_{\mathsf{N}}[k]} \big)
% % -
% % \underbrace{\begin{bmatrix} \tfrac{\Ts^2}{2}\idMat[3] \\ \Ts \idMat[3] \\ 0 \end{bmatrix} \R[k] \aIMUN[k]}_{u_{\mathsf{N}}[k]}
% \end{multline}
% Note that this approximation assumes that the attitude $\R$ is constant within one sampling step.
% If this would be the case, the discretization would be exact.
\begin{align}
 \aIMU = \R^\top(\rdd - \gravityAcc) + \aIMUB + \aIMUN,
\qquad
 \aIMUBd = 0
\end{align}
Introducing the state $z = [\r^\top, \rd^\top, \aIMUB^\top]^\top$ and neglecting the accelerometer noise $\aIMUN=0$, this can be written as
\begin{align}\label{eq:AccelerometerDynamicsState}
 \tdiff{t}
 \underbrace{\begin{bmatrix} \r \\ \rd \\ \aIMUB \end{bmatrix}}_{z}
 &=
 \underbrace{\begin{bmatrix} 0 & \idMat[3] & 0 \\ 0 & 0 & -\R \\ 0 & 0 & 0 \end{bmatrix}}_{A}
 \underbrace{\begin{bmatrix} \r \\ \rd \\ \aIMUB \end{bmatrix}}_{z}
 +
 \underbrace{\begin{bmatrix} 0 \\ \idMat[3] \\ 0 \end{bmatrix}}_{B}
 \underbrace{(\gravityAcc + \R \aIMU)}_{u}
\end{align}

\paragraph{Estimator implementation.}
The implemented estimation algorithm on the \microcontroller works as follows:
After having received 2 subsequent position measurements $\rVIC[1]$, $\rVIC[2]$ and their reconstructed time $\tVIC[1]$, $\tVIC[2]$ with $\Ts'[1]=\tVIC[2]-\tVIC[1]$ we set the initial value for the estimator state $\hat{z} = [\rObs^\top, \rdObs^\top, \aIMUBObs^\top]^\top$ as
\begin{align}
 \rObs(\tVIC[2]) = \rVIC[2], 
\qquad
 \rdObs(\tVIC[2]) = \tfrac{\rVIC[2]-\rVIC[1]}{\Ts'[1]}
\qquad
 \aIMUB(\tVIC[2]) = 0.
\end{align}
Then a discrete approximation of the accelerometer dynamics \eqref{eq:AccelerometerDynamicsState}:
\begin{align}\label{eq:PosPredictor}
 \underbrace{\begin{bmatrix} \rObs[\kPlusOne] \\ \rdObs[\kPlusOne] \\ \aIMUBObs[\kPlusOne] \end{bmatrix}}_{\hat{z}[\kPlusOne]}
 =
 \underbrace{\begin{bmatrix} \idMat[3] & \Ts\idMat[3] & -\tfrac{\Ts^2}{2}\RObs[k] \\ 0 & \idMat[3] & -\Ts\RObs[k] \\ 0 & 0 & \idMat[3] \end{bmatrix}}_{\hat{A}_\idxSamp[k]}
 \underbrace{\begin{bmatrix} \rObs[k] \\ \rdObs[k] \\ \aIMUBObs[k] \end{bmatrix}}_{\hat{z}[k]}
 +
 \underbrace{\begin{bmatrix} \tfrac{\Ts^2}{2}\idMat[3] \\ \Ts \idMat[3] \\ 0 \end{bmatrix}}_{B_\idxSamp}
 \underbrace{(\gravityAcc + \RObs[k] \aIMU[k])}_{\hat{u}[k]}
\end{align}
is used to integrate the current position and velocity.
Note that the measurement time $\tVIC[2]$ usually does not coincide with a sampling step, see \autoref{fig:ViconDelayClocks}.
Consequently the value $\Ts$ in \eqref{eq:PosPredictor} has to be adjusted to match the time from the initial value to the subsequent sampling step.
Furthermore the accelerometer measurement $\aIMU$ is linearly interpolated to approximate the acceleration at the initial time $\tVIC[2]$.
The attitude $\R$ at the initial point is $\RotMatQuat(\qVIC[2])$ which is received simultaneously with $\rVIC[2]$.

After this initial procedure, \eqref{eq:PosPredictor} is computed once every sampling step with the most recent accelerometer measurement $\aIMU[k]$ yielding the most recent predictions for the position and velocity.
The position estimates $\rObs[k]$ for the last 64 sampling steps are stored in a ring buffer.

When a new configuration measurement $(\rVIC[k'], \RVIC[k'])$ and its timestamp $\tVIC[k']$ are available, the corresponding estimate $\rObs(\tVIC[k'])$ is interpolated.
This is used to correct the estimate \textit{at the measurement time} $\tVIC[k']$, i.e.\
\begin{align}\label{eq:PosCorrector}
 \underbrace{\begin{bmatrix} \rObs(\tVIC[k']) \\ \rdObs(\tVIC[k']) \\ \aIMUBObs(\tVIC[k']) \end{bmatrix}}_{\hat{z}(\tVIC[k'])}
 \leftarrow 
 \underbrace{\begin{bmatrix} \rObs(\tVIC[k']) \\ \rdObs(\tVIC[k']) \\ \aIMUBObs(\tVIC[k']) \end{bmatrix}}_{\hat{z}(\tVIC[k'])}
 +
 \underbrace{\begin{bmatrix} L_r[k'] \\ L_v[k'] \\ L_a[k'] (\RVIC[k'])^\top \end{bmatrix}}_{L[k']}
 (\rVIC[k'] - \rObs(\tVIC[k'])).
\end{align}
After the correction the prediction for the current sampling step is again integrated from \eqref{eq:PosPredictor}.
Note that this also requires a (ring) buffer for the last accelerometer measurements.

\paragraph{Error dynamics and tuning.}
To chose reasonable correction gains $L_r$, $L_v$ and $L_a$ we investigate the resulting error dynamics of the proposed estimator.
Here we do the coarse assumption that the attitude is constant and perfectly known $\RObs = \bar{\R}$.
Then the dynamic matrix $A$ in the continuous model \eqref{eq:AccelerometerDynamicsState} is constant and is denoted by $\bar{A}$.

From measurement time $\tVIC[k']$ till the next one at $\tVIC[k'+1]$ there are several prediction steps \eqref{eq:PosPredictor}:
The first from the measurement time till the subsequent sampling step with integration time $T_\mathsf{I}$.
Then several steps with integration time $\Ts$ till the sampling step before $\tVIC[k'+1]$, and finally a step integration time $T_\mathsf{F}$ till $\tVIC[k'+1]$.
Overall this results in the dynamic matrix 
\begin{align}
 \bar{A}'_\idxSamp = \exp(\bar{A}\,T_\mathsf{I}) \exp(\bar{A}\,\Ts) \cdots \exp(\bar{A}\,\Ts) \exp(\bar{A}\,T_\mathsf{F}) = \exp(\bar{A}\,\Tsp), 
\end{align}
where $\Tsp[k'] = T_\mathsf{I}[k'] + \Ts + \ldots + \Ts + T_\mathsf{F}[k'] = \tVIC[k'] - \tVIC[k'-1]$.
Since $\Tsp[k']$ is known and exploiting this property of the exponential map, we do not need to worry about the different integration times of the predictions.

Finally we add one correction step \eqref{eq:PosCorrector} and subtract the estimator from the discretization of the model \eqref{eq:AccelerometerDynamicsState} to obtain the relation for the error $e = z - \hat{z}$ at one correction/measurement time $\tVIC[k']$ to the next one at $\tVIC[k'+1]$ as
\begin{align}
 e(\tVIC[k'\!+\!1]) = \underbrace{(\idMat[9] - \bar{L}[k'] C) \bar{A}'_\idxSamp[k']}_{\bar{A}'_e[k']} e(\tVIC[k'])
\end{align}
where
\begin{align}
 \bar{A}'_\idxSamp[k'] &=
 \begin{bmatrix}
  \idMat[3] & \Tsp[k'] \idMat[3] & -\tfrac{(\Tsp[k'])^2}{2} \bar{\R} \\
  0 & \idMat[3] & -\Tsp[k']\bar{\R} \\
  0 & 0 & \idMat[3]
 \end{bmatrix},&
 \bar{L}[k'] &= \begin{bmatrix} L_r[k'] \\ L_v[k'] \\ L_a[k'] \bar{\R}^\top \end{bmatrix},&
 C &= [\idMat[3] \ 0 \ 0]
\end{align}
and finally
\begin{align}
 \bar{A}_e[k'] = 
 \begin{bmatrix} 
  \idMat[3] - L_r[k'] & \Tsp(\idMat[3] - L_r[k']) & \tfrac{(\Tsp[k'])^2}{2}(L_r[k'] - \idMat[3]) \bar{\R} \\
  -L_v[k'] & \idMat[3] - \Tsp[k'] L_v[k'] & \Ts(\tfrac{\Tsp[k']}{2} L_v[k'] - \idMat[3])\bar{\R} \\
  -L_a[k'] \bar{\R}^\top & -\Tsp[k'] L_a[k'] \bar{\R}^\top & \idMat[3] + \tfrac{\Tsp[k']^2}{2} L_a[k']
 \end{bmatrix}
%  \begin{bmatrix} 
%   (1-l_r)\idMat[3] & \Ts'(1 - l_r)\idMat[3] & \tfrac{\Ts'^2}{2}(l_r - 1) \bar{\R} \\
%   -l_v \idMat[3] & (1-\Ts'l_v)\idMat[3] & \Ts(\tfrac{\Ts'}{2}l_v - 1) \bar{\R} \\
%   -l_a \bar{\R}^\top & -\Ts' l_a \bar{R}^\top & (1 + \tfrac{\Ts'^2}{2} l_a) \idMat[3]
%  \end{bmatrix}
\end{align}
Scaling the gains with the variable integration time $\Tsp[k']$ between two correction steps as
\begin{align}
 L_r[k'] &= l_r \idMat[3],& 
 L_v[k'] &= \tfrac{l_v}{\Tsp[k']} \idMat[3],&
 L_a[k'] &= \tfrac{2 l_a}{(\Tsp[k'])^2}
\end{align}
yields the characteristic polynomial with constant coefficients
\begin{align}
 \det(\lambda \idMat[9] - \bar{A}'_e[k'])
 = \Big( \lambda^3
  + \big(l_r + l'_v - l'_a - 3 \big) \lambda^2
  + \big(3 - 2 l_r - l'_v - l'_a \big) \lambda
  + l_r - 1
 \Big)^3.
\end{align}
However, it should be noted here that constant eigenvalues of the time varying matrix $\bar{A}'_e[k']$ can not conclude stability of the time-varying system.

In practice the eigenvalues $\eig(\bar{A}'_e) = \{ 0, 0.5, 0.9 \}$ and the resulting gains $l_r = 1$, $l_v = 0.575$, $l_a = -0.025$ did result in quite good estimation performance.
These rather fast eigenvalues also reflect a high confidence in the position measurement.


\subsection{Force bias}
Recall the translational \Multicopter dynamics and the accelerometer measurement $\aIMU$ equation
\begin{align}
 m (\rdd - \gravityAcc) + d \rd &= \R (F + \FB),&
 \aIMU &= \RT (\ddot{r} - \gravityAcc) + \aIMUN + \aIMUB.
\end{align}
Solving these equations for the force bias $\FB$ and elimination of the acceleration $\rdd$ yields
\begin{align}
 \FB &= m (\aIMU - \aIMUN - \aIMUB) + d \RT \dot{r} - F.
\end{align}
From the previous subsections we have estimates for the attitude $\RObs$, velocity $\rdObs$ and the accelerometer bias $\aIMUBObs$.
From the previous section on the actuator dynamics we have a an estimate for the \Multicopter force $\FObs$.
However, the accelerometer measurement $\aIMU$ contains the significant, unknown noise $\aIMUN$.
In order to suppress this noise we use a simple low-pass to obtain the estimate $\FBObs$ for the force bias
\begin{align}
 \FBObs[\kPlusOne] = l_F \FBObs[k] + (1 - l_F) \big( m (\aIMU[k] - \aIMUBObs[k]) + d (\RObs[k])^\top \rdObs[k] - \FObs[k] \big)
 .
\end{align}
In practice the low-pass gain $l_F = 0.98$ did yield reasonable results.
